Copyright \copyright{} 2002, 2010 Sam Williams\\
Copyright \copyright{} 2010 Richard M. Stallman\\
2011, 2012 Theo Walm

Permission is granted to copy, distribute and/or modify
this document under the terms of the GNU Free Documentation License,
Version 1.3 or any later version published by the Free Software
Foundation; with no Invariant Sections, no Front-Cover Texts, and no
Back-Cover Texts. A copy of the license is included in the section
entitled "`GNU Free Documentation License"'.

\comment{
	\bigskip

	Published by the Free Software Foundation\\
	51 Franklin St., Fifth Floor\\
	Boston, MA 02110-1335\\
	USA\\
	ISBN: 9780983159216\\
}
\bigskip

\ifnum\commercial=1
	Das Titelbild basiert auf einem flickr-Photo von \href{http://www.flickr.com/photos/redjar/474652905/}{redjar} und steht unter der Creative-Commons-Lizenz Attribution-ShareAlike 2.0 Generic (CC BY-SA 2.0).
\else
Das Titelbild stammt von Matías Subat und darf in Verbindung mit diesem Werk nichtkommerziell genutzt werden.
\fi

Das Photo der PDP-10 in Kapitel 7 stammt von Rodney Brooks. Das Photo von St. IGNUcius in Kapitel 8 stammt von Stian Eikeland. 

\subsection*{Abstammungshistorie}
\begin{itemize}
\item
\textit{Free as in Freedom: Richard Stallman's Crusade for Free Software}, 2002, Sam Williams, erschienen bei O'Reilly

\item
\textit{Free as in Freedom 2.0: Richard Stallman and the Free Software Revolution}, 2010, Richard Stallman, Sam Williams, erschienen bei GNU Press, \LaTeX-Quellcode unter \url{http://www.fsf.org/faif}

\item
\textit{Frei wie in Freiheit - Richard Stallmans Kreuzzug für freie Software, 2011, Theo Walm}, \LaTeX-Quellcode unter \url{http://github.com/yadayada/faif}

\end{itemize}

Die deutsche Ausgabe ist um das Vorwort und das Nachwort von Sam Willams sowie das Kapitel "`A Brief Journey through Hacker Hell"' gekürzt.

\subsection*{Danksagung}

Vielen Dank an Christian Mantey für die eingesandten Hinweise auf die Textfehler.
