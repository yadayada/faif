%\chapter{Es war einmal ein Drucker}
\chapter{Am Anfang war der Drucker}
\pagenumbering{arabic}

\begin{quotation}
Ich fürchte die Danaer, auch wenn sie Geschenke bringen.

- Vergils \textit{Aeneid}
\end{quotation}

% TODO: Jahreszahl herausfinden
\comment{Das AI Lab und das Laboratory for Computer Science wurden 2003 zum CSAIL zusammengelegt.}

Der neue Drucker hatte schon wieder einen Papierstau.
Richard M. Stallman, angestellter Programmierer am Artificial Intelligence Laboratory (AI~Lab) des Massachusetts Institute of Technology, sollte es noch herausfinden. 

Eine Stunde, nachdem er eine 50seitige Datei an den Laserdrucker geschickt hatte, unterbrach Stallman, 27, seine Arbeit, um sein Dokument holen zu gehen. Bei der Ankunft lagen nur vier Seiten in der Papierausgabe. Und dazu kam, dass die vier Seiten nicht einmal zu seiner gedruckten Datei gehörten. Stallmans und der Rest des Druckauftrags eines anderen waren immer noch in der Kabelage des Netzwerks gefangen.

Auf Geräte warten zu müssen ist ein Berufsrisiko, wenn man Programmierer ist, also nahm Stallman es hin. Trotzdem ist der Unterschied zwischen dem Warten auf ein Gerät und dem Warten an einem Gerät ein beträchtlicher. Es war nicht das erste Mal, dass er am Drucker stehen und den Ausdruck jeder einzelnen Seite überwachen musste. Als jemand, der den Großteil seiner Tage und Nächte damit verbringt, die Leistung von Geräten und die sie steuernde Software zu verbessern, fühlte Stallman den Drang, das Gerät zu öffnen, sich seine Innereien anzusehen und das Übel an seiner Wurzel zu packen.

Leider erstreckten sich Stallmans programmiererische Fähigkeiten nicht in den Bereich des Maschinenbaus. Während frische Ausdrucke aus dem Gerät schossen, hatte Stallman Gelegenheit, über Wege nachzudenken, wie man das Papierstauproblem lösen konnte. 

Wie lange war es her gewesen, dass die Angestellten des AI Labs den neuen Drucker in Empfang genommen hatten, fragte sich Stallman. Das Gerät war eine Schenkung der Xerox Corporation. Ein innovativer Prototyp, ein umgebauter Xerox-Photokopierer. Anstatt physisch Kopien zu machen, empfing er Daten über das Netzwerk und verwandelte sie in professionell aussehende Dokumente. Er war von den Ingenieuren in Xerox' weltbekannter Forschungseinrichtung in Palo Alto (PARC) entworfen worden, und war einfach gesagt ein Vorgeschmack auf die Revolution des Desktop-Printings, die den Rest der Computerindustrie Ende des Dezenniums erfassen würde.

%alter Druckertyp aus einem Interview in "Making It Big in Software"
Getrieben vom instinktiven Drang, an der neuen Technik herumzuspielen, hatten die Programmierer am AI~Lab das neue Gerät prompt in die ausgeklügelte Computerinfrastruktur integriert. Die Ergebnisse waren sofort zufriedenstellend. Der neue Xerox-Drucker war schnell, ganz im Gegensatz zum alten Labordrucker, einem XeroGraphic Printer. Die Seiten kamen im Sekundentakt aus dem Drucker, wodurch ein ehemals 20minütiger Druckauftrag nur noch 2 Minuten dauerte. Das neue Gerät war außerdem präziser. Ausgedruckte Kreise sahen auch so aus wie Kreise und nicht oval; gerade Linien waren gerade Linien, keine flachen Sinusschwingungen.

Er war, alles in allem, ein zu gutes Geschenk, um ihn auszuschlagen.

Als das Gerät dann im Einsatz war, kamen seine Mängel zum Vorschein. Der größte Nachteil war seine Anfälligkeit für Papierstaus. Die technisch denkenden Programmierer verstanden schnell den Grund hinter diesem Fehler. Als Photokopierer bedurfte das Gerät generell die Überwachung durch einen Bediener. Unter der Annahme, diese Bediener wären immer vor Ort, um Papierstaus zu beheben, falls einer auftritt, widmeten die Ingenieure bei Xerox ihre Zeit der Behebung anderer Probleme.\comment{In der engineering terms, user diligence was builtinto the system.}

Durch die Wandlung des Geräts in einen  Drucker hatten die Ingenieure bei Xerox die Mensch-Maschine-Beziehung auf eine subtile, aber profunde Art verändert. Statt einem einzelnen menschlichen Bediener zu gehorchen, war er nun Knecht aller Benutzer im gesamten Netzwerk. Statt direkt am Gerät zu stehen, schickte nun ein Benutzer am einen Ende des Netzwerks seinen Druckauftrag über eine Eimerkette aus Rechnern und erwartete, dass die gewünschten Daten an der richtigen Stelle in korrekter Form eintreffen.\comment{Erst wenn jemand schließlich das Resultat holen wollte, würde er merken, wie wenig wirklich gedruckt wurde.}

Stallman war wohl kaum der einzige Bewohner des AI~Labs, der das Problem erkannte, aber er dachte auch an eine Abhilfe. Jahre zuvor hatte Stallman ein ähnliches Problem bei dem alten Drucker behoben, indem er die Software auf einer kleinen PDP-11 abänderte, die den Drucker gesteuert hat, und das Incompatible Timesharing System, das auf dem PDP-10-Hauptrechner lief. Stallman konnte die Papierstaus nicht verhindern, aber er konnte Code schreiben, mit dem die PDP-11 den Drucker regelmäßig abfragt und der den Papierstau an die PDP-10 meldet. Stallman schrieb außerdem Code für die PDP-10, der jedem Nutzer mit einem ausstehenden Druckauftrag eine Nachricht schickt, dass der Drucker Papierstau hat. Die Nachricht war simpel, in etwa "`Der Drucker ist verklemmt, bitte beheben."' Und weil sie an die Leute geschickt wurde, die das größte Interesse daran hatten, das Problem zu beheben, standen die Chancen nicht schlecht, dass das Problem alsbald aus der Welt geschafft würde.

Stallmans Fehlerkorrektur war – wie sie es nun einmal an sich haben – indirekt, aber elegant. Sie brachte den mechanischen Fehlstand nicht wieder in Ordnung, aber durchbrach zumindest die Informationsbarriere zwischen Benutzer und Gerät. Dank einiger Zeilen Code konnten die AI-Lab-Angestellten sich die 10 bis 15 Minuten jede Woche mit dem Hin- und Herlaufen zum Prüfen des Druckers sparen. 
Stallmans Fehlerbehebung nutzte die kollektive Intelligenz des Netzwerks aus. "`Wenn man die Nachricht bekommen hat, konnte man nicht davon ausgehen, dass jemand anders sich darum kümmern würde"', erinnert sich Stallman an die Logik dahinter. "`Man musste zum Drucker gehen. Ein, zwei Minuten, nachdem der Drucker Schwierigkeiten machte, standen zwei, drei Leute, die die Nachricht bekommen hatten, am Drucker, um den Stau zu beheben. Von den zwei oder drei Leuten wusste meist mindestens einer, wie man das Problem behebt."'

Solche cleveren Fehlerausbügelungen waren ein Markenzeichen des AI Labs und seiner Ureinwohnerschaft aus Programmierern. Allerdings lehnten die besten Programmierer am AI~Lab den Begriff "`Programmierer"' ab und bevorzugten stattdessen die saloppe Berufsbezeichnung "`Hacker"'. Die Bezeichnung deckte eine Fülle an Tätigkeiten ab – vom kreativen Klamauk bis hin zur Verbesserung bestehender Software und Computersysteme. In der Bezeichnung schwingt auch etwas von einer Yankee-Bauernschläue mit.
Für einen Hacker ist das Schreiben von funktionierender Software nur der Anfang. Ein Hacker würde versuchen, seine eigene Finesse zum Ausdruck zu bringen (und andere Hacker zu beeindrucken), indem er eine zusätzliche Herausforderung angeht: das Programm besonders schnell, klein, mächtig, elegant oder eindrucksvoll zu machen.\footnote{Mehr zum Begriff "`Hacker"' gibt es im \nameref{Anhang A}.}

Bei Unternehmen wie Xerox war es gängige Praxis, ihre Produkte (und Software) an Institutionen zu verschenken, an denen Hacker zusammenkommen. Wenn Hacker ihre Produkte nutzten, würden sie vielleicht später für ihr Unternehmen arbeiten. In den 60ern und 70ern entwickelten sie auch gelegentlich Programme, die nützlich für den Hersteller waren und an andere Kunden ausgeliefert werden konnten. 

Als Stallman beim Xerox-Laserdrucker die Neigung zu Papierstaus bemerkte, dachte er daran, die alte Fehlerkorrektur – den "`Hack"' – bei ihm anzuwenden. Als er sich jedoch in Folge die Druckersoftware ansehen wollte, machte Stallman eine unschöne Entdeckung. Es gab für den Drucker keine Software, oder jedenfalls keine, die Stallman oder einer seiner Kollegen lesen konnte. Bis dahin gehörte es für die meisten Firmen zum guten Ton, ihre Quellcodedateien zu veröffentlichten, lesbare Textdateien, die die einzelnen Maschinenbefehle dokumentieren. Xerox hatte in diesem Fall nur die Software in kompilierter, also Binärform ausgeliefert. Wenn sich ein Programmierer diese Dateien ansieht, sieht er nur endlose Kolonnen aus Nullen und Einsen, chinesisch.

Es gibt Programme, sogenannte "`Disassembler"', die die Nullen und Einsen in maschinennahe Befehle übersetzen, aber herauszufinden, was diese Befehle nun  wirklich "`tun"', ist eine langwierige und komplizierte Aufgabe, bekannt als "`Reverse engineering"'. Dieses Programm zu reverse-engineeren, hätte gut mehr Zeitaufwand benötigt als 5 Jahre Drucken mit gelegentlichen Papierstaus. So frustriert war Stallman nicht deswegen und tat das Problem ab. 

Xerox' unfreundliche Firmenpolitik stand im krassen Gegensatz zu den üblichen Gepflogenheiten in der Hackergemeinschaft. Zum Beispiel brauchte das AI~Lab zur Entwicklung des Programms für die PDP-11, die den alten Drucker steuerte, und des Programms für eine andere PDP-11, die die Bildschirmterminals handhabt, zum Bauen der PDP-11-Programme auf dem PDP-10-Hauptrechner einen Cross-Assembler. Die Hacker im Lab hätten einen schreiben können, aber Stallman, ein Harvard-Student, fand ein solches Programm auf einem Rechner im Computerlabor von Harvard. Das Programm war für den gleichen Rechnertyp geschrieben, die PDP-10, aber für ein anderes Betriebssystem. Stallman hat nie herausgefunden, wer das Programm geschrieben hatte, weil es nicht im Quellcode stand. Aber er brachte eine Kopie mit ins AI Lab. Er änderte den Quellcode, damit es auf dem Incompatible Timesharing System (ITS)\index{ITS|(} lief, das im AI~Lab eingesetzt wurde. Ohne große Schwierigkeiten konnte das AI~Lab so an das Programm gelangen, das es für seine Softwareinfrastruktur benötigte. Stallman fügte sogar einige neue Funktionen hinzu, die es in der Ursprungsversion nicht gab. "`Wir haben es letztendlich mehrere Jahre eingesetzt"', sagt Stallman.

Aus der Sicht eines Programmierers der 70er war dieser Vorgang nichts anderes als wie ein vorbeikommender Nachbar, der sich ein Elektrowerkzeug oder eine Tasse Zucker von einem borgt. Der einzige Unterschied war, dass Stallman mit dem Kopieren der Software für das AI~Lab keinem anderen die Nutzung des Programms unmöglich machte. Wenn überhaupt, profitierten andere Hacker von dem Vorgang, weil Stallman es um neue Funktionalität ergänzt hatte, die andere Hacker gerne rückübernehmen konnten. Stallman erinnert sich zum Beispiel an einen Programmierer einer privaten Firma, Bolt, Beranek \& Newman, der sich das Programm entlehnt hat. Er brachte es auf Twenex zum Laufen und erweiterte es um einige Funktionen, die Stallman wiederum schließlich in den AI-Lab-Quellcode eingepflegt hat. Zwei der Programmierer hatten sich entschlossen, zusammen eine gemeinsame Version zu verwalten, die auf ITS\index{ITS|)} und Twenex läuft.

"`Ein Programm entwickelte sich so, wie sich eine Stadt entwickelt"', sagt Stallman, wenn er sich an die Infrastruktur des AI~Labs erinnert. "`Einige Teile wurden ausgetauscht und erneuert. Neue Dinge wurden hinzugefügt. Aber man konnte sich immer einen bestimmten Teil ansehen und sagen, \glq Hmm, dem Stil nach wurde dieser Teil in den frühen 60ern geschrieben und dieser Teil Mitte der 70er.\grq\,"'

Mit diesem System der geistigen Akkumulierung erschufen die Hacker am AI~Lab und anderswo robuste Werke. Nicht jeder Programmierer, der an dieser Kultur teilhatte, bezeichnete sich selbst als Hacker, aber die meisten teilten die Ansichten Stallmans. Wenn ein Programm oder ein Patch die eigenen Probleme löst, warum sollte es dann nicht auch die der anderen lösen können. Warum sollte man nicht selbstlos mit anderen teilen?

Dieses Kooperationssystem wurde durch kommerzielle Geheimhaltung und Gier unterminiert, was zu seltsamen Kombinationen aus Geheimhaltung und Zusammenarbeit führte. Zum Beispiel hatten die Informatiker an der UC Berkeley ein mächtiges Betriebssystem namens BSD geschrieben, das auf dem Unix-System basierte, das sie von AT\&T erworben hatten. Berkeley stellte BSD für den Preis einer Kopie auf Band zur Verfügung, gab solche Bänder aber nur an Schulen heraus, die eine für 50.000\$ gekaufte Lizenz von AT\&T vorweisen konnten. Die Berkeley-Hacker stellten weiterhin alles soweit zur Verfügung, wie AT\&T es zuließ, aber sie erkannten keinen Konflikt zwischen diesen beiden Verhaltensweisen.

Ebenso störte es Stallman, dass Xerox den Quellcode nicht mitgeliefert hatte, aber er war noch nicht wütend. Er hatte nie daran gedacht, Xerox nach dem Quellcode zu fragen. "`Sie hatten uns schon den Laserdrucker geschenkt"', sagt Stallman. "`Sie waren uns nicht mehr schuldig. Außerdem bin ich davon ausgegangen, dass das Fehlen des Quellcodes Ausdruck einer wissentlichen Entscheidung war, und sie darum zu bitten aussichtslos gewesen wäre."' 

Aber schließlich gab es gute Nachrichten: gerüchteweise hieß es, dass ein Wissenschaftler an der Informatik-Fakultät der Carnegie Mellon University den Quellcode zum Laserdrucker hatte. Der Bezug zur Carnegie Mellon verhieß nichts Gutes. 1979 hatte Brian Reid\index{Reid, Brian}, Doktorand an dieser Universität, die Gemeinschaft brüskiert, als er sich weigerte, sein Textformatierungsprogramm Scribe\index{Scribe|(} offenzulegen. Seine Formatierungssprache war die erste, die Formatierungsbefehle hatte, die an der gewünschten Semantik orientiert waren (wie etwa "`dieses Wort hervorheben"' oder "`dieser Absatz ist ein Zitat"'), anstatt konkrete Formatierungsdetails zu verwenden ("`dieses Wort kursiv setzen"' oder "`die Ränder dieses Absatzes verschmälern"').
Reid verkaufte Scribe an Unilogic, ein Softwareunternehmen aus dem Raum Pittsburgh. Zum Ende seiner Zeit als Doktorand sagte Reid, er wollte einfach einen Weg finden, wie er das Programm an Programmierer übergeben konnte, ohne dass es in die Gemeinfreiheit rutscht. (Warum jemand einen solchen Ausgang als besonders unerwünscht ansehen sollte, bleibt unklar.) Dem ganzen setzte Reid noch das Sahnehäubchen auf, als er dem Einbau zeitgesteuerter Funktionen zustimmte – von Programmierern "`Zeitbomben"' genannt – die die Gratiskopien des Programms nach einem Verfallsfrist von 90 Tagen deaktivieren. Um die Deaktivierung zu verhindern, würden Nutzer das Softwareunternehmen bezahlen, das dann einen Schlüssel herausgab, der die interne Zeitbomben-Antifunktion entschärfen konnte.

Für Stallman was das schlichtweg Verrat am Programmiererethos. Statt die Idee des Teilens und Weitergebens zu ehren, hatte Reid einen Weg für Unternehmen geschaffen, Programmierern für den Zugang zu Informationen Geld abzunötigen. Aber er dachte damals nicht weiter über dieses Problem nach, weil er Scribe nur selten benutzte.

Unilogic stellte dem AI Lab eine Gratisversion zur Verfügung, ließ aber die Zeitbombe aktiviert und erwähnte sie mit keinem Wort. Die Software funktionierte eine Zeit lang; dann kam eines Tages ein Bericht von einem Nutzer, dass Scribe\index{Scribe|)} nicht mehr funktioniert. Der Hacker Howard Cannon verbrachte einige Stunden mit dem Debuggen der Binärdatei, bis er die Zeitbombe fand und patchte sie heraus. Cannon war empört und machte keinen Hehl daraus. Er erzählte anderen Hackern, wie sauer er war, dass Unilogic seine Zeit mit einem absichtlichen Bug verschwendet hatte.

Stallman besuchte aus beruflichen Gründen einige Monate später den Campus der Carnegie Mellon. Während seines Aufenthalts wollte er die Person ausfindig machen, die angeblich den Quellcode zu der Druckersoftware hatte. Wie der Zufall so spielte, war der Mann gerade in seinem Büro. In bester Ingenieursmanier war das Gespräch höflich, aber direkt. Nachdem er sich kurz als Besucher vom MIT vorgestellt hatte, fragte Stallman nach dem Quellcode für den Laserdrucker, den er ändern wollte. Zu seinem Ärger verweigerte der Forscher die Herausgabe. "`Er sagte mir, er hätte versprochen, mir keine Kopie zu geben"', so Stallman. 

Das Gedächtnis ist ein seltsames Sieb\comment{seltsam' Ding}. Zwanzig Jahre nach dem Ereignis ist Stallmans geistiges Magnetband von dieser Erinnerung stellenweise leer. Nicht nur kann er sich nicht mehr an den Zweck der Reise entsinnen, oder die Jahreszeit, zu der er sie antrat, sondern kann sich auch nicht mehr erinnern, mit wem er das Gespräch geführt hat. Laut Reid ist derjenige, mit dem Stallman am wahrscheinlichsten zu tun hatte, Robert Sproull\index{Sproull, Robert}, ehemaliger Forscher am Xerox PARC und derzeitiger Leiter der Oracle Laboratories, einer Forschungsabteilung des Konglomerats Oracle Corporation. In den 70ern war Sproull der Hauptentwickler der besagten Laserdruckersoftware, als er am Xerox PARC arbeitete. Um 1980 nahm Sproull eine Forschungsstelle an der Carnegie Mellon an, wo er seine Arbeit rund um Laserdrucker fortsetzte und anderen Projekten nachging.

Wenn man ihn direkt danach fragt, kann sich Sproull nicht erinnern. "`Ich kann keine sachliche Aussage machen"', schreibt Sproull via E"~Mail. "`Ich kann mich an diesen Vorfall nicht im Geringsten erinnern."' "`Der Code, den Stallman wollte, war innovativer, modernster Code, den Sproull etwa im Jahr, bevor er an die Carnegie Mellon ging, geschrieben hatte"', erinnert sich Reid. Falls das der Fall ist, könnte daraus ein Missverständnis erwachsen sein, weil Stallman den Quellcode für ein Programm wollte, das das MIT schon längere Zeit einsetzte, und nicht die neuere Version. Aber die Frage, um welche Version es sich handelt, kam in dem kurzen Gespräch nicht auf.

In seinen Vorträgen vor Publikum hat Stallman oft von diesem Vorfall berichtet und erwähnt, dass die Weigerung, den Quellcode auszuhändigen, von einer Geheimhaltungserklärung herrührt, einem Vertrag zwischen jener Person und der Xerox Corporation, die dem Unterzeichner Zugang zum Quellcode ermöglicht, wenn er sich verpflichtet, ihn vertraulich zu behandeln. In der heutigen Zeit ist die Geheimhaltungserklärung (NDA) ein Standardinstrument in der Softwareindustrie; damals war sie ein Novum, ein Ausdruck des kommerziellen Werts des Laserdruckers und der Informationen über seine Funktionsweise für Xerox. "`Xerox hatte zu der Zeit versucht, ein kommerzielles Produkt aus dem Laserdrucker zu machen"', erinnert sich Reid. "`Sie wären verrückt gewesen, wenn sie den Quellcode herausgegeben hätten."'

Für Stallman jedoch stellte sich die Geheimhaltungserklärung ganz anders dar – als Weigerung eines CMU-Forschers, der Gemeinschaft beizutragen, die bis dahin Programmierer ermutigt hatte, Software als Gemeingut anzusehen. Wie ein Kleinbauer, dessen jahrhundertealter Bewässerungsgraben plötzlich ausgetrocknet war, war Stallman dem Verlauf des Grabens zu seinem Ursprung gefolgt, und fand dort einen brandneuen Staudamm vor, auf dem das Xerox-Logo prangte.

Für Stallman brauchte die Erkenntnis, dass Xerox einen Programmiererkollegen genötigt hatte, an diesem neumodischen System erzwungener Geheimhaltung teilzunehmen, eine Weile zum Sacken. Im ersten Moment konnte er die Weigerung nur auf sich persönlich beziehen. "`Ich war so wütend, dass ich nicht wusste, wie ich mich ausdrücken sollte. Also habe ich mich einfach umgedreht und bin ohne ein Wort gegangen"', erinnert sich Stallman. "`Vielleicht habe ich die Tür zugeschlagen, wer weiß. Ich kann mich bloß noch erinnern, dass ich nur noch weg wollte. Ich bin in sein Büro gekommen, weil ich davon ausgegangen bin, dass er hilft. Ich hatte nicht darüber nachgedacht, wie ich reagieren sollte, wenn er sich weigert. Als er das dann tat, war ich fassungs- und sprachlos und auch enttäuscht und wütend."' 

Zwanzig Jahre danach hält die Wut noch an und Stallman stellt den Vorfall als einen dar, der ihn das ethische Problem hat angehen lassen, aber nicht als den einzigen. In den folgenden Monaten sollte sich Stallman und der Hackergemeinde des AI~Labs eine Reihe an Ereignissen widerfahren, die die 30 Sekunden Spannung in dem fernen Büro an der Carnegie Mellon in den Schatten stellen würden. Wenn es jedoch darum geht, der Reihe an Ereignissen nachzugehen, die den solitären Hacker Stallman mit seinem instinkiven Misstrauen gegen zentrale Autorität zu dem kreuzzüglerhaften Aktivisten gemacht haben, der für die traditionellen Werte der Freiheit, Gleichheit und Brüderlichkeit in der Welt der Softwareentwicklung einsteht, schenkt Stallman dem Carnegie-Mellon-Vorfall besondere Aufmerksamkeit.
"`Es war meine erste Begegnung mit einer Geheimhaltungserklärung und sie hat mich gelehrt, dass es dabei Opfer gibt"', sagt Stallman mit Nachdruck. "`In dem Fall war ich das Opfer. [Das Lab und ich] waren die Opfer."'

Stallman erklärt später: "`Wenn er mir aus persönlichen Gründen seine Zusammenarbeit verwehrt hätte, wäre es kein größeres Problem gewesen. Ich hätte ihn vielleicht für einen Drecksack gehalten, mehr nicht. Der Umstand, dass die Weigerung nichts mit mir zu tun hatte, dass er im Vorhinein versprochen hatte, unsozial zu sein, nicht nur mir gegenüber, sondern gegen jeden, hat das Ganze zu einem wichtigen Thema werden lassen."'

Obwohl schon frühere Ereignisse Stallmans Geduld auf die Probe gestellt hatten, sagt er, bis zum Vorfall an der Carnegie Mellon war ihm nicht bewusst gewesen, dass diese Vorgänge anfingen, in eine Kultur einzudringen, die er lange als heilig angesehen hatte. Er sagte, "`Ich hatte schon die Vorstellung, dass Software ausgetauscht werden sollte, war mir aber nicht sicher, wie ich darüber denken sollte. Meine Gedanken waren unklar und unorganisiert, so dass ich sie nicht in einer präzisen Form der Welt hätte mitteilen können. Nach dieser Erfahrung fing ich an, zu erkennen, was das Problem und wie bedeutend es war."'

Als Eliteprogrammierer an einer Eliteeinrichtung war Stallman ohne Weiteres bereit gewesen, die Kompromisse und Kuhhandel zu ignorieren, die seine Programmiererkollegen eingingen, solange sie nicht seine eigene Arbeit behinderten. Bis zur Anlieferung des Xerox-Laserdruckers hatte Stallman sich damit zufriedengegeben, auf die Geräte und Programme herabzusehen, die andere Nutzer grimmig tolerierten.

Nun, da sich der Laserdrucker ins Netzwerk des AI Lab eingeschlichen hatte, hatte sich etwas geändert. Das Gerät funktionierte gut, abgesehen von den Papierstaus. Aber die Möglichkeit, die Software nach dem persönlichen Geschmack oder zum Nutzen der Allgemeinheit abzuändern, war ihnen genommen. Aus der Sicht der Softwareindustrie stellte die Druckersoftware eine Änderung in der Geschäftsstrategie dar. Software war ein so wertvolles Gut geworden, dass Unternehmen nicht länger die Notwendigkeit sahen, den Quellcode zu veröffentlichen, besonders wenn die Veröffentlichung potentiellen Konkurrenten die Möglichkeit einer billigen Nachahmung eröffnete. Aus Stallmans Sicht war der Drucker ein trojanisches Pferd. Nach einem Jahrzehnt des Misserfolgs sollte Software, die der Nutzer nicht verändern und weitergeben kann – in Zukunft würden Hacker sie "`proprietäre Software"' nennen – durch gewiefte Methoden im AI~Lab Fuß gefasst haben. Sie kam in der Gestalt eines Geschenks.

Dass Xerox einigen Programmierern Zugang zu weiteren Geschenken in Gegenleistung zur Geheimhaltung angeboten hat, war ebenso ärgerlich, aber Stallman merkt an\comment{takes pains to note that}, dass wenn ihm in jüngeren Jahren so ein Quid-pro-quo-Handel angeboten worden wäre, er ihn vielleicht angenommen hätte. Aber der Ärger mit dem Vorfall an der Carnegie Mellon hatte einen festigenden Effekt auf Stallmans eigene moralische Farblosigkeit. Nicht nur verhalf er ihm zu dem Zorn, zukünftig solche Angebote skeptisch zu sehen, er zwang ihn auch, die ganze Sache umzudrehen: Was, wenn ein Hackerkollege eines Tages in sein Büro kommen sollte und es nun Stallmans Aufgabe wäre, ihm den Quellcode vorzuenthalten?
"`Als mir jemand angeboten hat, all meine Kollegen auf diese Art zu verraten, habe ich mich daran erinnert, wie ärgerlich ich war, als man mit mir und dem ganzen Lab dasselbe gemacht hat"', sagt Stallman. "`Also hab ich gesagt \glq Vielen Dank, dass Sie mir dieses schöne Softwarepaket angeboten haben, aber ich kann es unter den Bedingungen, die Sie verlangen, nicht annehmen und muss wohl ohne es auskommen.\grq\,"'

Es war eine Lektion, die Stallman in den turbulenten 80er Jahren bleiben sollte, einem Jahrzehnt, in dem viele seiner MIT-Kollegen vom AI~Lab gehen und selbst Geheimhaltungserklärungen unterschreiben sollten. Sie haben vielleicht bei sich gedacht, dass das ein notwendiges Übel ist, um das Beste für ihre Projekte zu erreichen. Aber für Stallman zog es die moralische Rechtmäßigkeit der Projekte in Frage. Wozu soll ein spannendes technisches Projekt gut sein, wenn es der Gemeinschaft vorenthalten werden soll?

Stallman sollte bald lernen, dass das Ausschlagen solcher Angebote mehr als nur persönliche Opfer erforderte. Es erforderte die Absonderung von seinen Hackerkollegen, die zwar auch eine ähnliche Abneigung gegenüber Geheimhaltung hatten, sie aber in einer moralisch flexibleren Art ausdrückten. Jemandem die Herausgabe des Quellcodes zu verweigern, entschied sich Stallman, war nicht nur ein Verrat am wissenschaftlichen Auftrag, der die Softwareentwicklung seit dem Ende des Zweiten Weltkriegs gefördert hatte, sondern es war die Verletzung der Goldenen Regel, dem moralischen Grundsatz, andere so zu behandeln, wie man selbst behandelt werden will.

Daher rührt die Bedeutsamkeit des Laserdruckers und der Begegnung, die sich seinetwegen ergab. Ohne ihn, sagt Stallman, hätte sein Leben vielleicht einen gewöhnlicheren Verlauf genommen, ein Leben, in dem er mit den materiellen Vorzügen als kommerzieller Programmierer seine Frustration als Schreiber von unsichtbaren Quellcode ausgleichen würde. Es hätte keinen Sinn von Klarheit gegeben, keinen Drang, die Probleme anzugehen, die andere ignorieren. Und zu guter Letzt hätte es keinen Zorn der Rechtschaffenheit gegeben, ein Gefühl, wie wir bald sehen werden, das Stallmans Laufbahn sicherlich genauso vorangetrieben hat wie politische Ideologie oder moralische Überzeugung.

"`Von diesem Tag an war ich entschlossen, dass das etwas war, an dem ich niemals teilhaben konnte"', sagt Stallman zu der Handlungsweise, persönliche Freiheit gegen Annehmlichkeiten einzutauschen – Stallmans Beschreibung der Geheimhaltungsverpflichtung – und zur gesamten Kultur, die solche moralisch zweifelhaften Geschäfte überhaupt erst fördert. "`Ich habe mich entschieden, niemals andere Leute zu Opfern zu machen, so wie ich [zum Opfer] gemacht wurde."'
