\chapter{Anhang A – Hack, Hacker und Hacking}
\label{Anhang A}

Um die volle Bedeutung des Wortes "`Hacker"' zu verstehen, ist es hilfreich, sich die Etymologie des Worts über die Jahre zu betrachten.
Das \textit{New Hacker's Dictionary}, ein Buch über Programmiererjargon, listet offiziell neun verschiedene Bedeutungen des Worts "`Hack"' und eine ähnliche Anzahl für "`Hacker"'. Aber es enthält auch ein begleitendes Essay mit einem Zitat von Phil Agre, einen MIT-Hacker, der den Leser warnt, sich nicht von der wahrgenommenen Flexibilität des Worts beirren zu lassen. "`Hack hat nur eine Bedeutung"', meint Agre. "`Eine extrem subtile und profunde, die sich jedem Versuch einer Definition entzieht."' Richard Stallman versucht es mit den Worten "`spielerische Cleverness"' zu beschreiben. 

Egal, wie weit oder eng man den Begriff fassen will, die meisten modernen Hacker sehen den Ursprung des Worts beim MIT, wo es in den frühen 50ern im Studentenjargon aufkam. 1990 stellte das MIT-Museum ein Heft zusammen, das das Hackerphänomen dokumentiert. Laut dem Heft benutzten Studenten am Institute in den 50ern das Wort "`Hack"' wie ein heutiger Student das Wort "`goof"'.\footnote{goof [coll.], s: Patzer/Schnitzer, Trottel; v: herumblödeln, Mist bauen, etw. vermasseln} Eine alte Rostlaube aus dem Wohnheimfenster zu hängen, war ein "`Hack"', aber alles harsche oder böswillige – z.\,B. ein rivalisierendes Wohnheim mit Eiern zu bewerfen oder die Campus-Statue zu verunstalten – lag weit außerhalb der Grenze. Im Wort "`Hack"' lag eine Konnotation von Harmlosigkeit, von kreativem Spaß.

Dieser Geist inspirierte die Gerundiumform "`Hacking"'. Ein 50er-Jahre-Student, der den Großteil seines Nachmittags damit verbringt, zu telefonieren oder ein Radio auseinanderzunehmen, ging einer Tätigkeit nach, die man als "`Hacking"' bezeichnen konnte.
Wieder würde ein moderner Sprecher die Bezeichnung durch eine Form von "`goof"'  ersetzen – "`goofing"' oder "`goofing off"'\footnote{goof off [coll.]: Zeit vertrödeln, müßiggehen} – um dieselbe Aktivität zu beschreiben.

Im Laufe der 50er bekam das Wort "`Hack"' einen schärferen, rebellischeren Ton. Das MIT war in den 50ern sehr wettbewerbsorientiert und Hacking kam gleichermaßen als Reaktion auf sowie als Fortsetzung der Wettbewerbskultur auf. Unsinn zu bauen und Streiche zu spielen wurde eine Möglichkeit, Dampf abzulassen, der Univerwaltung eine lange Nase zu zeigen, und sich in kreativem Denken und Verhalten zu üben, was von dem rigorosen Grundstudium am MIT unterdrückt wurde. Mit seiner Myriade Fluren und unterirdischen Tunneln\comment{ XXX steam tunnels} bot das Institute reichlich Erkundungsmöglichkeiten für Studenten, die nicht vor verschlossenen Türen oder Betreten-Verboten-Schildern zurückschreckten. Die Studenten fingen an, ihre verbotenen Erkundungen als "`tunnel hacking"' zu bezeichnen. Über der Erde bot das Campus-Telefonsystem ähnliche Möglichkeiten. Durch ungezieltes Experimentieren und dem gebührenden Eifer lernten Studenten, wie man verschiedene humorvolle Tricks ausführt. Inspiriert von der traditionelleren Aktivität des tunnel hackings kamen Studenten bald auf den Namen "`phone hacking"'.

Die Kombination aus kreativem Spiel und unbeschränkter Erkundung diente als Basis für die zukünftigen Metamorphosen des Begriffs "`Hacking"'. Die ersten Computer-Hacker am MIT der 1960er, die sich selbst so nannten, hatten ihren Ursprung in einer Studentengruppe aus den späten 50ern namens Tech Model Railroad Club. Eine enge Clique in diesem Club war das Signals and Power (S\&P) Committee – die Gruppe, die unter den Modellbahnern für das elektrische Schaltungssystem verantwortlich war. Das System war eine ausgeklügelte Ansammlung aus Relais und Schaltern, ähnlich denen, die das örtliche Campus-Telefonnetz steuerten. Zur Steuerung wählte man als Gruppenmitglied die Befehle einfach mit einem angeschlossenen Telefon und schon konnte man die Züge fahren sehen.

Die angehenden Elektroingenieure, die für den Bau und die Wartung verantwortlich waren, sahen ihre Aktivität im Geiste des phone hackings. Mit der Übernahme des Begriffs des Hackings fingen sie an, ihn noch weiter einzugrenzen\comment{refine - weiterentwickeln}. Aus der Sicht der S\&P-Hacker bedeutete das Benutzen eines Relais weniger für einen bestimmten Schienenverlauf, dass man ein Relais mehr für später hatte. Die Bedeutung von Hacking verschob sich leicht von einem Synonymn für eine einfache Spielerei hin zu einer Spielerei, bei der man gleichzeitig die Gesamtleistung oder -effizienz des Schienensystems des Clubs verbessert. Schon bald bezeichneten die S\&P-Ausschussmitglieder die gesamte Tätigkeit des Umgestaltens der zugrundeliegenden Schalttechnik stolz als "`Hacking"' und Leute, die sie ausübten, als "`Hacker"'.

Bei ihrer Affinität zu ausgeklügelter Elektronik – nicht zu erwähnen ihrem Nichthaltmachen vor verschlossenen Türen und Betreten-Verboten-Schildern in bester MIT-Tradition – dauerte es nicht lange, bevor die Hacker von einer neuen Maschine auf dem Campus Wind bekamen. Die Rechenmaschine mit dem Namen TX-0 war einer der ersten kommerziell vertriebenen Computern. Zum Ende der 50er war die gesamte S\&P-Clique geschlossen in den TX-0-Steuerraum abgewandert und hatte ihren kreativen, spielerischen Geist mitgebracht.
Das unerschlossene Gebiet der Computerprogrammierung sollte noch eine weitere Mutation in der Etymologie hervorrufen. "`Hacken"' bedeutete nicht mehr "`Löten ungewöhnlich aussehender Schaltungen"', sondern "`Zusammenschustern von Softwareprogrammen"' ohne größere Beachtung der "`offiziellen"' Methoden der Softwareentwicklung. Es bedeutete auch, die Effizienz und Geschwindigkeit bestehender ressourcenfressender Programme zu verbessern. Getreu der Wortherkunft bedeutete es auch, Programme zu schreiben, die keinem Zweck außer dem Amüsement oder der Unterhaltung dienen.

% TODO: erstes Videospiel?
Ein klassisches Beispiel dieser erweiterten Hacking-Definition ist das Spiel Spacewar, das erste computerbasierte Videospiel. Es wurde Anfang der 60er von MIT-Hackern entwickelt und passte auf alle traditionellen Hacking-Definitionen: es war albern und ziellos, mit wenig praktischem Sinn außer zur nächtlichen Zerstreuung für circa ein Dutzend Hacker, die es mit Freude spielten. Aus der Softwareperspektive war es ein monumentaler Beweis von Innovation im Programmierbereich.
Außerdem war es komplett frei. Weil die Hacker es aus Spaß entwickelt hatten, sahen sie keinen Grund, ihre Schöpfung zu schützen, und sie teilten sie frei mit anderen Programmierern. Bis zum Ende der 60er war Spacewar ein Zeitvertreib für Hacker auf der ganzen Welt geworden, jedenfalls für solche mit einer (damals ziemlich seltenen) graphischen Anzeige.

Die Vorstellung von kollektiver Innovation und gemeinsamem Softwarebesitz grenzte den Vorgang des Computer-Hackings in den 60ern von dem tunnel hacking und phone hacking der 50er ab. Letztere Beschäftigungen tendierten dazu, Einzel- oder Kleingruppenaktivitäten zu sein. Tunnel- und Telefonhacker waren stark auf Campusüberlieferungen angewiesen, aber die unbefugte Natur ihrer Aktivitäten hielt sie von der offenen Verbreitung neuer Entdeckungen ab. Computerhacker andererseits machten ihre Arbeit auf einem Gebiet der Wissenschaft, das zur Zusammenarbeit und Belohnung von Innovation neigt. Hacker und "`offizielle"' Informatiker waren nicht immer die besten Verbündeten, aber bei der schnellen Evolution auf dem Gebiet entwickelten die beiden Spezies der Programmierer eine kooperative – vielleicht symbiotische – Beziehung zueinander.

Hacker hatten wenig Respekt für die Regeln der Bürokraten übrig. Sie sahen Sicherheitssysteme, die den Zugang zu Rechnern beschränken, nur als einen weiteren Bug an, den man umgehen musste oder beheben, falls möglich. Und so war das Knacken von Sicherheitsvorkehrungen (für nicht böswillige Zwecke) anerkannter Bestandteil des Hackens in den 70ern, nützlich für Streiche\comment{ (ein Opfer könnte sagen: "`Ich glaube, ich werde gehackt"')} sowie für die Erlangung von Zugriff auf Rechner. Aber es war nicht die Hauptidee des Hackens. Wo es ein Sicherheitshindernis gab, zeigten die Hacker stolz ihre Fähigkeiten bei ihrer Überwindung; jedoch, wenn sie die Wahl hatten, wie im AI~Lab, entschieden sie sich gegen die Hindernisse und hackten lieber andere Dinge. Wo es keine Sicherheitsvorkehrungen gibt, muss sie keiner knacken.

Dass später Programmierer, einschließlich Richard M. Stallman, sich denselben Hackermantel überwerfen wollten, ist Beweis der meisterhaften Fähigkeiten der ursprünglichen Hacker. Mitte der 70er bekam der Begriff "`Hacker"' eine elitäre Konnotation. Allgemein gesagt war jeder ein Computerhacker, der zum Selbstzweck Software schrieb. Im engeren Sinne war es ein Zeugnis von jemandes Programmierfähigkeiten. Wie der Begriff "`Künstler"' hatte die Bedeutung einen sippenartigen Unterton. Einen Programmiererkollegen als Hacker zu bezeichnen, war ein Zeichen von Respekt. Sich selbst als Hacker zu bezeichenen, war ein Zeichen von Selbstsicherheit. Jedenfalls schwand die ursprüngliche Laxheit der Hackerbezeichnung mit der wachsenden Verbreitung von Computern.

Mit der Einengung der Definition bekam "`Computer"'-Hacking zusätzliche Untertöne. Die Hacker von MITs AI~Lab teilten viele andere Charakteristika, inklusive die Vorliebe für chinesische Küche, die Abneigung gegenüber Tabakrauch und das Meiden von Alkohol, Tabak und anderen süchtig machenden Drogen. Diese Charakteristika wurden Teil von dem, was die Leute unter einem Hacker verstanden, und die Gemeinschaft übte einen Einfluss auf Neulinge aus, obwohl sie von ihnen nicht verlangte, sich anzupassen. Diese kulturellen Assoziationen verschwanden jedoch mit der AI-Lab-Hackergemeinde. Heutzutage ähneln die meisten Hacker in diesen Punkten ihrem gesellschaftlichen Umfeld.

Wenn Hacker von Eliteinstitutionen wie dem MIT, Stanford und Carnegie Mellon über Hacks sprachen, die sie bewunderten, machten sie sich auch Gedanken über die ihren Handlungen zugrundeliegende Ethik und fingen an, offen über eine "`Hackerethik"' zu sprechen: die noch ungeschriebenen Regeln, die das Alltagsverhalten der Hacker bestimmten. In dem Buch \citefield{shorttitle}{hackers} kodifiziert der Autor \citefield{shortauthor}{hackers} nach genauer Untersuchung und Rücksprache die Hackerethik mit fünf Kernsätzen.

\begin{enumerate}\item Zugang zu Computern – und allem, was einem etwas über Welt lehren kann – sollte unbeschränkt und vollständig sein. Der prakische Ansatz ist imperativ!
\item Alle Informationen sollten frei sein.
\item Autorität misstrauen – Dezentralisierung fördern.
\item Hacker sollten nach ihrem Hacking beurteilt werden, nicht nach Scheinkriterien wie akademischem Grad, Alter, Geschlecht, Rasse oder Position.
\item Man kann auf einem Computer Kunst und Anmut schaffen. Computer können dein Leben zum Besseren verändern.
\end{enumerate}

In den 80ern wuchs die Computernutzung stark, und auch das Knacken von Sicherheitsvorkehrungen nahm zu. Meistens wurde das Knacken von Insidern mit kriminellen Absichten ausgeführt, die meistens überhaupt keine Hacker waren. Manchmal konnten die Polizei oder Beamten\comment{, die jeden Ungehorsam als schlecht definieren, } einen Computer-"`Einbruch"' zu einem Hacker zurückverfolgen, dessen ethische Vorstellung es war, "`keiner Person zu schaden"'. Die Journalisten veröffentlichten Artikel, in denen "`Hacken"' das Brechen von Sicherheitsvorkehrungen bedeutete, und nahmen meist die staatliche Sicht zu der Angelegenheit ein. Obwohl Bücher wie \citefield{shorttitle}{hackers} viel zur Aufklärung über den ursprünglichen Erforschungsgeist beitrugen, der der Hackerkultur zu ihrem Aufstieg verholfen hatte, wurde für die meisten Zeitungsreporter und -leser der Begriff "`Computerhacker"' gleichbedeutend mit "`elektronischer Einbrecher"'.

Ende der 80er hatten viele Teenager in den USA Zugang zu Computern. Einige davon waren von der Gesellschaft ausgeschlossen; von dem journalistischen Zerrbild des "`Hackens"' inspiriert, drückten sie ihren Unmut durch das Einbrechen in Computersysteme aus, so wie andere ausgeschlossene Teenager vielleicht durch das Einschlagen von Fensterscheiben. Sie fingen an, sich "`Hacker"' zu nennen, aber sie hatten das MIT-Prinzip, sich nicht boshaft zu verhalten, nie kennengelernt.
Als jüngere Programmierer damit anfingen, ihre Computerfähigkeiten zum Anrichten von Schäden einsetzten – dem Schreiben und Verbreiten von Viren, dem Einbrechen in Computersysteme, um Unfug anzurichten, dem absichtlichen Verursachen von Abstürzen – bekam der Begriff "`Hacker"' einen revoluzzerhaften, nihilistischen Beiklang, der weitere Leute mit ähnlichen Einstellungen anzog.

Hacker wettern gegen diese als falsch wahrgenommene Verwendung ihrer Selbstbezeichnung schon seit mehr als zwei Jahrzehnten\comment{nearly two decades}. Stallman ist jemand, der Dinge nicht einfach so hinnimmt, und prägte den Begriff "`Cracking"' für das Knacken von Sicherheitssystemen.\comment{so that people could more easily avoid calling it "`hacking"'.} Aber der Unterschied zwischen Hacking und Cracking wird oft missverstanden. Die zwei Begriffe sollen sich nicht gegenseitig ausschließen. Es ist nicht so, dass Hacking und Cracking zwei getrennte Welten sind, die sich an keinem Punkt treffen. Hacking und Cracking sind verschiedene Merkmale von Tätigkeiten, so wie "`jung"' und "`groß"' verschiedene Merkmale von Personen sind.

Das meiste Hacking schließt keine Sicherheitsaspekte ein, deswegen ist es kein Cracking. Das meiste Cracking geschieht aus Profitgründen oder Böswilligkeit und nicht aus einem spielerischen Geist heraus, also ist es kein Hacking. Gelegentlich erfüllt eine Tätigkeit die Voraussetzungen für Cracking und Hacking, aber im Allgemeinen ist das nicht der Fall. Der Hackergeist umfasst die Missachtung von Regeln, aber die meisten Hacker brechen die Regeln nicht. Cracking ist per Definition Ungehorsam, aber nicht notwendigerweise bösartig oder schadhaft. Im Kryptologiebereich unterscheidet man zwischen "`Black-hat"'- und "`White-hat"'-Crackern – also Cracker, die zum Zerstörerischen neigen, und solche, die die Sicherheitsvorkehrungen untersuchen, um sie zu verbessern.

Das Hauptprinzip der Hacker, nicht bösartig zu sein, bleibt die primäre kulturelle Verbindung zwischen der Vorstellung von Hacking im frühen 21.\,Jahrhundert und in den 50ern. Es sollte erwähnt werden, dass bei der Entwicklung der Hacker-Idee über die letzten fünf Jahrzehnte die Ursprungsvorstellung vom Hacking – dem Spielen von Streichen oder Erkunden der unterirdischen Tunnel – erhalten geblieben ist. Im Herbst 2000 erwies das MIT Museum der uralten Hackingtradition mit einer speziellen Ausstellung, der Hall of Hacks, die Ehre. Die Ausstellung umfasst zahlreiche Photographien, einige datieren bis in die 20er zurück, darunter eines mit einem nachgeahmten Polizeiauto. Als Tribut an die ursprüngliche Bedeutung des Hackings bauten 1994 einige Studenten so einen Streifenwagen mit angestellten Warnleuchten oben auf der Großen Kuppel des Gebäudes 10 am MIT auf. Auf das Nummernschild des Autos war "`IHTFP"' geprägt, eine populäre Abkürzung am MIT mit vielen Bedeutungen. Die bekanntste Variante geht auf das stressreiche MIT-Studentenleben der 50er zurück und lautet "`I hate this fucking place"'. 1990 verwandte das Museum die Abkürzung als Grundlage für ein Heft über die Geschichte des Hackings:\comment{Unter dem Namen} \textit{The Journal of the Institute for Hacks, Tomfoolery, and Pranks}\comment{ bietet es eine geschickte Zusammenfassung des Hackings}.

"`In der Kultur des Hackings wird eine elegante, simple Schöpfung genauso hoch angesehen wie in der reinen Wissenschaft"', schreibt der Boston-Globe-Reporter Randolph Ryan in einem Artikel von 1993 über die Streifenwagen-Aktion. "`Ein Hack unterscheidet sich von dem normalen College-Streich darin, dass der Vorgang üblicherweise sorgfältige Planung, Ingenieurskunst und Finesse benötigt und einen zugrunde liegenden Witz und Einfallsreichtum hat"', schreibt Ryan. "`Die ungeschriebene Regel besagt, dass ein Hack gutmütig, nicht destruktiv und ungefährlich sein sollte. Manchmal sind die Hacker sogar bei der Zerlegung ihrer eigenen Werke behilflich."'

Der Drang, die Kultur des Computerhackings in denselben ethischen Grenzen zu halten, ist gutgemeint, aber unmöglich. Obwohl die meisten Softwarehacks dieselbe Eleganz und Simplizität anstreben, bietet das Medium Software weniger Möglichkeit zur Rückgängigmachung. Einen Streifenwagen zu zerlegen ist einfach im Vergleich zum Zerlegen einer Idee, besonders wenn die Zeit für eine Idee reif ist.

Was einst ein vager Begriff im obskuren Studentenjargon war, wurde zur linguistischen Billardkugel\comment{, subject to political spin and ethical nuances}. Vielleicht ist das der Grund, warum ihn so viele Hacker und Journalisten gern verwenden. Wir können nicht vorhersagen, wie man das Wort in Zukunft verwenden wird, aber wir können uns entscheiden, wie wir es selbst verwenden. Den Begriff "`Cracking"' statt "`Hacking"' zu verwenden, wenn man das Knacken von Sicherheitssystemen meint, zeigt Respekt gegenüber Stallman und den anderen Hackern in diesem Buch und unterstützt den Erhalt von etwas, von dem alle Computernutzer profitiert haben: dem Hackergeist.