\chapter{GNU/Linux}

1993 stand die Free-Software-Bewegung an einem Scheideweg. Für die Optimisten sah alles nach Erfolg für die Hackerkultur aus. \textit{Wired}, ein hippes neues Magazin mit Artikeln über Datenverschlüsselung, das Usenet und Softwarefreiheit, wurde den Zeitungsverkäufern geradezu aus den Händen gerissen. "`Internet"', früher ein Fachterminus unter Hackern und Forschern, hatte seinen Weg in den Wortschatz der breiten Masse gefunden. Selbst Präsident Clinton nutzte es. Der PC, einst ein Spielzeug für Liebhaber, hatte allgemein an Ansehen gewonnen und gab einer ganzen neuen Generation von Computernutzern Zugang zu von Hackern geschaffener Software. Während das GNU Project sein Ziel eines vollständigen, freien GNU-Betriebssystems noch nicht erreicht hatte, konnten Nutzer schon die GNU/Linux-Variante nutzen.

Wie man es dreht und wendet, es waren gute Neuigkeiten, oder es sah zumindest so aus. Nach einem Jahrzehnt des Kampfes wurden Hacker und ihre Werte schließlich von der breiten Masse akzeptiert. Man verstand sie. Oder doch nicht? Für die Pessimisten gab es für jedes Zeichen der Akzeptanz zwei problematische Gegenanzeichen. Sicherlich war es jetzt plötzlich cool, Hacker zu sein, aber zu was sollte das gut sein, in einer Gemeinschaft, die in Abgeschiedenheit gedeiht? Gewiss sagte man im Weißen Haus nette Dinge über das Internet, und meldete sogar einen eigenen Domainnamen an, \href{http://whitehouse.gov}{white-house.gov}, aber man traf sich dort auch mit Unternehmen, Fürsprechern von Zensur und Funktionären der Strafverfolgungsbehörden, die die Wildwest-Kultur des Internets bändigen wollten. Gewiss waren PCs leistungsfähiger geworden, aber Intel als Chiplieferant hatte eine Situation geschaffen, in der die proprietären Softwarehersteller nun an der Macht waren. Für jeden neuen Nutzer, den man mit GNU/Linux für freie Software gewonnen hatte, booteten Hunderte, wenn nicht Tausende zum ersten Mal Microsoft Windows. GNU/Linux hatte nur rudimentäre graphische Interfaces und war so kaum benutzerfreundlich. 1993 konnte es nur ein Experte nutzen. Der erste Versuch des GNU Projects, einen graphischen Desktop zu entwerfen, war gescheitert.

Dann war da noch die politische Situation. Der urheberrechtliche Schutz von graphischen Benutzeroberflächen stellte immer noch eine reale Bedrohung dar – die Gerichte hatten sich damals von der Vorstellung noch nicht entfernt. Inzwischen waren Patente auf Softwarealgorithmen und -funktionen eine wachsende Gefahr, die sich auf andere Länder auszubreiten drohte.
Außerdem war da noch die seltsame Natur von GNU/Linux selbst. Ungehindert von Rechtsstreitigkeiten (wie bei BSD) war GNU/Linux' schnelle Evolution so ungeplant, sein Erfolg so unbeabsichtigt, dass selbst die Kernprogrammierer nicht wussten, was sie davon halten sollten. Es war eher ein Compilation-Album als ein einheitliches Projekt und bestand aus einem Hacker-Potpourri der größten Hits: alles von GCC, GDB und glibc (der neu entwickelten C-Bibliothek des GNU Projects) über X (einem Unix-basierten graphischen Userinterface, entwickelt im Laboratory for Computer Science des MIT) bis zu den BSD-Tools wie BIND (der Berkeley Internet Naming Daemon, der für Nutzer die einfach zu merkenden Domainnamen in numerische IP-Adressen umsetzt) und TCP/IP. Zusätzlich enthielt es den als Ersatz für Minix entwickelten Linux-Kernel. Anstatt ein neues Betriebssystem zu entwickeln, hatten Torvalds und sein rasch wachsendes Linux-Entwicklerteam ihre Arbeit in das Rahmenwerk eingepasst. Oder wie Torvalds es später selbst ausdrücken würde, wenn er das Geheimnis seines Erfolgs beschreibt: "`Im Grunde bin ich ein sehr fauler Mensch, der die Lorbeeren für Dinge einheimst, die in Wirklichkeit andere Leute tun."'\footnote{Torvalds hat diesen Satz oft zu verschiedenen Anlässen fallenlassen. Bis heute ist jedoch das bedeutendste Vorkommen des Zitats bei \cite[][]{catb}.}

Solche Faulheit war, obwohl aus Sicht der Effizienz bewundernswert, aus der politischen Sicht problematisch. Zum einen unterstrich sie den Mangel an ideologischen Plänen seitens Torvalds. Anders als die GNU-Entwickler hatte Torvalds seinen Kernel nicht aus dem Wunsch heraus entwickelt, seinen Hackerkollegen Freiheit zu bringen; er hatte ihn entwickelt, um etwas für sich zum Spielen zu haben. Was genau war also dieses zusammengefügte System und welche Philosophie würden die Menschen mit ihm in Verbindung bringen? Die Manifestation der Free-Software-Philosophie, wie in Stallmans GNU Manifesto beschrieben? Oder war es einfach eine Amalgamierung ausgeklügelter Software, die jeder Nutzer mit ähnlicher Motivation auf seinem Heimsystem zusammenbauen konnte?

Gegen Ende 1993 tendierte eine wachsende Anzahl an GNU/Linux-Nutzern zu letzterer Definition und begann, sich ihre persönlichen Variationen zusammenzuschustern. Sie fingen an, ihre eigenen verschiedenen "`Distributionen"' von GNU/Linux zu entwickeln und zu verteilen, manche gratis und manche gegen Geld. Die Ergebnisse waren bestenfalls dürftig.

"`Das war noch vor Red Hat und den anderen kommerziellen Distributionen"', erinnert sich Ian Murdock, damals Informatikstudent an der Purdue University. "`Man blätterte duch Unix-Magazine und fand all diese visitenkartengroßen Anzeigen, die mit \glq Linux \grq geworben haben. Die meisten dieser Firmen waren zweifelhafte Unternehmungen, die nichts schlimm daran fanden, etwas eigenen [proprietären] Sourcecode beizumengen."'

Murdock, ein Unix-Programmierer, erinnert sich daran, ganz mitgerissen gewesen zu sein von GNU/Linux, als er es das erste Mal heruntergeladen und auf seinem Heim-PC installiert hatte. "`Es hat einfach sehr viel Spaß gemacht"', sagt er. "`Ich wollte da mitmachen."' Aber die Schwemme an schlecht gemachten Distributionen fing an, seine Begeisterung zu dämpfen. Er entschied, dass es das Beste wäre, sich einzubringen, indem er eine Version frei von den Beigaben Dritter entwickelt, und machte sich daran, eine Liste der besten freien Softwarewerkzeuge zusammenzustellen, um sie in seiner eigenen Distribution zusammenzufassen. "`Ich wollte etwas, das dem Namen Linux gerecht wird"', sagt Murdock. 

Um "`etwas Interesse zu erregen"', postete Murdock seine Absicht im Internet, einschließlich der Usenet-Newsgroup comp.os.linux. Eine der ersten Antworten war von \href{mailto:rms@ai.mit.edu}{rms@ai.mit.edu}. Als Hacker erkannte Murdock sofort die Adresse. Sie war von Richard M. Stallman, Gründer des GNU Projects und einem Mann, den Murdock selbst damals als den "`Hacker der Hacker"' kannte. Die Adresse in seiner Mail-Queue zu sehen, wunderte Murdock. Warum um alles in der Welt sollte Stallman, jemand, der sein eigenes Betriebssystemprojekt leitet, sich um seine Krittelei an den Linux-"`Distributionen"' scheren?
Murdock öffnete die Nachricht. "`Er sagte, die Free Software Foundation wollte sich Linux genauer ansehen und dass die FSF vielleicht an einem eigenen Linux-System [\textit{sic}] interessiert wäre. Im Grunde sah es für Stallman so aus, als wären unsere Ziele im Einklang mit ihrer Philosophie."'

\comment{TODO: Prüfen: Nur ein Hurd-Entwickler?!}
Ohne es überzudramatisieren, stellte die Nachricht einen Strategiewechsel seitens Stallmans dar. Bis 1993 war Stallman zufrieden damit, sich aus den Linux-Angelegenheiten herauszuhalten. Nachdem er zuerst von dem neuen Kernel gehört hatte, bat Stallman einen Freund, seine Eignung zu testen. Stallman erinnert sich: "`Er hat mir berichtet, dass die Software ein System-V-Nachbau war, was die schlechtere Unix-Version war. Er hat mir auch gesagt, dass er nicht portabel ist."'
Der Bericht des Freunds war korrekt. Da Linux für 386-basierte Rechner geschrieben wurde, war es stark in der Billig-Hardwareplattform verwurzelt.
Was der Freund ihm nicht berichtete, war der beträchtliche Vorteil, den Linux als einziger freier Kernel auf dem Markt genoss. Anders gesagt: während Stallman in den nächsten anderthalb Jahren die Statusberichte seines Hurd-Entwicklers bekam, die einen eher langsamen Fortschritt vermeldeten, gewann Torvalds Programmierer für sich\comment{, die später Linux und GNU auswurzeln und auf andere Plattformen umpflanzen sollten XXX}.

Im Jahre 1993 war das Unvermögen, einen lauffähigen Kernel zu bieten, das Hauptproblem innerhalb des GNU Projects und in der Free-Software-Bewegung als Ganzes. In einem Artikel von Simson Garfinkel in der Zeitschrift \textit{Wired} vom März 1993 beschreibt er das GNU Project als "`festgefahren"', trotz der Erfolge vieler Werkzeuge des Projekts.\footcite[Vgl.][]{stalled} Die Leute im Project und seinem gemeinnützigen Auswuchs, der Free Software Foundation, erinnern sich an eine noch schlechtere Stimmung als in Garfinkels Artikel beschrieben. "`Es war ganz klar, zumindest mir damals, dass es eine begrenzte Zeit gibt, um ein neues Betriebssystem einzuführen"', sagt Chassell\index{Chassell, Robert|(}. "`Wenn die Zeit dann herum wäre, würden die Leute das Interesse verlieren. Und das ist auch exakt so passiert."'\footnotemark

\footnotetext{Mit seinen Bedenken über ein "`Zeitfenster"' von 36 Monaten für ein neues Betriebssystem ist Chassell\index{Chassell, Robert|)} nicht der einzige im GNU Project. Anfang der 90er wurden die Free-Software-Versionen der Berkeley Software Distribution durch eine Klage der Unix System Laboratories aufgehalten, die die Veröffentlichung von von BSD abstammender Software einschränkte. Obwohl viele Nutzer die BSD-Ableger wie FreeBSD und OpenBSD nachweislich als GNU/Linux überlegen in Sachen Leistung und Sicherheit ansehen, bleibt die Anzahl der FreeBSD- und OpenBSD-Nutzer nur ein Bruchteil der der GNU/Linux-Nutzer. Für eine exemplarische Analyse des relativen Erfolgs von GNU/Linux in Bezug zu anderen freien Betriebssystemen, \cite[vgl.][]{linsucc}.}

Viel Lärm wurde um die Mühen des GNU Projects in der Zeit von 1990-1993 gemacht. Während einige Stallman die Schuld daran geben, sagt Eric Raymond, ein alter Freund Stallmans, der das GNU Project halbherzig unterstützt hat, dass das Problem größtenteils institutionell bedingt war. "`Die FSF ist arrogant geworden"', sagt Raymond. "`Sie haben sich von dem Ziel entfernt, ein Betriebssystem für den Produktiveinsatz zu entwickeln und sind auf Betriebssystemforschung umgeschwenkt."' Und noch schlimmer: "`Sie dachten, nichts von außerhalb könnte ihnen etwas anhaben."'

Murdock vertritt eine gemäßigtere Ansicht. "`Ich denke, es war Teil des Problems, dass sie etwas zu ehrgeizig waren und gutes Geld schlechtem hinterhergeworfen haben"', sagt er. "`Microkernel waren in den späten 80ern und frühen 90s ein heißes Thema. Leider war das etwa zu der Zeit, als das GNU Project mit seinem Kerneldesign anfing. Sie hatten schließlich einen Haufen Ballast, den sie nur durch Zurückrudern losgeworden wären."' 

Stallman antwortet: "`Obwohl die Emotionen, die Raymond hier anführt, seiner Einbildung entspringen, hat er recht mit einem Grund der Hurd-Verzögerung: die Hurd-Entwickler haben die Codebasis mehrmals umgestaltet und in großen Teilen umgeschrieben, nach den Erfahrungen, die sie gemacht haben, anstatt zu versuchen, die Hurd so schnell wie möglich zum Laufen zu bekommen. Das war eine gute Entwurfspraktik, aber nicht die richtige Praktik, um unser Ziel zu erreichen, so schnell wie möglich etwas funktionierendes zu bekommen."'

Stallman nennt andere Gründe, die außerdem zur Verzögerung beigetragen haben. Die Gerichtsverfahren mit Lotus und Apple haben seiner vollen Aufmerksamkeit bedurft; das, und die Probleme mit seiner Hand, die Stallman drei Jahre lang vom Tippen abgehalten haben, schlossen Stallman größtenteils vom Programmieren aus. Stallman führt auch die schlechte Kommunikation zwischen den verschiedenen Teilen des GNU Projects an. "`Wir hatten viel Mühe, die Debugging-Umgebung lauffähig zu bekommen"', erinnert er sich.

% Fußnote hinzugefügt
"`Und die GDB-Maintainer waren zu der Zeit auch nicht so aktiv."' Ihre Priorität war die Unterstützung der bestehenden Plattformen der damaligen GDB-Nutzer, und nicht das Fernziel eines kompletten GNU-Systems. Aber fundamental war, so Stallman, dass er und die Hurd-Entwickler die Schwierigkeiten unterschätzt haben, die eine Entwicklung von Unix-Kernel-Funktionen auf einem Mach-Microkernel machen. "`Ich habe mir gedacht, O.\,k., der [Mach]-Teil, der mit dem Rechner kommuniziert, ist schon debuggt"', sagt Stallman, wenn er sich an die Schwierigkeiten des Hurd-Teams in einer Rede im Jahr 2000 erinnert. "`Mit diesem Vorsprung sollten wir in der Lage sein, schneller fertig zu werden. Aber es stellte sich heraus, dass das Debuggen dieser asynchronen Multithread-Programme sehr schwierig ist. Es gab Timing-Fehler,\footnote{Gemeint sind wahrscheinlich Race Conditions.} die Dateien überschreiben, und das war kein Vergnügen. Im Endeffekt hat es viele lange Jahre gedauert, bis wir eine Testversion fertigstellen konnten."'%\footnote{Vgl. Rede am Maui High Performance Computing Center. In folgenden E-Mails fragte ich Stallman, was genau er mit dem Begriff "`Timing-Bugs"' meint. Stallman sagte, "`Timing-Fehler"' fasse das Problem besser zusammen und er gab aufklärende technische Informationen, wie ein Timing-Fehler die Funktion eines Systems beeintächtigen kann: \quote{"`Timing-Fehler"' treten in asynchronen Systemen auf, wo parallele Jobs theoretisch in jeder Reihenfolge auftreten können, aber eine bestimmte Reihenfolge Probleme verursacht. Man hat Programm A, das X ausführt, und Programm B, das Y ausführt. X und Y sind kurze Routinen, die dieselbe Datenstruktur lesen und verändern. Fast immer führt der Computer X vor Y aus, oder Y vor X, und dann gibt es keine Probleme. Aber in seltenen Fällen lässt der Scheduler zufällig Programm A laufen, bis es X zur Hälfte abgearbeitet hat, und lässt dann Programm B laufen, was Y ausführt. Dann ist Y fertig und X halbfertig. Da sie auf dieselbe Datenstruktur schreiben, kommen sie sich in die Quere. Zum Beispiel hat vielleicht X gerade die Daten gelesen, und es erkennt dann nicht, dass es eine Änderung gab. Es kommt dann zu nicht reproduzierbaren Fehlern, weil sie nur durch den Zufall bedingt sind (wenn der Scheduler entscheidet, welches Programm wie lange laufen soll). Man vermeidet solche Fehler, indem man Locking verwendet, damit X und Y nicht zur gleichen Zeit laufen. Programmierer, die asynchrone Systeme schreiben, wissen generell, dass man Locks benötigt, aber manchmal übersehen sie den Bedarf eines Lockings an einer bestimmten Stelle oder einer bestimmten Datenstruktur. Dann kommt es zu einem Timing-Fehler im Programm.}}

Mit der Zeit und dem wachsenden Erfolg von GNU in Verbindung mit Linux wurde es klar, dass das GNU Project auf den fahrenden Zug aufspringen und nicht auf die Hurd warten sollte. Abgesehen davon gab es Schwachstellen in der Gemeinde um GNU/Linux. Natürlich stand Linux unter der GPL, aber wie Murdock selbst feststellte, war der Wunsch, GNU/Linux als rein freies Betriebssystem zu behandeln, absolut nicht einstimmig. Bis Ende 1993 war die Gemeinde der GNU/Linux-Nutzer von etwa einem Dutzend Enthusiasten auf 20.000 bis 100.000 gewachsen.\footnote{Die GNU/Linux-Nutzerzahlen sind bestenfalls vage, deswegen habe ich eine so große Spanne angegeben. Die Zahl 100.000 stammt von Red Hats "`Milestones"'-Seite, \url{http://www.redhat.com/about/corporate/milestones.html}.}

Was einst ein Hobby gewesen ist, war nun ein Markt, aus dem man Kapital schlagen konnte, und einige Entwickler hatten keine Bedenken, ihn mit unfreier Software zu erschließen. Wie Winston Churchill, der die sowjetischen Truppen in Berlin einmarschieren sah, hatte Stallman verständlicherweise ambivalente Gefühle, wenn es darum ging, den "`Erfolg"' von GNU/Linux zu feiern.\footnote{Ich habe diesen Churchill-Vergleich geschrieben, bevor Stallman selbst mir unaufgefordert einen Kommentar zu Churchill schickte:
\quote{
Der Zweite Weltkrieg und die Entschlossenheit, die notwendig war, um ihn zu gewinnen, war eine lebhafte Erinnerung als ich aufgewachsen bin. Äußerungen wie Churchills
"`We will fight them in the landing zones, we will fight them on the beaches\ldots we will never surrender"' sind mir für immer im Gedächtnis geblieben.}
}

Obwohl Stallman etwas spät dran war, hatte er immer noch Schlagkraft. Als die FSF angekündigt hatte, Murdocks freies Softwareprojekt monetär und moralisch zu unterstützen, kamen auch weitere Unterstützungsangebote. Murdock nannte das neue Projekt Debian\index{Debian|(} – eine Kombination aus seinem und dem Namen seiner Frau Deborah – und innerhalb einiger Wochen brachte er die erste Version heraus. "`[Richards Unterstützung] katapultierte Debian fast über Nacht von einem interessanten kleinen Projekt in die Position, in der die Leute in der Gemeinde es einfach beachten mussten"', sagt Murdock.

Im Januar 1994 veröffentlichte Murdock sein \textit{Debian Manifesto}\index{Debian Manifesto}. Es erklärte im Geiste des ein Jahrzehnt älteren \textit{GNU Manifestos} von Stallman die Wichtigkeit einer engen Zusammenarbeit mit der Free Software Foundation. Murdock schrieb:
\begin{quote}
Die Free Software Foundation spielt eine extrem wichtige Rolle für Debians Zukunft. Einfach durch den Umstand, dass sie es verbreiten werden, schickt es die Botschaft an die Welt, dass Linux [\textit{sic}] kein kommerzielles Produkt ist und niemals sein soll, aber das heißt nicht, dass Linux niemals kommerziell konkurrenzfähig sein wird. Denen, die anderer Meinung sind, sage ich, sie sollen sich den Erfolg von GNU Emacs und GCC klarmachen, beides nichtkommerzielle Softwareprodukte, die trotzdem einen starken Einfluss auf den kommerziellen Markt hatten. Es ist an der Zeit, sich auf die Zukunft von Linux [\textit{sic}] zu konzentrieren, statt sich dem destruktiven Ziel zu widmen, sich auf Kosten der gesamten Linux-Gemeinde und ihrer Zukunft zu bereichern. Die Entwicklung und Verbreitung von Debian mag nicht die Lösung für die Probleme sein, die ich im Manifesto dargelegt habe, aber ich hoffe, sie wird hoffentlich zumindest genug Aufmerksamkeit auf diese Probleme ziehen, dass man sie lösen kann.\footcite[Vgl.][\textit{Appendix A - The Debian Manifesto}]{debhist}
\end{quote}

Kurz nach der Veröffentlichung des \textit{Manifestos} stellte die Free Software Foundation ihre erste große Forderung. Stallman wollte, das Murdock seine Distribution "`GNU/Linux"' nennt. Zuerst hatte Stallman den Begriff "`Lignux"' vorgeschlagen,\comment{– eine Kombination der Namen Linux und GNU –} aber die Reaktionen darauf waren sehr negativ und das überzeugte Stallman vom längeren, aber weniger kritisierten GNU/Linux.
Einige Leute taten Stallmans Bemühungen um den "`GNU"'-Präfix als spätes Heischen nach Anerkennung ab, egal ob verdient oder nicht. Aber Murdock sah die Dinge etwas anders, rückblickend sieht er es als Versuch, der wachsenden Spannung zwischen den Entwicklern des GNU Projects und denen, die GNU-Programme für den Linux-Kernel angepasst haben, entgegenzuwirken. "`Es war eine Spaltung im Gange"', erinnert sich Murdock, "`Richard war besorgt."'

Bis zum Jahr 1990 hatte jedes GNU-Programm einen designierten Maintainer.
Einige GNU-Programme liefen auf vielen verschiedenen Systemen und die Nutzer lieferten oft Änderungen, damit sie auf weiteren Systemen laufen. Oftmals kannten diese Nutzer nur das System, auf dem sie arbeiteten, und dachten nicht daran, wie man den Code auch für andere Systeme sauberhält. Man musste viele der Änderungen größtenteils umschreiben, um das neue System dann zu unterstützen und den Code verständlich zu halten, so dass er für alle Systeme zuverlässig gewartet werden kann. Dem Maintainer oblag die Verantwortung, die Änderungen zu kritisieren und den Schreibern zu sagen, dass sie Teile ihres Ports neu schreiben sollten. Im Allgemeinen waren sie absolut bereit dazu, damit ihre Änderungen in die Standardversion einfließen. Der Maintainer würde dann die umgeschriebenen Änderungen bearbeiten und sie in Zukunft warten. Bei einigen GNU-Programmen war das dutzende Male für dutzende Systeme passiert.
Die Programmierer, die die verschiedenen GNU-Programme für den Linux-Kernel anpassten, gingen meist den einfachen Weg und betrachteten nur ihre eigene Plattform. Als die Projektleiter sie dann gebeten haben, ihre Änderungen zu bereinigen\comment{for future maintenance}, zeigten einige von ihnen kein Interesse daran. Sie haben sich nicht darum gekümmert, das Richtige zu tun oder die zukünftige Wartung der GNU-Pakete zu erleichtern, sondern nur um ihre eigenen Versionen und wollten sie als Forks weiterführen.

In der Hackerwelt sind Forks ein interessantes Phänomen. Obwohl die Hackerethik es dem Programmierer erlaubt, mit dem Quellcode eines Programms zu machen, was er will, wird es als guter Ton angesehen, dem ursprünglichen Entwickler seine Unterstützung anzubieten, damit eine gemeinsame Version erhalten bleibt. Hacker finden es im Allgemeinen nützlich und zweckmäßig, wenn ihre Verbesserungen in die Hauptversion einfließen. Eine freie Softwarelizenz gibt jedem Hacker das Recht, ein Programm zu forken, und manchmal ist das auch nötig, aber ohne Notwendigkeit oder Grund wird der Akt etwas unhöflich.

Als Leiter des GNU Projects hatte Stallman die negativen Auswirkungen eines Forks schon 1991 spüren müssen. Stallman sagt, "`Lucid hatte einige Leute eingestellt, die Verbesserungen schreiben sollten, welche als Beiträge zu GNU Emacs geplant waren; aber die Entwickler haben mich nicht über das Projekt informiert. Stattdessen haben sie selbst mehrere neue Funktionen entworfen. Wie man sich denken kann, war ich mit einigen Entscheidungen einverstanden und mit anderen wiederum nicht. Sie hatten mich gebeten, ihren gesamten Code einzupflegen, aber als ich sagte, ich wolle nur etwa die Hälfte verwenden, haben sie sich geweigert, mir mit den Anpassungen zu dieser Hälfte allein zu helfen. Ich musste alles selbst machen."' Der Fork war der Beginn für die Paralellversion Lucid Emacs und für allerhand Feindseligkeiten.\footnote{Jamie Zawinski, ehemaliger Lucid-Programmer, der später das Mozilla-Entwicklerteam anführen sollte, hat dem Lucid/GNU-Emacs-Fork eine Webseite gewidmet, "`The Lemacs/FSFmacs Schism"',  \url{http://www.jwz.org/doc/lemacs.html}. Stallmans Antwort auf diese Anschuldigungen findet man unter \url{http://stallman.org/articles/
xemacs.origin}.}

Jetzt waren verschiedene der GNU-Hauptpakete geforkt. Zuerst sah Stallman die Forks nur als Ausdruck der Ungeduldigkeit an. Im Gegensatz zur schnellen und zwanglosen Dynamik im Linux-Team tendierten die GNU-Programmmaintainer dazu, langsamer zu sein und vorsichtiger bei Änderungen, die die langfristige Existenzfähigkeit eine Programms beeinträchtigen könnten. Sie hatten auch keine Scheu, anderer Leute Code harsch zu kritisieren. Im Laufe der Zeit fing Stallman beim Lesen ihrer E-Mails an, zu merken, dass es vielen Linux-Entwicklern an Kenntnis über das GNU Project und seiner Ziele fehlte.

"`Wir fanden heraus, dass den Leute, die sich als \glq Linux-Nutzer\grq{} betrachteten, das GNU Project egal war"', sagt Stallman. "`Sie meinten \glq Warum soll ich mich mit diesen Dingen befassen? Mich interessiert das GNU Projekt nicht. Bei mir läuft es doch. Es läuft bei den Linux-Nutzern und alles andere ist uns egal.\grq{} Und das war ziemlich überraschend, weil die Leute im Grunde eine Variante des GNU-Systems benutzt haben und sie das wenig interessiert hat. Sie haben sich weniger darum gekümmert als irgendwer sonst um GNU."' Wegen ihrer eigene Angewohnheit, die Kombination "`Linux"' zu nennen, haben sie nicht erkannt, dass das System mehr GNU war als Linux.

Um der Einigkeit willen bat Stallman die Projektleiter, die Arbeit zu übernehmen, die normalerweise die Autoren der Veränderungen gemacht haben. In den meisten Fällen war das machbar, aber nicht bei glibc. glibc, kurz für GNU C Library, ist das Paket, dass alle Programme nutzen, um "`Systemaufrufe"' an den Kernel zu machen, in diesem Fall Linux. Programme im Benutzer-Modus kommunizieren in unixähnlichen Systemen nur durch die C-Bibliothek mit dem Kernel.

Die Änderungen, die nötig waren, um glibc zum Kommunikationskanal zwischen Linux und den anderen Systemprogrammen zu machen, waren fundamental und sehr speziell, und wurden ohne Rücksicht auf die Effekte auf andere Plattformen geschrieben.
Für den glibc-Maintainer war die Vorstellung entmutigend, die Änderungen bereinigen zu müssen. Die Free Software Foundation bezahlte ihn stattdessen dafür, dass er fast ein Jahr diese Änderungen völlig neu implementiert, so dass Version 6 auf GNU/Linux ohne Probleme läuft.

Murdock sagt, das war der Auslöser für Stallmans Beharren auf den GNU-Präfix zu der Zeit, als er Debian gestartet hat. "`Der Fork ist seitdem wieder ins Hauptprojekt konvergiert. Aber zu der Zeit gab es die Sorge, dass wenn sich die Linux-Gemeinde als etwas anderes als die GNU-Gemeinde ansieht, das vielleicht eine spalterische Kraft ist."'

Obwohl es einige als politisch anmaßend ansahen, die Kombination aus GNU und Linux als GNU-"`Variante"' zu bezeichnen, betrachtete Murdock, der schon mit dem Free-Software-Gedanken sympathisierte, Stallmans Bitte, die Debian-Version "`GNU/Linux"' zu nennen, als angemessen. "`Es ging eher um die Einheit als um Anerkennung"', sagt er.
Bitten eher technischerer Natur sollten schnell folgen. Obwohl Murdoch bei den politischen Themen entgegenkommend war, nahm er eine festere Position ein, wenn es ums Design- und Entwicklungsmodell der eigentlichen Software ging. Was als Demonstration der Solidarität angefangen hatte, war bald zu einer laufenden Uneinigkeit geworden.
"`Man kann schon sagen, dass ich reichlich Auseinandersetzungen mit ihm hatte"', sagt Murdock und lacht. "`Ganz ehrlich, Richard kann ein ziemlich schwieriger Mensch sein, wenn man mit ihm arbeitet."' Die Hauptdiskrepanz bestand beim Debuggen. Stallman wollte Debugging-Informationen in allen ausführbaren Programmen, damit die Nutzer sofort alle auftretenden Fehler untersuchen können. Murdock dachte, das würde die Systemdateien zu groß machen und die Verbreitung behindern. Keiner von beiden wollte seine Meinung ändern.

1996 entschied sich Murdock nach seinem Abschluss and der Purdue University, die Zügel des wachsenden Debian-Projekts abzugeben. Er hatte schon vorher Verwaltungsaufgaben an Bruce Perens\index{Perens, Bruce|(} übergeben, seinerseits Hacker, am besten bekannt als Schöpfer von Busybox.\comment{"`Electric Fence"', einem unter der GPL erschienenen Unix-Werkzeug.} Perens war wie Murdock Unix-Programmierer, der sofort von GNU/Linux begeistert war, als sich die unix\-artigen Züge des Betriebssystems zu manifestieren begannen. Wie Murdock sympathisierte Perens mit der politischen Agenda Stallmans und der Free Software Foundation, obwohl nur aus der Ferne.

"`Ich erinnere mich, nachdem Stallman das GNU Manifesto, GNU Emacs und GCC herausgebracht hatte, einen Artikel gelesen zu haben, in dem stand, dass er Berater für Intel war"', sagt Perens zu seiner ersten Berührung mit Stallman in den späten 80ern. "`Ich habe ihm geschrieben und ihn gefragt, wie er einersetis als Befürworter freier Software andererseits für Intel arbeiten könne. Er schrieb zurück \glq Ich arbeite als Berater für die Entwicklung freier Software.\grq{} Er war absolut höflich bei dem Thema, und ich hielt seine Antwort für völlig einleuchtend."'

Als einer der führenden Debian-Entwickler sah Perens Murdocks Design-Streitigkeiten mit Stallman jedoch mit Sorge. Als er die Führungsposition über das Entwicklerteam von Debian übernahm, sagt Perens, habe er sich entschlossen, sich von der Free Software Foundation zu distancieren. "`Ich hatte mich entschieden, dass wir den Micromanaging-Stil von Richard nicht wollen"', sagt er.
Laut Perens\index{Perens, Bruce|)} war Stallman überrascht von der Entscheidung, war aber so weise, sich damit abzufinden. "`Er gab mir etwas Zeit, mich abzuregen und schickte mir eine Nachricht, dass wir unbedingt in Kontakt bleiben sollen. Er bat darum, dass wir es GNU/Linux nennen, und beließ es dabei. Ich habe mich entschieden, dass das ok ist. Ich habe die Entscheidung allein getroffen. Alle haben erleichtert aufgeatmet."'
Mit der Zeit entwickelte sich für Debian der Ruf als die Hacker-Version von GNU/Linux, gemeinsam mit Slackware\index{Slackware}, einer anderen populären Distribution, die in derselben Zeit um 1993-1994 entstand. Jedoch enthielt Slackware einige unfreie Programme und Debian begann nach seiner Trennung von GNU auch mit der Verbreitung unfreier Programme.\footnote{Debian Buzz enthielt ab Juni 1996 das unfreie Netscape 3.01 in seinem Contrib-Zweig.} Trotz der Markierung als "`non-free"' und der Beteuerung, dass sie "`kein offizieller Bestandteil von Debian"' seien, stellte die Aufnahme dieser Programme für den Nutzer eine Art Billigung dar. Als das GNU Project von dieser Praxis Wind bekam, wurde ihnen klar, dass sie der Allgemeinheit weder Slackware noch Debian als GNU/Linux-Distro empfehlen konnten.

Außerhalb der Hackerwelt gewann GNU/Linux an Schwung auf dem kommerziellen Unix-Markt. In North Carolina krempelte ein Unix-Unternehmen namens Red Hat sein Geschäft um, um sich auf GNU/Linux zu konzentrieren. Sein Hauptgeschäftsführer war Robert Young\index{Young, Robert|(}, der frühere \textit{Linux-Journal-}Redakteur, der 1994 Linus Torvalds fragte, ob er es bereue, den Kernel unter die GPL gestellt zu haben. Für Young hatte Torvalds' Antwort eine "`profunde"' Auswirkung auf seine eigene Sicht auf GNU/Linux.

Statt mit traditionellen Softwaretaktiken zu versuchen, den GNU/Linux-Markt zu beherrschen, machte Young sich Gedanken darüber, was passieren würde, wenn man denselben Ansatz wie Debian\index{Debian|)} wählt – d.\,h. ein Betriebssystem komplett aus freier Software zu bauen. Cygnus Solutions\index{Cygnus Solutions}, die 1990 von Michael Tiemann\index{Tiemann, Michael} und John Gilmore\index{Gilmore, John} gegründete Firma, hatte schon bewiesen, dass man qualitativ hochwertige und anpassbare freie Software verkaufen konnte. Was wäre, wenn Red Hat dieselbe Herangehensweise für GNU/Linux übernehmen würde?

"`Nach der westlichen Wissenschaftstradition stehen wir auf den Schultern von Riesen"', sagt Young und zitiert Torvalds und Sir Isaac Newton. "`In der Geschäftswelt bedeutet das, dass wir auf unserem Weg nicht das Rad neu erfinden müssen. Die Schönheit [des GPL]-Modells liegt darin, dass man den Code in die Public domain gibt.\footnote{Young nutzt den Begriff "`Public domain"' hier sehr frei. Genauer gesagt bedeutet es "`nicht urheberrechtlich geschützt"'. Code, der unter der GNU GPL veröffentlicht wird, kann nicht gemeinfrei sein, weil er geschützt sein muss, damit die GNU GPL greifen kann.} Wenn man ein unabhängiges Softwareunternehmen ist und man eine Applikation erstellen will, und dazu eine Modemeinwahlsoftware braucht, warum sollte man dann die Einwahlsofware neu erfinden? Man kann einfach PPP von Red Hat [GNU/]Linux entwenden und es als Basis für sein eigenes Einwahltool verwenden. Wenn man ein graphisches Toolkit braucht, muss man nicht seine eigene Bibliothek schreiben, nur GTK herunterladen. So hat man schnell die Möglichkeit, das Beste von dem zu benutzen, was es schon gibt. Und dann kann man sich als Softwarehändler plötzlich weniger auf Softwaremanagement konzentrieren und mehr auf das Schreiben von Anwendungen speziell den Kundenwünschen entsprechend."' Jedoch war Young kein Aktivist für freie Software und hat unbesehen unfreie Programme in Red Hats Distribution aufgenommen.

%XXX Schwall/Welle?
Young\index{Young, Robert|)} war nicht der einzige Geschäftsmann in einem Softwareunternehmen, der die Wirtschaftlichkeit von freier Software gesehen hat. Ende 1996 waren die meisten Unix-Firmen aufgewacht und rochen den Sourcecode-Braten. Der GNU/Linux-Sektor war noch gut ein, zwei Jahre vom kommerziellen Durchbruch entfernt, aber diejenigen, die der Hackergemeinde nahe standen, konnten es merken: etwas Großes war im Gange. Intels 386-Prozessor, das Internet und das World Wide Web waren wie riesige Wellen auf den Markt eingeschlagen; freie Software schien die bis dahin größte Welle.

Für Ian Murdock schien die Welle gleichzeitig eine passende Anerkennungsbezeugung und eine passende Bestrafung für den Mann zu sein, der soviel Zeit darauf verwendet hatte, der Free-Software-Bewegung ein Gesicht zu geben. Wie viele Linux-Anhänger hatte Murdock die anfänglichen Postings gelesen. Er hatte einst Torvalds' Warnung gesehen, dass Linux "`nur ein Hobby"' sei. Er hatte auch Torvalds' Eingeständnis gegenüber dem Schöpfer von Minix, Andrew Tanenbaum, gesehen: "`Wenn der GNU-Kernel schon im letzten Frühling fertig gewesen wäre, hätte ich mir nicht die Mühe gemacht, das Projekt überhaupt zu starten."'\footnotemark{} Wie viele andere wusste auch Murdock, dass einige Möglichkeiten verschenkt worden waren. Er kannte auch die Aufregung, neue Möglichkeiten im Internet auftauchen zu sehen.
"`In den Anfangstagen an Linux beteiligt gewesen zu sein, hat Spaß gemacht"', erinnert sich Murdock. "`Und gleichzeitig hatte man etwas zu tun, etwas, um die Zeit rumzukriegen. Wenn man sich noch mal die alten Diskussionen auf [comp.os.minix] durchliest, sieht man die [vorherrschende] Meinung: das ist was, an dem wir rumbasteln können, bis Hurd fertig ist. Die Leute waren ungeduldig. Es ist witzig, aber ich vermute, in vielerlei Hinsicht, dass Linux nie passiert wäre, wenn die Hurd schneller vorangekommen wäre."'

%oder \url{http://oreilly.com/catalog/opensources/book/appa.html}
\footnotetext{Das Zitat stammt aus dem sehr bekannten Torvalds/Tanenbaum-"`Flame-war"' nach der Erstveröffentlichung von Linux. Während der Verteidigung seiner Wahl eines nichtportablen monolithischen Kerneldesigns sagt Torvalds, er habe nur mit Linux angefangen, um mehr über seinen neuen 386er zu lernen. 
\citet[Vgl.][S.\,224, \textit{Appendix A: The Tanenbaum-Torvalds Debate}]{opensrc}.}

% Ende: Stallman
Bis zum Ende 1996 waren solche Was-wäre-wenn-Fragen jedoch schon irrelevant, weil Torvalds' Kernel eine kritische Masse an Nutzern gewonnen hatte. Das 36-Monats-Fenster hatte sich geschlossen, was bedeutete, dass selbst wenn das GNU Project den Hurd-Kernel herausgebracht hätte, die Chancen gering gewesen wären, dass irgendjemand außerhalb der Hardcore-Hackergemeinde es mitbekommen hätte. Mit Linux als Lückenfüller für das GNU-System war das Ziel des GNU Projects erreicht, ein unixoides, freies Betriebssystem zu entwickeln. Jedoch erkannten die meisten Nutzer nicht, was hier passiert war: sie dachten, das ganze System wäre Linux, und dass Torvalds es komplett selbst entwickelt hätte. Die meisten von ihnen installierten Distributionen, die unfreie Software ausliefern; mit Torvalds als moralische Leitfigur sahen sie keinen fundamentalen Grund, diese Software abzulehnen. Trotzdem war eine prekäre Freiheit für diejenigen zu haben, die sie zu schätzen wussten.
