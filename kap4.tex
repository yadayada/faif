\chapter{Gott absetzen!}

\comment{
Obwohl ihr Verhältnis angespannt war, hatte Richard Stallman eine bemerkenswerte Qualität von seiner Mutter geerbt: eine Leidenschaft für progressive Politk.

It was an inherited trait that would take several decades to emerge, however. In den ersten paar Jahren seines Lebens lebte Stallman lived in what he now admits was a "`politischem Vakuum"'\footcite{rmshsm} Wie die meisten Familien während der Eisenhower-Zeit, versuchte die Familie Stallman in den 50ern, die Normalität, die in den Kriegsjahren der 40 verloren gegangen ist, wieder aufleben zu lassen.

"`Richards Vater und ich waren Demokraten und waren zufrieden, es dabei zu belassen"', sagt Lippman, wenn sie sich an ihr Familienleben in Queens erinnert. "`Wir haben uns nicht sehr engagiert in lokaler und nationaler Politik."'

Das sollte sich jedoch ändern, als sich Alice Ende der 50er von Daniel Stallman scheiden ließ. Das Zurückziehen nach Manhattan stellte mehr als nur eine Änderung der Anschrift dar; es spiegelte eine neue, unabhängige Identität wider und einen erschreckenden Verlust von Gelassenheit.

"`Ich glaube, ich habe den Vorgeschmack für politischen Aktivismus bekommen, als ich einmal in die öffentliche Bibliothek in Queens gegangen bin und entdeckt habe, dass es dort nur ein einziges Buch über Scheidung in der ganzen Bibliothek gab"', erinnert sich Lippman. "`Es war sehr von der katholischen Kirche kontrolliert, jedenfalls in Elmhurst, wo wir gelebt haben. Ich denke, dass war die erste Ahnung, die ich bekommen habe, von den Kräften, die heimlich unser Leben steuern."'

Returning to her childhood neighborhood, Manhattan's Upper West Side, Lippman was shocked by the changes that had taken place since her departure to Hunter College a decade and a half before. The skyrocketing demand for post-war housing had turned the neighborhood into a political battleground. On one side stood the pro-development city-hall politicians and businessmen hoping to rebuild many of the neighborhood's blocks to accommodate the growing number of white-collar workers moving into the city. On the other side stood the poor Irish and Puerto Rican tenants who had found an affordable haven in the neighborhood.

At first, Lippman didn't know which side to choose. As a new resident, she felt the need for new housing. As a single mother with minimal income, however, she shared the poorer tenants' concern over the growing number of development projects catering mainly to wealthy residents. Indignant, Lippman began looking for ways to combat the political machine that was attempting to turn her neighborhood into a clone of the Upper East Side.

Lippman says her first visit to the local Democratic party headquarters came in 1958. Looking for a day-care center to take care of her son while she worked, she had been appalled by the conditions encountered at one of the city-owned centers that catered to low-income residents. "`All I remember is the stench of rotten milk, the dark hallways, the paucity of supplies. I had been a teacher in private nursery schools. The contrast was so great. We took one look at that room and left. That stirred me up."'

The visit to the party headquarters proved disappointing, however. Describing it as "`the proverbial smoke-filled room,"' Lippman says she became aware for the first time that corruption within the party might actually be the reason behind the city's thinly disguised hostility toward poor residents. Instead of going back to the headquarters, Lippman decided to join up with one of the many clubs aimed at reforming the Democratic party and ousting the last vestiges of the Tammany Hall machine. Dubbed the Woodrow Wilson/FDR Reform Democratic Club, Lippman and her club began showing up at planning and city-council meetings, demanding a greater say.

"`Unser Hauptziel war die Bekämpfung von Tammany Hall, Carmine DeSapio und seinen Handlangern"',\footnote{Carmine DeSapio holds the dubious distinction of being the first Italian-American boss of Tammany Hall, the New York City political machine. For more information on DeSapio and the politics of post-war New York, see John Davenport, "`Skinning the Tiger: Carmine DeSapio and the End of the Tammany Era,"' \textit{New York Affairs} (1975): 3:1.} sagt Lippman. "`I was the representative to the city council and was very much involved in creating a viable urban-renewal plan that went beyond simply adding more luxury housing to the neighborhood."'

Such involvement would blossom into greater political activity during the 1960s. By 1965, Lippman had become an "`outspoken"' supporter for political candidates like William Fitts Ryan, a Democrat elected to Congress with the help of reform clubs and one of the first U.S. representatives to speak out against the Vietnam War.

Nicht lange später wurde auch Lippman ein öffentlicher Gegner der Einmischung der USA in Indochina. "`Ich war gegen den Vietnamkrieg vom Tag an als Kennedy Truppen geschickt hat"', sagt sie. "`Ich hatte die Artikel von den Reportern und Journalisten gelesen, die man dorthin geschickt hatte, um von der Frühphase des Konflikts zu berichten. I really believed their forecast that it would become a quagmire."'

%Kommentarende
Such opposition permeated the Stallman-Lippman household.} 1967 heiratete Lippman erneut. Ihr neuer Mann, Maurice Lippman, Major in der Air National Guard, gab seine Offiziersposition auf, um seinen Protest gegen den Krieg auszudrücken. Lippmans Stiefsohn, Andrew Lippman, war am MIT und für eine vorübergehende Rückstellung vom Wehrdienst berechtigt. Trotzdem bestand die Gefahr, eingezogen zu werden, wenn die Rückstellung endete, was das Risiko einer US-Eskalation umso imminenter machte. Außerdem gab es da noch Richard, der, obwohl jünger, auch der Gefahr einer\comment{ möglichen} Einberufung gegenüberstand, als der Krieg sich bis in die 70er zog.

"`Vietnam war ein wichtiges Thema in unserem Haushalt"', sagt Lippman. "`Wir haben ständig darüber gesprochen: was wir tun würden, wenn der Krieg noch länger geht, was Richard und sein Stiefbruder tun würden, wenn sie einberufen werden. Wir waren alle gegen den Krieg und den Kriegsdienst. Wir haben wirklich gedacht, es ist unmoralisch."'

Bei Stallman löste der Vietnamkrieg eine Mischung aus Gefühlen aus: Verwirrung, Grauen und schließlich ein tiefgehendes Gefühl von politischer Ohnmacht. Als Kind, das kaum die leicht autoritäre Welt einer Privatschule ertragen konnte, fuhr Stallman jedes mal ein Schauer über den Rücken, wenn er an ein Ausbildungslager der Army dachte. Er glaubte nicht, dass er es geistig gesund überstehen hätte können.

"`Ich war am Boden vor Angst, aber ich wusste nicht, was ich machen sollte, und hatte nicht den Mut, demonstrieren zu gehen"', erinnert sich Stallman, dessen Geburtstag am 16.\,März ihm eine niedrige Nummer im Einberufungslotto einbrachte. Es hat ihn nicht sofort betroffen, er hatte eine Rückstellung wegen seines Studiums bekommen, eine der letzten, die die USA überhaupt vergeben hat; aber ein paar Jahre später würde es ihn betreffen. "`Ich konnte mir nicht vorstellen, nach Kanada oder Schweden zu ziehen. Die Vorstellung, mich aufzuraffen und allein irgendwohin zu ziehen. Wie hätte ich das machen sollen? Ich wusste nicht, wie ich allein zurechtkommen sollte. Ich war nicht so jemand, der solche Dinge mit Zuversicht angeht."'

Stallman sagt, er war beeindruckt von den Familienmitgliedern, die sich gegen den Krieg ausgesprochen haben. Er erinnert sich an einen Aufkleber, den sein Vater gedruckt und verteilt hat, der das My-Lai-Massaker mit ähnlichen Gräueltaten der Nazis während des Zweiten Weltkriegs verglich, und sagt, er sei "`aufgeregt"' gewesen von der Geste der Empörung seines Vaters. "`Ich habe ihn dafür bewundert"', sagt Stallman. "`Aber ich konnte mir nicht vorstellen, selbst etwas zu machen. Ich hatte Angst, dass mich die Marter des Kriegsdienst zerstört."'

Jedoch sagt Stallman, er sei von dem Ton und der Richtung, in die viele in der Bewegung gingen, abgeschreckt worden. Wie die anderen Mitglieder des Science Honors Program sah er die Wochenenddemonstrationen an der Columbia University als kaum mehr als ein ablenkendes Spektakel an.\footnote{Chess, auch Abgänger des Columbia Science Honors Program, beschreibt die Proteste als "`Hintergrundrauschen"'. "`Wir waren alle politisch"', sagt er, "`aber das SHP war wichtig. Wir hätten es nie für eine Demonstration geschwänzt."'} Letzten Endes, sagt Stallman, wurden die irrationalen Kräfte hinter der Antikriegsbewegung ununterscheidbar von den irrationalen Kräften hinter dem Rest der Jugendkultur. Statt die Beatles zu vergöttern, vergötterten die Mädchen in Stallmans Altersgruppe plötzlich Aufwiegler wie Abbie Hoffman und Jerry Rubin. Für ein Kind, dass ohnehin schon kaum seine Teenager-Altersgenossen verstand, hatten Sprüche wie "`Make love, not war"' einen spöttischen Unterton. Stallman wollte keinen Krieg, jedenfalls nicht in Südostasien, aber es wollte auch niemand mit ihm Liebe machen.

"`Ich konnte die Gegenkultur nicht sehr leiden"', erinnert sich Stallman. "`Ich mochte die Musik nicht, ich mochte die Drogen nicht. Ich hatte Angst vor den Drogen. Und besonders den Antiintellektualismus konnte ich nicht leiden, und auch nicht die Vorurteile gegen Technologie. Schließlich habe ich Computer geliebt. Und ich mochte den grundlosen Antiamerikanismus nicht, der mir oft begegnete. Es gab Leute, deren Denke so simpel war, dass sie dachten, wenn sie das Verhalten der USA im Vietnamkrieg nicht für gut befinden, sie die Nordvietnamesen unterstützen müssen. Sie konnten sich eine kompliziertere Position gar nicht vorstellen, glaube ich."'

Solche Kommentare unterstreichen eine Eigenschaft, die der Schlüssel in Stallmans eigenem politischen Reifeprozess werden sollte. Für Stallman war politisches Selbstvertrauen direkt proportional zu dem persönlichen Selbstvertrauen. 1970 war Stallman nur in wenigen Dingen selbstsicher außerhalb des Reichs der Mathematik und Naturwissenschaften. Trotzdem bot ihm seine Selbstsicherheit in Mathematik Grundlage genug, um die Extreme der Antikriegsbewegung auf eine rein logische Weise zu untersuchen. Stallman befand die Logik für unzulänglich. Obwohl er gegen den Krieg in Vietnam war, sah er keinen Grund, dem Krieg die Berechtigung als Mittel zur Verteidigung von Freiheit oder der Behebung von Unrecht abzusprechen.

In den 80ern entschied sich ein selbstsicherer Stallman, seine frühere Untätigkeit wettzumachen, indem er an Massendemos für Abtreibungsrechte in Washington DC teilnimmt. "`Ich wurde unzufrieden über mein früheres Ich, weil ich meine Pflicht zum Protest am Vietnamkrieg nicht erfüllt habe"', erklärt er.

1970 ließ Stallman die abendlichen Gespräche am Esstisch über Politik und den Vietnamkrieg hinter sich, als er nach Harvard ging. Im Rückblick sieht Stallman den Übergang von der Wohnung seiner Mutter in Manhattan zum Leben im Wohnheim in Cambridge als "`Entkommen"' an. In Harvard konnte er immer in sein Zimmer gehen und seine Ruhe haben, wenn er wollte. Freunde, die den Übergang beobachtet haben, konnten jedoch wenig erkennen, was nach einer befreienden Erfahrung aussah.

"`Er sah ziemlich unglücklich aus in der ersten Zeit in Harvard"', erinnert sich Dan Chess\index{Chess, Dan|(}, Klassenkamerad im Science Honors Program, der auch in Harvard immatrikuliert war. "`Man konnte sehen, dass soziale Interaktion sehr schwer für ihn war, und in Harvard konnte man ihr nicht ausweichen. Harvard war ein stark gesellschaftlicher Ort."'

Um sich den Übergang zu erleichtern, besann Stallman sich auf seine Stärken: Mathe und Naturwissenschaften. Wie für die meisten Mitglieder des Science Honors Program war der Aufnahmetest für Math~55\index{Math 55|(} ein Leichtes für ihn. Harvards Math~55 ist das legendäre "`Ausbildungslager"' für Mathe-"`Konzentratoren"' im ersten Studienjahr. Im dem Kurs bildeten die Ehemaligen des Science Honors Program eine währende Einheit. "`Wir waren die Mathemafia"', lacht Chess. "`Harvard war nichts im Vergleich zum SHP."'

Um sich das Recht zu prahlen zu verdienen, mussten Stallman, Chess und die anderen SHP-Alumni Math~55 jedoch erst einmal bestehen. Mit den in zwei Semester gepackten zwei Jahren Mathestoff war der Kurs nur etwas für die wirklichen Enthusiasten. "`Es war ein unglaublicher Kurs"', sagt David Harbater\index{Harbater, David|(}, ehemaliger "`Mathemafioso"' und heutiger Mathematikprofessor an der University of Pennsylvania. "`Ich glaube, man kann sagen, dass es nie einen anderen Kurs für College-Anfänger gegeben hat, der so intensiv und fortgeschritten war. Den Satz, den ich Leuten sage, um ihnen das klarzumachen, ist, dass wir unter anderem im zweiten Semester Differentialgeometrie in Banach-Mannigfaltigkeiten besprochen haben. Dann reißen sie meistens ihre Augen auf, weil die meisten über Banach-Mannigfaltigkeiten nicht vor dem zweiten Semester im postgradualen Studium reden."'

Von den 75 Studenten am Anfang verkleinerte sich der Kurs schnell auf 20 bis zum Ende des zweiten Semesters. Und von den 20, sagt Harbater, "`wussten nur 10 wirklich Bescheid, was sie da machen."' Von den 10 waren 8 zukünftige Matheprofessoren, einer sollte später Physik lehren.

"`Und der andere"', betont Harbater\index{Harbater, David|)}, "`war Richard Stallman."'

Seth Breidbart\index{Breidbart, Seth|(}, Kommilitone im Math-55-Kurs, erinnert sich, dass Stallman sich selbst damals von seinesgleichen abgehoben hat.

"`Er war auf eine sehr seltsame Art ein Pedant"', sagt Breidbart. "`Es gab eine Standardtechnik in der Mathematik, die jeder falsch macht. Es ist ein falscher Gebrauch der Notation, wo man eine Funktion für etwas definieren muss; man definiert eine Funktion und beweist dann, dass sie wohldefiniert ist. Außer beim ersten Mal, als er es vorgeführt hat, hat er eine Relation definiert und bewiesen, dass es eine Funktion ist. Es ist genau derselbe Beweis, aber er hat die korrekte Notation benutzt, die sonst keiner benutzt. Er war einfach so."'

Es war auch im Math-55-Kurs, wo Richard Stallman sich einen Ruf für Scharfsinn aufbauen konnte. Breidbart stimmt zu, aber Chess, dessen wettkämpferische Natur nicht fruchtete, sagt, die Einsicht, dass Stallman vielleicht der beste Mathematiker im Kurs war, kam bei ihm erst im nächsten Jahr. "`Es war in einem Kurs über Reelle Analysis"', so Chess, heute Mathematikprofessor am Hunter College. "`Ich erinnere mich an einen Beweis über komplexe Maße, wo Richard eine Idee aufgestellt hat, die im Grunde eine Metapher aus der Variationsrechnung war. Das war das allererste Mal, dass ich jemanden ein Problem auf eine genial originelle Weise lösen sehen habe."'

Für Chess war es ein problematischer Moment. Wie ein Vogel, der gegen eine durchsichtige Scheibe fliegt, sollte es eine Weile dauern, zu merken, dass einem einige Einsichten schlicht verwehrt bleiben.

"`Das ist Wichtigste in der Mathematik"', sagt Chess\index{Chess, Dan|)}. "`Man muss nicht selbst ein erstklassiger Mathematiker sein, um ein erstklassiges Mathematiktalent zu erkennen. Ich wusste, dass ich zu den besten gehörte, aber ich wusste auch, dass ich nicht die erste Garde war. Hätte Richard sich entschieden, Mathematiker zu werden, wäre er ein erstklassiger Mathematiker geworden."'\footnote{Stallman selbst bezweifelt das. "`Einer der Gründe, warum ich von Mathe und Physik zum Programmieren gewechselt bin, war, dass ich nie gelernt habe, wie man in den ersten zwei [Disziplinen] etwas Neues entdeckt. Ich habe nur gelernt, wie man studiert, was andere getan hatten. In der Programmierung konnte ich jeden Tag etwas Nützliches machen."'}

Für Stallman glich sich der Erfolg im Klassenzimmer mit dem fehlenden Erfolg in der Sozialwelt aus. Selbst wenn sich die anderen Mitglieder der Mathe-Mafia zusammenfanden, um die Math-55-Übungsblätter anzugehen, bevorzugte Stallman es, alleine zu arbeiten. Dasselbe galt für die Wohngewohnheiten. Auf seinem Antrag für ein Zimmer in Harvard hatte Stallman seine Wünsche deutlich gemacht. "`Ich habe gesagt, ich bevorzuge einen unsichtbaren, unhörbaren und nicht greifbaren Zimmermitbewohner."' In einem seltenen Anflug von bürokratischer Voraussicht erfüllte der Wohnheimträger die Bitte und gab Stallman für sein erstes Semester ein Einzelzimmer.

Breidbart, das einzige Math-55-Mitglied, das im ersten Jahr mit Stallman im selben Wohnheim wohnte, sagt, Stallman hätte langsam, aber sicher gelernt, mit den anderen Studenten zu interagieren. Er erinnert sich, dass die anderen Wohnheimbewohner von Stallmans Scharfsinn beeindruckt waren und anfingen, seine Meinung zu schätzen, immer wenn es eine intellektuelle Debatte in der Mensa oder in den Aufenthaltsräumen des Wohnheims ausbrach.

"`Wir haben die üblichen Diskussionen über das Lösen der Probleme auf der Welt geführt oder was die Konsequenzen von irgendwas wären"', erinnert sich Breidbart. "`Angenommen, jemand entdeckt ein Wasser für ewiges Leben, was würdest du dann tun? Was sind die politischen Konsequenzen? Wenn man es jedem gibt, dann wird die Welt übervölkert und alle sterben. Wenn man es beschränkt, wenn man es jedem gibt, der jetzt lebt, aber nicht ihren Kindern, dann hat man am Ende eine Unterschicht ohne es. Richard war einfach besser darin als die meisten, die unvorhergesehenen Eventualitäten von Entscheidungen zu erkennen."'

Stallman erinnert sich lebhaft an die Diskussion. "`Ich war immer ein Befürworter der Unsterblichkeit"', sagt er. "`Wie sollte man sonst in der Lage sein, zu sehen, wie die Welt in 200 Jahren ausschaut?"' Neugierig fragte er verschiedene Bekannte, ob sie Unsterblichkeit wollen würden, wenn man sie ihnen anböte. "`Ich war überrascht, dass die meisten Unsterblichkeit als schlecht ansahen."' Viele sagten, dass der Tod eine gute Sache ist, weil es keinen Sinn hat, ein klappriges Leben zu führen, und dass Altern gut ist, weil es die Menschen auf den Tod vorbereitet, ohne den Zirkelschluss zu bemerken.

Obwohl als erstklassiger Mathematiker und informeller Debattant anerkannt, drückte sich Stallman vor eindeutigen Wettbewerbssituationen, die seinen brillanten Ruf vielleicht hätten besiegeln können. Breidbart erinnert sich, wie Stallman sich zum Ende des ersten Harvard-Jahres auffällig um den Putnam-Test gedrückt hat, einen renommierten Test für Mathestudenten in den USA und Kanada. Zusätzlich zu der Gelegenheit, sein Wissen mit dem seiner Kommilitonen zu messen, diente der Putnam-Test als wichtiges Einstellungskriterium für die mathematischen Fakultäten. Der Campuslegende nach würde sich derjenige mit der höchsten Punktzahl automatisch für ein Stipendium an einer Hochschule seiner Wahl qualifizieren, einschließlich Harvard.

Wie Math~55 war der Putnam ein brutaler Exzellenztest. Die sechsstündige Prüfung in zwei Teilen scheint eindeutig die Spreu vom Weizen trennen zu wollen. Breidbart, Veteran des Science Honors Program und Math 55\index{Math 55|)}, beschreibt ihn als den sicherlich schwersten Test, den er je geschrieben hat. "`Um Ihnen eine Vorstellung davon zu geben, wie schwer er war"', sagt Breidbart, "`die maximal erreichbare Punktzahl war 120, und meine Punktzahl lag im ersten Jahr im 30er-Bereich. Und das Ergebnis war immer noch so gut, dass ich der 101.-beste im Land war."'

Überrascht darüber, dass Stallman, der beste Student im Kurs, den Test nicht mitgeschrieben hatte, sagt Breidbart, hätten er und ein Kommilitone ihn in der Mensa in die Enge getrieben und zur Rede gestellt. "`Er sagte, er hätte Angst gehabt, nicht so gut abzuschneiden"', erinnert sich Breidbart.

Breidbart und ein Freund schrieben ihm schnell einige Probleme auf, die sie sich gemerkt hatten, und gaben sie Stallman. "`Er konnte alle lösen"', sagte Breidbart, "`was mich zu der Schlussfolgerung führte, dass er mit \glq nicht so gut\grq{} meint, dass er entweder zweiter wird oder irgendwas falsch macht."'

Stallman erinnert sich etwas anders an den Vorfall. "`Ich erinnere mich, dass sie mir einige Fragen gebracht haben, und ich möglicherweise eine davon gelöst habe, aber ich bin mir ziemlich sicher, dass ich nicht alle gelöst habe"', sagt er. Trotzdem stimmt Stallman Breidbarts\index{Breidbart, Seth|)} Erinnerung zu, dass seine Angst der Hauptgrund für die Nichtteilnahme am Test war. Trotz seiner Bereitschaft, in den Vorlesungen auf die geistigen Schwächen seiner Kommilitonen und Professoren aufmerksam zu machen, hasste und fürchtete Stallman die Vorstellung von direktem Wettkampf\comment{head-to-head competition} – warum ihn dann nicht einfach vermeiden?

"`Aus demselben Grund konnte ich Schach nie leiden"', sagt Stallman. "`Immer wenn ich [Schach] gespielt habe, hatte ich so sehr Angst, dass ich einen einzigen Fehler mache und verliere, dass ich [dann] anfing, sehr früh im Spiel dumme Fehler zu machen. Die Angst wurde zu einer selbsterfüllenden Prophezeiung."' Er umging das Problem, indem er kein Schach mehr spielte.

Ob solche Furcht Stallman letzten Endes von einer Karriere in der Mathematik abgebracht hat, ist irrelevant. Zum Ende seines ersten Jahrs in Harvard hatte Stallman andere Interessen entwickelt, die ihn von diesem Feld abbrachten. Das Programmieren, eine latente Faszination während seiner High-School-Jahre, wurde zu einer ausgewachsenen Leidenschaft. Während andere Mathestudenten gelegentlich Zuflucht in Kunst- und Geschichtskursen suchten, suchte Stallman Zuflucht im Computerlabor.

Für Stallman hatte der erste Vorgeschmack auf echte Programmierung am IBM New York Scientific Center den Wunsch geweckt, mehr zu lernen. "`Zum Ende meines ersten Jahrs an Harvard hatte ich genug Mut gesammelt, in die Computerlabors zu gehen, um zu sehen, was sie da haben. Ich habe jemanden gefragt, ob sie Handbücher übrighaben, die ich lesen konnte."' In den Handbüchern sollte Stallman über Hardwarespezifikationen lesen und so verschiedene Rechnerarchitekturen kennenlernen.

Eines Tages, fast am Ende seines ersten Jahrs, hatte Stallman etwas über ein besonderes Labor am MIT gehört. Das Labor war im neunten Stock eines Gebäudes am Tech Square, dem hauptsächlich kommerziell genutzten Bürokomplex, den das MIT gegenüber der Straße vom Campus gebaut hatte. Den Gerüchten zufolge widmete sich das Labor der innovativen Disziplin der Künstlichen Intelligenz und war dazu mit der passenden hochmoderne Hard- und Software ausgestattet.

Seine Neugier war geweckt und Stallman entschied sich, vorbeizuschauen.
Der Weg war kurz, etwa 3 Kilometer zu Fuß, 10 Minuten per Zug, aber wie Stallman bald herausfinden sollte, konnten das MIT und Harvard wirken wie die Gegenpole eines Planeten. Mit dem labyrinthartigen Knäuel verbundener Bürogebäude bildete der Campus des MIT das ästhetische Yin zu Harvards ausgedehntem Kolonialdorf-Yang. Von den beiden war das Labyrinth des MIT viel eher Stallmans Ding. Dasselbe konnte man über die Studentenschaft sagen, eine Ansammlung von Geeks, ehemalige Highschool-Außenseiter, die eher für Streiche bekannt waren als für politisch einflussreiche Alumni.

Das Yin-Yang-Verhältnis erstreckte sich auch auf das AI~Lab. Anders als bei Harvards Computerlabors gab es keinen Postgraduierten als Pförtner, keine Warteliste für Terminalzugang am Klemmbrett, keine "`Berühren-Verboten"'-Atmosphäre. Stattdessen fand Stallman nur eine Ansammlung offener Terminals und Roboterarme vor, letztere wahrscheinlich Artefakte eines KI-Experiments. Als er einem Laborangestellten begegnete, fragte er ihn, ob sie Handbücher hätten, die sie einem interessierten Studenten ausleihen können. "`Sie hatten ein paar, aber viele Sachen waren nicht dokumentiert"', sagt Stallman. "`Schließlich waren sie ja Hacker."', fügt er trocken an, und meint den Hang von Hackern, zum nächsten Projekt zu wechseln, ohne das letzte dokumentiert zu haben.

Stallman ging wieder, mit etwas besserem als einem Handbuch, und zwar einer Anstellung. Sein erstes Projekt bestand darin, einen PDP-11-Simulator zu schreiben, der auf einer PDP-10 läuft. Er kam in der nächsten Woche zurück ins AI~Lab, ging an das nächste freie Terminal und fing an, zu programmieren.

Im Rückblick findet Stallman nichts Ungewöhnliches an der Bereitwilligkeit des AI~Labs, so ohne Weiteres einen unerprobten Außenstehenden einzustellen. "`Früher war das einfach so"', sagt er. "`Es ist heute immer noch so. Ich stelle jemanden ein, wenn ich ihn treffe und merke, dass er gut ist. Warum warten? Spießige Leute, die darauf bestehen, alles mit Bürokratie zu beladen, verstehen es einfach nicht. Wenn jemand gut ist, sollte er nicht durch ein langes, ausführliches Einstellungsverfahren gehen müssen; er sollte an einem Computer sitzen und Code schreiben."'

Wenn er "`Bürokratie"' und "`Spießigkeit"' haben wollte, musste Stallman nur die Computerlabors in Harvard besuchen. Dort wurde der Zugang zu den Terminals nach akademischem Grad vergeben. Als Bachelorstudent musste Stallman manchmal stundenlang warten.\comment{The waiting wasn't difficult, but it was frustrating.} Auf den Zugang zu einem öffentlichen Terminal zu warten, wenn man weiß, dass ein halbes Dutzend ebenfalls funktionaler Rechner in den Büroräumen der Professoren eingeschlossen war, schien wohl der Gipfel einer unsinnigen Ressourcenverschwendung. Obwohl Stallman die Computerlabors in Harvard gelegentlich aufsuchte, bevorzugte er die egalitären Grundsätze am AI~Lab. "`Es wehte [dort] ein frischer Wind"', sagt er. "`Im AI~Lab schienen sich die Leute mehr mit ihrer Arbeit zu befassen als mit ihrem Status."'

Stallman lernte bald, dass die Wer-zuerst-kommt-mahlt-zuerst-Praxis im AI~Lab sehr den Bestrebungen einiger wachsamer Leute zu verdanken war. Viele waren Veteranen aus der Zeit des Project MAC, einem vom Verteidigungsministerium finanzierten Forschungsprogramm, das das erste Timesharing-Betriebssystem hervorgebracht hatte. Einige waren schon Legenden in der Computerwelt. Da wäre Richard Greenblatt, der laborinterne Lisp-Experte und Autor von MacHack, dem Schachprogramm, das einst den AI-Kritiker Hubert Dreyfus geschlagen hat. Da wäre Gerald Sussman, Autor des Programms HACKER, das Blöcke mit einem Roboterarm bewegen konnte. Und dann wäre da noch Bill Gosper\index{Gosper, Ralph William \glq Bill\grq}, das hauseigene Mathe-Ass, der schon mitten in einer 18monatigen Hackorgie war, angestoßen von den philosophischen Fragen, die das Spiel LIFE\footnote{Auch: Conways Spiel des Lebens.} aufwarf.\footnote{\cite[Vgl.][S.\,144]{hackers}. Levy widmet etwa fünf Seiten der Beschreibung Gospers Fazination mit LIFE, einer mathelastigen Software des britischen Mathematikers John Conway. Ich kann dieses Buch als Ergänzung zu diesem oder vielleicht als Grundlagenlektüre empfehlen.}

Mitglieder dieser eng verbundenen Gruppe nannten sich "`Hacker"'. Mit der Zeit erweiterten sie die Bezeichnung "`Hacker"' auch auf Stallman. Dafür bleuten sie ihm die\comment{ ethischen} Traditionen der "`Hackerethik"' ein. In ihrer Faszination mit der Erkundung der Grenzen des Möglichen von Computern saßen Hacker auch schon mal 36~Stunden am Stück an einem Terminal, wenn sie eine Herausforderung packte. Ihre wichtigste Forderung war nach Zugang zum Computer (wenn ihn gerade niemand sonst benutzt) und zu den nützlichsten Informationen über ihn. Hacker sprachen offen darüber, die Welt über Software zu verändern, und Stallman erfuhr von der instinktiven Verachtung aller Hindernisse, die den Hacker von der Erfüllung dieses noblen Ziels abhalten. Die bedeutendsten dieser Hindernisse waren schlechte Software, akademische Bürokratie und egoistisches Verhalten.

Stallman lernte auch die Überlieferungen: Geschichten, wie Hacker, wenn sie vor einem Hindernis standen, es kreativ umgangen haben. Darunter auch verschiedene Wege, wie Hacker sich Zugang zu den Professorenbüros verschafft haben, um die konfiszierten Terminals zu "`befreien"'. Anders als ihre verwöhnten Harvard-Pendants wussten es die MIT-Dozenten besser, als die begrenzte Anzahl an Terminals am AI~Lab als Privateigentum zu betrachten. Wenn ein Dozent den Fehler machte, ein Terminal über Nacht einzuschließen, machten die Hacker es schnell wieder zugänglich – auch, um gegen den Professor zu protestieren, weil er die Gemeinschaft schlecht behandelt hat. Einige Hacker machten das, indem sie Schlösser knackten ("`Lock hacking"'), und andere, indem sie Deckenplatten herausnahmen und über die Wand kletterten. Im 9. Stock krochen einige in den Doppelboden, in dem die Verkabelung für die Computer verlegt war. "`Mir hat sogar jemand einen Wagen mit einem schweren Metallzylinder darauf gezeigt, den man dazu benutzt hatte, um die Tür eines Professorenbüros einzubrechen"',\footnote{Gerald Sussman, Dozent am MIT und Hacker, der schon länger am AI~Lab arbeitet als Stallman, bestreitet diese Geschichte. Laut Sussman haben die Hacker nie irgendwelche Türen aufgebrochen, um Terminals aus den Räumen zu herauszuholen.} sagt Stallman.

%Du-Sie?
Die Beharrlichkeit der Hacker diente dem nützlichen Zweck, die Professoren von egoistischem Verhalten abzuhalten, das den Arbeitsfortschritt im Labor behindert. Die Hacker missachteten nicht die Bedürfnisse der Einzelnen, aber sie bestanden darauf, dass man sie so auslebt, dass man allen anderen dadurch nicht im Wege steht. Zum Beispiel hatten die Professoren manchmal etwas im Büro, dass sie vor Diebstahl schützen wollten. Die Hacker sagten: "`Niemand hat etwas dagegen, wenn Sie Ihr Büro abschließen, obwohl das nicht sehr freundlich ist, solange Sie nicht das Labor-Terminal darin einschließen."'

Obwohl im AI~Lab die Akademiker den Hackern deutlich in der Überzahl waren, herrschte die Hackerethik vor. Die Hacker waren das Laborpersonal und die Studenten, die Teile der Computer entworfen und gefertigt hatten, und fast alle Software geschrieben hatten, die die Anwender nutzten. Sie hielten auch alles am Laufen. Ihre Arbeit war unentbehrlich, und sie weigerten sich, unterdrückt zu werden. Sie arbeiteten an persönlichen Lieblingsprojekten und an Funktionen, die die Anwender nachfragten, aber bei einigen handelte es sich bei ihren Lieblingsprojekten auch um die weitere Verbesserung der Hard- und Software. Wie jugendliche Autoschrauber sahen die meisten Hacker das Basteln an den Rechnern als Form der Unterhaltung.

Nirgendwo gab es etwas, das diesen Drang nach Bastelei besser widerspiegelte als das Betriebssystem, das den zentralen PDP-10-Rechner des Labors steuerte. Das Betriebssystem ITS,\index{ITS|(} kurz für "`Incompatible Time Sharing system"', hatte die Hackerethik direkt im Design verankert. Die Hacker hatten es als Protest gegen das ursprüngliche Betriebssystem des Project MAC geschrieben, dem Compatible Time Sharing System, CTSS, und es entsprechend benannt. Damals fanden die Hacker die CTSS-Architektur zu restriktiv, sie beschränkte die Fähigkeiten der Programmierer, bei Bedarf die interne Architektur zu verändern und zu verbessern\comment{modify and improve the program's own internal architecture if needed}. Laut einer Legende hatte die Entscheidung, ITS zu entwickeln, auch politische Untertöne. Anders als CTSS, welches für die IBM 7094 entwickelt wurde, war ITS speziell für die PDP-6 entwickelt. Da das System von Hackern geschrieben wurde, konnten sich die AI-Lab-Administratoren gewiss sein, dass nur Hacker sicher im Umgang mit der PDP-6 sein könnten. Der Schachzug auf dem feudalen Brett der akademischen Forschung ging auf. Obwohl die PDP-6 im Mitbesitz anderer Fakultäten stand, hatten die AI-Forscher sie schnell für sich allein. Mit ITS und der PDP-6 als Unterbau konnte das AI~Lab kurz vor Stallmans Ankunft Unabhängigkeit vom Project MAC ausrufen.\footcite{hackers}

%MMU?!
Bis 1971 war ITS auf die neuere, aber kompatible PDP-10 umgezogen und die PDP-6 blieb für besondere eigenständige Zwecke. Die PDP-10 des Labs hatte sehr viel Speicher für das Jahr 1971, er entsprach etwas mehr als einem Megabyte; Ende der 70er wurde er verdoppelt. Das Project MAC hatte zwei weitere PDP-10-Rechner gekauft; alle standen im 9.\,Stock, und auf allen lief ITS. Die hardwareinteressierten Hacker entwarfen und bauten eine bedeutende Hardwareerweiterung für diese PDP-10-Rechner, die eine virtuelle Speicherverwaltung realisiert, eine Funktion, die der Standard-PDP-10 fehlte.\footnote{Ich bitte um Verzeihung wegen dieser blitzartigen Zusammenfassung der Entstehung von ITS, einem Betriebssystem, das viele Hacker immer noch als Musterbeispiel des Hackerethos ansehen. Für weitere Informationen über die politische Bedeutsamkeit \cite[siehe][]{architects}.}

Als auszubildender Hacker war Stallman sehr schnell angetan von ITS.\comment{Obwohl es für einige Nicht-Hacker abschreckend war,} ITS wies Funktionen auf, die kommerzielle Betriebssysteme erst Jahre später anbieten konnten (oder bis heute nicht), Funktionen wie Multitasking, das Debuggen beliebiger laufender Anwendungen und Bildschirmeditierfähigkeiten. 

"`ITS hatte einen sehr eleganten internen Mechanismus, wie Programme einander untersuchen können"', sagt Stallman. "`Man konnte alle möglichen Zustände von einem anderen Programm auf eine sehr saubere, gut spezifizierte Art untersuchen."' Das war nicht nur fürs Debugging nützlich, sondern auch zum Starten, Beenden und Steuern anderer Programme.

Eine andere Lieblingsfunktion war die Möglichkeit, den Job eines anderen Programms sauber anzuhalten, zwischen Befehlen. In anderen Betriebssystemen konnten vergleichbare Befehle ein Programm in der Mitte eines Systemaufrufs unterbrechen, und der innere Zustand war undefiniert und für den Nutzer nicht ersichtlich. Bei ITS\index{ITS|)} stellte diese Funktion sicher, dass die Überwachung der Schritt-für-Schritt-Ausführung eines Programms zuverlässig und konsistent war.

"`Wenn man sagen würde, \glq Job anhalten\grq, dann würde er immer im Nutzer-Modus angehalten. Er hielt zwischen zwei Nutzer-Befehlen an, und an diesem Punkt war dann [der Zustand] des Jobs konsistent"', sagt Stallman. "`Wenn man sagen würde, \glq Job fortsetzen\grq, dann würde er immer korrekt weitergehen. Nicht nur das, außerdem konnte man den (explizit sichtbaren) Zustand eines Jobs ändern und ihn fortsetzen, und ihn dann zurückändern, und alles blieb konsistent. Es gab nirgendwo einen versteckten Zustand."'

Ab dem September 1971 war das Hacken im AI~Lab zu einem festen Bestandteil in Stallmans Terminplan geworden. Sonntags bis freitags war Stallman in Harvard. Aber sobald es Freitag Nachmittag war, machte er sich mit der U-Bahn auf den Weg zum MIT. Stallman sollte es meist so arrangieren, dass er gut vor dem rituellen Essengehen ankam. Mit fünf oder sechs anderen Hackern auf der nächtlichen Suche nach chinesischer Küche fuhr er in einem heruntergekommenen Auto mit über die Harvard Bridge ins nahe gelegene Boston. In der folgenden Stunde etwa würden er und seine Hackerkollegen dann über alles von ITS bis hin zur chinesischen Sprache und ihr Schriftsystem diskutieren. Nach dem Abendessen fuhr die Gruppe zurück ans MIT und hackte bis zum Sonnenaufgang oder ging vielleicht noch mal um 3 Uhr nach Chinatown.

% TODO: Überleitung?
Stallman würde dann den ganzen Morgen aufbleiben und hacken oder sonntagmorgens auf einer Couch schlafen. Wenn er dann aufwachte, würde er noch eine Weile hacken, wieder zum Chinesen gehen und dann zurück nach Harvard. Manchmal blieb er auch sonntags da. Die Abendessen beim Chinesen waren nicht nur köstlich, sie lieferten auch die Nährung, die er in den Harvard-Mensen nicht bekommen konnte, wo es im Durchschnitt nur eine Mahlzeit am Tag gab, die er herunterbringen konnte. (Das Frühstück zählte nicht dazu, weil er die meisten Frühstücksgerichte nicht mochte und zu der Zeit meistens schlief.)

Für den Geek-Außenseiter, der selten mit seinen Schulkameraden in der Highschool Kontakt hatte, war es eine berauschende Erfahrung, mit anderen Leuten rumzuhängen, die dieselbe Schwäche für Computer, Science Fiction und die chinesische Küche hatten. "`Ich erinnere mich an viele Sonnenaufgänge, die ich aus dem Auto heraus gesehen habe, auf dem Weg zurück von Chinatown"', sagt Stallman nostalgisch, 15 Jahre danach, in einer Rede an der Königlich Technischen Hochschule Stockholm. "`Es war wirklich eine sehr schöne Sache, einen Sonnenaufgang zu sehen, weil das so eine ruhige Tageszeit ist. Es ist eine wundervolle Tageszeit, um sich aufs Zubettgehen vorzubereiten. Es ist so schön, nach Hause zu gehen, wenn es gerade hell wird und die Vögel anfangen zu singen; man bekommt ein echtes Gefühl milder Zufriedenheit, von Gelassenheit über die Arbeit, die man die Nacht geleistet hat."'\footcite[Vgl.][]{rmskth}

Je mehr Stallman mit den Hackern rumhing, desto mehr nahm er ihre Weltanschauungen an. Schon vorher von dem Gedanken der persönlichen Freiheit überzeugt, begann Stallman, in seine Taten etwas Gemeinschaftsgeist einfließen zu lassen. Wenn andere gegen den Gemeinschaftskodex verstießen, dauerte es nicht lange, bis Stallman sich dagegen aussprach. Innerhalb eines Jahrs nach seinem ersten Besuch war Stallman derjenige, der die verschlossenen Büros aufmachte und die eingeschlossenen Terminals zurückholte, die der ganzen Laborgemeinde gehörten. In echter Hackermanier versuchte Stallman auch, seinen eigenen Beitrag zum Handwerk zu leisten. Bei einem der kunstvolleren Türöffnungstricks, der meist Greenblatt zugeschrieben wird, nahm man sich einen festen Draht und bog einige rechte Winkel hinein und befestigte ein Stück Klebeband am Ende. Man schob den Draht unter der Tür durch und drehte und bog den Draht so, dass das Klebeband von innen den Türknauf berührte. Wenn das Band kleben blieb, konnte der Hacker den Türknauf drehen, indem er am Draht zog\comment{a hacker could turn the doorknob by pulling the handle formed from the outside end of the wire}.

Als Stallman den Trick versuchte, fand er ihn schwer durchführbar. Das Klebeband zum Klebenbleiben zu bringen, war nicht immer so einfach, und das Biegen des Drahts, so dass er den Türknauf dreht, war ähnlich schwierig. Stallman dachte sich eine andere Methode aus: die Deckenplatten verschieben und über die Wand klettern. Das funktionierte immer, wenn ein Schreibtisch da war, auf den man herunterspringen konnte, aber man hatte danach meist überall Glaswolle am Körper. Gab es eine Möglichkeit, diesen Nachteil zu beseitigen? Stallman dachte sich eine alternative Herangehensweise aus. Was, wenn man, statt einen Draht unter der Tür durchzuschieben, zwei Deckenplatten beiseite schöbe und mit dem Draht über die Wand reichen könnte?

Stallman nahm einen Versuch auf sich. Statt eines Drahts bereitete er sich eine lange U-förmige Schlaufe Magnetband zurecht, an der am Bogen ein Stück Klebeband mit der haftenden Seite nach oben befestigt war. Er langte über den Türpfosten und ließ das Band solange baumeln, bis es unter dem Türknauf eine Schlinge machte. Dann zog er das Band nach oben, bis das Klebeband haften blieb, zog dann an einem Ende des Bands und drehte so den Knauf. Und natürlich öffnete sich die Tür.\comment{Stallman had added a new twist to the art of getting into a locked room.}

"`Manchmal musste man gegen die Tür treten, nachdem man den Knauf gedreht hatte"', erinnert sich Stallman an eine kleine Schwäche in der neuen Methode. "`Es war schon etwas Balance nötig, um das zu machen, während man auf einem Stuhl auf einem Schreibtisch stand."'

% AI lab spirit of direct action
Solche Umtriebe spiegelten Stallmans wachsende Bereitschaft wider, aufzubegehren und seine politischen Ansichten zu äußern. Die Art im AI Lab, sofort zu handeln, gab Stallman genug Inspiration, um aus der ängstlichen Machtlosigkeit seiner Teenagerjahre auszubrechen. Ein Büro aufzumachen und ein Terminal zu befreien war nicht dasselbe wie die Teilnahme an einem Protestmarsch, aber es war in einer Weise effektiv, die die meisten Proteste nicht waren: es löste das vorliegende Problem.

In seinen letzten Harvard-Jahren fing Stallman an, die skurrilen und respektlosen Dinge, die er im AI~Lab gelernt hatte, in der Hochschule anzuwenden.

"`Hat er Ihnen von der Schlange erzählt?"', fragt seine Mutter einmal während eines Gesprächs. "`Er und seine Wohnheimkollegen haben eine Schlange für die Studentenwahl aufgestellt. Sie hatte wohl eine beträchtliche Anzahl Stimmen bekommen."'

%zur Wahl schlängeln / Mäandat
Die Schlange war Kandidat für die Wahl im Currier House, Stallmans Wohnheim, nicht den campusweiten Studentenrat. Stallman erinnert sich, dass die Schlange recht viele Stimmen erringen konnte, teilweise dank ihres Nachnamens, der mit ihrem Besitzer übereinstimmte. "`Die Leute haben vielleicht für sie gestimmt, weil sie dachten, sie würden für ihren Besitzer stimmen"', sagt Stallman. "`Auf den Wahlplakaten stand, die Schlange \glq windet\grq{} sich zur Wahl.\footnote{"`Slither for office"' – statt "`run for office"'.} Wir sagten auch, sie sei ein Kandidat \glq auf freiem Fuß\grq, weil sie ein paar Wochen davor durch die Belüftungsanlage in eine Wand gekrochen war und niemand wusste, wo sie war."'

Stallman und seine Freunde "`nominierten"' außerdem den 3jährigen Sohn des Heimleiters. "`Sein Programm war Zwangsverrentung im Alter von 7."', erinnert sich Stallman. Diese Scherze wurden jedoch von denen am MIT-Campus in den Schatten gestellt. Einer der erfolgreichsten Scherzkandidaten war eine Katze namens Woodstock, die es geschafft hatte, ihre meisten menschlichen Gegner in der campusweiten Wahl auszustechen. "`Sie haben nie bekanntgegeben, wieviel Stimmen Woodstock bekommen hat und sie haben diese Stimmen als ungültig angesehen"', erinnert sich Stallman. "`Aber die hohe Anzahl ungültiger Stimmen in der Wahl deutete an, dass Woodstock eigentlich gewonnen hatte. Einige Jahre später wurde Woodstock verdächtigerweise von einem Auto überfahren. Niemand weiß, ob der Fahrer nicht im Auftrag der MIT-Verwaltung gehandelt hat\comment{Nobody knows if the driver was working for the MIT administration}."' Stallman sagt, er hatte nichts mit der Kandidatur von Woodstock zu tun, "`aber ich habe es bewundert."'\footnote{In einer E-Mail kurz nach der dem Beginn der endgültigen Bearbeitung des Buchs schreibt Stallman in einer E-Mail, dass er auch politische Inspiraton aus Harvard gezogen hat. "`In meinem ersten Jahr in Harvard habe ich in einem Kurs über chinesische Geschichte von dem ersten Aufstand gegen die Qin-Dynastie gelesen."'  (Das ist die, dessen Gründer die Bücher verbrannt hat und mit der Terracotta-Armee beigesetzt wurde.)  "`Die Geschichte ist nicht historisch anerkannt, aber sie hat mich sehr bewegt."'}

Im AI~Lab schlugen die politischen Aktivitäten einen schärferen Ton an. In den 70ern standen die Hacker der ständigen Gefahr von Dozenten und Administratoren gegenüber, die das ITS\index{ITS} und sein hackerfreundliches Design unterminieren wollten. Bei ITS war es jedem möglich, sich an eine Konsole zu setzen und wirklich alles zu machen, auch dem System den Befehl zum Herunterfahren in 5 Minuten zu geben. Wenn jemand ohne guten Grund so einen Shutdown-Befehl absetzte, würde ein anderer Nutzer ihn aufheben. Mitte der 70er fingen einige Dozenten an (meistens die, die sihc ihre Meinung anderswo gebildet hatten), ein Sicherheitssystem für das Dateisystem zu verlangen, damit sie den Zugriff auf ihre Dateien beschränken konnten. Andere Betriebssysteme hatten solche Funktionen, und diese Dozenten hatten sich an ein Leben "`in Sicherheit"' gewöhnt und an das Gefühl der Beschütztheit vor etwas Gefährlichem. Aber das AI~Lab blieb durch die Beharrlichkeit Stallmans und der anderen Hacker eine sicherheitsfreie Zone.

Stallman führte ethische und praktische Gründe an, warum keine Sicherheitsfunktionen hinzugefügt werden sollten. Von der ethischen Seite her appellierte Stallman an die Tradition der geistigen Offenheit und des Vertrauens in der AI-Lab-Gemeinde. Bei den praktischen Gründen führte er die interne ITS-Struktur an, die darauf angelegt war, Hacken und Zusammenarbeit zu fördern statt jeden Nutzer unter Kontrolle zu stellen. Für jeden Versuch, das umzukehren, würde eine Generalüberholung nötig sein. Um es noch schwieriger zu machen, hatte er eine Änderung eingebaut, die im letzten Feld des Dateideskriptors vermerkt, welcher Nutzer die Datei zuletzt geändert hat. Diese Funktion ließ keinen Platz mehr für Dateisicherheitsinformationen, und sie war so nützlich, dass niemand ernstlich vorschlagen konnte, sie wieder zu entfernen.

"`Die Hacker, die das Incompatible Timesharing System geschrieben haben, waren überzeugt, dass Dateiberechtigungen üblicherweise von einem selbsternannten Systemverwalter dazu verwendet werden, Macht über die anderen auszuüben"', erklärt Stallman später. "`Sie wollten nicht, dass irgendjemand auf diese Weise Macht über sie bekommt, deshalb wollten sie so eine Funktion nicht implementieren. Das Resultat war, dass man immer, wenn etwas im System kaputt war, es reparieren konnte"' (weil keine Zugriffsrechte im Wege standen).\footcite{rmskth}

%Dynamic Modeling group
Durch solche Bestrebungen hielten die Hacker die Rechner am AI~Lab sicherheitsfrei. Eine Gruppe im nahe gelegenen MIT Laboratory for Computer Sciences jedoch trugen die sicherheitsorientierten Dozenten den Sieg davon. Die DM group installierte 1977 ihr erstes Passwortsystem. Wieder nahm Stallman es auf sich, das richtigzustellen, was er als moralische Laxheit betrachtete. Er erlangte Zugriff zu dem Quellcode, das das Passwortsystem steuert und schrieb ein Programm, das die\comment{ vom System gespeicherten} Passwörter dekodiert. Dann startete er eine E-Mail-Kampagne, in der er die Nutzer bat, den Nullstring als Passwort zu verwenden. Für einen Nutzer mit dem Passwort "`Seestern"' würde die E-Mail etwa so aussehen:

\begin{quote}
Ich habe mitbekommen, dass Sie das Passwort "`Seestern"' benutzen. Ich schlage vor, sie ändern das Passwort auf "`Wagenrücklauf"', das benutze ich auch. Es ist einfacher einzutippen und es widersetzt sich der Idee von Passwörtern und Sicherheitsvorkehrungen.
\end{quote}

Die Nutzer, die "`Wagenrücklauf"' verwandten – also nur die Enter- oder Return-Taste drückten und den leeren String statt eines Passwort eingaben – ließen ihre Konten für alle zugänglich, so wie es nicht lange vorher bei allen Konten gewesen war. Das war auch der Sinn: durch die Weigerung, sich mit den glänzenden neuen Schlössern ihre Konten zu verriegeln, zogen sie die Vorstellung, Schlösser zu verwenden, ins Lächerliche. Sie wussten, dass die schwach implementierte Sicherheit auf der Maschine keine echten Angreifer außen vor lassen würde, und das interessierte sie auch nicht, weil es keinen Grund gab, sich vor Angreifern zu fürchten, weil sowieso niemand in das System einbrechen wollte, sondern nur vorbeischauen.

Stallman erwähnt in einem Interview für das 1984 erschienene Buch \citefield{shorttitle}{hackers} stolz, dass ein Fünftel des LCS-Personals seinem Vorschlag gefolgt ist und das leere Passwort genutzt hat.\footcite[Vgl.][S.\,417]{hackers}

Stallmans Nullstring-Kampagne und sein Widerstand gegen Sicherheitsvorkehrungen überhaupt wurde schließlich niedergeschlagen. Anfang der 80er prangten auch auf einigen Rechnern am AI~Lab Passwortsicherheitssysteme. Trotzdem stellte es einen Meilenstein in Stallmans persönlichem und politischem Reifevorgang dar. Aus dem Zusammenhang seiner späteren Laufbahn gesehen, war es ein wichtiger Schritt in der Entwicklung vom schüchternen Teenager, der sich fürchtete, selbst in lebenswichtigen Dingen das Wort zu erheben, zu dem erwachsenen Aktivisten, der sein Querulantentum und Beschwatzen zur Vollzeitbeschäftigung machen sollte.

Mit dem Aussprechen gegen Sicherheitssysteme stützte sich Stallman auf viele der Vorstellungen, die sein junges Leben geprägt hatten: Wissensdurst, Abneigung gegen Autorität und Frustration über Vorurteile und geheime Regeln, die andere Leute zu Außenseitern machen. Er stützte sich auch auf ethische Konzepte, die sein Leben als Erwachsener geprägt haben: Verantwortung in der Gemeinschaft, Vertrauen und das direkte Eingreifen im Sinne des Hackergeists.\comment{Expressed in software-computing terms,} Der Nullstring stellt gewissermaßen die Version 1.0 in Richard Stallmans politischer Weltanschauung dar – in Teilen unvollständig, aber zum Großteil völlig ausgereift.

Im Nachhinein zögert Stallman, dieser Begebenheit im Anfang seiner Hackerkarriere zuviel Bedeutung zuzusprechen. "`In dieser Frühphase gab es eine Menge Leute, die gleich empfanden wie ich"', sagt er. "`Die große Anzahl an Leuten, die den Nullstring als Passwort übernommen haben, war ein Zeichen, dass viele Leute mir zustimmten. Ich war nur bereit, mich als Aktivist zu betätigen."'

Stallman erkennt dem AI~Lab jedoch an, seine aktivistischen Tendenzen erweckt zu haben. Als Teenager hatte Stallman die politischen Ereignisse nur beobachtet, ohne eine Vorstellung davon, wie er etwas Bedeutendes dazu sagen oder tun könnte. Als junger Erwachsener äußerte sich Stallman über Themen, über die er sich absolut sicher war, Themen wie Softwaredesign, Verantwortung gegenüber der Gemeinschaft und persönliche Freiheit. "`Ich bin dieser Gemeinschaft beigetreten, die eine [eigene] Lebensweise hatte, die die Freiheit der anderen respektiert"', sagt er. "`Es hat mich nicht lange gebraucht, zu begreifen, dass das eine gute Sache ist. Es hat mich länger gebraucht, zum Schluss zu kommen, dass es eine moralische Frage ist."'

Das Hacken am AI~Lab war nicht die einzige Aktivität, die Stallmans Selbstvertrauen stärkte. Zu Beginn seines ersten Semesters an Harvard war Stallman einer neugegründeten internationalen Volkstanzgruppe im Currier House beigetreten. Er wollte es erst nicht probieren, und dachte, er könne nicht tanzen, aber ein Freund wies ihn auf etwas hin: "`Du weißt nicht, dass du es nicht kannst, wenn du es nicht probiert hast."' Zu seinem Erstaunen konnte er es gut und es machte ihm Spaß. Was als Experiment gestartet hatte, wurde zu einer neuen Leidenschaft neben dem Hacken und Studieren; außerdem gelegentlich ein Weg, Frauen zu treffen, obwohl nicht während seiner Studienzeit. Beim Tanzen fühlte sich Stallman nicht länger wie der unbeholfene, unkoordinierte 10jährige, dessen Versuche, Fußball zu spielen, in Frustration geendet waren. Er fühlte sich selbstsicher, agil und lebendig. In den frühen 80ern ging Stallman einen Schritt weiter und trat der MIT Folk Dance Performing Group bei. Beim Tanzen vor Publikum, gekleidet in einer nachgeahmten traditionellen Tracht eines balkanischen Kleinbauern, empfand er das Auftreten vor Publikum als Spaß und entdeckte eine Begabung für Bühnenauftritte, die ihm später bei seinen öffentlichen Reden helfen sollte.

Obwohl das Tanzen und Hacken Stallman nur wenig halfen, seinen sozialen Status zu verbessern, so halfen sie ihm doch, das Gefühl der Ausgeschlossenheit zu überwinden, das sein Leben vor Harvard verfinstert hatte. 1977, auf seiner ersten Sci-Fi-Convention, traf er die Buttonmacherin Nancy, die Buttons nach Wunsch kalligraphisch beschriftete. Begeistert gab Stallman einen Button mit dem Schriftzug "`Gott absetzen"' in Auftrag.

Für Stallman war die Botschaft "`Gott absetzen"' vielschichtig. Als Atheist von Kindesbeinen an sah Stallman sie erstens als Möglichkeit, eine "`zweite Front"' in der Religionsdebatte zu eröffnen. "`Damals, als jeder darum debattiert hat, ob ein Gott existiert"', sagt Stallman, ging \glq Gott absetzen\grq{} das Thema von einer ganz anderen Weise an. Wenn ein Gott so mächtig war, die Welt zu erschaffen, aber nichts unternahm, um die Probleme darauf richtigzustellen, warum sollte man dann so einen Gott verehren? Wäre es nicht gerechter, ihm den Prozess zu machen?"'

Außerdem war "`Gott absetzen"' ein Seitenhieb auf die Watergate-Affäre in den 70ern,\footnote{"`Impeach God"' ist eine Abwandlung des häufig auf Demoschildern und Buttons vertretenen Spruchs "`Impeach Nixon"'. "`Impeachment"' ist ein Amtsenthebungsverfahren.} mit dem er eine tyrannische Gottheit mit Nixon\index{Nixon, Richard} verglich. Watergate hatte Stallman sehr beeinflusst. Stallman war als ein autoritätsverachtendes Kind aufgewachsen. Jetzt, als Erwachsener, hatte sich sein Misstrauen durch die Kultur der Hackergemeinschaft im AI~Lab gefestigt. Für die Hacker war Watergate schlicht eine Shakespearsche Aufführung der alltäglichen Machtkämpfe, die das Leben für unprivilegierte Leute so mühsam machte. Es war eine überdimensionale Parabel dafür, was passiert, wenn man Freiheit und Offenheit gegen Sicherheit und Bequemlichkeit eintauscht. %Komfort

Von seinem wachsenden Selbstbewusstsein gestützt, trug Stallman den Button mit Stolz. Leute, die davon neugierig gemacht wurden und ihn danach fragten, bekamen ein wohlpräpariertes Lamento zu hören. "`Mein Name ist Jehovah"', würde Stallman dann sagen, "`Ich habe einen geheimen Plan, der Ungerechtigkeit und dem Leiden ein Ende zu setzen, aber aus Gründen der himmlischen Sicherheit kann ich dir nicht vom Walten meines Plans berichten. Ich sehe das große Ganze, und du nicht, und du weißt, dass ich gütig bin, weil ich es dir sage. Also lege deinen Glauben in mich und gehorche mir ohne Frage. Wenn du nicht gehorchst, heißt das, du bist böse, und ich setze dich auf meine Feindesliste und werfe dich in ein Loch, wo das Finanzamt bis in alle Ewigkeit deine Steuererklärung prüfen wird."'

Diejenigen, die die Leier als Parodie auf die Watergate-Anhörungen verstanden, hatten nur die Hälfte begriffen. Für Stallman war der andere Teil der Botschaft einer, den nur seine Hackerkollegen zu verstehen schienen. Hundert Jahre, nachdem Lord Acton davor gewarnt hatte, dass absolute Macht absolut korrumpiert, schienen die Amerikaner den ersten Teil des Spruchs vergessen zu haben: Macht selbst korrumpiert. Statt die zahlreichen Beispiele geringfügiger Korruption aufzuzeigen, genügte sich Stallman damit, seine Empörung gegenüber eines Systems auszudrücken, das der Macht überhaupt vertraut.

"`Ich dachte mir, warum bei den kleinen Fischen aufhören"', sagt Stallman zum Button und seiner Botschaft. "`Wenn wir Nixon verfolgen, warum dann nicht auch den großen Mann? So wie ich das sehe, sollte jedem, der Macht besitzt und sie missbraucht, diese Macht weggenommen werden."'
