\chapter{Die GNU General Public License}
\label{kap09}

Im Frühling 1985 hatte Richard Stallman sein erstes nützliches Resultat für das GNU Project zustande gebracht – eine Lisp-basierte Version von Emacs für unixoide Betriebssysteme. Um sie anderen als freie Software zugänglich zu machen, musste er einen Weg finden, sie zu verbreiten – einen Nachfolger der Emacs-Kommune.

Die Spannungen zwischen der Freiheit zur Modifikation und dem Vorrecht des Autors haben sich schon vor Gosmacs\index{Gosmacs|(} angefangen aufzubauen. Der Copyright Act von 1976 hatte das geltende US-Copyright reformiert und die geschützten Werke auf Software ausgeweitet. Nach Section 102(b) konnten natürliche und juristische Personen sich ein  Computerprogramm als Form der "`Äußerung"'\comment{"`expression"'} schützen lassen, aber nicht die "`eigentlichen Vorgänge oder Methoden, die im Programm eingeschlossen sind."'\footcite[Vgl.][]{swcprl}

Das heißt, dass ein Programm so ähnlich wie ein Mathe-Lehrbuch behandelt wird: sein Autor kann das Urheberrecht auf den Text beanspruchen, aber nicht auf die mathematischen Ideen oder die didaktische Technik, die zum Erklären angewandt wird. Folglich waren andere Programmierer berechtigt, ihre eigenen Implementierungen der Ideen und Befehle von Emacs zu schreiben, egal was Stallman über den Code des Original-Emacs sagte, und das taten sie auch. Gosmacs war eine von einem paar und dreißig Imitationen des Original-Emacs für verschiedene Computersysteme.

Die Emacs-Kommune bestand nur aus den Leuten, die Code von Stallmans Original-Emacs verwandten. Separat entwickelte Imitationen wie Gosmacs waren rechtlich an nichts gebunden. Aber Gosmacs unfrei zu machen, war in den Augen der Free-Software-Bewegung unethisch, weil es (als proprietäre Software) nicht die Freiheit der Nutzer achtet\comment{, aber das Thema hatte nichts damit zu tun, von wo die Ideen von Gosmacs hergekommen sind}.

Nach dem Copyright mussten Programmierer, die Code aus einem bestehenden Programm kopieren wollten (auch mit Änderungen), erst die Genehmigung des ursprünglichen Entwicklers einholen. Das neue Gesetz schützte selbst Werke ohne Copyright-Vermerk\footnote{Bis zum Eintritt der USA zum Berner Übereinkommen 1989 waren Werke ohne Copyright-Vermerk nicht automatisch urheberrechtlich geschützt.} – obwohl die Hacker das im Allgemeinen nicht wussten – und Copyright-Vermerke fingen an, aufzutauchen.

Stallman sah diese Vermerke als Flaggen einer einmarschierenden Besatzungsarmee. Nur selten gab es ein Programm, das sich keinen Sourcecode von alten Programmen entlieh, und trotzdem\comment{with a single stroke of the president's pen,} hatte die US-Regierung den Programmierern und Firmen das Recht gegeben, solche Wiederverwertungen zu verbieten. Das Copyright injizierte außerdem eine Dosis Formalität in das sonst informelle System. Streitigkeiten, die früher von Hackern Angesicht zu Angesicht beigelegt wurden, gingen nun über Anwälte. In so einem System waren die Unternehmen und nicht die Hacker automatisch im Vorteil. Manche betrachteten das Setzen seines Namens im Copyright-Vermerk als Übernahme von Verantwortung für die Codequalität, aber solche Vermerke trugen meist die Namen der Unternehmen, und es gibt andere Möglichkeiten für den Einzelnen, wie man ausdrücken kann, welchen Code man geschrieben hat.

Jedoch hatte Stallman in den Jahren vor dem GNU Project auch bemerkt, dass das Copyright es einem Autor ermöglicht, gewisse copyrightrelevante Aktivitäten zu erlauben und auch Bedingungen an sie zu knüpfen. "`Ich hatte E-Mails mit Copyright-Vermerken gesehen und dazu ein einfaches \glq wortwörtliches Kopieren erlaubt\grq\,"', erinnert er sich. "`Die waren definitiv [eine] Inspiration."' Die Lizenzen enthielten eine Bedingung, dass man die Lizenz nicht entfernen darf. Stallman wollte die Idee noch weitertreiben. Zum Beispiel könnte eine Klausel es den Nutzern erlauben, auch modifizierte Versionen weiterzuverbreiten, unter der Voraussetzung, dass die Versionen dieselbe Klausel tragen.

Und Stallman schloss daraus, dass das Gebrauchmachen von den Copyrechten nicht notwendigerweise unethisch ist. Was schlecht am Copyright auf Software war, war die Art, auf die es typischerweise eingesetzt wurde, für die es auch geschaffen wurde: den Nutzern ihre grundlegenden Freiheiten zu versagen. Die meisten Autoren kam es gar nicht in den Sinn, es anders einzusetzen. Aber das Copyright konnte anders eingesetzt werden: um ein Programm frei zu machen und seine fortwährende Freiheit zu sichern.

Zur Fertigstellung von GNU Emacs 16 Anfang 1985 hatte Stallman eine Lizenz verfasst, die den Nutzern das Recht gibt, Kopien zu machen und zu verbreiten. Sie gab den Nutzern auch das Recht, modifizierte Versionen zu machen und zu verbreiten, aber nur unter derselben Lizenz. Sie hatten keine uneingeschränkten Rechte über die modifizierten Versionen und so konnten sie ihre Versionen nicht nachträglich proprietär machen wie Gosmacs\index{Gosmacs|)}. Und sie mussten den Quellcode zur Verfügung stellen. Diese Beschränkungen schlossen die rechtliche Lücke, die es ansonsten ermöglicht hätte, dass unfreie Versionen von GNU Emacs auftauchen.

Obwohl hilfreich, den Sozialvertrag der Emacs-Kommune zu kodifizieren, sei die anfängliche GNU-Emacs-Lizenz zu "`informell"' für ihren Zweck gewesen, sagt Stallman. Kurz nach der Gründung der Free Software Foundation begann er die Arbeit an einer stichfesteren Version und beriet sich mit den anderen Vorstandsmitgliedern und mit den Anwälten, die bei ihrer Erstellung geholfen hatten.

Mark Fischer\index{Fischer, Mark}, ein bostoner Urheberrechtsanwalt, der Stallman anfangs Rechtsbeistand leistete, erinnert sich, die Lizenz mit Stallman in dieser Zeit erörtert zu haben. "`Richard hatte eine sehr feste Meinung darüber, wie sie funktionieren soll"', sagt Fischer, "`Er hatte zwei Prinzipien. Das erste war, die Software so offen wie möglich zu machen."' (Zu der Zeit, als er das gesagt hat, scheint Fischer schon von Open-Source-Befürwortern beeinflusst gewesen zu sein; Stallman hat niemals versucht, Software "`offen"' zu machen.) "`Das zweite war, andere zu ermutigen, dieselbe Lizenzpraktik zu verwenden."' \comment{Die Bedingungen in der Lizenz waren für das zweite Ziel vorgesehen.}

Die revolutionäre Natur der Lizenzbedingungen brauchte eine Weile, bis sie einem bewusst wurde. Zu der Zeit, sagt Fischer, habe er die GNU-Emacs-Lizenz einfach als simplen Tauschhandel angesehen. Sie hängte ein Preisschild an die Benutzung von GNU Emacs. Statt Geld verlangte Stallman von den Nutzern den Zugang zu ihren eigenen zukünftigen Änderungen.\comment{That said, Fischer does remember the license terms as unique.}

"`Ich denke, von Leuten diesen Preis zu verlangen, war, wenn nicht einzigartig, [doch zumindest] hochgradig unüblich zu dieser Zeit"', sagt er.

Bei der Ausarbeitung der GNU-Emacs-Lizenz hat Stallman nur eine große Änderung gegenüber den Lehren der alten Emacs-Kommune vorgenommen. Statt wie einst zu verlangen, dass die Kummunenmitglieder ihm alle Änderungen schicken, die sie geschrieben haben, verlangte Stallman jetzt nur, dass sie Quellcode und Freiheit weitergeben, wenn sie das Programm weiterverbreiten. Anders gesagt mussten Programmierer, die Emacs nur für den Privatgebrauch veränderten, ihre Quellcodeänderungen nicht mehr an Stallman zurückschicken. Mit dieser seltenen Abänderung der Free-Software-Doktrin kürzte Stallman den Preis für freie Software drastisch. Die Nutzer konnten innovieren, ohne dass Stallman ihnen dabei über die Schulter schaut, und ihre Versionen nur veröffentlichen, wenn sie es wollten, solange alle Kopien mit derselben Erlaubnis für ihre Empfänger zur Weiterentwicklung und Weitergabe ausgestattet sind.

Stallman sagt, diese Änderung rührte von seiner Unzufriedenheit mit dem Großen-Bruder-Aspekt vom ursprünglichen Sozialvertrag der Emacs-Kommune her.\comment{So sehr er es auch nützlich fand, dass ihm jeder seine Änderungen geschickt hat, he came to feel that requiring this was unjust.}

"`Es war ungerecht, von den Leuten zu verlangen, alle Änderungen zu veröffentlichen"', sagt Stallman. "`Es war ungerecht, zu verlangen, sie an einen zentralen privilegierten Entwickler zu schicken. Diese Art von Zentralisation und Privilegiertheit für einen einzelnen passte nicht zu der Gesellschaft, in der alle gleiche Rechte haben."'

Die GNU Emacs General Public License machte ihr Debüt in einer Version von GNU Emacs im Jahr 1985. Nach der Veröffentlichung bat Stallman die Hackergemeinde um Hinweise zur Verbesserung der Lizenz. Ein Hacker, der die Gelegenheit wahrnahm, war der zukünftige Softwareaktivist John Gilmore\index{Gilmore, John|(}, damals Berater für Sun Microsystems. Im Zuge seiner Beratertätigkeit hatte Gilmore Emacs auf SunOS portiert, die hausinterne Unix-Version der Firma. Dabei hatte Gilmore die veränderte Version unter die GNU-Emacs-Lizenz gestellt. Statt die Lizenz als Verpflichtung anzusehen, sah Gilmore sie als klaren und deutlichen Ausdruck des Hackerethos. "`Bis dahin waren die meisten Lizenzen sehr informell"', erinnert sich Gilmore.

Als Beispiel dieses Informellen führt Gilmore die Lizenz von trn aus der Mitte der 80er an. trn war ein Newsreader von Larry Wall\index{Wall, Larry}, der später als Autor des Unix-Programms patch und der Programmiersprache Perl zu Berühmtheit kommen sollte. Wall verfasste sein Copyright-Vermerk in der Hoffnung, die allgemeine Hacker-Höflichkeit und das Autorenrecht, die Möglichkeiten zur kommerziellen Veröffentlichung diktieren zu können, in Waage zu halten:

\begin{quote}
Copyright (c) 1985, Larry Wall\\
Sie dürfen das trn kit vollständig oder teilweise kopieren, solange Sie nicht versuchen, damit Geld zu verdienen, oder vorgeben, es selbst geschrieben zu haben.\footcite[Vgl.][(neuere Version)]{trnread}
\end{quote}

Solche Aussagen reflektieren zwar die Hackerethik, aber auch die Schwierigkeiten, die lose, informelle Natur dieser Ethik in die starre Rechtssprache zu übersetzen. Mit dem Schreiben der GNU-Emacs-Lizenz hatte Stallman mehr getan, als nur das Schlupfloch zu schließen, das proprietäre Ableger ermöglicht. Er hatte die Hackerethik in einer Art und Weise ausgedrückt, die Rechtsanwälte wie Hacker gleichermaßen verstehen können.

Nicht viel später, sagt Gilmore, hätten Hacker angefangen, die Möglichkeiten zu diskutieren, wie man die GNU-Emacs-Lizenz auf ihre eigenen Programme "`portiert"'. Ausgehend von einem Gespräch im Usenet schickte Gilmore im November 1986 eine E-Mail an Stallman, in der er eine Änderung vorschlug:

\begin{quote}
Sie sollten vielleicht das "`EMACS"' von der Lizenz entfernen und es mit "`SOFTWARE"' oder so ersetzen. Bald, hoffen wir, wird Emacs nicht mehr der größte Teil des GNU-Systems sein und die Lizenz wird auf alles davon angewandt.\footnote{Aus einer E-Mail von John Gilmore an den Autor.}
\end{quote}

Gilmore war nicht der einzige, der einen umfassenderen Ansatz vorschlug. Zum Ende 1986 arbeitete Stallman selbst an dem nächsten Meilenstein des GNU Projects, dem Debugger GDB. Für seine Veröffentlichung musste er die GNU-Emacs-Lizenz so anpassen, dass sie auf GDB\index{GDB} anwendbar ist\comment{instead of GNU Emacs}. Es war keine schwierige Aufgabe, aber es war eine Quelle möglicher Fehler. 1989 fand Stallman heraus, wie man den spezifischen Bezug auf Emacs entfernen und die Verbindung zwischen dem Programmcode und der Lizenz nur über die Quellcodedateien eines Programms ausdrücken konnte. Auf diese Art konnte jeder Entwickler die Lizenz auf sein Programm anwenden, ohne die Lizenz selbst zu ändern. Die GNU General Public License, kurz GNU GPL, war geboren. Das GNU Project benutzte sie bald darauf als ihre offizielle Lizenz für alle bestehenden GNU-Programme.

Bei der Veröffentlichung der GPL folgte Stallman der Konvention aus der Softwarewelt, in der Version mit einer Ganzzahl die Hauptversion anzugeben und mit dem Dezimalanteil die Unterversion. Die erste Version von 1989 mit der Nummer 1.0 legte in der Präambel ihre politischen Absichten dar:

\begin{quote}
Die General Public License ist so entworfen, dass sichergestellt ist, dass man die Freiheit hat, Kopien von freier Software zu verschenken oder zu verkaufen, dass man den Quellcode empfängt oder ihn bekommen kann, wenn man will, dass man die Software ändern kann oder Teile davon in einem neuen Programm verwenden; und dass man weiß, dass man diese Dinge tun kann.

Um diese Rechte zu schützen, müssen wir Einschränkungen machen, die es jedem verbieten, einem diese Rechte zu nehmen oder zu verlangen, auf die Rechte zu verzichten. Diese Einschränkungen drücken sich in gewissen Verantwortungen für denjenigen aus, der Kopien der Software verbreitet oder die Software modifiziert.\footcite[Vgl.][]{gplv1}
\end{quote}

Die GPL gilt als einer von Stallmans besten Hacks. Sie hat ein System des Gemeineigentums in den normalerweise proprietären Grenzen des Copyrights geschaffen. Und, was noch wichtiger ist, sie demonstrierte die Ähnlichkeit zwischen Gesetzestext und Software. In der Präambel der GPL steckt eine profunde Botschaft: statt das Copyright mit Misstrauen zu sehen, sollte man als Hacker es als gefährliches System betrachten, das gehackt werden kann.

"`Die GPL hat sich sehr stark wie jede andere freie Software entwickelt, hinter der eine große Gemeinschaft steht, die ihre Struktur diskutiert, ihre Achtung oder das Gegenteil ihrer Meinung nach [ausdrückt], den Bedarf nach Anpassung und sogar kleine Kompromisse einzugehen für eine bessere Aufnahme [der Lizenz]"', sagt Jerry Cohen, ein anderer Anwalt, der Stallman nach dem Abgang Fischers beraten hat. "`Der Vorgang hat sehr gut funktioniert und die GPL hat es in ihren verschiedenen Versionen von allgemein skeptischen und teilweise feindlichen Reaktionen zu allgemeiner Anerkennung gebracht."'

In einem Interview von 1986 mit dem Magazin \textit{BYTE} fasst Stallman die GPL schillernd zusammen. Zusätzlich zur Proklamierung der Hackerwerte, sagt Stallman, sollten die Leser sie als "`Form des intellektuellen Jui-Jitsus ansehen, die das Rechtssystem ausnutzt, das die Software-Horter gegen sie aufgebaut haben."'\footnote{\cite[Vgl.][]{byte}

Das Interview bietet einen interessanten, außerdem freizügigen Blick auf Stallmans politische Ansichten in den Anfangstagen des GNU Projects. Es ist auch hilfreich bei der Analyse der Entwicklung von Stallmans Rhetorik.

Bei der Beschreibung des Zwecks der GPL sagt Stallman "`Ich versuche, die Weise zu ändern, in der Leute an Wissen und Informationen im Allgemeinen herangehen. Ich glaube, dass der Versuch, Wissen zu besitzen oder zu kontrollieren, ob Leute es benutzen dürfen, oder Leute davon abzuhalten, es weiterzugeben, Sabotage ist."'

Im Gegensatz dazu eine Aussage des Autors vom August 2000: "`Ich bitte Sie, nicht den Begriff \glq geistiges Eigentum\grq{} in Ihren Überlegungen zu benutzen. Er verleitet dazu, Dinge misszuverstehen, weil der Begriff Urheberrecht, Patente und Markenrecht zusammenwirft. Und diese Dinge sind so verschieden in ihren Auswirkungen, dass es völlig albern ist, über alles auf einmal zu reden zu versuchen. Wenn Sie jemanden etwas über "`geistige Eigentum"' reden hören, ohne [es in] Anführungszeichen [zu setzen], dann denkt derjenige nicht sehr klar und und Sie sollten nicht mitreden."'

[RMS: Den Gegensatz, den es zeigt, ist, dass ich gelernt habe, vorsichtiger bei Verallgemeinerungen zu sein. Ich würde heute wahrscheinlich nicht über das "`Besitzen von Wissen"' reden, weil das ein sehr breites Konzept ist. Aber "`Besitz von Wissen"' ist nicht dieselbe Generalisierung wie "`geistiges Eigentum"' und der Unterschied zwischen den drei Gesetzen ist essenziell, wenn man egal welches rechtliches Thema zum Besitz von Wissen verstehen will. Patente sind direkte Monopole auf die Anwendung eines besonderen Wissens; das ist wirklich eine Form von "`Wissensbesitz"'. Copyrechte sind eine Methode zur Vereitelung der Verbreitung von Werken, die Wissen enthalten oder erklären, was etwas völlig anderes ist. Und Marken hingegen haben nur sehr wenig mit dem Thema Wissen zu tun.]} Jahre später würde Stallman die Entstehung der GPL in weniger feindseligen Worten beschreiben. "`Ich dachte an Probleme, die in gewisser Weise ethisch waren und in gewisser Weise politisch und in gewisser Weise juristisch"', sagt er. "`Ich musste versuchen, mit dem auszukommen, was das Rechtssystem zulässt, in dem wir uns befinden. Gedanklich war es die Aufgabe, die Grundlage für eine neue Gesellschaft gesetzlich zu regeln, aber weil ich keine Regierung bin, konnte ich ja keine Gesetze ändern. Ich musste versuchen, auf dem existierenden Rechtssystem aufzubauen, was nicht für so etwas ausgelegt war."'

Zu der Zeit, als Stallman über die ethischen, politischen und rechtlichen Fragen im Zusammenhang mit freier Software nachdachte, schickte ihm ein kalifornischer Hacker namens Don Hopkins\index{Hopkins, Don} ein Handbuch für den 68000er Mikroprozessor. Hopkins, Unix-Hacker und ebenfalls Sci-Fi-Fan, hatte sich das Handbuch eine Weile zuvor von Stallman geborgt. Als Zeichen seiner Dankbarkeit verzierte Hopkins den Rückumschlag mit einigen Aufklebern, die er auf einer lokalen Sci-Fi-Veranstaltung erhalten hatte. Ein Aufkleber fiel Stallman besonders ins Auge. Darauf stand "`Copyleft (L), All Rights Reversed"'.\footnote{"`alle Rechte umgedreht"', statt "`alle Rechte vorbehalten"'} Von diesem Aufkleber inspiriert, gab Stallman der rechtlichen Technik in der GNU-Emacs-Lizenz (und später der GNU GPL) den Beinamen "`Copyleft"' mit dem scherzhaften Symbol aus einem gespiegelten "`C"' in einem Kreis. Mit der Zeit wurde der Spottname allgemeine Terminologie der Free Software Foundation als Bezeichnung für jede Urheberrechtslizenz, die "`ein Programm zu freier Software macht und [rechtlich] erfordert, dass alle veränderten und erweiterten Versionen des Programms ebenfalls freie Software werden."'

Der deutsche Soziologe Max Weber hat einmal die These aufgestellt, dass alle großen Religionen aus der "`Routinisierung"' und "`Institutionalisierung"' von Charisma entstanden sind. Jede erfolgreiche Religion, argumentiert Weber, wandelt das Charisma oder die Botschaft des ursprünglichen religiösen Führers in einen sozialen, politischen und ethischen Apparat, der sich einfacher übertragen lässt auf Kulturen und über die Zeit.

Obwohl sie nicht per se religiös ist, qualifiziert sich die GNU GPL doch zweifellos als interessantes Beispiel des "`Routinisierungs"'-Vorgangs in der modernen dezentralisierten Welt der Softwareentwicklung. Seit ihrer Vorstellung vor der Öffentlichkeit haben Programmierer und Firmen, die sonst wenig Loyalität zu Stallman ausgedrückt hatten, den GPL-Handel bereitwillig akzeptiert\comment{have willingly accepted the GPL bargain at face value}. Tausende haben die GPL auch als präemptiven Schutzmechanismus für ihre eigene Software akzeptiert. Selbst jene, die die Bedingungen der GPL als zu restriktiv ablehnen, erkennen sie dennoch als einflussreich an.

Ein Hacker, auf den letzteres zutrifft, ist Keith Bostic\index{Bostic, Keith|(}, zur Zeit der Veröffentlichung der GPL 1.0 Angestellter an der University of California. Bostics Forschungsgruppe, die Computer Systems Research Group (CSRG), hatte seit Ende der 70er mit Unix-Entwicklung zu tun und war verantwortlich für viele Hauptbestandteile von Unix, einschließlich der TCP/IP-Netzwerkunterstützung, dem Grundpfeiler der Kommunikation im modernen Internet. In den späten 80ern begann AT\&T, der Rechtsinhaber an der Unix-Software, mit der Kommerzialisierung von Unix und sah die Berkeley Software Distribution, BSD, die akademische Unix-Version von Bostic und seinen Berkeley-Kollegen, als Schlüsselquelle für kommerziell verwertbare Technologie.

Der Code von Bostic und seinen Freunden war für fast alle unzugänglich, weil er mit proprietärem Code von AT\&T vermischt war. Die Berkeley-Distributionen waren deswegen nur für Institutionen erhältlich, die schon eine Lizenz von AT\&T für die Unix-Quellen hatten. Als AT\&T die Lizenzgebühren anhob, wurde, was erst harmlos erschien (für die, die nur an die akademische Welt dachten), selbst dort immer belastender. Um Berkeleys Code in GNU verwenden zu können, musste Stallman Berkeley überzeugen, seinen Code von dem von AT\&T zu trennen und ihn als freie Software zu veröffentlichen. 1984 und 1985 traf er sich mit den Leitern des BSD-Projekts und machte darauf aufmerksam, dass AT\&T keine Wohltätigkeitsorganisation ist und dass es einer Universität nicht ansteht, ihre Arbeit (letzten Endes) an AT\&T zu verschenken. \comment{Er bat sie, ihren Code zu isolieren und ihn als freie Software zu veröffentlichen.}

\comment{argue - streiten/diskutieren?}
Der 1986 eingestellte Bostic hatte das Projekt persönlich auf sich genommen, die neueste BSD-Version auf die PDP-11 zu portieren. Während dieser Zeit, so Bostic, sei er mit Stallman in engen Kontakt gekommen, als er einen seiner gelegentlichen Überfalle auf die Westküste machte. "`Ich erinnere mich lebhaft daran, mit Stallan über Copyright diskutiert zu haben, während er an geliehenen Workstations der CSRG saß"', sagt Bostic. "`Wir gingen danach Abendessen und diskutierten weiter über das Copyright beim Abendessen."'

Die Argumente sollten schließlich den Ausschlag geben, obwohl nicht auf die Art, wie Stallman gehofft hatte. Im Juni 1989 hatte Berkeley seinen Netzwerk-Code von dem Rest des von AT\&T besessenen Betriebssystems getrennt und fing an, ihn unter einer freien Lizenz zu verbreiten. Die Lizenzbedingungen waren freizügig. Alles, was der Lizenznehmer tun musste, war, die Universität zu nennen, wenn er Werbung für ein abgeleitetes Programm macht.\footnote{Die "`unsägliche Werbeklausel"' der University of California sollte sich später als Problem herausstellen. Bei der Suche nach einer laxeren Alternative zur GPL stießen einige Hacker auf die frühe BSD-Lizenz und ersetzten "`University of California"' mit ihren eigenen Namen oder den Namen ihrer Organisationen. Das Resultat war, dass freie Softwaresysteme, die viele dieser Programme nutzten, dutzende Namen in ihrer Werbung nennen mussten. 1999, nach einigen Jahren Zureden von Stallmans Seite, stimmte die University of California zu, die Klausel zu streichen. \cite[Vgl.][]{bsdprob}.} Im Gegensatz zur GPL erlaubte diese Lizenz proprietäre Ableger. Ein Problem des BSD Networkings war, dass es kein komplettes Betriebssystem war, sondern nur die netzwerkbezogenen Teile eines Betriebssystems. Obwohl der Code ein großer Beitrag für jedes freie Betriebssystem sein würde, konnte er zu der Zeit nur in Verbindung mit anderem, proprietärem Code zum Laufen gebracht werden.

In den nächsten paar Jahren arbeiteten Bostic und andere Angestellte der University of California an den fehlenden Komponenten und der Entwicklung von BSD zu einem vollständigen, frei weiterverbreitbaren Betriebssystem. Obwohl die Anstrengungen durch juristische Angriffe von den Unix Systems Laboratories – der AT\&T-Ausgründung, die die Rechte am Unix-Code erbte – verzögert wurden, sollten sie Anfang der 90er ihre Früchte tragen. Selbst davor sollten viele der BSD-Netzwerkprogramme ihren Weg in Stallmans GNU-System gefunden haben.

"`Ich glaube, es ist sehr unwahrscheinlich, dass wir jemals so stark geworden wären ohne den GNU-Einfluss"', sagt Bostic dazu. "`Es war eindeutig eine Sache, bei der sie Druck gemacht haben und wir die Idee gut fanden."'

Zum Ende der 80er begann die GPL eine Anziehungskraft auf die Free-Software-Gemeinde auszuüben. Ein Programm musste nicht unter der GPL stehen, um als freie Software zu gelten – man bedenke die BSD-Netzwerktools – aber ein Programm unter die GPL zu stellen, vermittelte eine klare Botschaft. "`Ich glaube, die bloße Existenz der GPL hat Leute inspiriert, darüber nachzudenken, ob sie freie Software machen und wie sie sie lizenzieren wollen"', sagt Bruce Perens\index{Perens, Bruce}, Programmierer von Electric Fence, einem beliebten Unix-Werkzeug, und späterer Leiter des Entwicklungsteams von Debian GNU/Linux. Einige Jahre nach der Veröffentlichung der GPL hat sich Perens entschieden, die selbstgestrickte Lizenz von Electric Fence zu verwerfen und Stallmans rechtlich überprüfte Lizenz zu verwenden. "`Es war wirklich ziemlich einfach zu machen"', erinnert sich Perens.

Rich Morin\index{Morin, Richard \glq Rich\grq|(}, ein Programmierer, der Stallmans GNU-Ankündigung mit Skepsis gesehen hatte, erinnert sich, von der Software beeindruckt gewesen zu sein, die sich unter dem Dach der GPL zu entwickeln begann. Als Leiter einer Sun\-OS User Group in den 80ern war es eine der Hauptaufgaben für Morin, als Verteiler Bänder mit den besten Freeware- oder Free-Software-Programmen zu verschicken. Dazu musste er oft die Autoren der Programme anrufen, um zu prüfen, ob die Programme urheberrechtlich geschützt waren oder in der Public Domain lagen. Um 1989, sagt Morin, bemerkte er, dass die besten Programme typischerweise unter die GPL fielen. "`Als Softwareverteiler wusste ich, sobald ich das Wort \glq GPL\grq{} gesehen habe, dass ich aus dem Schneider bin"', erinnert sich Morin.

Um den Aufwand zu kompensieren, den er beim Zusammenstellen der Bänder für die Verteilung an die Sun User Group hatte, stellte Morin den Empfängern eine angemessene Gebühr\comment{convenience fee} in Rechnung. Nun, da Programmierer\comment{O-Ton: Programme} zur GPL wechselten, konnte Morin plötzlich seine Bänder in der halben Zeit zusammenstellen, und machte einen ordentlichen Profit damit. Morin\index{Morin, Richard \glq Rich\grq|)} sah eine gute Geschäftsmöglichkeit und modelte sein Hobby zum Geschäft um: Prime Time Freeware.

Eine solche kommerzielle Nutzung lag komplett im Rahmen der Free-Software-Agenda. "`Wenn wir von freier Software sprechen, beziehen wir uns auf Freiheit, nicht auf den Preis"', informiert Stallman in der Präambel der GPL. Bis zum Ende der 1980er hatte Stallman das zu einem einfacheren Merkspruch aufbereitet: "`Man sollte nicht an \glq frei\grq{} wie in \glq Freibier\grq{} denken; sondern an \glq frei\grq{} wie in \glq freie Rede\grq.\,"'
%Doppeldeutigkeit

Größtenteils wurden Stallmans Appelle von den Unternehmen ignoriert. Denn für die meisten Unternehmer war die Freiheit, die mit freier Software in Verbindung gebracht wird, dieselbe Freiheit wie auf den freien Märkten. Wenn man das Softwareeigentum aus der kaufmännischen Gleichung nimmt, dann hat man eine Situation, in der selbst das kleinste Softwarehaus mit den IBMs und DECs der Welt konkurrieren konnte.

Einer der ersten Unternehmer, der das Konzept verstand, war Michael Tiemann\index{Tiemann, Michael|(}, Programmierer und Postgraduierter an der Stanford University. In den 80ern verfolgte Tiemann das GNU Projekt wie ein aufstrebender Jazzmusiker seinen Lieblingsinterpreten. Jedoch wurde ihm erst mit der Freigabe des GNU C~Compilers, GCC\index{GCC|(}, im Jahr 1987 klar, wie groß das Potential von freier Software ist. Er nennt GCC einen "`Bombenschlag"' und sieht die Existenz des Compilers selbst als Ausdruck von Stallmans Entschlossenheit als Programmierer.

"`Genauso, wie jeder Autor davon träumt, eine Great American Novel zu schreiben, redete jeder Programmierer in den 80ern davon, einen Great American Compiler zu schreiben"', erinnert sich Tiemman. "`Plötzlich hatte Stallman das gemacht. Das war ziemlich beschämend."'

"`Wenn man über den einzigen Knackpunkt sprechen will, dann war es GCC"', stimmt Bostic\index{Bostic, Keith|)} zu. "`Niemand hatte damals einen Compiler, bis es den GCC gab."'

Statt mit Stallman zu konkurrieren, entschloss er sich, auf seiner Arbeit aufzubauen. Die Originalversion von GCC kam auf 110.000 Zeilen Code, aber Tiemann erinnert sich, dass sie überraschend leicht zu verstehen war. Sogar so einfach, dass Tiemann sagt, er hätte nur weniger als fünf Tage gebraucht, um ihn zu beherrschen und noch eine Woche, um die Software auf eine neue Hardwareplattform zu portieren, nämlich auf National Semiconductors 32032. Im nächsten Jahr fing Tiemann an, mit dem Quellcode herumzuprobieren, und schrieb den ersten "`nativen"' oder direkten Compiler für die Programmiersprache C++ durch Erweiterung des GCCs. (Die bestehenden proprietären Implementierungen arbeiteten so, dass sie den C++-Code nach C konvertieren und die Ausgabe dann in einen C-Compiler schicken.) Eines Tags, als er eine Vorlesung über das Programm an den Bell Labs hielt, stieß Tiemann auf einige AT\&T-Entwickler, die vor derselben Aufgabe standen und sich mit dem Problem weniger erfolgreich abmühten.

"`Es waren etwa 40 oder 50 Leute im Raum, und ich habe gefragt, wieviele von ihnen an einem nativen Compiler arbeiten"', erinnert sich Tiemann. "`Der Veranstalter hat gesagt, die Information wäre geheim, aber wenn ich mich mal im Raum umsehe, würde ich eine ganz guten Vorstellung davon bekommen."'

Nicht lange danach, sagt Tiemann, sei ihm ein Licht aufgegangen. "`Ich hatte sechs Monate an dem Projekt gearbeitet"', sagt Tiemann. Ich habe mir gedacht, ob es nun an mir liegt oder dem Code; dieser Effizienzgrad ist etwas, was der freie Markt bereitwillig belohnen müsste."'

Tiemann fand weitere Inspiration im \textit{GNU Manifesto}: obwohl es die Gier der proprietären Softwareunternehmen scharf anprangerte, ermutigte es auch Firmen, freie Software zu benutzen und weiterzuverbreiten, solange sie die Freiheit der Nutzer bei ihren kommerziellen Tätigkeiten respektieren. Mit der Entfernung der Monopolmacht\comment{aus der kommerziellen Softwaregleichung} ermöglicht es die GPL selbst kleinen Unternehmen, auf Dienstleistungsbasis zu konkurrieren, von der einfachen Supporttätigkeit hin bis zur Erweiterung freier Programme nach Kundenwunsch.

In einem Essay von 1999 erzählt Tiemann von dem Einfluss von Stallmans \textit{Manifesto}. "`Es liest sich wie eine sozialistische Polemik, aber ich habe etwas anderes darin gesehen. Ich sah ein verstecktes Geschäftskonzept."'\footcitet[Vgl.][S.\,139:  Michael Tiemann, \textit{Future of Cygnus Solutions: An Entrepreneur's Account}]{opensrc}

%TODO: ausbaufähig
Das Geschäftskonzept war nicht neu; Stallman hatte sich im kleinen Rahmen Ende der 80er selbst damit über Wasser gehalten. Aber Tiemann\index{Tiemann, Michael|)} wollte noch einen Schritt weiter gehen.
Er schloss sich mit John Gilmore\index{Gilmore, John|)} und David Vinayak Wallace zusammen und gründete eine Beratungsfirma für die Anpassung von GNU-Programmen. Die Cygnus Support genannte Firma (inoffiziell war "`Cygnus"' ein rekursives Akronym für "`Cygnus, Your GNU Support"') schloss ihren ersten Entwicklungsvertrag im Februar 1990 ab. Bis zum Ende des Jahres hatte die Firma Support- und Entwicklungsverträge im Wert von 725.000\$. 
%1999 wurde Cygnus von Red Hat, Inc., aufgekauft.

Das komplette GNU-System, dass sich Stallman vorgestellt hat, brauchte mehr als nur Softwareentwicklungswerkzeuge. In den 90ern entwickelte GNU einen Befehlszeileninterpreter, eine "`Shell"', die ein erweiterter Ersatz für die Bourne Shell war (geschrieben vom FSF-Angestellten Brian Fox\index{Fox, Brian} und auf den Namen "`bash"' getauft, Bourne Again Shell), des weiteren den PostScript-Interpreter Ghostscript, das Dokumentationsanzeigesystem Texinfo, die C-Bibliothek, die C-Programme benötigen, um zu laufen und mit dem Systemkernel zu kommunizieren, die Tabellenkalkulation Oleo ("`besser für dich als die teurere Tabellenkalkulation"') und sogar ein ziemlich gutes Schachspiel. Die Programmierer jedoch waren meistens hauptsächlich an den GNU-Programmierwerkzeugen interessiert.

GNU Emacs, GDB und GCC waren die "`großen drei"' entwicklerorientierten Werkzeuge, aber nicht die einzigen vom GNU Project in den 80ern entwickelten. Bis 1990 hatte das GNU Project auch noch GNU-Versionen der Buildsteuerung Make, des Parser-Generators YACC (auf den Namen Bison umgetauft) und awk (alias gawk); und viele andere mehr. Wie GCC waren GNU-Programme meist auf mehreren Plattformen lauffähig\comment{not just a single vendor's platform}. Mit der Erhöhung der Flexibilität der Programme durch Stallman und seine Mitarbeiter ging oft auch eine Erhöhung der Nützlichkeit einher.

Morin\index{Morin, Richard \glq Rich\grq} von Prime Time Freeware erinnert sich an die universalistische Herangehensweise bei GNU und weist auf ein nutzloses, aber unerlässliches Softwarepaket hin, GNU Hello, das als Beispiel für Programmierer dient, wie man ein Programm korrekt für GNU verpackt. "`Es ist das Hallo-Welt-Programm, fünf Zeilen C, so verpackt, als wäre es ein GNU-[Paket]"', sagt Morin. "`Und deswegen hat es das Texinfo-Zeug und das configure-Zeug. Es hat den ganzen restlichen Softwareentwicklungsleim, den sich das GNU Project hat einfallen lassen, damit man die Pakete reibungslos auf die anderen Umgebungen portieren kann. Das ist eine unheimlich wichtige Arbeit und betrifft nicht nur die ganze Software von [Stallman], sondern auch die ganze andere GNU-Project-Software."'

Laut Stallman war die technische Verbesserung der Unix-Komponenten dem Ersatz durch freie Software untergeordnet. "`Bei jedem Teil könnte ich eine Möglichkeit finden, ihn zu verbessern, oder auch nicht"', sagt Stallman in der \textit{BYTE}. "`Zu einem gewissen Maß habe ich den Vorteil der Neuimplementierung, was viele Systeme sehr verbessert. Zu einem gewissen Maß liegt es daran, dass ich schon lange Zeit auf dem Gebiet tätig bin und auf vielen anderen Systemen gearbeitet habe. Ich habe deswegen viele Ideen [die ich von ihnen gelernt habe], die ich zur Anwendung bringen kann."'\footcite[Vgl.][]{byte}

Als sich die GNU-Werkzeuge in den späten 80ern einen Namen machten, wurde Stallmans AI-Lab-geschliffener Ruf für Design-Akkuratesse schnell in der gesamten Welt der Softwareentwickler legendär.
Jeremy Allison\index{Allison, Jeremy}, in den späten 80ern Sun-User und Programmierer, der in den 90ern sein eigenes freies Softwareprojekt – Samba – starten sollte, erinnert sich mit einem Lachen an diesen Ruf. Ende der 80 fing Allison an, Emacs zu benutzen. Von dem gemeinschaftlichen Entwicklungsmodell inspiriert, schickte er ein Stück Code ein – und Stallman lehnte es prompt ab.
"`Es war wie die eine \textit{Onion}-Schlagzeile"', sagt Allison. "`\,\glq Gott erhört Gebete eines Kinds, Antwort: Nein.\grq\,"'

Wie das GNU Project bei der Schaffung von Programmen und Bibliotheken auf der Endbenutzerebene von Erfolg nach Erfolg eilte, schob es die Entwicklung eines Kernels immer weiter hinaus, dem zentralen "`Verkehrspolizisten"'-Programm, das den Zugriff auf den Prozessor und andere Rechnerressourcen für andere Programme steuert.

Wie bei verschiedenen anderen wichtigen Systemkomponenten versuchte sich Stallman einen Vorsprung bei der Kernelentwicklung zu holen, indem er sich nach einem bestehenden Programm umsah, das er anpassen konnte. Die Durchsicht der "`GNUsletters"' aus den späten 80ern lässt erkennen, dass der Ansatz wie bei dem Versuch, GCC auf Pastel aufzubauen, so seine Probleme hatte. In einem GNUsletter vom Januar 1987 wird die Absicht des GNU Projects bekanntgegeben, TRIX zu überarbeiten, einen am MIT entwickelten Kernel. Jedoch hat Stallman das nie wirklich versucht, weil er zu der Zeit am GCC gearbeitet hat; später kam er zum Schluss, dass TRIX zu viele Änderungen braucht, als dass man ihn als guten Ausgangspunkt verwenden könnte. Bis Februar  1988 hatte das GNU Project laut seinem Newsletter aus dem Monat seine Kernel-Pläne auf Mach geändert, einen schlanken "`Micro-Kernel"', der an der Carnegie Mellon entwickelt wurde. Mach war keine freie Software, aber seine Entwickler sagten unter vier Augen, dass sie ihn "`befreien"' würden; als das 1990 passierte, konnte die Kernelentwicklung im GNU Project wirklich beginnen.\footcite[Vgl.][]{hurdhist}

% + Mosaic
Die Verzögerungen in der Kernelentwicklung waren nur eines von vielen Problemen, die auf Stallman zu der Zeit lasteten. 1989 hatte die Lotus Development Corporation Klagen gegen ihre Konkurrenten im Softwarebereich eingereicht, Paperback Software International, Mosaic Software und Borland, weil sie Menübefehle von Lotus' bekannter Tabellenkalkulation 1-2-3 kopiert hatten. Lotus' Klage in Kombination mit dem Kampf Apple vs. Microsoft über das "`Look and Feel"' gefährdeten die Zukunft des GNU-Systems. Obwohl keine Klage direkt das GNU Project angriff, gefährdeten beide das Recht der Entwicklung von zu bestehenden Programmen kompatibler Software, was viele GNU-Programme betraf. Die Gerichtsverfahren konnten einen lähmenden Einfluss auf die gesamte Softwareentwicklungskultur haben. Entschlossen, etwas zu unternehmen, setzten Stallman und ein paar Professoren eine Anzeige in \textit{The Tech} (die MIT-Studentenzeitung), in der sie die Klagen heftig attackierten und zu einem Boykott von Lotus und Apple aufriefen. Nach der Anzeige half er bei der Bildung einer Protestgruppe gegen die Unternehmen. Sie nannte sich \textit{League for Programming Freedom} und hielt ihre Proteste vor dem Bürogebäude von Lotus ab.

\comment{
	Die Proteste waren bemerkenswert.\footnote{Laut einer Pressemitteilung der League for Programming Freedom unter  \url{http://progfree.org/Links/prep.ai.mit.edu/demo.final.release} waren die Proteste wegen der hexadezimalen Protestgesänge bemerkenswert:
	
	\begin{verse}
	1-2-3-4, toss the lawyers out the door\\
	5-6-7-8, innovate don't litigate\\
	9-A-B-C, 1-2-3 is not for me\\
	D-E-F-O, look and feel have got to go\\
	\comment{
	\begin{verse}
	1-2-3-4, werft die Richter aus der Tür\\
	5-6-7-8, Neuerung statt Rechteschlacht\\
	9-A-B-C, 1-2-3 – mir einerlei\\
	D-E-F-0, Look and Feel schießt übers Ziel\\
	}
	\end{verse}
	}
}
Die Proteste waren ein Zeichen der evolutionären Natur der Softwareindustrie. Die Anwendungen hatten die Betriebssysteme still vom Hauptschlachtfeld der Unternehmen verdrängt. Mit seinem unvollendeten Kreuzzug für ein freies Betriebssystem schien das GNU Project hoffnungslos in der Zeit zurück zu sein im Vergleich zu denen, deren Grundwerte Modernität und Erfolg waren. Der alleinige Umstand, dass Stallman es für nötig hielt, eine ganz neue Gruppe zu bilden, die die "`Look-and-Feel"'-Klagen bekämpft, führte einige Beobachter zu dem Schluss, dass die FSF antiquiert ist.

Jedoch hatte Stallman strategische Gründe für die Gründung einer eigenständigen Organisation, die Durchsetzung neuer Monopole in der Softwareindustrie zu bekämpfen: damit auch proprietäre Softwareentwickler beitreten konnten. Die Ausweitung des Copyrights auf Interfaces würde viele Entwickler proprietärer als auch freier Software betreffen.  Die proprietären Entwickler würden wohl kaum die Free Software Foundation unterstützen, aber es gab nichts an der League for Programming Freedom, was sie vertreiben könnte.  Aus diesem Grunde gab Stallman die Führung der LFP an andere ab, sobald es praktikabel war.

% TODO
% Klage SAPC (Visicalc) vs. Lotus abgewiesen, Apple vs. Microsoft gescheitert, Xerox vs. Apple gescheitert
Paperback, Mosaic und Borland wurden schließlich in erster Instanz schuldig gesprochen. Allerdings zog Borland vor ein Berufungsgericht, welches Lotus' 1-2-3-Menüs als nicht copyrechtlich schützbar erklärte. Die Abstimmung des Supreme Courts über das Urteil endete unentschieden, und das Urteil des Berufungsgerichts blieb intakt.

% genius grant
1990 vergab die John D. and Catherine T. MacArthur Foundation\index{MacArthur Foundation} eine MacArthur-Fellowship an Stallman, umgangssprachlich "`Genie-Stipendium"', das auf 240.000\$ über 5 Jahre dotiert war. Obwohl die Stiftung keine Gründe für die Gewährung der Zuschüsse nennt, sah man es als Auszeichnung für die Gründung des GNU Projects an und dafür, der Free-Software-Philosophie eine Stimme gegeben zu haben. Der Zuschuss linderte viele von Stallmans kurzfristigen Sorgen. Zum Beispiel ermöglichte er ihm, seine Beratertätigkeit einzustellen, durch die er sich in den 80ern sein Einkommen gesichert hatte, und konnte mehr Zeit für den Free-Software-Zweck aufbringen.

Der Preis ermöglichte es Stallman außerdem, sich als normaler Wähler zu registrieren. Durch ein Feuer in dem Gebäude, in dem er wohnte, hatte er seit 1985 keinen offiziellen Wohnsitz. Es hatte noch dazu die meisten seiner Bücher mit Asche verdreckt und ihre Reinigung brachte keine zufriedenstellenden Resultate. Seitdem lebte er als "`Hausbesetzer"' am 545 Technology Square und musste als "`obdachlose Person"' wählen.\footcite[Vgl.][]{macarth} "`[Der Wahlausschuss von Cambridge] wollte das nicht als meine Anschrift akzeptieren"', erinnert sich Stallman später. "`In einem Zeitungsartikel über den MacArthur-Preis stand [diese Adresse] und dann konnte ich mich [damit] registrieren."'\footcite[Vgl.][]{rmshsm}

Und, was am wichtigsten war, erweckte Stallman durch die MacArthur-Fellowship Aufmerksamkeit durch die Presse und bekam so Einladungen, Reden zu halten, was er dazu nutzte, die Kunde über GNU, freie Software und die Gefahren wie die "`Look-and-Feel"'-Prozesse und Softwarepatente zu verbreiten.

% TODO Welche polytechnische Hochschule?
Interessanterweise sollte die Vervollständigung des GNU-Systems auf einen dieser Trips zurückgehen. Im April 1991 hielt Stallman einen Gastvortrag an einer polytechnischen Hochschule in Helsinki, Finnland. Unter den Zuhörern war der 21jährige Linus Torvalds\index{Torvalds, Linus|(}, der gerade mit der Entwicklung des Linux-Kernels begonnen hatte – dem freien Kernel, der dazu bestimmt war, das größte Loch im GNU-System zu stopfen.

%Stallmanscher Nachsatz
Torvalds, damals Student an der nahe gelegenen Universität Helsinki, betrachtete Stallman mit Verwunderung\comment{with bemusement}. "`Ich habe zum ersten Mal in meinem Leben einen stereotypen langhaarigen, bärtigen Hacker-Typus gesehen"', schreibt Torvalds in seiner Autobiographie aus dem Jahre 2001, \textit{Just for Fun}. "`Von denen haben wir nicht viele in Helsinki."'\footnote{\cite[Vgl.][S.\,58--59]{tojff}.\footnotemark}
\footnotetext{Obwohl es wahrscheinlich, was Torvalds' Leben anbelangt, korrekt ist, sind im Buch einige Dinge über Stallman falsch. Zum Beispiel steht dort, dass Stallman "`alles zu Open Source machen will"', und dass er "`sich über Leute beschwert, die die GPL nicht einsetzen"'. In Wahrheit ist Stallman Befürworter von freier Software, nicht Open Source. Er bittet Autoren mit Nachdruck, in den meisten Fällen die GNU GPL zu verwenden, aber er sagt, dass alle freie Lizenzen ethisch vertretbar sind.}

Obwohl er nicht unbedingt mit der "`soziopolitischen"' Seite von Stallmans Agenda in Einklang steht, schätzt Torvalds nichtsdestotrotz einen Aspekt ihrer zugrundeliegenden Logik: kein Programmierer schreibt fehlerfreien Code. Selbst wenn die Nutzer nicht den Wunsch haben, ein Programm nach ihren spezifischen Vorstellungen anzupassen, hat doch jedes Programm Raum für Verbesserung. Durch die Weitergabe des Quellcodes stellt der Hacker die Verbesserung des Programms über seine individuellen Beweggründe wie Gier oder Selbstwertschutz.

Wie viele Programmierer seiner Generation hatte Torvalds seine ersten Erfahrungen nicht auf Mainframe-Computern wie der IBM 7094 gesammelt, sondern auf einer bunten Mischung selbstgebauter Computersysteme. Als Universitätsstudent hatte Torvalds den Schritt von der PC-Programmierung zu Unix gemacht, auf der MicroVAX der Universität. Dieser leiterartige Fortschritt hatte Torvalds eine andere Sichtweise auf die Barrieren zum Zugang zu Rechnern gegeben. Für Stallman waren die größten Barrieren Bürokratie und Privilegientum. Torvalds größte Barrieren waren die Entfernung und die harschen Winter in Helsinki. Er war gezwungen, über den Campus zu wandern, nur um sich in seinen Unix-Account einzuloggen, und suchte bald nach einem Weg, sich von seiner warmen Wohnung außerhalb des Campus aus anzumelden.

Torvalds benutzte Minix, ein schlankes unfreies System, das als Lehrbeispiel vom niederländischen Universitätsprofessor Andrew Tanenbaum\index{Tanenbaum, Andrew}\phantomsection\label{Tanenbaum} entwickelt wurde.\footnote{Es war 1991 unfrei. Heute ist Minix freie Software.} Es enthielt das unfreie Free University Compiler Kit, dazu einige Werkzeuge von der Art, die Tanenbaum 1983 verächtlich Stallman vorgeschlagen hatte, zu schreiben.\footnote{Im Buch \citefield[]{title}{tanos} beschreibt Tanenbaum Minix als "`Betriebssystem"', meint damit aber nur den Teil des Systems, der mit dem Unix-Kernel korrespondiert. Es gibt zwei geläufige Definitionen des Begriffs "`Betriebssystem"'; eine davon entspricht dem, was man in der Unix-Terminologie "`Kernel"' nennt. Aber das ist noch nicht alles. Dieser Teil von Minix besteht aus einem Microkernel plus den Servern, die darauf laufen; ein Design wie bei GNU Hurd plus Mach. Microkernel plus Server sind vergleichbar zu dem Kernel von Unix. Wenn im Buch vom "`Kernel"' die Rede ist, ist nur der Microkernel gemeint. \cite[Vgl.][]{tanos}}

Minix passte in den spärlichen Speicher von Torvalds' 386er, aber es war eher zum Studieren als zum Nutzen gedacht. Das Minix-System hatte außerdem keine Terminal-Emulation, welche ein typisches Bildschirmterminal nachbildet, wodurch Torvalds sich nicht von zu Hause auf der MicroVAX anmelden konnte.

Anfang 1991 begann Torvalds einen Terminal-Emulator zu schreiben. Er nutzte Minix bei der Entwicklung des Emulators, aber der Emulator lief nicht unter Minix; er war ein betriebssystemunabhängiges Programm. Dann implementierte er Funktionen für den Zugriff auf das Dateisystem von Minix. Ungefähr zu der Zeit bezeichnete Torvalds sein im Entstehen begriffenes Werk als das "`GNU/Emacs unter den Terminal-Emulatoren"'.\footcite[][S.\,78]{tojff}

Weil Minix viele wichtige Funktionen fehlten, fing Torvalds an, seinen Terminalemulator zu einem Minix-ähnlichen Kernel zu erweitern, nur war seiner monolithisch. In seinem Ehrgeiz bat er in einer Minix-Newsgroup um ein Exemplar des POSIX-Standards, die Spezifikation für einen Unix-kompatiblen Kernel.\footnote{POSIX wurde später um viele Spezifikationen zu Befehlszeilenfunktionen erweitert, aber die gab es 1991 noch nicht.} Einige Wochen später, als er seinen Kernel mit einigen GNU-Programmen zum Laufen gebracht hatte, postete Torvalds eine Nachricht, die an Stallmans Usenet-Posting von 1983 erinnerte:

\begin{quote}
Hallo an alle da draußen, die minix benutzen --

Ich mache ein (freies) betriebssystem (nur als hobby, nichts großes und professionelles wie gnu) für 386- (486-) AT-clones. Das läuft schon seit april, und wird bald fertig sein. Ich hätte gern feedback, was leute an minix mögen/nicht mögen, weil mein OS ziemlich ähnlich ist (unter anderem gleiches physisches dateisystemlayout (aus praktischen gründen)). Bis jetzt habe ich schon bash (1.08) und gcc\index{GCC|)} (1.40) portiert\ldots\footcite[][S.\,85]{tojff}
\end{quote}

Das Posting erhielt nur wenige Antworten und einen Monat später hatte Torvalds die Version 0.01 auf einem FTP-Server bereitgestellt.\comment{ – i.e., the earliest possible version fit for outside review – } Dazu musste sich Torvalds einen Namen für den neuen Kernel ausdenken. Auf seiner Festplatte auf dem PC hatte Torvalds das Programm als "`Linux"' abgespeichert, in der Manier, jede Unix-Variante mit einem Namen zu versehen, der auf "`x"' endet. Er hielt den Namen für zu "`egotistisch"' und änderte ihn in "`Freax"'\index{Freax}, doch einer der Administratoren des FTP-Servers entschied sich, den Namen zurückzuändern.

Torvalds sagte, er schreibe ein freies Betriebssystem, und der Vergleich mit GNU zeigt, dass er ein komplettes System meinte. Jedoch hat er ganz einfach nur einen Kernel geschrieben. Torvalds musste nicht mehr als einen Kernel schreiben, weil, wie er wusste, die anderen Komponenten schon verfügbar waren, dank der GNU-Entwickler und anderer freier Softwareprojekte. Weil das GNU Project sie alle im GNU-System einsetzen wollte, mussten sie sie notgedrungenerweise so gestalten, dass sie zusammenarbeiten. Und Torvals (und seine späteren Helfer) brachten die Programme bei seiner Weiterentwicklung auch dazu, mit dem Kernel zusammenzuarbeiten.

Anfangs war Linux keine freie Software: die Lizenz, unter der es stand, konnte nicht als frei gezählt werden, weil sie keine kommerzielle Verbreitung erlaubte. Torvalds hatte sich Sorgen gemacht, dass eine Firma zupacken und ihm Linux wegnehmen könnte. Doch als die GNU/Linux-Kombination an Popularität gewann, sah Torvalds ein, dass es für die Gemeinschaft nützlich ist, wenn Kopien verkauft werden dürfen, und war weniger besorgt wegen einer möglichen Übernahme.\footcite[][S.\,94f]{tojff} Das führte ihn dazu, sich die Lizenzierung von Linux noch einmal zu überlegen.

Weder das Compilieren mit GCC noch das Einsetzen von GCC unter Linux zwangen ihn dazu, Linux unter der GNU GPL zu veröffentlichen, aber mit Torvalds' Verwendung von GCC ging auch die unterschwellige Verpflichtung einher, andere wiederum Code von sich borgen zu lassen. Oder wie Torvalds es später ausgedrückt hat: "`Ich hatte mich auf die Schultern von Riesen gehievt."'\footcite[][S.\,95-97]{tojff} Wenig überraschend begann er darüber nachzudenken, was passieren würde, wenn andere Leute ihn um ähnliche Unterstützung ersuchen würden. Ein Jahrzehnt nach der Entscheidung wiederholt Torvalds die Worte von Robert Chassell von der FSF, wenn er seine Gedanken zu dieser Zeit rekapituliert:

\begin{quote}
Da steckt man sechs Monate seines Lebens in diese Sache und man will es verfügbar machen und man will etwas rausbekommen, aber man will nicht, dass es Leute ausnutzen. Ich wollte, dass die Leute [Linux] sehen können, und dass sie nach Herzenslust Änderungen und Verbesserungen machen. Aber ich wollte auch sichergehen, dass ich [aus der Sache] rausbekomme, dass ich sehe, was sie da gemacht haben. Ich wollte immer Zugang zu den Quellen haben, damit, wenn sie eine Verbesserung gemacht haben, ich auch diese Verbesserung machen kann.\footcite[][S.\,94f]{tojff}
\end{quote}

Als die Zeit der Veröffentlichung der Linux-Version 0.12 herankam, der ersten Version, die komplett mit GCC arbeitete, entschied sich Torvalds, sich der Free-Software-Bewegung anzuschließen. Er gab die alte Linux-Lizenz auf und ersetzte sie mit der GPL. Innerhalb von drei Jahren brachten die Linux-Entwickler Version 1.0 heraus; der Kernel funktionierte reibungslos mit dem fast vollständigen GNU-System, das aus Programmen des GNU Projects und weiteren bestand. Im Grunde haben sie das GNU-Betriebssystem vervollständigt, als sie Linux hinzugefügt haben. Das daraus resultierende System war im Grunde GNU plus Linux. Torvalds und seine Freunde bezeichneten es jedoch verwirrenderweise als "`Linux"'.

Bis 1994 hatte sich das amalgamierte System in der Hackerwelt genug Respekt verdient, dass einige Beobachter zur Frage kamen, ob Torvalds nicht Haus und Hof verspielt hätte, als er das Projekt in seinen Anfangsmonaten auf die GPL umstellte. In der ersten Ausgabe des \textit{Linux Journal} führt der Herausgeber Robert Young ein Interview mit Torvalds. Als Young den finnischen Programmierer fragte, ob er es bereue, den "`Privatbesitz"' am Linux-Quellcode aufgegeben zu haben, verneint Torvalds. "`Selbst hinterher, wenn man immer klüger ist"', betrachte er die GPL als "`eine der allerbesten Designentscheidungen"' in der Anfangsphase des Linux-Projekts.\footcite[][]{yiwto}

Dass diese Entscheidung nicht auf Bitten oder aus Achtung vor Stallman oder der Free Software Foundation getroffen wurde, spricht für die wachsende Portabilität der GPL. Obwohl es ein paar Jahre dauern sollte, bis er sie bemerkte, hatte die Explosivität der Linux-Entwicklung Ähnlichkeiten mit der von Emacs.\comment{ This time around, however, the innovation triggering the explosion wasn't a software hack like Control-R but the novelty of running a Unix-like system on the PC architecture.} Die Motivation mag zwar eine andere gewesen sein, aber das Endresultat passte auf die ethische Vorgabe, ein voll funktionales Betriebssystem zu erschaffen, das komplett aus freier Software zusammengesetzt ist.

Wie seine anfängliche Mail an die Newsgroup comp.os.minix zeigt, sollte es einige Monate dauern, bevor Torvalds Linux als mehr als nur ein Provisorium ansah für die Zeit, bis die GNU-Entwickler ihren Hurd-Kernel fertigstellen.\comment{Was Torvals betrifft, dachte er einfach, dass er der jüngste aus einer Reihe Jugendlicher war, die Dinge zum Spaß auseinandernehmen und wieder zusammensetzen.} Wenn er jedoch den großen Erfolg des Projekts zusammenfasst, das sein Dasein genauso gut auch auf einer ausgemusterten Festplatte hätte fristen können, schreibt Torvalds seinem jüngeren Selbst zugute, dass es die Weisheit hatte, die Kontrolle aufzugeben\comment{ und das GPL-Abkommen zu akzeptieren}.

"`Ich habe vielleicht nicht das Licht gesehen"', schreibt Torvalds\index{Torvalds, Linus|)} über die Rede Stallmans 1991 an der Polytechnichnischen Hochschule und seine Entscheidung, zur GPL zu wechseln. "`Aber ich glaube, etwas von seiner Rede ist hängengeblieben."'\footcite[][S.\,59]{tojff}
