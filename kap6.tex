\chapter{Die Emacs-Kommune}

%\TPGrid{10}{10}
%\begin{textblock}{3.2}(6,0.5)
%\includegraphics[width=\textwidth]{emacs.pdf}
%\end{textblock}

Das AI~Lab der 70er war in allen Belangen ein besonderer Ort. Modernste Projekte und erstklassige Forscher verliehen ihm sein Ansehen in der Welt der Informatik. Die Hackerkultur im Labor und ihre anarchischen Grundsätze verliehen ihm außerdem eine rebellische Mystik. Erst später, als die meisten Forscher und Softwaresuperstars vom Labor gegangen waren, sollte den Hackern völlig bewusst werden, welche einzigartige und flüchtige Welt sie einst bewohnt hatten.

"`Es war ein bisschen wie der Garten Eden"', fasst Stallman das Lab und seinen Ethos des Softwareaustauschs in einem \textit{Forbes}-Artikel von 1998 zusammen. "`Es war uns nicht in den Sinn gekommen, nicht zusammenzuarbeiten."'\footcite[Vgl.][]{loveofhacking}

Solche mythologischen Beschreibungen unterstreichen eine wichtige Tatsache, auch wenn sie extrem sind. Der neunte Stock am 545 Tech Square war für viele mehr als nur ein Arbeitsplatz. Für Hacker wie Stallman war er ihr Zuhause.

Das Wort "`Zuhause"' ist ein belasteter Begriff im Wortschatz Stallmans. Mit einem gezielten Schlag gegen seine Eltern weigert er sich bis heute, ein Zuhause vor dem Currier House anzuerkennen, dem Wohnheim, in dem er zu seinen Harvard-Tagen gelebt hat. Er ist auch dafür bekannt, das Verlassen seines Zuhauses in tragikomischer Weise zu beschreiben. Als er einmal seine Harvard-Jahre beschrieb, sagte Stallman, das einzige, was er bereut, ist dass er rausgeworfen wurde. Erst nachdem ich ihn nach dem Grund für seinen Rausschmiss frage, merke ich, dass ich gerade die Überleitung für eine klassische Stallman-Pointe geliefert habe.

"`An Harvard hatten sie eine Richtlinie, die besagt, dass wenn man zuviele Kurse besteht, man gebeten wird, zu gehen"', sagt Stallman.

Ohne Wohnheim und ohne Verlangen, nach New York zurückzukehren, folgte Stallman dem Pfad Greenblatts, Gospers, Sussmans und der vielen anderen Hacker vor ihm. Er schreibt sich für ein postgraduales Studium am MIT ein und mietet ein Zimmer in einer Wohnung\comment{room in an apartment} im nahe gelegenen Cambridge, aber sieht schon bald das AI~Lab selbst als sein De-facto-Zuhause an. In einer Rede von 1986 erinnert sich Stallman an seine Erlebnisse im AI~Lab während dieser Zeit:

\begin{quote}
Ich habe vielleicht etwas mehr im Lab gewohnt als die meisten Leute, weil ich jedes oder jedes zweite Jahr aus irgendeinem Grund keine Wohnung hatte und dann ein paar Monate im Lab wohnte. Und ich fand es immer sehr gemütlich, außerdem angenehm und kühl im Sommer. Es war überhaupt nicht unüblich, dass man im Lab eingeschlafene Leute gesehen hat, wieder wegen ihres Enthusiasmus; man blieb so lange wach, wie man hacken konnte, weil man einfach nicht aufhören wollte. Und dann, wenn man komplett erschöpft war, legte man sich auf die nächstgelegene horizontale Fläche. Eine sehr informelle Atmosphäre.\footcite[Vgl.][]{rmskth}
\end{quote}

Die heimelige Atmosphäre konnte manchmal zum Problem werden. Was einige als Wohnheim ansahen, sahen andere als elektronische Opiumhölle. Im 1976 veröffentlichten Buch \textit{Computer Power and Human Reason} äußert MIT-Forscher Joseph Weizenbaum eine vernichtende Kritik am "`Computerpenner"', Weizenbaums Bezeichnung für die Hacker, die die Computerräume wie das AI~Lab bevölkerten. "`Ihre zerknitterte Kleidung, ihre ungewaschenen Haare und unrasierten Gesichter und ihr ungekämmtes Haar waren Zeugnisse davon, dass sie sich ihrer Körper und der Welt um sich gar nicht bewusst waren"', sagt Weizenbaum. "`[Computerpenner] existieren, jedenfalls wenn sie so beschäftigt sind, nur durch und für ihre Computer."'\footcite[Vgl.][S.\,116 oder \url{http://www.sacbusiness.org/cs/hesterj/HACKER.htm}]{comppower}

Fast ein Vierteljahrhundert nach seiner Veröffentlichung echauffiert sich Stallman immer noch, wenn er Weizenbaums Beschreibung des "`Computerpenners"' hört, und redet darüber in der Gegenwartsform, als ob Weizenbaum selbst im Raum wäre. "`Er will, dass Leute nur professionell sind, es nur für das Geld machen und dann so schnell wie möglich davon [von der Arbeit] wegkommen und [sie] vergessen sollen"', sagt Stallman. "`Was er als normalen Stand der Dinge ansieht, sehe ich als Tragödie."'

Doch auch das Hackerleben ist nicht frei von Tragödien. Stallman charakterisiert den Übergang vom Wochenendhacker zum Vollzeit-AI-Lab-Bewohner als eine Reihe schmerzlicher Unglücksfälle, die nur durch die Euphorie des Hackens gelindert werden konnte. Wie Stallman selbst gesagt hat, war das erste Unglück seine Graduierung von Harvard. Um seine Studien der Physik fortsetzen zu können, schrieb er sich für ein postgraduales Studium am MIT ein. Die Wahl der Hochschulen war selbstverständlich. Nicht nur gab sie Stallman die Möglichkeit, in den Fußstapfen berühmter MIT-Abgänger zu folgen wie William Shockley ('36), Richard P. Feynman ('39)  und Murray Gell-Mann ('51), er war auch drei Kilometer näher am AI~Lab und seiner neuen PDP-10. "`Mein Hauptaugenmerk war die Programmierung, aber ich dachte immer noch, na ja, dass ich vielleicht beides machen kann"', so Stallman.

Mit der harten Arbeit auf dem Feld der Physik auf der Stufe eines postgradualen Studenten bei Tage und dem Programmieren in den mönchsartigen Grenzen des AI~Labs bei Nacht versuchte Stallman einen Balanceakt. Das Zünglein\comment{fulcrum of this geek teeter-totter} war sein wöchentlicher Ausflug zum Volkstanzclub, sein einziges soziales Ventil, das ihm wenigstens ein bisschen Interaktion mit dem schönen Geschlecht garantierte. Zum Ende des ersten Jahrs am MIT hin schlug jedoch ein Unheil zu. Eine Knieverletzung zwang Stallman, mit dem Tanzen aufzuhören. Zuerst sah er die Verletzung als vorübergehendes Problem an; er ging weiter tanzen und plauderte mit seinen Freunden zu der Musik, die er so liebte. Am Ende des Sommers, als das Knie immer noch schmerzte und die Kurse wieder anfingen, fing Stallman an, sich Sorgen zu machen. "`Mit meinem Knie war es nicht besser geworden"', erinnert er sich, "`das hieß, dass ich davon ausgehen konnte, dass ich dauerhaft nicht mehr tanzen kann. Ich war todunglücklich."'

Ohne Wohnheim und ohne das Tanzen implodierte Stallmans Sozialwelt. Beim Tanzen war die einzige Situation, in der er erfolgreich Frauen treffen und gelegentlich mit einer ausgehen konnte. Nie mehr zu tanzen war schon schmerzlich genug, aber es bedeutete wohl auch nie wieder Rendezvous.

"`Ich habe mich im Grunde so gefühlt, als ob ich all meine Kraft verloren hätte"', erinnert sich Stallman. "`Ich habe die Energie verloren, irgendwas zu machen, außer was unmittelbar greifbar war. Die Energie für alles andere war weg. Ich war völlig verzweifelt."'

Stallman zog sich noch mehr von der Außenwelt zurück und konzentrierte sich ganz auf seine Arbeit am AI~Lab. Im Oktober 1975 brach er sein Physikstudium am MIT ab; es sollte sein endgültiges Studienende sein. Hacken, einst ein Hobby, war seine Berufung geworden.

Wenn sich Stallman an diese Zeit erinnert, sieht er den Übergang vom Vollzeitstudenten zum Vollzeithacker als unausweichlich an. Früher oder später hätte der Sirenenruf des Hackens sein Interesse an anderen beruflichen Aktivitäten bezwungen. "`Bei Physik und Mathe wusste ich nie so richtig, wie ich etwas beitragen kann"', berichtet Stallman von seinem  Hadern vor seiner Knieverletzung. "`Ich wäre stolz gewesen, wenn ich in eines der beiden Felder vorangebracht hätte, aber ich habe nie einen Weg gesehen, wie ich das machen könnte. Ich wusste nicht, wo ich ansetzen sollte. Bei Software wusste ich sofort, wie ich Dinge schreiben konnte, die laufen und nützlich sind.\comment{The pleasure of that knowledge led me to want to do it more}"'

Stallman war nicht der erste, der Hacken mit Vergnügen gleichgesetzt hat. Viele Hacker am AI~Lab rühmen sich ähnlich unvollständigen akademischen Werdegängen. Viele strebten einen Abschluss in Mathe oder E-Technik an, nur um später ihre akademische Laufbahn und beruflichen Ambitionen zu tauschen gegen das Hochgefühl beim Lösen von Problemen, die noch niemand vorher angegangen war. Wie der heilige Thomas von Aquin, der dafür bekannt ist, so lange an seinen theologischen Abhandlungen gearbeitet zu haben, dass er manchmal göttliche Visionen hatte,\footnote{Er soll die Stimme Jesu gehört haben.} erreichten Hacker übersinnliche Zustände durch die schiere geistige Konzentration und physische Erschöpfung. Obwohl Stallman sich von Drogen fernhielt, mochte er, wie die meisten Hacker, das "`High"' nach einem 20stündigen Coding-Marathon.

Vielleicht das angenehmste Gefühl jedoch war die persönliche Erfüllung. Wenn es ums Hacken ging, war Stallman ein Naturtalent. Das häufige Lernen bis spät in die Nacht als Kind hat ihn auf lange Arbeitszeiten mit wenig Schlaf vorbereitet. Als gesellschaftlicher Außenseiter ab dem Alter von 10 hatte er wenig Schwierigkeiten damit, allein zu arbeiten. Als Mathematiker mit einem angeborenen Talent für Logik und Weitblick hatte Stallman die Fähigkeiten, Barrieren zu durchbrechen, an denen andere sich ohne Ergebnis verausgabten.

"`Er war etwas Besonderes"', erinnert sich Gerald Sussman\index{Sussman, Gerald}, Professor am AI~Lab und (seit 1985) Vorstandsmitglied der Free Software Foundation und beschreibt ihn als "`klaren Denker und klaren Designer"'. Sussman bat Stallman in den Jahren 1973 und 1975, ihn bei KI-Forschungsprojekten zu unterstützen. Beide Projekte hatten das Ziel, KI-Programme zu entwerfen, die Schaltsysteme so analysieren können wie ein menschlicher Ingenieur. Das Projekt erforderte die Lisp-Kenntnisse eines Experten, einer Programmiersprache, die speziell für KI-Anwendungen geschrieben wurde\comment{??}, und ein Verständnis davon, wie ein Mensch dieselbe Aufgabe angeht (für den Teil sorgte Sussman). Das Projekt aus dem Jahr 1975 bereitete den Weg für eine KI-Technik namens dependency-directed Backtracking oder Truth Maintenance, die darin besteht, vorläufige Annahmen zu postulieren, zu bemerken, ob sie zu Widersprüchen führen, und die relevanten Annahmen neu zu prüfen, falls das auftritt.

Wenn er nicht gerade an offiziellen Projekten wie diesen arbeitete, widmete Stallman seine Zeit seinen Lieblingsprojekten. Es lag im Eigeninteresse eines Hackers, die Softwareinfrastruktur des Labors zu verbessern, und eines von Stallmans am meisten bevorzugten Projekten zu dieser Zeit war das Editorprogramm des Labors, TECO.

Die Geschichte von Stallmans Arbeit an TECO in den 70ern ist untrennbar mit Stallmans späterer Führung der Free-Software-Bewegung verbunden. Sie ist außerdem eine signifikante Phase in der Geschichte der Computerentwicklung und verdient eine kurze Zusammenfassung. In den 50ern und 60ern, als Computer Einzug an den Universitäten hielten, war Programmieren eine unglaublich abstrakte Tätigkeit. Um mit dem Rechner zu kommunizieren, mussten Programmierer eine Reihe von Lochkarten stanzen\comment{, with each card representing an individual software command}. Die Programmierer gaben ihre Lochkarten an einen Operator, der sie\comment{one by one – einzeln???} an den Rechner verfütterte, und auf den Rechner wartete, bis er einen neuen Stapel Lochkarten ausspuckt, von denen der Programmierer dann die Ausgabe entziffern konnte. Dieser Vorgang hieß "`Stapelverarbeitung"' und war schwerfällig und zeitaufwendig.\comment{Außerdem war es anfällig für Autoritätsmissbrauch. One of the motivating factors behind hackers' inbred aversion to centralization was the power held by early system operators in dictating which jobs held top priority.}

1962 unternahmen die Informatiker und Hacker vom Project MAC am MIT, einem Vorläufer des AI~Labs, Schritte, um Abhilfe zu schaffen. Das Time-sharing,\footnote{Mehrbenutzerbetrieb} ursprünglich "`time stealing"', des Betriebssystems CTTS\index{CTTS} machte es möglich, dass mehrere Programme [verschiedener Benutzer] "`gleichzeitig"' die Betriebsmittel eines Rechners nutzen. Die Verwendung von Fernschreibern als Eingabegeräte ermöglichte es, mit echtem Text und ohne Stapel von Lochkarten mit den Rechnern zu kommunizieren. Ein Programmierer konnte nun Befehle direkt eintippen und die Ausgabe des Rechners zeilenweise lesen.

In den späten 60ern machten die Benutzerschnittstellen große Sprünge. In einer berühmten Rede von 1968 enthüllte Doug Engelbart\index{Engelbart, Douglas \glq Doug\grq}, damals Wissenschaftler am Stanford Research Institute, einen Prototyp einer modernen graphischen Benutzerschnittstelle. Mit dem Anschließen eines Fernsehers an den Computer und dem Hinzufügen eines Zeigegeräts, das Engelbart "`Maus"' nannte, hatte der Wissenschaftler ein noch interaktiveres System geschaffen als das Timesharing-System des MIT. Mit der Verwendung des Bildschirms als eine Art Hochgeschwindigkeitsdrucker gab Engelbarts System dem Nutzer die Möglichkeit, den Cursor in Echtzeit auf dem Schirm hin und her zu bewegen\comment{gekürzt}. Der Nutzer konnte nun Text beliebig auf dem Bildschirm plazieren.

Solche Innovationen sollten erst in zwei Jahrzehnten ihren Weg auf den kommerziellen Markt finden. Trotzdem wurden in den 70ern die Fernschreiber schon allmählich durch Bildschirme als Terminal ersetzt, und so war die Möglichkeit für bildschirmfüllende – statt zeilenweiser – Textbearbeitung geschaffen.

Eines der ersten Programme, die diesen Vorteil nutzten, war TECO aus MITs AI Lab. Der \textit{Text Editor and COrrector} war aus einem alten Zeileneditor für die PDP-6 entstanden, der von den Hackern angepasst wurde.\footnote{Der Name TECO stand ursprünglich für Tape Editor and Corrector. \cite[Vgl.][Glossary: "`TECO"']{jargonf}.}

TECO war eine wesentliche Verbesserung den alten Editoren gegenüber, hatte aber immer noch einige Nachteile. Um ein Textdokument zu erstellen und zu editieren, musste man eine Reihe an Befehlen für jeden Editierschritt angeben. Es war ein abstrakter Vorgang. Anders als die modernen Textverarbeitungsprogramme, die den Text mit jedem Tastendruck aktualisieren, erforderte TECO, dass der Nutzer eine lange Folge von Editierbefehlen und eine "`Ende-des-Befehlsstrings"'-Sequenz eingibt, nur um Text zu ändern. Mit der Zeit war ein Hacker geübt genug darin, dass er lange Änderungen elegant in einem Befehlsstring machen konnte, aber wie Stallman selbst später betont, war für den Vorgang "`ein geistiges Können wie beim Blindschach"' vonnöten.\footcite[Vgl.][ich zitiere aus der aktualisierten HTML-Version]{emacspaper}

Um den Vorgang zu erleichtern, hatten die Hacker am AI~Lab ein System erstellt, das den Text und den Befehlsstring gleichzeitig in verschiedenen Bildschirmbereichen anzeigt\comment{on a split screen}. Trotz des innovativen Hacks bedurfte Editieren mit TECO immer noch Geschick und Planung.

% Definition WYSIWYG?
TECO war nicht der einzige Bildschirmeditor in der Computerwelt zu dieser Zeit. Während eines Besuchs am Stanford Artificial Intelligence Lab 1976 stieß Stallman auf ein Editierprogramm namens E\index{E (Editor)}. Das Programm hatte die Funktion, dass der Text auf dem Bildschirm nach jedem Tastendruck aktualisiert wurde\comment{???allowed a user to update display text after each command keystroke}. In den 70ern war E einer der ersten rudimentären WYSIWYG-Editoren.\index{WYSIWYG} WYSIWYG steht kurz für "`what you see is what you get"' und bedeutet, dass der Nutzer die Datei bearbeiten kann, wie er sich durch den angezeigten Text bewegt, statt mit einem Backend-Editierprogramm arbeiten zu müssen.\comment{as opposed to working through a back-end editor program.}"'\footcite[Vgl.][]{rmsetfse}

Von dem Programm beeindruckt und wieder am MIT angekommen, suchte Stallman nach einem Weg, TECOs Funktionalität auf ähnliche Weise zu erweitern. Er fand eine TECO-Funktion namens Control-R, geschrieben von Carl Mikkelson, die nach der Tastenkombination benannt war, durch die sie ausgelöst wurde. Mikkelsons Hack schaltete TECO von seinem gewöhnlichen Befehlsmodus in einen intuitiveren tastenorientierten Modus. Die einzigen Unzulänglichkeiten waren, dass sie nur fünf Zeilen auf dem Bildschirm darstellte und zu ineffizient für den Produktiveinsatz war. Stallman implementierte die Funktion neu, um den ganzen Schirm auszunutzen, und erweiterte sie dann auf eine feine, aber profunde Weise. Er schuf die Möglichkeit, TECO-Befehlsfolgen, sogenannten "`Makros"', Tastenkombinationen zuzuweisen. Fortgeschrittene TECO-Nutzer speicherten Makros schon in Dateien ab; Stallmans Hack machte es möglich, sie schnell aufzurufen. Das Ergebnis war ein nutzerprogrammierbarer WYSIWYG-Editor. "`Das war wirklich ein Durchbruch"', sagt Guy Steele\index{Steele, Guy|(}, damaliger AI-Lab-Kollege.\footcite[][]{rmsetfse}

Laut Stallmans Erinnerung löste der Macro-Hack eine Explosion in der Weiterentwicklung aus. "`Wirklich jeder hat seine eigene Sammlung von umgeänderten Editorbefehlen geschrieben, einen Befehl für alles, was er typischerweise gern macht"', so Stallman. "`Die Leute gaben sie herum und verbesserten sie, meistens machten sie sie leistungsfähiger und allgemeiner. Die Sammlungen der Umänderungen wurden allmählich eigenständige Systemprogramme."'\footcite[][]{rmsetfse}

So viele Leute fanden die Makro-Neuerung nützlich und hatten sie in ihre eigenen TECO-Versionen integriert, dass der Editor durch die ausgebrochene Makromanie nebensächlich geworden ist. "`Wir fingen an, ihn geistig eher als Programmiersprache einzustufen denn als einen Editor"', sagt Stallman. Die Nutzer hatten selbst ihre Freude daran, an der Software zu feilen und neue Ideen auszutauschen.\footcite[][]{rmsetfse}

Zwei Jahre nach der Explosion begann die hohe Innovationsrate unangenehme Nebeneffekte an den Tag zu legen. Das explosive Wachstum hatte die Machbarkeit des kollaborativen Ansatzes eindrucksvoll bewiesen, aber es führte auch zu Inkompatibilitäten. "`Wir hatten ein babylonisches Sprachgewirr\comment{Turm-von-Babylon-Effekt}"', sagt Guy Steele.

Das Gewirr drohte den Geist zu ersticken, der es hervorgerufen hat, sagt Steele. Hacker hatten das ITS entworfen, um es Programmierern leichter zu machen, Wissen auszutauschen und gegenseitig die Arbeitsergebnisse zu verbessern. Das bedeutete, dass man in der Lage war, sich an den Arbeitsplatz eines anderen Programmierers zu setzten, seinen\comment{dessen?} Arbeitsstand zu laden und direkt Kommentare und Änderungen in der Software zu machen. "`Manchmal war es am einfachsten, jemandem zu zeigen, wie man etwas programmiert oder debuggt, indem man sich einfach an das Terminal gesetzt und es ihnen vorgemacht hat."', erklärt Steele.

Die Makrofunktion begann nach ihrem zweiten Jahr diese Fähigkeit zu verbauen. In ihrem Eifer, die neuen Editiermöglichkeiten auszureizen, hatten die Hacker ihre TECO-Versionen soweit individuell angepasst, dass wenn ein Hacker an dem Terminal eines anderen saß, er normalerweise eine Stunde nur damit beschäftigt war, herauszufinden, was die Makrobefehle machen.

Frustriert davon wollte Steele das Problem selbst angehen. Er suchte sich die vier verschiedenen Makropakete zusammen und fing an, ein Diagramm über die nützlichsten Makrobefehle zu erstellen. Im Verlauf der Implementierung des Designs nach dem Diagramm, sagt Steele, hatte er Stallmans Aufmerksamkeit erregt.

"`Er fing an, mir über die Schulter zu schauen, und fragte, was ich da mache"', erinnert sich Steele.

Für Steele, einem Hacker mit milder Stimme, der mit Stallman nur selten zu tun hatte, bleibt das Ereignis in wacher Erinnerung. Einem anderen Hacker über die Schulter zu schauen, während er arbeitet, war nichts Ungewöhnliches am AI~Lab. Stallman, der TECO-Maintainer am Lab, erachtete Steeles Arbeit als "`interessant"' und machte sich bald daran, sie zu vervollständigen.

"`Oder wie ich gern sage, ich habe die ersten 0,001 Prozent der Implementierung gemacht und Stallman den Rest"', sagt Steele lachend.

Der neue Projektname "`Emacs"' kam von Stallman. Die Abkürzung für "`editing macros"' sollte zum Ausdruck bringen, welche evolutionäre Transzendenz bei der Makro-Welle in den zwei Jahren davor stattgefunden hatte. Außerdem sollte sie eine Lücke im Programmiererwortschatz schließen. Stallman hatte einen Mangel an ITS-Programmen mit dem Anfangsbuchstaben "`E"' bemerkt und entschied sich für "`Emacs"'\comment{, dadurch konnte man sich mit nur einem Buchstaben auf das Programm beziehen. Wieder einmal hatte die Vorliebe der Hacker für Effizienz seine Spuren hinterlassen}.\footcite[][]{rmsetfse}

Natürlich wechselte nicht jeder zu Emacs, oder nicht sofort. Den Nutzern stand es frei, ihre eigenen TECO-basierten Editoren weiter zu benutzen und zu pflegen. Aber die meisten zogen es vor, zu Emacs zu wechseln, besonders weil Emacs dafür entworfen war, das Ersetzen und Hinzufügen von Komponenten einfach zu machen, während andere Komponenten unverändert verwendet werden.

"`Einerseits haben wir versucht, wieder einen einheitlichen Befehlssatz zu schaffen; andererseits wollten wir ihn ausbaufähig halten, weil die Programmierbarkeit wichtig gewesen ist"', erinnert sich Steele\index{Steele, Guy|)}.

Stallman stand nun vor einem weiteren Problem: wenn die Nutzer Änderungen machten, sie aber dem Rest der Gemeinschaft nicht mitteilten, würde der Babel-Effekt schlicht an anderen Stellen auftreten. Stallman griff auf die Hackerdoktrin des Innovationsaustauschs zurück und brachte einen Kommentar in den Quellcode ein, der die Nutzungsbedingungen festlegt. Es stand den Nutzern frei, den Code zu verändern und weiterzuverbreiten, unter der Bedingung, dass sie alle vorgenommenen Änderungen zurückfließen lassen. Stallman nannte das "`der Emacs-Kommune beitreten"'. So wie aus TECO mehr als nur ein einfacher Texteditor geworden ist, so war aus Emacs mehr als nur ein einfaches Programm geworden. Für Stallman war es ein Gesellschaftsvertrag geworden. In einem Memo von 1981, das das Projekt dokumentiert, legte Stallman die Vertragsbedingungen dar. "`EMACS"', schreibt er, "`wird auf der Basis von kommunalem Austausch verbreitet, das bedeutet, dass alle Verbesserungen an mich zurückgegeben werden müssen, damit sie eingepflegt und weiterverbreitet werden können."'\footcite[Vgl.][\href{http://www.gnu.org/software/emacs/emacs-paper.html\#SEC34}{\#SEC34}]{emacspaper}

Das ursprüngliche Emacs lief nur auf der PDP-10, aber bald wollten auch Nutzer anderer Computer Emacs zum Editieren nutzen. Die explosionsartigen Innovationen setzten sich in diesem Jahrzehnt weiter fort, was zur Entstehung einer Unmenge an emacsähnlichen Programmen führte, mit variierender Kompatibilität untereinander. Die Regeln der Emacs-Kommune galten für sie nicht, weil sie eine getrennte Codebasis hatten. Einige erwähnten ihren Bezug zum Stallmanschen Original mit spaßigen rekursiven Namen: SINE (Sine is not Emacs), EINE (Eine is not Emacs) und ZWEI (Zwei was Eine initially). Ein echtes Emacs musste eine Erweiterbarkeit bieten wie das Original; Editoren mit ähnlichem Befehlsumfang, aber ohne die Nutzerprogrammierbarkeit nannte man "`ersatz Emacs"'.\footnote{"`Ersatz"' im Englischen hat in etwa dieselbe Konnotation wie "`Surrogat"' im Deutschen.} Ein Beispiel dazu war Mince (Mince is Not Complete Emacs). 

Während Stallman im AI Lab Emacs entwickelte, gab es andere, beunruhigende Entwicklungen anderswo in der Hackerwelt. Brian Reids Entscheidung 1979, eine "`Zeitbombe"' in Scribe einzubauen, wodurch Unilogic die unbezahlte Nutzung der Software limitieren konnte, war ein dunkles Omen für Stallman. "`Er hat es als die nazihafteste Sache angesehen, die ihm je untergekommen ist"', erinnert sich Reid. Trotz seiner späteren Internetberühmtheit als Mitschöpfer der alt-Hierarchie im Usenet, sagt Reid, muss er seinen Namen immer noch von der Entscheidung im Jahr 1979 reinwaschen, jedenfalls in Stallmans Augen. "`Er sagte, dass alle Software frei sein soll und die Vorstellung, Geld für Software zu verlangen, ein Verbrechen gegen die Menschheit ist."'\footnote{In einem Interview von 1996 im Online-Magazin \textit{MEME} gibt sich Stallman sehr verärgert über Scribe, ohne aber den Namen der Software oder Reids Namen zu erwähnen. "`Das Problem war, dass diesen Studenten niemand kritisiert oder bestraft hat für das, was er getan hat"', sagt Stallman. "`Im Ergebnis wurden andere Leute dazu verleitet, seinem Beispiel zu folgen."' \cite[Vgl.][]{memeint}}

Obwohl Stallman machtlos war, Reids Verkauf der Software an eine Firma abzuwenden, hatte er doch die Möglichkeit, andere Verhaltensweisen einzuschränken, die im Widerspruch zum Hackerethos standen. Als zentraler Maintainer des Original-Emacs fing Stallman an, seinen Einfluss für politische Ziele einzusetzen. Im Endstadium bei dem Konflikt mit den Administratoren vom Laboratory for Computer Science wegen des Passwortsystems startete Stallman einen Software-"`Streik"' und weigerte sich, den LCS-Mitarbeitern die neueste Version von Emacs zu schicken, bis sie das Sicherheitssystem von den Laborcomputern entfernen.\footcite[Vgl.][S.\,419]{hackers} Es war eher eine Geste als eine Sanktion, weil sie nichts davon abhielt, die Software selbst zu installieren. Aber es brachte den Punkt zum Ausdruck, dass das Setzen eines Passworts auf einem ITS-System zur Verdammung und zu Konsequenzen führt.

"`Eine Menge Leute waren mir böse, sie sagten, ich würde sie als Geisel halten oder erpressen, was auf eine Art stimmte"', sollte Stallman später dem Autor Steven Levy erzählen. "`Ich habe Gewalt gegen sie angewandt, weil ich dachte, dass sie Gewalt gegen alle insgesamt ausüben."'\footcite[][]{hackers}

Mit der Zeit wurde Emacs ein Verkaufsinstrument für die Hackerethik. Die Flexibilität, die Stallman in die Software eingebaut hatte, begünstigte nicht nur Zusammenarbeit, sondern verlangte sie. Nutzer, die sich nicht mit der Entwicklung auf dem Laufenden hielten oder ihre Beiträge nicht an Stallman zurücklieferten, gingen das Risiko ein, den neuesten Durchbruch zu verpassen. Und es gab viele Durchbrüche. Zwanzig Jahre später haben die Benutzer von GNU Emacs (eine zweite Implementierung ab 1984) es für so viele völlig verschiedene Dinge angepasst – für die Verwendung als Tabellenkalkulation, Taschenrechner, Datenbank und Webbrowser – dass die Emacs-Entwickler später den Vergleich mit einem überlaufenden Waschbecken aufgegriffen haben, um die vielseitige Funktionalität auszudrücken. "`Das ist die Idee, die wir übermitteln wollten"', sagt Stallman. "`Die Menge an Dingen, die es enthalten hat, ist gleichermaßen wunderbar als auch furchtbar."'

Stallmans Zeitgenossen am AI~Lab sind nachsichtiger. Hal Abelson\index{Abelson, Harold \glq Hal\grq}, ein Postgraduierter am MIT, der in den 70ern mit Sussman gearbeitet hat und Stallman später als Gründungs- und Vorstandsmitglied\comment{charter board member // Satzungsausschussmitglied?} der Free Software Foundation unterstützen sollte, beschreibt Emacs als "`eine absolut brillante Kreation"'. Mit der Möglichkeit, die er anderen Programmierern gegeben hat, neue Bibliotheken und Funktionen hinzuzufügen, ohne dabei das System zu beeinträchtigen, sagt Abelson, hat Stallman den Weg geebnet für zukünftige großangelegte kollaborative Softwareprojekte. "`Seine Struktur war robust genug, dass Leute aus aller Welt lose zusammenarbeiten [und] zu [Emacs] beitragen konnten"', sagt Abelson. "`Ich weiß nicht, ob es das vorher schon einmal gegeben hat."'\footnote{In diesem Kapitel habe ich mich entschieden, mich mehr auf die soziale Signifikanz von Emacs zu konzentrieren als die softwareseitige Signifikanz. Um Weiteres über die Softwareseite zu erfahren, empfehle ich Stallmans Memo von 1979, besonders den Abschnitt \textit{Research Through Development of Installed Tools} (\href{http://www.gnu.org/software/emacs/emacs-paper.html\#SEC27}{\#SEC27}). Nicht nur ist er für den nichttechnischen Leser zugänglich, er wirft auch Licht darauf, wie stark verflochten Stallmans politische Philosophie mit seiner Softwaredesignphilosophie ist. Hier ein Beispielauszug:

\begin{quote}
EMACS hätte nicht durch einen sorgfältigen Designprozess erreicht werden können, weil solche Prozesse nur bei Zielen angelangen, die zum Anfang sichtbar sind, und deren Erwünschtheit im Fazit der Anfangsphase etabliert ist. Weder ich noch irgendjemand sonst hatte sich einen erweiterbaren Editor vorgestellt, bis ich einen fertig hatte, noch wusste man ihn zu schätzen, bis man ihn selbst erlebt hat\comment{nor appreciated its value until he had experienced it}. EMACS existiert, weil ich freie Hand hatte, individuell kleine nützliche Verbesserungen zu machen auf einem Pfad, dessen Ende noch nicht abzusehen war.
\end{quote}}

Guy Steele\index{Steele, Guy}, heute Forscher der Oracle Corporation, drückt ähnliche Bewunderung aus. Er erinnert sich an Stallman hauptsächlich als einen "`brillanten Programmierer mit der Fähigkeit, in große Mengen relativ fehlerfreien Code zu produzieren."' Obwohl sich ihre Persönlichkeiten nicht unbedingt deckten, arbeiteten Steele und Stallman lange genug zusammen, dass Steele einen Eindruck von Stallmans intensivem Programmierstil zu bekommen. Er erinnert sich an ein denkwürdiges Erlebnis Ende der 70er, als die zwei Programmierer sich zusammengetan haben, um die "`Prettyprint"'-Funktion für den Editor zu schreiben. Prettyprint wurde ursprünglich von Steele entworfen, als weitere durch Tastenkombination startbare Funktion, die Quellcode in Emacs so umformatiert, dass er gleichzeitig lesbarer und platzsparender wird\comment{, further bolstering the program's WYSIWYG qualities}. Die Funktion war strategisch wichtig genug, um Stallmans Interesse zu wecken, und nicht viel später schrieb Steele, dass er und Stallman eine verbesserte Version planen.

"`Wir haben uns an einem Morgen hingesetzt"', erinnert sich Steele. "`Ich war an der Tastatur, und er war an meinem Ellbogen."', sagt er. "`Er war völlig damit einverstanden, dass ich tippe, aber er hat mir auch gesagt, was ich tippen soll."'

Die Programmiersitzung dauerte 10 Stunden an. In der ganzen Zeit, so Steele, hätten weder er noch Stallman eine Pause gemacht oder nicht über die Arbeit geredet. Zum Ende hatten sie es geschafft, den Quellcode für den Prettyprinter auf knapp unter 100 Zeilen zu drücken. "`Meine Finger lagen die ganze Zeit auf der Tastatur"', erinnert sich Steele, "`aber es fühlte sich so an, als ob unsere Ideen gemeinsam auf den Bildschirm geflossen sind. Er hat mir gesagt, was ich eintippen soll, und ich habe es getippt."'

Die Länge der Sitzung wurde offensichtlich, als Steele schließlich aus dem AI~Lab ging. Er stand vor dem Gebäude am 545 Tech Square und war überrascht, von der Dunkelheit der Nacht umgeben zu sein. Als Programmierer war Steele Coding-Marathons gewöhnt. Trotzdem war diesmal etwas anders gewesen. Mit Stallman zu arbeiten, hatte Steele gezwungen, alle äußeren Reize auszublenden und seine gesamten geistigen Kräfte auf die Aufgabe zu konzentrieren. Im Rückblick sagt Steele, er fand die Gedankenverschmelzung mit Stallman gleichzeitig aufregend und unheimlich. "`Mein erster Gedanke danach war, dass es eine großartige Erfahrung gewesen ist, sehr intensiv, und dass ich das nie wieder im Leben machen wollte."'
