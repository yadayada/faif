\chapter{Open Source}
\label{open_source}

[RMS: Dies ist das einzige Kapitel, in dem ich einige Zitate entfernt habe. Im gelöschten Material ging es um Open Source, es hatte aber keinen Bezug auf meine Person oder mein Schaffen.]

Im November 1995 rief Peter Salus\index{Salus, Peter|(}, Mitglied der Free Software Foundation und Autor des 1994 erschienenen Buches \citefield{title}{quartcent}, die Abonnenten der Mailingliste "`system-discuss"' des GNU Projects auf, Papers einzureichen. Salus, der planmäßige Konferenzvorsitzende, wollte die Hackerkollegen über die kommende \textit{Conference on Freely Redistributable Software} in Cambridge, Massachusetts, informieren. Angesetzt für Februar 1996 und gesponsert von der Free Software Foundation, sollte sie die erste Technikkonferenz werden, die sich rein der freien Software widmet. Als Zeichen der Verbundenheit mit anderen Free-Software-Programmierern waren Papers zu "`allen Aspekten von GNU, Linux, NetBSD, 386BSD, FreeBSD, Perl, Tcl/tk und anderen Werkzeugen, bei denen der Code zugänglich und weiterverbreitbar ist"', willkommen. Salus schrieb: 
\begin{quote}
Im Laufe der letzten 15 Jahre ist freie und kostengünstige Software allgegenwärtig geworden. Diese Konferenz soll die Implementierer verschiedener Arten frei verbreitbarer Software und die Verteiler solcher Software (auf verschiedensten Medien) zusammenbringen. Es wird Tutorials und refereed papers geben, außerdem Keynotes von Linus Torvalds und Richard Stallman.\footcite[][]{callfpap}
\end{quote}
% TODO: refereed papers?

Unter den Empfängern Salus' E-Mail war das Konferenzkomiteemitglied Eric S. Raymond\index{Raymond, Eric|(}. Obwohl er kein Projektleiter oder Firmenchef war, wie die anderen Mitglieder auf der Liste, hatte sich Raymond mit einigen Softwareprojekten und als Herausgeber des \textit{New Hacker's Dictionary}, einer stark erweiterten Ausgabe des \textit{Hacker's Dictionary} von Guy Steele aus dem davorigen Jahrzehnt, einen sauberen Ruf in der Hackergemeinde aufgebaut. Für Raymond war die Konferenz 1996 eine willkommene Veranstaltung. Obwohl er die Ansichten der Free-Software-Bewegung nicht vollständig teilte, hatte er zu einigen GNU-Programmen beigetragen, besonders GNU Emacs. Diese Beiträge hörten 1992 auf, als Raymond die Befugnis verlangte, direkt Änderungen an der offiziellen GNU-Version von GNU Emacs machen zu dürfen, ohne sie erst mit Stallman zu besprechen, welcher direkt für die Emacs-Entwicklung verantwortlich war. Stallman lehnte Raymonds Forderung ab und Raymond warf Stallman "`Micromanagement"' vor. "`Richard machte ein Theater, von wegen unbefugter Änderungen, obwohl ich doch bloß die Emacs-LISP-Bibliotheken aufgeräumt habe"', erinnert sich Raymond. "`Das hat mich so sehr frustriert, dass ich nicht mehr mit ihm arbeiten wollte."'

Trotz der Meinungsverschiedenheit blieb Raymond in der Free-Software-Gemeinschaft aktiv. Und als Salus das Team Stallman-Torvalds als Keynotespeaker vorschlug, unterstützte Raymond diese Idee. Mit Stallman als Repräsentant der älteren, weisen Gruppe der ITS/Unix-Hacker und Torvalds als Repräsentant der jüngeren, energetischen Menge der Linux-Hacker stellte das Paar eine symbolische Einheit dar, die nur von Vorteil sein könnte, besonders für ambitionierte jüngere (d.\,h. unter 40jährige) Hacker wie Raymond. "`Ich hatte irgendwie einen Fuß in beiden Lagern"', sagt Raymond.

Zum Ende der Konferenz wurden die Spannungen zwischen den beiden Lagern offensichtlich. Beide Gruppen hatten dennoch eines gemein, und zwar war die Konferenz ihre erste Möglichkeit, das finnische Wunderkind in Fleisch und Blut zu erleben. Überraschenderweise stellte Torvalds sich als charmanter, umgänglicher Redner heraus. Torvalds erstaunte mit seinem kaum merklichen schwedischen Akzent und seinem schnellen, bescheidenen Witz die Hörerschaft.\footnotemark{}
Noch verblüffender war laut Raymond Torvalds' Bereitschaft, Seitenhiebe an andere prominente Hacker auszuteilen, einschließlich dem prominentesten von allen, Richard Stallman. Zum Ende der Konferenz hatte Torvalds mit seiner halb hackerhaften, halb saloppen Art die älteren und jüngeren Konferenzteilnehmer für sich gewonnen.

\footnotetext{Obwohl Linus finnisch ist, ist seine Muttersprache Schwedisch. Die \citefield{title}{linusfaq} liefert eine kurze Erklärung:
\begin{quote}
Finnland hat eine beträchtliche schwedischsprachige Minderheit. Sie nennen sich \glq finlandssvensk\grq{} oder \glq finlandssvenskar\grq{} und sehen sich selbst als Finnen; viele dieser Familien leben seit Jahrhunderten in Finnland. Schwedisch ist eine der zwei Amtssprachen Finnlands.
\end{quote}
}

"`Es war ein Wendepunkt"', erinnert sich Raymond. "`Vor 1996 war Richard der einzige glaubhafte Anwärter auf die Position als ideologischer Anführer der gesamten Kultur. Leute mit anderer Meinung äußerten sie nicht öffentlich. Derjenige, der das Tabu gebrochen hat, war Torvalds."'

Der ultimative Tabubruch kam zum Ende der Show. Während einer Diskussion über die wachsende Marktdominanz von Microsoft Windows oder einem ähnlichen Thema gab Torvalds zu, ein Fan von Microsofts Präsentationssoftware PowerPoint zu sein. Aus der Sicht der konservativen Softwarepuristen war das gerade so, als ob man auf einer Sklavenbefreiungskonferenz mit seinen Sklaven prahlt. Aus der Sicht Torvalds und seiner wachsenden Gefolgschaft war das nur gesunder Menschenverstand. Warum sollte man zweckdienliche proprietäre Software meiden, nur aus Prinzip? Sie waren schon mit dem Prinzip nicht einverstanden. Wenn die Freiheit ein Opfer verlangt, sehen diejenigen, denen die Freiheit egal ist, das Opfer als Selbstkasteiung an, und nicht als Methode, etwas Wichtiges zu erreichen. Einem Hacker ging es nicht um Selbstkasteiung, es ging darum, seine Sache zu erledigen, und "`die Sache"' für sie war praktisch definiert.

"`Das war schon eine ziemliche schockierende Sache"', erinnert sich Raymond. "`Aber er konnte das machen, weil er bis 95/96 rasch an Einfluss gewonnen hatte."'
Stallman seinerseits erinnert sich an keine Spannungen auf der Konferenz 1996; er war, während Torvalds diese Aussage machte, wohl nicht anwesend. Aber er erinnert sich, später den Stachel von Torvalds' gefeierter "`Vorwitzigkeit"' gespürt zu haben. "`Es stand da was in der Linux-Dokumentation, dass man die GNU-Programmierrichtlinien ausdrucken und sie dann zerreißen sollte"',\footnotemark{} erinnert sich Stallman an ein Beispiel. "`Bei genauerer Betrachtung war der Teil, mit dem er nicht einverstanden war, der unwichtigste Teil, nämlich die Empfehlung, wie man seinen C-Code einrücken soll."'

"`O.\,k., dann ist er eben mit einigen Konventionen von uns nicht einverstanden. Schön und gut, aber er hat sich eine besonders gemeine Ausdrucksweise gewählt. Er hätte auch einfach sagen können \glq [Seht her], hier die ist Art, auf die ihr meiner Meinung nach euren Code einrücken solltet.\grq{} Und gut. Da sollte es keine Feindseligkeiten geben."'

%eigen Fußnote
\footnotetext{Torvalds spricht von verbrennen. \cite[Vgl.][]{lincodst}.}

Für Raymond hat der herzliche Empfang, mit dem andere Hacker Torvalds' Kommentare aufgenommen haben, eine Vermutung bestätigt: die Trennlinie zwischen den Linux- und GNU-Entwicklern war größtenteils generationsbedingt. Viele Linux-Hacker sind, wie Torvalds, in der Welt der proprietären Software aufgewachsen. Sie hatten angefangen, Beiträge zu freier Software zu leisten, ohne irgendeine Ungerechtigkeit in unfreier Software wahrzunehmen. Für die meisten war nichts außer der Komfort in Gefahr. Solange ein Programm nicht technisch minderwertig war, sahen sie wenig Anlass, es nur wegen der Lizenzfrage abzulehnen. Eines Tages würden vielleicht einige Hacker eine freie Alternative zu PowerPoint entwickeln. Warum sollte man bis dahin PowerPoint oder Microsoft kritisieren und die Software nicht nutzen?

Das war ein Beispiel für die wachsende Kontroverse innerhalb der Free-Software-Gemeinschaft, zwischen denen, die die Freiheit als solche schätzen, und denen, die hauptsächlich mächtige, zuverlässige Software schätzen. Stallman bezeichnete die zwei Lager als politische Parteien innerhalb der Bewegung und nannte erstere "`Freiheitspartei"'. Die Anhänger des anderen Lagers wollten sich keinen Namen geben, also nannte Stallman sie die "`Trittbrettpartei"' oder die "`Erfolgspartei"', weil einige von ihnen "`mehr Nutzer"' zum Hauptziel erklärt hatten.

In dem Jahrzehnt seit dem Start des GNU Projects hatte sich Stallman einen Namen als furchteinflößender Programmierer gemacht. Er hatte sich auf einen Ruf der Unnachgiebigkeit in Bezug auf Softwaredesign und Menschenführung aufgebaut. Das war zum Teil wahr, aber der Ruf stellte für jeden eine willkommene Ausrede dar, auf die man sich berufen konnte, wenn Stallman nicht das tat, was man selbst wollte. Der Ruf wurde durch falsche Annahmen verzerrt.

Zum Beispiel war der Free Software Foundation kurz vor der Konferenz 1996 fast die gesamte Belegschaft abtrünnig geworden. Brian Youmans\index{Youmans, Brian}, derzeitiges FSF-Mitglied und von Salus nach den Kündigungen angestellt, erinnert sich an die Vorgänge: "`Zu einem Zeitpunkt war Peter [Salus] der einzige Angestellte, der im Büro arbeitete."' Die einstigen Angestellten waren unzufrieden mit dem Executive Director, wie Bryt Bradley\index{Bradley, Bryt} ihren Freunden im Dezember 1995 mitteilt:

\begin{quote}
[Name entfernt] (Executive Director des FSF) entschied sich letzte Woche, aus ihrem Urlaub wegen Krankheit/politischer Gründe wiederzukommen. Wir Büromitarbeiter (Gena Bean, Mike Drain und ich) hatten uns entschieden, dass wir nicht länger unter ihrer Führung arbeiten wollen, weil sie in der Vergangenheit viele berufliche Fehlentscheidungen getroffen hatte, bevor sie ihren Urlaub nahm. Auch gab es zahlreiche Vorfälle, bei denen einzelnen auf unangemessene Weise mit der Kündigung gedroht wurde und wir ALLE aus unserer Sicht oftmals ihren Beschimpfungen ausgesetzt waren. Wir haben (vielfach) darum gebeten, dass sie nicht mehr als unsere Vorgesetzte wiederkommt, aber erklärten uns bereit, mit ihr als Kollegin zu arbeiten. Unsere Bitte wurde ignoriert. Wir haben gekündigt.
\end{quote}

Der besagte Executive Director stellte Stallman ein Ultimatum, ihr völlige Autonomie über das Büro zu geben, sonst würde sie kündigen. Stallman als Präsident der FSF weigerte sich, ihr uneingeschränkte Kontrolle über alle Vorgänge zu geben. Sie trat zurück und er stellte Peter Salus\index{Salus, Peter|)} an ihrer Stelle an.

Als Raymond als Außenstehender davon hörte, dass all diese Leute die FSF verlassen hatten, nahm er an, es wäre Stallmans Schuld gewesen. Es gab ihm die Bestätigung für seine Theorie, dass Stallmans Persönlichkeit der einzige Grund für alle Probleme im GNU Project war. Raymond hatte noch eine  Theorie: die neuerlichen Verzögerungen wie bei Hurd und Schwierigkeiten wie die Lucid-Emacs-Abspaltung wären Zeichen von Problemen beim Projektmanagement, nicht bei der Softwareentwicklung.

Kurz nach der Freely Redistributable Software Conference begann Raymond mit der Arbeit an seinem eigenen Lieblings-Softwareprojekt, einem Mailprogramm namens fetchmail. Nach Torvalds' Vorbild veröffentlichte Raymond ein Programm mit dem Versprechen, den Sourcecode so früh und so oft wie möglich zu aktualisieren. Als Nutzer dann anfingen, ihm Bugreports und Funktionswünsche zu schicken, erhielt Raymond, der erst ein wirres Chaos erwartete, eine resultierende Software, die erstaunlich stabil war. Der Erfolg mit Torvalds' Herangehensweise ließ Raymond zu der schnellen Analyse kommen: mit dem Internet als seine "`Petrischale"' und der strengen Prüfung durch die Hackergemeinde als Form der natürlichen Auslese hatte Torvalds ein evolutionäres Modell frei von zentraler Planung geschaffen.

Außerdem, entschied Raymond, hatte Torvalds einen Weg gefunden, wie man das Brooksche Gesetz umgehen konnte. Das von Fred P. Brooks, Manager des OS/360-Projekts von IBM und Autor des 1975 erschienenen Buchs, \citefield{title}{manmonth},\footnote{Vom Mythos des Mann-Monats} erstmals geäußerte Gesetz besagt, dass das Zuweisen zusätzlicher Entwickler zu einem Projekt nur zu weiteren Verzögerungen führt. Raymond, der wie die meisten Hacker daran glaubte, dass Software, wie Brei, von vielen Köchen verdorben wird, sah eine Revolution im Gange. Mit dem Einladen von immer mehr Köchen in die Küche hatte Torvalds wirklich einen Weg gefunden, die daraus entstehende Software \textit{besser} zu machen.\footnote{Brooks' Law ist die kurze Zusammenfassung des folgenden Zitats aus seinem Buch:
\begin{quote}
Weil Softwareherstellung von Natur aus eine Gemeinschaftleistung ist – eine Übung in komplexen Wirkungszusammenhängen – ist der Kommunikationsaufwand groß, und er übersteigt schnell die Zeitersparnis des einzelnen in seiner Arbeit, die durch die Aufteilung erreicht wird. Mehr Leute zum Projekt hinzuzufügen, verschiebt dann den Zeitplan nach hinten, nicht nach vorn.
\end{quote}
\cite[Vgl.][]{manmonth}
}

Raymond brachte seine Beobachtungen zu Papier. Er arbeitete sie in eine Rede ein, die er prompt vor einer Gruppe von Freunden und Nachbarn in Chester County, Pennsylvania, hielt. Unter dem Titel \citefield{title}{catb} stellte die Rede den "`Basar"'-Stil Torvalds' dem "`Kathedralen"'-Stil gegenüber, den jeder sonst anwandte. Raymond sagt, es gab begeisterte Reaktionen, aber nicht annähernd so begeistert wie die auf dem Linux Kongress 1997, einem Treffen von GNU/Linux-Nutzern in Deutschland im folgenden Sommer.

"`Auf dem Kongress habe ich stehende Ovationen am Ende der Rede bekommen"', erinnert sich Raymond. "`Ich habe das aus zwei Gründen als signifikant angesehen. Zum einen bedeutete es, dass sie begeistert darüber waren, was sie gehört hatten. Und zum anderen bedeutete es, dass sie von der Rede begeistert waren, trotz der Sprachbarriere."'

Später sollte Raymond die Rede unter demselben Titel, benannt nach der zentralen Analogie, als Essay aufsetzen. Das Essay hatte seinen Namen von Raymonds zentraler Analogie erhalten. Zuvor waren Programme "`Kathedralen"': beeindruckende, zentral geplante Monumente, gebaut, um die Ewigkeit zu überdauern. Linux andererseits war wie ein "`großer brabbelnder Basar"', eine Software, die mit der losen, dezentralen Dynamik des Internets entwickelt worden war.

% TODO: einarbeiten: Raymond erwähnt GNU Emacs' Lisp
% nennt Stallman Design-Genie
Raymonds Essay brachte den Kathedralenstil, den er selbst, Stallman und viele andere verwandt hatten, ausdrücklich mit dem GNU Project und Stallman in Verbindung und stellte die gegensätzlichen Entwicklungsmodelle als Vergleich zwischen Stallman und Torvalds dar. Wobei Stallman als Beispiel für den klassischen Kathedralenarchitekten gewählt wurde – d.\,h. ein Programmiergenie, das für 18 Monate von der Bildfläche verschwinden und mit einem GNU-C-Compiler wieder auftauchen konnte – Torvalds war eher ein genialer Gastgeber einer Abendgesellschaft. Das Entwicklungsmodell, in dem er andere die Diskussionen zum Linux-Design führen ließ und nur einschritt, wenn ein ganzer Tisch einen Schiedsrichter brauchte, reflektierte sehr stark Torvalds' eigene entspannte Persönlichkeit. Aus Torvalds' Perspektive war die wichtigste Führungsaufgabe nicht die Durchsetzung von Kontrolle, sondern den Ideenfluss am Laufen zu halten.

Raymond fasst zusammen: "`Ich glaube, Linus' cleverster und konsequentester Hack war nicht der Linux-Kernel an sich, sondern die Erfindung des Linux-Entwicklungsmodells."'\footcite[Vgl.][]{catb}
Wenn auch die Beschreibung dieser beiden Entwicklungsstile im Essay scharfsinnig war, so war doch die spezifische Verbindung des Kathedralenmodells mit Stallman (statt mit allen, die dieses Modell genutzt hatten, einschließlich Raymond selbst) reine Verleumdung. Tatsächlich hatten die Entwickler einiger GNU-Pakete, inklusive GNU Hurd, von Torvalds' Methoden gelesen und sie übernommen, bevor Raymond sie ausprobiert hatte, jedoch ohne sie weiter zu analysieren und sie öffentlich zu belobhudeln wie in Raymonds Essay. Tausende Hacker, die Raymonds Artikel gelesen haben, müssen durch diese Verunglimpfungen zu einer negativen Haltung gegenüber GNU gekommen sein.

% + Mozilla
Mit der Zusammenfassung der Geheimnisse hinter Torvalds' Führungserfolgen erregte Raymond die Aufmerksamkeit anderer Mitglieder der Free-Software-Gemeinschaft, für die Freiheit keine Priorität war. Sie wollten das Interesse der Industrie an der Nutzung und Entwicklung freier Software erwecken, und entschlossen sich, das Thema mit den Begriffen zu präsentieren, die für die Wirtschaft attraktiv sind: mächtig, zuverlässig, billig, fortschrittlich. Raymond wurde zum bestbekannten Befürworter dieser Ideen und erreichte mit ihnen die Managementebene von Netscape\index{Netscape|(}, deren proprietärer Browser Marktanteil gegenüber Microsofts ebenfalls proprietärem Internet Explorer verlor. Fasziniert von einer Rede Raymonds brachte ein leitender Angestellter bei Netscape die Kunde mit in den Firmenhauptsitz. Einige Monate später, im Januar 1998, kündigte das Unternehmen seinen Plan an, den Quellcode seines Flagschiff-Webbrowsers Navigator unter dem Codenamen Mozilla\index{Mozilla|(} zu veröffentlichen, in der Hoffnung, für zukünftige Versionen Unterstützung durch Hacker zu gewinnen.

% + Treffen fand bei VA statt (späterer Bezug darauf, aber von Stallman hier weggekürzt)
Als sich Netscapes CEO Jim Barksdale\index{Barksdale, Jim} auf Raymonds Essay \citefield{title}{catb} als einen großen Einfluss zur Entscheidung berief, hob das Unternehmen Raymond sofort auf das Level der Hacker-Prominenz. Er lud einige Leute zum Reden ein, darunter Larry Augustin, Gründer von VA Research, einer Firma, die Workstations mit vorinstalliertem GNU/Linux verkaufte; Tim O'Reilly\index{O'Reilly, Tim|(}, Gründer des Verlagshauses O'Reilly \& Associates, und Christine Peterson, Präsidentin des Foresight Institute, einer gemeinnützigen Organisation im Silicon Valley, die sich in Nanotechnologie spezialisiert hat. "`Bei dem Treffen ging es im Wesentlichen um eine Sache: wie man sich Netscapes Entscheidung zunutze machen konnte, so dass andere Firmen vielleicht dasselbe tun."'

Raymond erinnert sich nicht an die Gespräche, die stattfanden, aber er erinnert sich an die erste aufgebrachte Beschwerde. Trotz der Anstrengungen, die Stallman und andere Hacker unternommen haben, Leute darauf hinzuweisen, dass das Wort "`frei"' in "`freier Software"' für Freiheit steht und nicht für Kostenfreiheit, kam die Botschaft immer noch nicht an. Die meisten Manager interpretierten den Begriff beim ersten Hören als synonym zu "`gratis"' und der Hammer war gefallen, sie blendeten alles Weitere aus. Bis Hacker einen Weg fanden, dieses Missverständnis aus dem Weg zu räumen, stand die Free-Software-Bewegung vor einem Berg, auch nach Netscape.

%TODO: Freund/Freundin?
Peterson, deren Organisation ein reges Interesse am Vorwärtsbringen von freier Software gezeigt hatte, bot eine Alternative an: "`Open Source"'\index{Open Source}. Im Rückblick sagt Peterson, ihr sei der Begriff "`Open Source"' eingefallen, als sie mit einer befreundeten Person aus der PR-Industrie über Netscapes Entscheidung sprach. Sie kann sich nicht erinnern, wie der Begriff aufgekommen ist, oder ob sie ihn aus einem anderen Feld entlehnt hat, aber sie erinnert sich, dass ihr Gesprächspartner den Begriff nicht mochte.\footcite{profitmot}

Bei dem Treffen, sagt Peterson, war die Reaktion völlig anders. "`Ich habe gezögert, ihn vorzuschlagen"', erinnert sie sich. "`Ich hatte keinen Status in der Gruppe, und habe ihn beiläufig fallen lassen, nicht als \textit{den} neuen Begriff hervorgehoben."' Zu Petersons Überraschung kam der Begriff an. Zum Ende des Treffens schienen die meisten Teilnehmer, auch Raymond, damit zufrieden.

Raymond sagt, er habe den Begriff "`Open Source"' als Ersatz für "`freie Software"' das erste Mal ein, zwei Tage nach der Mozilla-Launch-Party öffentlich genutzt, als  O'Reilly ein Meeting angesetzt hatte, um über freie Software zu reden. Mit der Namensgebung "`the Freeware Summit"'\index{Freeware Summit} wollte O'Reilly die Aufmerksamkeit der Medien und der Gemeinschaft auf die anderen verdienstvollen Projekte ziehen, die Netscape\index{Netscape|)} auch zur Veröffentlichung von Mozilla geraten hatten.
"`Diese Leute hatten so viel gemeinsam und ich war überrascht, dass sie sich nicht alle kannten"', sagt O'Reilly. "`Ich wollte die Welt wissen lassen, welche großen Auswirkungen die Free-Software-Kultur schon gehabt hatte. Die Leute hatten von dem größten Teil der Free-Software-Tradition noch gar nichts mitbekommen."'

Als er aber die Einladungsliste zusammenstellte, machte O'Reilly eine Entscheidung, die langfristige politische Konsequenzen haben sollte. Er entschied sich, nur Entwickler von der Westküste auf die Liste zu setzen – Leute wie Larry Wall, Eric Allman, Schöpfer von sendmail und Paul Vixie, Schöpfer von BIND. Es gab natürlich auch Ausnahmen: Raymond als Bürger Pennsylvanias, der schon wegen des Mozilla-Starts\index{Mozilla|)} in der Stadt war, bekam eine Einladung. Und auch Guido van Rossum, der Erfinder von Python, aus Virginia. "`Frank Willison, mein leitender Herausgeber und Python-Champion in der Firma, hat ihn eingeladen, ohne mich vorher zu fragen"', erinnert sich O’Reilly. "`Ich habe mich gefreut, ihn dabeizuhaben, aber anfangs war das alles nur ein lokales Treffen."'

Einige sahen in der Ablehnung, Stallmans Namen auf die Liste zu setzen, einen Affront. "'Ich habe mich deswegen dagegen entschieden, die Veranstaltung zu besuchen"', sagt Perens\index{Perens, Bruce}. Raymond, der dort hinging, sagt, er hätte sich vergeblich für die Einladung Stallmans eingesetzt. Das Gerücht um einen Affront wurde bestärkt durch den Umstand, dass Gastgeber O'Reilly sich öffentlich mit Stallman über das Thema des Urheberrechts an Softwarehandbüchern befehdet hatte. Vor der Veranstaltung hatte Stallman sich dafür ausgesprochen, dass die Handbücher zu freier Software genauso frei kopierbar und veränderbar sein sollten wie die freie Software selbst. O'Reilly seinerseits argumentierte, dass ein Wertschöpfungsmarkt für unfreie Bücher den Nutzen der freien Software erhöhe, indem sie sie einer größeren Allgemeinheit besser zugänglich machen. Die zwei hatten auch über den Titel der Veranstaltung gestritten. Stallman insistierte auf "`Free Software"' statt "`Freeware"'. Letzterer ist ein Begriff, der meist Programme bezeichnet, die kostenlos verfügbar, aber keine freie Software sind, weil der Quellcode nicht veröffentlicht wird.

Im Nachhinein sieht O'Reilly die Entscheidung, Stallman nicht auf die Liste der Eingeladenen zu setzen, nicht als Affront an. "`Zu der Zeit hatte ich Richard nie persönlich getroffen, aber in unseren E-Mail-Wechseln hatte er sich als unbeugsam erwiesen und war nicht willens gewesen, einen Dialog zu führen. Ich wollte sichergehen, dass die GNU-Tradition beim Treffen vertreten ist, also habe ich John Gilmore und Michael Tiemann\index{Tiemann, Michael|(} eingeladen, die ich persönlich kannte und von denen ich wusste, dass sie hinter den Werten der GPL standen, aber williger schienen, eine ehrliche Diskussion über die Stärken und Schwächen der verschiedenen Free-Software-Projekte und -Traditionen zu führen. Bei all dem Buhei danach wünschte ich, ich hätte Richard auch eingeladen, aber glaube nicht, dass die Nichteinladung als mangelnder Respekt dem GNU-Projekt oder Richard persönlich gegenüber ausgelegt werden kann."'

Affront oder nicht, O'Reilly und Raymond sagen, dass der Begriff "`Open Source"' gerade genug Teilnehmer überzeugt hatte, um als Erfolg gerechnet werden zu können. Die Teilnehmer tauschten Ideen und Erfahrungen aus und machten ein Brainstorming, wie man das Image der freien Software verbessern kann. Ein Hauptanliegen war das Herausstellen der Erfolge freier Software, besonders im Bereich der Internet-Infrastruktur, um nicht wieder das Thema GNU/Linux versus Microsoft Windows hochzuspielen. Aber wie in dem früheren Treffen bei VA schlug die Diskussion schnell zu den Problemen um, die mit dem Begriff "`freie Software"' verbunden waren. O'Reilly, der Gastgeber, erinnert sich an eine Bemerkung Torvalds', Teilnehmer des Treffens:

"`Linus war zu dem Zeitpunkt gerade ins Silicon Valley gezogen und erklärte, dass er erst kürzlich erfahren hatte, dass das Wort \glq frei\grq{} im Englischen zwei Bedeutungen hatte – frei wie \glq libre\grq{} und frei wie \glq gratis\grq.\,"'

Michael Tiemann, Gründer von Cygnus, schlug eine Alternative zum problembelasteten Begriff "`free Software"' vor: Sourceware. "`Niemand war großartig davon begeistert”, so O'Reilly. "`Dann hat Eric den Begriff \glq Open Source\grq{} fallen lassen."' Obwohl einigen der Begriff gefiel, gab es nicht annähernd eine Mehrheit, die die Änderung der offiziellen Terminologie zu diesem Begriff unterstützt hätte. Zum Ende der eintägigen Konferenz stimmten die Teilnehmer über die drei Begriffe ab – Free Software, Open Source und Sourceware. Laut O'Reilly stimmten 9 von 15 Teilnehmern für "`Open Source"'. Obwohl einige immer noch Einwände gegen den Begriff hatten, stimmten alle Teilnehmer zu, ihn in allen zukünftigen Diskussionen mit der Presse zu benutzen. "`Wir wollten mit einer einheitlichen Botschaft nach außen treten"', sagt O'Reilly. Es dauerte nicht lange, bis der Begriff Einzug in den nationalen Wortschatz fand. Kurz nach dem Gipfel schickte O'Reilly seine Teilnehmer zu einer Pressekonferenz mit Reportern von der \textit{New York Times}, dem \textit{Wall Street Journal} und anderen führenden Zeitungen. Binnen einiger Monate war Torvalds' Gesicht auf der Titelseite des \textit{Forbes} Magazine und die Konterfeis von Stallman, Perl-Erfinder Larry Wall und Apache-Teamleiter Brian Behlendorf auf der Doppelseite. Das Geschäft mit Open Source war eröffnet.

Für die Gipfelteilnehmer wie Tiemann war eine einheitliche Botschaft das Wichtigste. Obwohl sein Unternehmen einen großen Teil seines Erfolgs dem Verkauf freier Software und zugehörigen Dienstleistungen verdankte, sah er die Schwierigkeiten, vor denen andere Programmierer und Unternehmer standen.
"`Es steht außer Frage, dass das Wort \glq frei\grq{} in vielen Situationen verwirrend ist"', sagt  Tiemann\index{Tiemann, Michael|)}. "`Open Source positioniert sich als geschäftsfreundlich und geschäftlich sinnvoll. Freie Software positioniert sich als moralisch gerecht. Egal, wie es ausgehen sollte, dachten wir, es wäre am vorteilhaftesten, sich der Open-Source-Gruppe anzuschließen."'
Raymond rief Stallman nach dem Treffen an, um ihm von dem neuen Begriff "`Open Source"' zu erzählen und ihn zu fragen, ob er ihn benutzen würde. Raymond sagt, Stallman hätte kurz mit dem Gedanken gespielt, den Begriff zu übernehmen, aber dann doch verworfen. "`Ich weiß es, weil ich direkte persönliche Gespräche darüber geführt habe"', sagt Raymond. 

%Absatz von Stallman eingefügt
Stallmans direkte Antwort war "`Ich muss mir das noch überlegen."' Am folgenden Tag war er zum Schluss gekommen, dass die Werte Raymonds und O'Reillys in der Zukunft den Diskurs um "`Open Source"' dominieren würden und es der beste Weg sei, die Ideen der Free-Software-Bewegung in der Öffentlichkeit zu erhalten, indem man bei dem traditionellen Begriff bleibt.

Später im Jahr 1998 stellte Stallman seine Position zum Terminus "`Open Source"' dar: obwohl hilfreich bei der Vermittlung der technischen Vorteile freier Software, bringt er seine Benutzer aber auch dazu, das Thema der Softwarefreiheit zu bagatellisieren. Er vermeidet die ungewollte Bedeutung "`Gratissoftware"' und die gewollte Bedeutung "`freiheitsrespektierende Software"' gleichermaßen. Als Mittel zur Verdeutlichung der letzteren Bedeutung ist er deswegen nicht geeignet. Im Grunde hatten  Raymond und O'Reilly der nicht idealistischen politischen Partei in der Gemeinschaft einen Namen gegeben, der Partei, mit der Stallman nicht übereinstimmt.

Außerdem dachte Stallman, dass die Vorstellungen von "`Open Source"' die Leute dazu führen würden, der Unterstützung vom wirtschaftlichen Umfeld zu viel Gewicht beizumessen. Obwohl eine solche Unterstützung an sich nichts Schlechtes wäre, erwartete er, dass es zu inakzeptabelen Zugeständnissen kommt, wenn man zu sehr danach bettelt. "`Die 101. Verhandlung würde einen lehren, dass wenn man sich zu verzweifelt um die Zustimmung eines anderen bemüht, man freiwillig den Kürzeren zieht"', sagt er.
"`Man muss darauf vorbereitet sein, nein zu sagen"'. Auf der LinuxWorld Convention and Expo 1999, einer Veranstaltung, von Torvalds als "`Coming-out-Party"' der "`Linux"'-Community angekündigt, fasst Stallman seine Position zusammen und beschwört seine Hackerkollegen, den Verlockungen von bequemen Kompromissen zu widerstehen.

"`Weil wir gezeigt haben, was wir leisten können, müssen wir nicht [darum betteln], mit Firmen zusammenzuarbeiten oder unsere Ziele kompromittieren"', sagt Stallman während einer Gesprächsrunde. "`Lasst sie doch die Angebote machen, und wir akzeptieren. Wir müssen nicht[s daran] ändern, was wir tun, damit sie uns helfen. Man kann einen Schritt zum Ziel hin machen, dann noch einen und immer so weiter, bis man schließlich sein Ziel erreicht. Oder man kann eine halbe Sache machen und dann nie wieder einen weiteren Schritt, und man kommt nie an."'

Stallman hatte jedoch schon vor der LinuxWorld-Show zunehmende Bereitschaft bewiesen, die Open-Source-Befürworter zu verprellen. Einige Monate vor dem Freeware Summit hatte O'Reilly\index{O'Reilly, Tim|)} seine zweite jährliche Perl Conference abgehalten. Diesmal war Stallman unter den Teilnehmern. Während einer Podiumsdiskussion, in der IBMs Entscheidung gelobt wurde, den freien Apache-Webserver in seinen kommerziellen Angeboten einzusetzen, ergriff Stallman die Gelegenheit, sich über ein Zuhörermikrophon Gehör zu verschaffen, und prangerte das Podiumsmitglied John Ousterhout\index{Ousterhout, John|(} stark an, seinerseits Erfinder der Skriptsprache Tcl. Stallman markte Ousterhout als "`Parasiten"' der Free-Software-Gemeinschaft, weil er mit seinem neugegründeten Unternehmen Scriptics eine proprietäre Version von Tcl auf den Markt brachte. Ousterhout hatte vorgebracht, dass Scriptics nur das absolute Minimum der Verbesserungen in die freie Version von Tcl einfließen lassen würde\comment{, um im Endeffekt diese kleinen Beiträge zur Gewinnung dazu zur meaning it would in effect use that small contribution to win community approval for much a larger amount of nonfree software development}.
Stallman lehnte diese Position ab und verurteilte Scriptics' Pläne. "`Ich glaube nicht, dass Scriptics für das Überleben von Tcl notwendig ist"', sagt Stallman und wird von der Besucherschaft ausgezischt.\footcite{profitmot}

"`Es war kein schöner Anblick"', erinnert sich Rich Morin\index{Morin, Richard \glq Rich\grq} von Prime Time Freeware. "`John hat einige respektabele Dinge gemacht: Tcl, Tk, Sprite. Er hat wirklich etwas beigetragen."' Trotz seines Verständnisses für Stallman und seiner Position versteht Morin auch diejenigen, die sich von Stallmans dissonanten Worten beunruhigt fühlten\comment{ for those troubled by Stallman’s discordant words}.

%von Stallman hinzugefügt
Stallman will sich nicht entschuldigen. "`Proprietäre Software zu kritisieren ist übel – proprietäre Software ist übel. Ousterhout hat in der Tat in der Vergangenheit echte Beiträge geleistet, aber der Punkt ist der, dass Scriptics ein zu fast 100\% proprietäres Softwareunternehmen werden sollte.\footnotemark{} Auf der Konferenz bedeutete das Eintreten für die Freiheit, mit fast jedem anderen anderer Meinung zu sein. Als Teil des Publikums konnte ich nur einige wenige Sätze sagen. Die einzige Möglichkeit, das Thema anzusprechen, so dass es nicht sofort wieder in Vergessenheit gerät, war mit scharfen Worten."'

% den auch
"`Wenn Leute mich tadeln, dass ich \glq eine Szene mache\grq, wenn ich eine ernsthafte Kritik an jemandes Verhalten übe, aber Torvalds \glq frech\grq{} nennen, wenn er gemeinere Sachen über geringere Themen sagt, dann sieht das für mich nach Doppelmoral aus."'
\footnotetext{Scriptics änderte 2000 seinen Namen zu "`Ajuba Solutions"' und wurde im selben Jahr von Interwoven aufgekauft und die meisten Entwickler haben die Firma verlassen. Vgl. \url{http://wiki.tcl.tk/912}}


Stallmans kontroverse Kritik an Ousterhout\index{Ousterhout, John|)} hatte kurzzeitig einen möglichen Sympathisanten verprellt, Bruce Perens\index{Perens, Bruce|(}. Eric Raymond schlug 1998 die Gründung der Open Source Initiative vor, kurz OSI, einer Organisation, die die Benutzung des Begriffs "`Open Source"' überwachen und eine Definition für Firmen herausgeben sollte, die ihre eigenen Programme machen wollten. Raymond rekrutierte Perens, der einen Definitionsentwurf erstellen sollte.\footcite[][]{osdef}

Perens sollte später aus der OSI austreten und sein Bedauern darüber ausdrücken, dass die Organisation sich in Opposition zu Stallman und dem FSF aufgestellt hatte.
Trotzdem versteht Perens, wenn er sich die Notwendigkeit einer Definition von freier Software außerhalb der Schirmherrschaft der Free Software Foundation im Nachhinein betrachtet, warum andere Hacker das Bedürfnis nach Distanzierung haben. "`Ich mag und bewundere Richard wirklich"', sagt Perens\index{Perens, Bruce|)}. "`Ich glaube, Richard würde seine Sache besser machen, wenn er ausgeglichener wäre. Das schließt eine Auszeit von der freien Software für einige Monate ein."'

Stallmans Anstrengungen sollten wenig Gegenwirkung zur PR-Welle der Open-Source-Befürworter haben. Als im August 1998 der Chiphersteller Intel einen Anteil am GNU/Linux-Distributor Red Hat kaufte, beschrieb ein Artikel in der \textit{New York Times} das Unternehmen als ein Produkt der Bewegung, die "`als freie Software oder Open Source bekannt ist."'\footcite[][]{forsale} Sechs Monate später kündigte ein Artikel von John Markoff Apples Entscheidung für den "`Open-Source-"'-Webserver Apache in der Schlagzeile an.\footcite[][]{appleado} Diese Dynamik sollte mit der wachsenden Dynamik der Firmen zusammenfallen, die den Begriff "`Open Source"' bereitwillig aufgriffen. Ab August 1999 verkaufte Red Hat, eine Firma, die sich nun begierig mit der Bezeichnung "`Open Source"' schmückte, Anteile an der Börse. Im Dezember machte VA Linux – ehemals VA Research – mit seinem Börsengang Geschichte. Vom Ausgabekurs von 30\$ pro Aktie schoss der Kurs anfangs auf über 300\$ und lag zum Börsenschluss bei 239\$. Aktionäre, die Glück hatten, konnten den Anfangswert auf dem Papier um 698\% steigern, ein NASDAQ-Rekord.
Eric Raymond als Vorstandsmitglied hatte Anteile im Wert von 36 Millionen Dollar. Diese hohen Preise waren jedoch nur vorübergehend; sie stürzten ein, als die Dot-Com-Blase platzte.

%Fußnote hinzugefügt
Die Botschaft der Open-Source-Befürworter war einfach: man muss es nur geschäftsfreundlich machen, um das Konzept "`freie Software"' zu verkaufen. Sie sahen Stallman und die Free-Software-Bewegung als marktfeindlich an; sie wollten Druck auf sie ausüben.\comment{ Statt die Außenseiter an der Schule zu sein, spielten sie die Rolle der Berühmtheiten, EEE magnifying their power in the process.}
Diese Methoden waren ein großer Erfolg für Open Source, aber nicht für die Ideale der freien Software. Um die "`Kunde zu verbreiten"', hatten sie ihren wichtigsten Teil ausgelassen: die Vorstellung von Freiheit als ethisches Thema. Die Auswirkungen dieser Auslassung sind heute sichtbar: anno 2009 enthalten fast alle GNU/Linux-Distributionen proprietäre Programme, Torvalds' Version des Linux-Kernels enthält proprietäre Firmware und das Unternehmen, das sich früher VA Linux nannte, baut sein Geschäftsmodell auf proprietärer Software auf. Auf über der Hälfte der Webserver der Welt läuft eine Version von Apache und das Standard-Apache ist freie Software, aber auf vielen dieser Servern läuft eine veränderte proprietäre Version von IBM.\footnote{IBM HTTP Server}

"`An seinen schlechten Tagen glaubt Richard, dass Linus Torvalds und ich sich gegen ihn verschwören, um seine Revolution zu kapern"', so Raymond. "`Richard's Ablehnung des Begriffs \glq Open Source\grq{} und seine absichtliche Erzeugung einer ideologischen Spaltung %his deliberate creation of an ideological fissure
kommen in meinen Augen von einer Mischung aus Idealismus und Territorialverhalten. Es gibt Leute, die denken, dass es ganz an Richards Ego liegt. Ich glaube das nicht. Es ist eher so, dass er persönlich so sehr mit dem Free-Software-Gedanken verbunden hat, dass er jede Gefahr für sie auch als Gefahr für sich ansieht."'

%Stallman
Stallman antwortet: "`Raymond stellt meine Ansichten falsch dar: Ich glaube nicht, dass Torvalds sich mit irgendjemandem \glq verschwört\grq, weil Verstohlenheit nicht seine Art ist. Aber Raymonds schlechtes Benehmen ist aus diesen Aussagen selbst ersichtlich. Statt zu meinen Ansichten (oder das, was er dafür hält) Stellung zu nehmen und ihre Vorteile anzusprechen, bringt er psychologische Interpretationen zu ihnen. Er schreibt die harschen Interpretationen ungenannten Dritten zu, und \glq verteidigt\grq{} mich dann, indem er selbst eine geringfügig weniger abfällige wählt. Er hat mich schon oft so \glq verteidigt\grq.\,"'

Ironischerweise sollte der Erfolg von Open Source und den Open-Source-Befürwortern wie Raymond\index{Raymond, Eric|)} Stallmans Rolle als Anführer nicht schmälern – aber er sollte bei vielen zu Missverständnissen führen, wovon er der Anführer ist. Da die Free-Software-Bewegung nicht die Anerkennung in Wirtschaft und den Medien hat wie Open Source, haben die meisten GNU/Linux-Nutzer noch nie von ihrer Existenz gehört, geschweige denn von ihren Ansichten. Sie haben von den Ideen und Werten von Open Source gehört, und können sich gar nicht vorstellen, dass Stallman andere Ansichten haben könnte. Deswegen empfängt er Nachrichten, in denen ihm für sein Eintreten für "`Open Source"' gedankt wird und muss in seinen Antworten erklären, dass er nie Unterstützer davon gewesen ist, und nutzt den Anlass, den Absender über freie Software aufzuklären.

%Stallman
Einige Autoren erkennen den Begriff der "`freien Software"' an, indem sie "`FLOSS"' verwenden, was für "`Free/Libre and Open Source Software"' steht. Jedoch sagen sie oft, es gebe eine einzige "`FLOSS"'-Bewegung, was geradezu so ist, als würde man sagen, in den USA gäbe es eine "`liberal-konservative"' Bewegung. Und die Ansichten, die man mit dieser vermeintlich einzigen Bewegung verbindet, sind die Open-Source-Ansichten, von denen sie schon gehört haben.

Trotz all dieser Hürden verschafft die Free-Software-Bewegung ihren Ideen manchmal Gehör und wächst weiterhin. Indem sie nicht nachgibt und ihre Vorstellungen denen von Open Source gegenüberstellt, gewinnt sie an Boden. "`Eine von Stallmans Hauptcharakterzügen ist, dass er nicht nachgibt"', sagt Ian Murdock\index{Murdock, Ian}. "`Er wartet auch schon mal ein Jahrzehnt, bis die Leute sich seiner Meinung anschließen, wenn es darauf ankommt."'

Murdock z.\,B. findet seine unbeugsame Natur gleichermaßen erfrischend wie wertvoll. Stallman mag nicht mehr der einzige Anführer der Free-Software-Bewegung sein, aber er ist immer noch der Nordstern der Free-Software-Gemeinde. "`Man weiß immer, dass der in seinen Ansichten konsequent ist"', sagt Murdock. "`Die meisten Leute sind nicht so. Ob man ihm nun zustimmt oder nicht, dafür muss man ihn achten."'
