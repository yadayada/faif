\chapter{Weiter im Kampf}

Für Richard Stallman heilt die Zeit nicht alle Wunden, aber sie ist ihm ein zweckdienlicher Verbündeter.

Vier Jahre nach \citefield{title}{catb} ist Stallman immer noch über Raymonds Kritik verärgert. Auch echauffiert er sich über Linus Torvalds' Erhebung in die Rolle des weltbekanntesten Hackers. Er erinnert sich an ein populäres T-Shirt, dass zur Zeit der 1999er Linux-Messen auftauchte. Es sollte das Star-Wars-Filmplakat nachahmen und bildete Torvalds mit einem geschwungenen Lichtschwert als Luke Skywalker ab, und Stallmans Gesicht hatte man auf den R2-D2 gepappt. Das Shirt regt Stallman immer noch auf, weil es ihn als Torvalds' Handlanger darstellt, aber auch weil es Torvalds in die Führerrolle der Free-Software-Gemeinde erhebt, eine Rolle, die Torvalds selbst nur widerwillig akzeptiert. "`Es ist ironisch"', sagt Stallman traurig. "`Das Schwert zu ergreifen ist genau das, was Linus sich weigert zu tun. Er bringt alle dazu, sich auf ihn als Symbol der Bewegung zu konzentrieren, und dann kämpft er nicht. Was soll das?"'

Andererseits ist es der Widerwille Torvalds', "`das Schwert zu ergreifen"', der die Tür für Stallman offengelassen hat, seinen Ruf als der moralische Vermittler innerhalb der Gemeinschaft zu stärken. Trotz der Gründe zur Beschwerde muss Stallman zugeben, dass die letzten Jahre ziemlich gut verlaufen sind, für ihn selbst und für seine Organisation. Vom ironischen Erfolg des GNU/Linux-Systems, das alle als "`Linux"' bezeichnen, auf die Auswechselbank geschoben, hat Stallman trotzdem wieder die Initiative ergriffen. Sein Vortragsprogramm zwischen Januar 2000 und Dezember 2001 hat ihn auf sechs Kontinente verschlagen und in Länder gebracht, in denen der Gedanke der freien Software einen bitteren Unterton hat – zum Beispiel China und Indien.

Außerhalb der Kanzel\comment{bully pulpit} nutzte Stallman den Einfluss der GNU General Public License (GPL), von der er der Verwalter blieb. Im Sommer 2000, als die Luft langsam aus den Linux-Börsengängen des Jahrs 1999 ging, konnten Stallman und die Free Software Foundation zwei große Siege erringen. Im Juli 2000 hatte Trolltech, ein norwegisches Softwareunternehmen und Entwickler von Qt, einer Graphikbibliothek für das GNU/Linux-Betriebbssystem, angekündigt, dass es seine Software unter der GPL lizenzieren wird. Einige Wochen später hat Sun Microsystems, ein Unternehmen, das bis dahin misstrauisch auf dem Open-Source-Zug mitgefahren war, ohne selbst Code beizutragen, schließlich nachgegeben und angekündigt, dass es seine neue OpenOffice-Suite\footnote{Sun war durch ein bestehendes Trademark gezwungen, den plumpen Namen "`OpenOffice.org"' zu verwenden.} unter der Doppellizenz LGPL (Lesser GNU Public License) und SISSL (Sun Industry Standards Source License) veröffentlicht.

Im Fall von Trolltech war der Sieg das Resultat einer langwierigen Anstrengung durch das GNU Project. Die Unfreiheit von Qt war ein ernsthaftes Problem für die Free-Software-Gemeinschaft, weil KDE, eine freie graphische Desktopumgebung mit wachsender Popularität, davon abhing. Qt war unfreie Software, aber Trolltech erlaubte freien Softwareprojekten (wie KDE), es gratis zu nutzen. Obwohl KDE selbst frei war, konnten Benutzer, die auf ihre Freiheit bestanden, es nicht einsetzen, weil sie Qt ablehnen mussten. Stallman erkannte an, dass viele GNU/Linux-Nutzer einen graphischen Desktop nutzen würden wollen, und die meisten würden ihre Freiheit nicht so ernst nehmen, dass sie auf KDE verzichten würden\comment{, with Qt hiding within}. Die Gefahr war, dass GNU/Linux ein Vehikel für die Installation von KDE und somit auch vom unfreien Qt werden könnte. \comment{This would undermine the freedom which was the purpose of GNU.}

Um mit dieser Gefahr umzugehen, engagierte Stallman Leute, die zwei parallele Gegenprojekte starten sollten. Das eine war GNOME, GNUs freie graphische Desktopumgebung. Das andere war Harmony, ein freier, kompatibler Ersatz für Qt. Sollte GNOME erfolgreich werden, wäre KDE nicht mehr unumgänglich; sollte Harmony erfolgreich sein, würde KDE nicht mehr von Qt abhängen. So oder so wären die Nutzer in die Lage versetzt, einen graphischen Desktop ohne das unfreie Qt zu haben.

1999 waren diese zwei Projekte gut vorangekommen und das Management von Trolltech begann den Druck zu spüren. Also veröffentlichte Trolltech Qt unter seiner eigenen freien Lizenz, der QPL. Die QPL erfüllte die Voraussetzungen für eine freie Lizenz, aber Stallman wies auf den Nachteil der Inkompatibilität mit der GPL hin: man konnte der GPL unterliegenden Code nicht mit Qt in einem Programm kombinieren, ohne eine der Lizenzen zu verletzen. Das Management von Trolltech sah ein, dass die GPL genauso gut ihren Zielen entspricht und veröffentlichte Qt unter einer Doppellizenz: derselbe Qt-Code war gleichzeitig unter der GNU GPL und der QPL lizenziert. Der Sieg war nach drei Jahren gekommen.

Nachdem Qt frei war, war die Motivation hinter der Entwicklung von Harmony (was für den Einsatz noch nicht weit genug fortgeschritten war) weggefallen und die Entwickler gaben es auf. GNOME hatte erheblich an Fahrt gewonnen und seine Entwicklung ging weiter, und es bleibt bis heute die Desktopumgebung für GNU.

Sun wollte nach den Regeln der Free Software Foundation spielen. Auf der 1999er O'Reilly Open Source Conference verteidigte Sun Microsystems Mitgründer und wissenschaftlicher Leiter Bill Joy die "`Community-Source"'-Lizenz seiner Firma, die im Grunde ein verwässerter Kompromiss war, der die Nutzer die Software von Sun kopieren und modifizieren lässt, aber es verbietet, Kopien der Software zu verkaufen, ohne vorher mit Sun ein Lizenzabkommen zu treffen. (Mit diesen Beschränkungen konnte man die Lizenz nicht als frei gelten lassen, und auch nicht als Open Source.) Ein Jahr nach Joys Rede erschien Sun Microsystems Vizepräsident Marco Boerries auf derselben Bühne und legte den neuen Lizenzkompromiss für seine Office-Suite OpenOffice dar. \comment{O-Ton: designed specifically for the GNU/Linux operating system}

"`Ich kann es in drei Buchstaben ausdrücken"', sagt Boerries. "`G-P-L"'.

Zu der Zeit, sagt Boerries, hatte die Entscheidung seiner Firma wenig mit Stallman zu tun, sondern mehr mit der Dynamik hinter GPL-geschützten Programmen. "`Was im Grunde passiert ist, war die Einsicht, dass verschiedene Produkte verschiedene [Benutzergruppen] anziehen, und die Lizenz, die man nutzt, davon abhängt, welche Art [Gruppe] man anziehen will"', sagt Boerries. "`Bei [OpenOffice] war es klar, dass wir die höchste Korrelation mit der GPL-Gemeinde hatten."'\footnote{Marco Boerries in einem Gespräch mit dem Autor (Juli 2000).}  Leider war der Sieg nicht vollumfänglich, weil OpenOffice die Verwendung unfreier Plug-ins empfiehlt.

Solche Kommentare heben die unterschätzten Stärken der GPL hervor und indirekt auch das politische Genie des Mannes, der den Hauptteil zu seiner Erschaffung beigetragen hat. "`Es gibt keinen Anwalt auf der Welt, der die GPL so ausgearbeitet hätte, wie sie jetzt ist"', sagt Eben Moglen, \index{Moglen, Eben|(} Professor für Rechtswissenschaften an der Columbia University und Chef-Justiziar der Free Software Foundation. "`Aber sie funktioniert. Und sie funktioniert wegen Richards Entwurfsphilosophie."'

Moglen, früher professioneller Programmierer, kann seine Pro-bono-Arbeit mit Stallman auf das Jahr 1990 zurückverfolgen, als er von ihm um Rechtsbeistand in einer persönlichen Angelegenheit gebeten wurde. Moglen, der damals mit dem Verschlüsselungsexperten Phillip Zimmerman während seiner juristischen Auseinandersetzungen mit den staatlichen Behörden zusammenarbeitete, sagt, er war von der Anfrage geehrt.\footnote{Für weitere Informationen zu Zimmermans rechtlichen Schwierigkeiten, siehe \cite[][S.\,287f]{crypto}. \comment{In der Originalversion des Buchs \textit{Free as in Freedom} habe ich berichtet, dass Moglen Zimmerman bei der Verteidigung gegen die National Security Agency unterstützt hat. Levy zufolge wurde gegen Zimmerman vom U.S. Attorney's Office und den U.S. Customs ermittelt, nicht dem NSA.}}

"`Ich habe ihm gesagt, dass ich Emacs jeden Tag meines Lebens verwende, und dass schon eine ganze Menge rechtliche Unterstützung meinerseits nötig wäre, um meine Schuld zu begleichen."'

Seitdem hatte Moglen, vielleicht mehr als jeder andere, die besten Möglichkeiten gehabt, die Überlappungen von Stallmans Hackerphilosophie in den rechtlichen Bereich zu beobachten. Moglen sagt, Stallmans Herangehensweise an den Rechtscodex und an Softwarecode sind größtenteils gleich. "`Ich muss sagen, als Anwalt ist der Gedanke, dass man all seine Fehler aus seinen Rechtsdokumenten bekommen sollte, kein sehr sinnvoller"', sagt Moglen. "`Es gibt Ungewissheiten bei jedem Gerichtsprozess, und die meisten Anwälte hoffen, die Zweifel zum Vorteil ihrer Klienten ausnutzen zu können. Richards Ziel ist genau das Gegenteil. Er will alle Ungewissheit beseitigen, was von Natur aus unmöglich ist. Es ist unmöglich, eine Lizenz aufzusetzen, die alle Umstände in allen möglichen Rechtssystemen der Welt abdeckt. Aber wenn man es versuchen wollte, müsste man so vorgehen wie er. Und die entstandene Eleganz und Einfachheit im Ergebnis erreicht fast, was sie erreichen soll. Und von da aus bringt einen ein wenig Jura ziemlich weit."'

Als jemand, der mit dem Durchdrücken von Stallmans Agenda beauftragt ist, versteht Moglen die Frustration von möglichen Verbündeten. "`Richard ist ein Mann, der bei Angelegenheiten, die er für fundamental hält, keine Kompromisse machen will"', sagt er, "`und er nimmt es nicht leicht, Wörter zu verdrehen oder selbst listige Mehrdeutigkeit anzustreben, was die menschliche Gesellschaft oft von vielen Leuten verlangt."'

Außer der Free Software Foundation zu helfen, hat Moglen Angeklagten in anderen Copyrightfragen beigestanden, beispielsweise Dmitri Skljarow und Verbreitern des DVD-Entschlüsselungsprogramms deCSS.

Skljarow hatte ein Programm geschrieben und veröffentlicht, das den digitalen Kopierschutz von Adobes e-Books knackt, und da in Russland kein Gesetz dagegen existierte, geschah das als Teil seiner Arbeit als Angestellter einer russischen Firma. Er wurde dann verhaftet, als er die USA besuchte, um eine Rede über seine Arbeit auf einer Konferenz zu halten.\comment{to give a scientific paper about his work} Stallman war eifriger Teilnehmer an den Protesten gegen Adobe wegen der Festnahme Skljarows und die Free Software Foundation prangerte den Digital Millennium Copyright Act als "`Softwarezensur"' an, konnte sich aber nicht für Skljarows Programm einsetzen, weil es keine freie Software war. Deswegen arbeitete Moglen im Auftrag der Electronic Frontier Foundation für die Verteidigung Skljarows. Die FSF vermied eine Involvierung in die Verbreitung von deCSS, weil das illegal war, aber Stallman verurteilte die US-Regierung für das Verbot von deCSS und Moglen arbeitete als Verteidiger der Angeklagten.

Moglen hat mit dem Folgen der von der FSF getroffenen Entscheidung, sich nicht in diese Fälle einzumischen, Stallmans Dickköpfigkeit zu schätzen gelernt. "`Es gab über die Jahre Zeiten, wo ich zu Richard gegangen bin und gesagt habe: \glq Wir müssen dieses machen. Wir müssen jenes machen. \textit{So} sieht die strategische Situation aus. \textit{Das} ist unser nächster Zug. \textit{Das} müssen wir machen.\grq{} Und Richards Antwort war immer: \glq Wir müssen überhaupt nichts machen.\grq{} Nur warten. Was getan werden muss, wird getan."'

"`Und wissen Sie was?"', sagt Moglen\index{Moglen, Eben|)}, "`Normalerweise hat er recht."'

Solche Kommentare stellen Stallmans Selbsteinschätzung in Abrede: "`Ich bin kein guter Spieler"', sagt er zu den vielen Kritikern, die ihn als scharfsinnigen Strategen sehen. "`Ich kann nicht gut in die Zukunft blicken und voraussehen, was ein anderer vielleicht machen wird. Mein Ansatz war immer, mich auf das Fundament [der Ideen] zu konzentrieren, zu sagen: \glq Lasst uns das Fundament so solide machen, wie es möglich ist.\grq\,"'

Die wachsende Popularität der GPL und ihre fortwährende Anziehungskraft zollen dem von Stallman und seinen GNU-Kollegen gelegten Fundament den besten Tribut. Obwohl Stallman niemals der einzige auf der Welt war, der freie Software veröffentlicht hat, gebührt ihm trotzdem als einzigem Anerkennung für die Schaffung des ethischen Rahmenwerks für die Free-Software-Bewegung. Ob die heutigen Programmierer sich in diesem Rahmenwerk wohlfühlen oder nicht, ist unerheblich. Der Umstand, dass man überhaupt eine Wahl hat, ist Stallmans größtes Vermächtnis.

Stallmans Vermächtnis zu dieser Zeit zu diskutieren, scheint etwas voreilig. Stallman, [zur Zeit der Erstauflage] 48, hat noch ein paar Jahre, um sein Vermächtnis zu vergrößern oder zu schmälern. Trotzdem ist es durch die Schwungkraft der Free-Software-Bewegung verlockend, Stallmans Leben außerhalb der täglichen Schlachten in der Softwareindustrie zu untersuchen und innerhalb einer hehreren, geschichtlichen Umgebung.

Man muss Stallman zugute halten, dass er sich weigert, jedwede Möglichkeit zur Spekulation wahrzunehmen. "`Ich konnte noch nie detaillierte Pläne ausarbeiten, wie die Zukunft aussehen wird"', sagt Stallman und liefert seinen eigenen vorzeitigen Nachruf: "`Ich habe nur gesagt \glq Ich werde kämpfen. Wer weiß, wohin mich das führen wird.\grq\,"'

Keine Frage, dass Stallman mit der Anzettelung seiner Streits genau die Leute verprellt hat, die sonst vielleicht seine größten Verfechter gewesen wären, wäre er willens gewesen, für ihre Ansichten statt die seinen zu kämpfen. Es ist außerdem ein Beweis für seine direkte ethische Art, für die viele von Stallmans Gegnern trotzdem noch ein gutes Wort einlegen, wenn man sie drängt. Die Spannung zwischen Stallman, dem Ideologen, und Stallman, dem Hackergenie, lässt einen Biographen sich jedoch wundern: wie werden die Leute Stallman beurteilen, wenn Stallmans eigene Persönlichkeit nicht mehr da ist, nicht mehr im Wege stehen kann?

\comment{In den frühen Entwürfen dieses Buchs habe ich sie die "`100-Jahre"'-genannt.} In der Hoffnung, eine objektive Sicht auf Stallman und sein Werk zu bekommen, frage ich verschiedene Koryphäen der Softwareindustrie, sich aus dem aktuellen Zeitrahmen herauszudenken und sich in die Position eines Historikers zu versetzen, der in 100 Jahren die Free-Software-Bewegung untersucht. Aus heutiger Sicht ist es einfach, Ähnlichkeiten zwischen Stallman und Amerikanern der Vergangenheit zu finden, die es trotz einer eher marginalen Rolle zu Lebzeiten zu historischem Ruhm gebracht haben. Einfache Vergleiche wären mit Henry David Thoreau, Transzendentalphilosoph und Autor von \textit{Civil Disobedience}, und John Muir, Gründer des Sierra Clubs und Vater der heutigen Umweltbewegung. Man kann auch leicht Ähnlichkeiten mit Männern wie William Jennings Bryan alias "`The Great Commoner"' erkennen, dem Anführer der Populistenbewegung, Feind der Monopole und ein Mann, der trotz seiner Macht in die historische Bedeutungslosigkeit verschwunden ist.

%TODO: altertümliches Politwort Frei...
Obwohl er nicht die erste Person ist, die Software als Gemeingut ansieht, ist Stallman dank der GPL eine Fußnote in den Geschichtsbüchern der Zukunft wert. Unter diesem Umstand scheint es sinnvoll, einen Schritt zurückzutun und Richard Stallmans Vermächtnis außerhalb des aktuellen Zeitrahmens zu betrachten. Wird die GPL im Jahr 2102 immer noch von Programmierern benutzt werden oder wird sie längst auf der Strecke geblieben sein? Wird der Begriff "`freie Software"' genauso politisch altertümlich klingen wie heutzutage das "`freie Silber"' oder wird er im Licht der kommenden politischen Ereignisse wie eine unheimliche Vorahnung erscheinen?

Die Zukunft vorherzusagen ist ein riskantes Unterfangen. Stallman lehnt es ab, sagt, dass es voraussetzt, dass man keinen Einfluss darauf hat, wenn man danach fragt, was die Leute in 100 Jahren von einem denken werden. Die Frage, die er bevorzugt, ist "`Was sollten wir tun, um die Zukunft besser zu machen?"' Aber die meisten Leute, denen die spekulative Frage gestellt wurde, haben angebissen.
 
"`In Hundert Jahren sind Richard und einige andere Leute mehr als nur eine Fußnote wert"', sagt Moglen\index{Moglen, Eben|(}. "`Sie werden zur Haupthandlung gezählt werden."'

Die einigen anderen Leute, die Moglen für die zukünftigen Geschichtsbücher nominiert, sind John Gilmore, der, abgesehen von seinen verschiedenartigen Beiträgen zur freien Software, die Electronic Frontier Foundation gegründet hat, und Theodor Holm Nelson, alias Ted Nelson, Autor des 1982 veröffentlichten Buchs \textit{Literary Machines}. Moglen sagt, Stallman, Nelson und Gilmore stechen in disjunkten Gebieten als historisch signifikant heraus. Nelson, dem gemeinhin die Prägung des Begriffs "`Hypertext"' zugeschrieben wird, rechnet er das Erkennen der misslichen Lage des Informationseigentums im digitalen Zeitalter an. Gilmore und Stallman hingegen schenkt er beträchtliche Anerkennung für das Erkennen der negativen politischen Auswirkungen der Informationskontrolle und das Gründen der Organisationen – die Electronic Frontier Foundation im Falle Gilmores und die Free Software Foundation im Falle Stallmans – um diesen Auswirkungen entgegenzuwirken. Von beiden sieht Moglen Stallmans Aktivitäten eher als persönlich motiviert denn als politisch.

"`Richard war dahingehend einzigartig, dass ihm die ethischen Konsequenzen von unfreier Software schon zu einem frühen Moment besonders klar waren"', sagt Moglen. "`Das hat viel mit Richards Persönlichkeit zu tun, die viele Leute, wenn sie über in schreiben, als Begleiterscheinung oder sogar als Nachteil in Richard Stallmans Lebenswerk darstellen."'

Gilmore, der seinen Einschluss zwischen dem sprunghaften Nelson und dem jähzornigen Stallman als etwas zweifelhafte Ehre ansieht, unterstützt nichtsdestotrotz Moglens\index{Moglen, Eben|)} Argument. Gilmore schreibt:

\begin{quote}
Meine Vermutung ist, dass Stallmans Schriften der Zeit standhalten werden wie die von Thomas Jefferson; er ist ein deutlicher Autor und auch deutlich in seinen Prinzipien\ldots Ob Richard so einflussreich sein wird wie Jefferson, wird davon abhängen, ob die Abstraktionen, die wir "`Bürgerrechte"' nennen, in hundert Jahren wichtiger werden als die Abstraktionen, die wir "`Software"' nennen oder "`technisch auferlegte Beschränkungen"'.
\end{quote}

Ein weiteres Element in Stallmans Vermächtnis, das man nicht vergessen sollte, schreibt Gilmore, ist das kollaborative Softwareentwicklungsmodell, auf dem das GNU Project  Pionierarbeit geleistet hat. Obwohl zuweilen fehlerhaft, hat sich das Modell dennoch zu einem Standard in der Softwareindustrie entwickelt. Alles in allem, sagt Gilmore, könnte dieses Entwicklungsmodell letztendlich sogar erfolgreicher sein als das GNU Project, die GPL oder irgendeine sonst von Stallman entwickelte Software:

\begin{quote}
Vor dem Internet war es ziemlich schwer, mit Leuten über Distanz an Software zu arbeiten, selbst bei Teams, die sich untereinander kennen und vertrauen. Richard hat bei der kollaborativen Entwicklung von Software Pionierarbeit geleistet, besonders bei unorganisierten Freiwilligen, die sich selten treffen. Richard hat keine der zugrundeliegenden Werkzeuge dafür geschaffen (das TCP-Protokoll, E-Mail-Listen, diff und patch, tar-Dateien, RCS oder CVS oder Remote-CVS), aber er hat diejenigen genutzt, die zur Verfügung standen, und soziale Gruppen von Programmierern gebildet, die effektiv zusammenarbeiten können.
\end{quote}

Stallman glaubt, dass die – zwar positive – Bewertung am Thema vorbeigeht. "`Sie betont die Entwicklung mehr als die Freiheit, was eher die Werte von Open Source wiedergibt als von freier Software. Wenn die Nutzer in Zukunft so auf das GNU Project zurückschauen, fürchte ich, dass das zu einer Welt führen wird, in der Entwickler Nutzer in Fesseln halten werden und sie zum Lohn gelegentlich die Entwicklung unterstützen lassen, aber ihnen niemals die Ketten abnehmen werden."'

Lawrence Lessig, Professor für Rechtswissenschaften an Stanford und Autor des 2001 erschienenen Buchs \citefield{title}{futofideas}, ist ähnlich optimistisch. Wie viele Rechtsgelehrte sieht Lessig die GPL als Hauptbollwerk der derzeitigen sogenannten "`digitalen Allmende"', der riesigen Zusammenballung von Software, Netzwerk- und Telekommunikationsstandards im Gemeineigentum, die das exponentielle Wachstum des Internets in den letzten drei Jahrzehnten ausgelöst hat. Statt Stallman mit anderen Internet-Pionieren in Verbindung zu bringen, Männern wie Vannevar Bush, Vinton Cerf und J.\,C.\,R. Licklider, die andere überzeugt haben, Computertechnologie in einem größeren Rahmen zu sehen, betrachtet Lessig Stallmans Einfluss als persönlicher, introspektiver und letzten Endes einzigartig:

\begin{quote}
[Stallman] hat die Debatte um das "`Ist"' zum "`Sollte"' gewandelt. Er hat Leute aufmerksam gemacht, was auf dem Spiel steht, und er hat ein Instrument geschaffen, um diese Ideale voranzubringen\ldots Wo das gesagt ist, weiß ich nicht recht, wie ich ihn im Kontext von Cerf oder Licklider einordnen sollte. Die Innovation ist anders. Es geht nicht nur um eine bestimmte Art Code, oder die Ermöglichung des Internets. [Es geht] viel eher darum, die Leute dazu zu bringen, den Wert einer bestimmten Art des Internets zu erkennen. Ich glaube nicht, dass es irgendwen sonst in dieser Kategorie gibt, vor oder nach ihm.
\end{quote}

Natürlich sieht nicht jeder Stallmans Vermächtnis schon in Stein gemeißelt. Eric Raymond, der Open-Source-Befürworter, der das Gefühl hat, Stallmans Rolle als Anführer sei seit 1996 signifikant geschwunden, sieht gemischte Zeichen, wenn er mit der Glaskugel einen Blick ins Jahr 2102 wirft:

\begin{quote}
Ich glaube, Stallmans Artefakte (GPL, Emacs, GCC) werden als revolutionäre Werke angesehen werden, als Grundstein der Informationswelt. Ich glaube, die Geschichte wird weniger gnädig blicken auf einige seiner Theorien, von denen [sein] Handeln ausging, und überhaupt nicht gnädig auf seine persönliche Tendenz zum Territorial- und Kultführer-Verhalten.
\end{quote}

Und auch Stallman selbst sieht gemischte Zeichen:

\begin{quote}
Was die Geschichtsbücher in zwanzig Jahren über das GNU Project sagen werden, wird davon abhängen, wer den Kampf um die Freiheit des öffentlichen Wissens gewinnen wird. Wenn wir verlieren, werden wir nur eine Fußnote sein. Wenn wir gewinnen, ist es ungewiss, ob die Leute die Rolle des GNU-Betriebssystems kennen werden – wenn sie denken, dass das System "`Linux"' ist, dann werden sie ein falsches Bild davon haben, was passiert ist und warum.

Aber selbst wenn wir gewinnen, das was die Geschichtsleute in hundert Jahren lehren werden, hängt wahrscheinlich davon ab, wer politisch dominieren wird.
\end{quote}

Auf der Suche nach einer historischen Analogie für sich holt sich Stallman die Figur des John Brown zu Hilfe, einen militanten Abolitionisten, der von der einen Seite\comment{of the Mason Dixon line} als Held angesehen wurde und von der anderen als Verrückter.

John Browns Sklavenrevolte ist nie passiert, aber im Verfahren gegen ihn hat er erfolgreich die nationale Forderung nach der Abschaffung der Sklaverei geweckt. Im Amerikanischen Bürgerkrieg war John Brown ein Held; 100 Jahre später und fast bis zum Ende der 90er lehrten die Geschichtsbücher, er sei verrückt gewesen. In der Ära der Rassentrennung, zur Zeit der schamlosen Bigotterie, akzeptierten die USA teilweise die Geschichte, die der Süden über sich selbst erzählen wollte und die Geschichtsbücher druckten viele Unwahrheiten über den Bürgerkrieg und mit ihm verbundene Ereignisse.

Solche Vergleiche belegen gleichermaßen die selbstwahrgenommene nebensächliche Natur Stallmans aktueller Arbeit als auch die dichotomische Natur seines aktuellen öffentlichen Ansehens. Schwerlich kann man sich Stallmans Ansehen auf das Niveau von Browns zur Zeit der Reconstruction fallen vorstellen. Stallman hat, abgesehen von seinen gelegentlichen kriegsähnlichen Gleichnissen, wenig getan, was Gewalt herausfordern würde. Trotzdem kann man sich leicht eine Zukunft vorstellen, in der Stallmans Ideen auf dem Müllhaufen der Geschichte landen.\footnote{RMS: Sam Williams schrieb dann weiter "`Durch die Gestaltung des Free-Software-Anliegens nicht als Massenbewegung, sondern als Ansammlung persönlicher Gefechte gegen die Kräfte der proprietären Versuchung [\ldots]"', und das stimmt nicht mit den Fakten überein. Seit der ersten Ankündigung des GNU Projects habe ich die Öffentlichkeit um Unterstützung gebeten. Die Free-Software-Bewegung will eine Massenbewegung sein und die einzige Frage, die sich stellt, ist, ob sie genug Unterstützter hat, um als "`Masse"' gezählt werden zu können. 2009 hatte die Free Software Foundation über 3000 Mitglieder, die die saftigen Gebühren zahlen, und über 20.0000 Abonnenten des monatlichen Newsletters.}

Stallmans Wille wird sich vielleicht eines Tages als sein größtes bleibendes Vermächtnis herausstellen. Moglen,\index{Moglen, Eben|(} aufmerksamer Beobachter über das letzte Jahrzehnt, warnt jene, die Stallmans Persönlichkeit als kontraproduktiv oder als Begleiterscheinung der "`Artefakte"' von Stallmans Leben fehlverstehen. Ohne diese Persönlichkeit, meint Moglen, gäbe es herzlich wenige Artefakte, über die man sprechen könnte. Moglen, ehemaliger wissenschaftlicher Mitarbeiter\comment{Rechtsreferendar} am Supreme Court, sagt:

\begin{quote}
Sehen Sie, der großartigste Mann, für den ich je gearbeitet habe, war Thurgood Marshall. Ich weiß, was ihn zu einem großartigen Mann gemacht hat. Ich weiß, wie er die Welt verändert hat auf seine mögliche Art. Ich würde mich etwas aus dem Fenster lehnen, wenn ich einen Vergleich machen würde, weil die beiden unterschiedlicher nicht sein könnten: Thurgood Marshall war ein Mann in der Gesellschaft, der eine Gesellschaft von Ausgestoßenen in der Gesellschaft repräsentiert, die sie einschließt, aber trotzdem ein Mann in der Gesellschaft. Seine Kompetenz war soziale Kompetenz. Aber er war auch aus einem Guss. So verschieden sie auch in jeder anderen Hinsicht waren, ist die Person, mit der ich ihn heute in dem Sinne am meisten vergleiche – aus einem Guss, kompakt, aus dem Stoff, der Stars macht, jemand, der Sachen durchzieht – Stallman.
\end{quote}

Um das Bild zu untermauern, berichtet Moglen von einem Moment im Frühling 2000. Der Erfolg von VA Linux hallte immer noch in den Wirtschaftsmedien nach und ein halbes Dutzend Themen mit Bezug auf freie Software ging durch die Nachrichten. Umgeben von dem Berg Themen und Artikeln, die alle darauf warten, erörtert zu werden, saß Moglen mit Stallman beim Essen und fühlte sich wie ein Schiffbrüchiger, der im Auge des Sturms ausgesetzt wurde. In der nächsten Stunde drehte sich das gelassene Gespräch um einen einzigen Gegenstand: die Stärkung der GPL.

"`Wir haben dort gesessen und darüber gesprochen, was wir gegen einige Probleme in Osteuropa unternehmen können und was wir tun werden, wenn das Problem des Besitztums an Inhalten anfängt, freie Software zu bedrohen"', erinnert sich Moglen. "`Als wir so redeten, kam mir kurz der Gedanke, wie wir wohl ausgesehen haben müssen für die Leute, die an uns vorbeigehen. Da sitzen wir, die zwei kleinen bärtigen Anarchisten, schmieden Ränke und planen unsere nächsten Schritte. Und natürlich rupft Richard sich die Knäuele aus seinen Haaren und wirft sie in seine Suppe und verhält sich auf seine übliche Art. Jeder, der das Gespräch mitgehört hat, muss gedacht haben, dass wir verrückt sind, aber ich habe gewusst: Ich wusste, die Revolution ist direkt hier an diesem Tisch. Das hier macht sie möglich. Und dieser Mann macht sie möglich."'

Moglen sagt, dieser Moment hat ihm wie kein anderer die elementare Simplizität des Stallmanschen Stils klargemacht.

"`Es war witzig"', erinnert sich Moglen\index{Moglen, Eben|)}. "`Ich habe zu ihm gesagt: \glq Richard, weißt Du, wir sind die einzigen Leute, die an dieser Revolution nichts verdient haben.\grq{} Und dann habe ich für das Mittagessen bezahlt, weil ich wusste, dass er nicht das Geld dafür hat."'\footnote{RMS: Ich weigere mich nie, wenn mir jemand anbietet, mir ein Essen auszugeben, weil ich meinen Selbstwert nicht daraus ziehe, die Rechnung zu bezahlen. Aber ich hatte sicherlich das Geld, um für mein Essen zu bezahlen. Mein Einkommen aus dem in etwa zur Hälfte bezahlten Teil meiner Vorträge ist zwar viel geringer als das Gehalt eines Jura-Professors, aber ich bin nicht arm.}
