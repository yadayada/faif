\chapter{2001: Odyssee im Hackall}

Die Informatik-Fakultät der New York University befindet sich in der Warren Weaver Hall, einem festungsähnlichen Gebäude zwei Häuserblocks östlich des Washington Square Parks. Riesige Abluftöffnungen der Klimaanlage erzeugen einen Burggraben aus heißer Luft, der herumlungernde Leute und Anwälte fernhält. Besucher, die den Graben durchschreiten, treffen auf eine weitere hohe Hürde, eine Sicherheitsabfertigung direkt hinter dem einzigen Eingang des Gebäudes.

Nach der Sicherheit herrscht eine leicht entspannte Atmosphäre. Trotzdem sind über das gesamte Erdgeschoss Schilder verteilt, die vor den Gefahren von offengelassenen Türen und mit Keilen offengehaltenen Brandschutztüren warnen. Insgesamt mahnen die Schilder: selbst in den relativ friedlichen Grenzen New Yorks vor dem 11. September 2001 kann man nie zu vorsichtig oder misstrauisch sein.

Die Schilder bilden einen interessanten thematischen Kontrapunkt zu der wachsenden Anzahl an Besuchern, die sich im Atrium des Gebäudes sammeln. Ein paar sehen wie NYU-Studenten aus. Die meisten sehen wie zottelhaarige Konzertgänger aus, die vor einer Konzerthalle herumlaufen und auf den Auftritt der Hauptgruppe warten. Einen kurzen Morgen lang haben die Massen die Warren Weaver Hall übernommen, und der Sicherheitsperson bleibt nichts anderes zu tun, als Ricki Lake im Fernsehen zu schauen und ihren Kopf in Richtung des nahegelegenen Auditoriums zu neigen, wenn sie Besucher nach der Rede fragen.

Im Auditorium angekommen, findet der Besucher die Person, die für das zeitweilige Aussetzen der Sicherheitsbestimmungen verantwortlich ist. Die Person ist Richard M. Stallman, Gründer des GNU Projects, Präsident der Free Software Foundation, Gewinner der MacArthur Fellowship (1990), Gewinner des Grace Murray Hopper Award der Association of Computing Machinery (auch 1990), einer der Empfänger des Takeda Award for Social/Economic Betterment der Takeda Foundation (2001) und früherer Hacker am AI~Lab. Wie auf einer Menge hackerspezifischer Webseiten angekündigt, einschließlich der eigenen des GNU Projects, http://www.gnu.org, ist Stallman in Manhattan, seiner ehemaligen Heimatstadt, um eine heißerwartete Gegenrede in Bezug auf Microsofts jüngste Kampagne gegen die GNU General Public License zu halten.

Das Thema Stallmans Rede ist die Geschichte und Zukunft der Free-Software-Bewegung. Der Ort ist von besonderer Bedeutung. Weniger als einen Monat zuvor war Microsofts Senior Vice President Craig Mundie\index{Mundie, Craig} in der nahe gelegenen NYU Stern School of Business aufgetreten und hielt eine Rede, in der er die GNU General Public License, kurz GNU GPL\index{GPL}, heftig angriff, ein von Stallman 16 Jahre zuvor erdachtes juristisches Instrument. Als Vereitelungsmechanismus gegen die wachsende Welle von Geheimhaltung von Software in der Computerindustrie\comment{– a wave first noticed by Stallman during his 1980 troubles with the Xerox laser printer –-} hat sich die GPL zu einem zentralen Werkzeug der Free-Software-Gemeinschaft entwickelt. Einfach ausgedrückt etabliert die GPL eine Form von Gemeingut – heutzutage "`digitale Allmende"' genannt – durch das juristische Gewicht des Urheberrechts. Mit der GPL geschieht das unwiderruflich; hat der Urheber einmal der Gemeinschaft auf diesem Weg den Code gegeben, kann der Code nicht wieder von jemandem proprietär gemacht werden. Abgeleitete Versionen müssen unter derselben Lizenz veröffentlicht werden, wenn sie einen wesentlichen Teil des Ursprungsquellcodes verwenden. Aus diesem Grund nennen die Kritiker der GPL sie eine "`virale"' Lizenz, was fälschlicherweise suggeriert, sie würde sich auf jede Software übertragen, auf die sie stößt.\footnote{In Wirklichkeit ist der Einfluss der GPL nicht ganz so groß: seinen Code nur mit einem GPL-lizenzierten Programm auf denselben Computer zu haben, heißt noch lange nicht, dass er ebenfalls den Bedingungen der GPL unterliegt. "`Etwas mit einem Virus zu vergleichen, ist sehr harsch"', sagt Stallman. "`Ein Vergleich mit einer Grünlilie wäre angemessener; sie wächst an einem anderen Ort weiter, wenn man einen Ableger pflanzt."' \cite[Vgl.][]{gpl}.}

In einer Informationswirtschaft, die immer stärker von Softwarestandards abhängig wird, ist die GPL das sprichwörtliche "`schwere Geschütz"'. Selbst Unternehmen, die sie früher als Auswuchs des "`Software-Kommunismus"' verspottet haben, haben nun ihre Vorzüge erkannt. Linux, der 1991 vom finnischen Studenten Linus Torvalds entwickelte Kernel, ist unter der GPL lizenziert, und auch die meisten Teile des GNU-Systems: GNU Emacs, der GNU Debugger, der GNU C Compiler etc. Zusammen bilden diese Werkzeuge die Komponenten des freien Betriebssystems GNU/Linux, das von der weltweiten Hackergemeinschaft entwickelt und gehegt wird und ihr gehört. Statt die Gemeinschaft als eine Bedrohung anzusehen, verlassen sich mittlerweile Hightech-Unternehmen wie IBM, Hewlett Packard und Oracle auf sie und verkaufen Anwendungen und Dienstleistungen, die auf der stetig wachsenden Infrastruktur der freien Software aufbauen.\footnote{Obwohl diese Anwendungen unter GNU/Linux laufen, folgt daraus nicht, dass sie selbst freie Software sind. Im Gegenteil: die meisten dieser Anwendungen sind proprietäre Software und respektieren Ihre Freiheit genauso wenig wie Windows. Sie tragen vielleicht zum Erfolg von GNU/Linux bei, aber nicht zum Ziel der Freiheit, wegen der es überhaupt erst existiert.}

Es ist auch dazu gekommen, dass sie sich auf freie Software als strategische Waffe im jahrelangen Krieg der Hackergemeinde gegen Microsoft verlassen, jene in Redmond, Washington, ansässige Firma, die den PC-Software-Markt seit den späten 80ern dominiert. Als Besitzer des populären Betriebssystems Windows hätte Microsoft bei einem industrieweiten Wechsel zur GPL das meiste zu verlieren. Jedes Programm im Windows-Koloss ist durch Copyright und Verträge geschützt (End User License Agreements, oder EULAs), die den proprietären Status der ausführbaren Datei und den Quellcode schützen, wobei man an letzteren als Nutzer ohnehin nicht gelangen kann. Code in eines dieser Programme zu übernehmen, der durch die "`virale"' GPL geschützt ist, ist verboten; um mit den Forderungen der GPL in Einklang zu stehen, wäre es für Microsoft rechtlich erforderlich, das ganze Programm zur freien Software machen. Konkurrenten könnten es dann kopieren, modifizieren und verbesserte Versionen davon verkaufen, und Microsofts Bindung an den Benutzer stören.

Daher die wachsende Sorge um die Entscheidung vieler Entwickler für die GPL. Daher auch die jüngste Rede Mundies\index{Mundie, Craig|(}, die die GPL und den "`Open-Source"'-Ansatz in der Softwareentwicklung und im Verkauf angreift. (Microsoft erkennt nicht einmal den Begriff der "`freien Software"' an und bevorzugt es, seine Angriffe auf das unpolitische "`Open-Source"'-Lager zu richten (wie in \autoref{open_source} beschrieben) und nicht auf die Free-Software-Bewegung.) Und daher Stallmans Entscheidung, heute hier am selben Campus eine öffentliche Gegenrede zu halten.

20 Jahre sind in der Softwareindustrie eine Ewigkeit. Wenn man nur einmal bedenkt, dass 1980, als Richard Stallman den Xerox-Laserdrucker im AI~Lab verfluchte, Microsoft, das heute die weltweite Softwareindustrie dominiert, damals noch ein junges Unternehmen in Privatbesitz war. IBM, die Firma, die zu jener Zeit als die stärkste Kraft in der Hardwareindustrie angesehen wurde, hatte seinen ersten PC noch nicht eingeführt, der den Startschuss für den heutigen Billigsektor im PC-Markt gab. Viele Technologien, die wir heute als selbstverständlich hinnehmen – das World Wide Web, Satellitenfernsehen, 32-Bit-Spielekonsolen – gab es damals nicht. Dasselbe gilt für viele Firmen, die heutzutage die höheren Ränge in der Unternehmenslandschaft einnehmen, Firmen wie AOL, Oracle\comment{Sun Microsystems}, Amazon.com, Compaq und Dell. Die Liste lässt sich beliebig fortsetzen.

Unter denen, die den Fortschritt als wichtiger als die Freiheit erachten, wird der Umstand, dass der Hightech-Markt in so kurzer Zeit so weit vorangeschritten ist, für und gegen die GNU GPL ausgelegt. Einige argumentieren für die GPL und führen die kurze Lebenszeit der meisten Hardwareplattformen an. Wegen des Risikos, ein veraltetes Produkt zu besitzen, entschieden sich Verbraucher oft für Firmen mit der höchsten Langlebigkeit. Infolge dessen sei der Softwaremarkt zu einem Alles-oder-nichts-Spiel geworden.\footcite[Vgl.][]{mssrc}
Das Umfeld der proprietären Software, sagen sie, führe zu Monopolmissbräuchen und Stagnation. Starke Unternehmen entzögen dem Markt allen Sauerstoff und erstickten so Konkurrenten und innovative Jungunternehmen. 

Andere argumentieren das genaue Gegenteil. Der Absatz von Software sei genauso riskant, wenn nicht riskanter, wie der Kauf von Software. Ohne die Rechtssicherheit, die die proprietären Softwarelizenzen garantieren, ganz zu schweigen von den wirtschaftlichen Aussichten auf eine eigene "`killer app"' (d.\,h. einer bahnbrechenden Technologie, die einen ganz neuen Markt eröffnet),\footnote{Killer apps müssen nicht proprietär sein. Trotzdem glaube ich, der Leser versteht, was gemeint ist: der Softwaremarkt ist wie eine Lotterie. Je größer die potentiellen Gewinne, desto mehr Leute werden daran teilnehmen wollen. Eine gute Zusammenfassung des Killer-app-Phänomens bietet \cite{killer}.}
verliere der Markt für Firmen den Anreiz. Dann würde der Markt stagnieren und Innovationen seltener werden. Wie Mundie selbst in seiner Ansprache vom 3. Mai auf demselben Campus sagt, stellt die "`virale"' Natur "`eine Gefahr"' für jedes Unternehmen dar, das sich auf die Einzigartigkeit ihrer Software als Wettbewerbsvorteil verlässt. Mundie ergänzt:
\begin{quote}
Sie unterminiert auch den unabhängigen kommerziellen Softwaresektor auf fundamentale Weise, weil sie es praktisch unmöglich macht, Software auf eine Art zu verbreiten, in der die Empfänger für das Produkt bezahlen, anstatt nur für die Kosten der Vervielfältigung.\footcite[Vgl.][]{mundie}
\end{quote}

Der gleichzeitige Erfolg von GNU/Linux und Windows in den letzten 10 Jahren deutet darauf hin, dass beide Seiten manchmal richtig liegen. Jedoch denken Free-Software-Aktivisten wie Stallman, man müsse auf einer Seite stehen. Die wirkliche Frage sei nicht, ob freie oder proprietäre Software erfolgreicher ist, es gehe darum, was ethischer ist. Ungeachtet dessen ist der Kampf um Anhängerschaft in der Softwareindustrie wichtig. Selbst einflussreiche Softwarehersteller wie Microsoft sind abhängig von der Unterstützung durch andere Entwickler, deren Werkzeuge, Programme und Computerspiele die zugrundeliegende Softwareplattform wie Windows attraktiver für den durchschnittlichen Verbraucher machen. Er spielt auf die rasche Entwicklung im Technologiemarkt der letzten 20 Jahre an, ganz zu schweigen von der eindrucksvollen Erfolgsbilanz seines eigenen Unternehmens während dieser Zeit, und rät den Zuhörern, sich nicht zu sehr von der in letzter Zeit entwickelten Dynamik der Free-Software-Bewegung mitreißen zu lassen:
\begin{quote}
Die Erfahrungen zweier Jahrzehnte haben gezeigt, dass das wirtschaftliche Modell, das das geistige Eigentum schützt, und ein Geschäftsmodell, das Forschungs- und Entwicklungskosten wieder einbringt, beeindruckende wirtschaftliche Vorteile bringen kann und diese weit streuen kann.\footcite[][]{mundie}\index{Mundie, Craig|)}
\end{quote}

Solche Mahnungen dienen Stallman heute als Hintergrund für seine Rede. Weniger als ein Monat, nachdem sie geäußert wurden, steht Stallman mit seinem Rücken zu einer der Tafeln im Rednerbereich des Saals, bereit anzufangen.

Wenn in den letzten zwei Jahrzehnten dramatische Änderungen in der Softwareindustrie geschehen sind, so hat sich Stallman noch drastischer verändert. Der dürre, glattrasierte Hacker von damals, der seine Tage damit verbracht hat, mit seiner geliebten PDP-10 Zwiesprache zu halten, ist nicht mehr. An seiner Stelle steht der untersetzte Mann mittleren Alters mit den langen Haaren und dem Rabbi-Bart, ein Mann, der nun den Großteil seiner Zeit mit dem Schreiben und Beantworten von E-Mails verbringt, seinen Programmiererkollegen predigt und wie heute Reden hält. Mit seinem aquamarinfarbenen T-Shirt und braunen Polyesterhosen sieht Stallman aus wie ein Einsiedler aus der Wüste, der gerade aus der Kleiderkammer der Heilsarmee kommt.


Die Hörerschaft besteht zu großen Teilen aus Besuchern, die Stallmans Geschmack in Sachen Kleidung und Haartracht teilen. Viele kommen mit Notebooks unter dem Arm und Mobilfunkmodems, um Stallmans Worte besser aufzeichnen und ins Internet übertragen zu können. Das Geschlechterverhältnis ist etwa 15 Männer auf eine Frau und eine von den 7 oder 8 Frauen kommt mit einem Plüschtierpinguin in den Saal, dem offiziellen Linux-Maskottchen, eine andere kommt mit einem Teddybären.

Angespannt verlässt Stallman seinen Posten im vorderen Teil des Saals und nimmt Platz in einem Stuhl in der ersten Reihe, um Befehle in den schon offenen Laptop zu tippen. In den nächsten 10 Minuten bemerkt Stallman die wachsende Anzahl an Studenten, Professoren und Fans nicht, die vor ihm im Rednerbereich umherlaufen.

Bevor die Rede beginnt, muss den altertümlichen Ritualen der Akademiker gehuldigt werden. Stallmans Anwesenheit verdient nicht nur eine, sondern zwei Ankündigungen\comment{XXX intruduction: Einleitung/Vorstellung/Ankündigung??}. Mike Uretsky, Codirector des Center for Advanced Technology der Stern School, hält die erste. 

"`Die Rolle einer Universität ist die Förderung von Debatten und interessanten Diskussionen"', sagt Uretsky. "`Diese besondere Vorlesung, dieses Seminar fällt genau in diese Kategorie. Ich finde die Diskussion um Open Source besonders interessant."' 

Bevor Uretsky zum nächsten Satz anheben kann, steht Stallman auf und winkt ihn beiseite wie ein Motorradfahrer mit einer Panne. "`Ich mache freie Software"', sagt Stallman unter anschwellendem Gelächter. "`Open Source ist eine andere Bewegung."'

Das Gelächter weicht dem Applaus. Der Saal ist voller Stallman-Partisanen, Menschen, die um seinen Ruf für verbale Akkuratesse wissen, ganz zu schweigen von der breit veröffentlichten Fehde mit den Open-Source-Befürwortern im Jahre 1998. Die meisten sind in der Erwartung solcher Ausbrüche hergekommen, so wie einst Radiohörer jede Sendung auf Jack Bennys legendären Spruch "`Now cut that out!"' gewartet haben.

Uretsky beendet hastig seine Vorrede und gibt die Bühne frei für Edmond Schonberg, Professor an der Informatikfakultät der NYU. Als Programmierer und Mitarbeiter am GNU-Projekt weiß Schonberg die linguistischen Landminen zu vermeiden. Er fasst Stallmans Werdegang geschickt aus der Perspektive eines modernen Programmierers zusammen.

"`Richard ist das perfekte Beispiel von jemandem, der aus seinem lokalen Handeln heraus angefangen hat, über die globale Probleme der Unzugänglichkeit von Sourcecode nachzudenken"', sagt Schonberg. "`Er hat eine einheitliche Philosophie entwickelt, die uns alle unsere Vorstellungen darüber überdenken lässt, wie Software hergestellt wird, was geistiges Eigentum bedeutet und was die Softwaregemeinschaft wirklich bedeutet."'\footnote{Wenn das heute jemand so sagte, würde Stallman gegen den Begriff "`geistiges Eigentum"' protestieren, da er einseitig und verwirrend ist. \cite[Vgl.][]{ipr}}

Schonberg heißt Stallman unter noch mehr Applaus willkommen. Stallman braucht einen Moment, sein Notebook herunterzufahren, erhebt sich aus seinem Stuhl und tritt auf die Bühne. Zuerst wirkt seine Ansprache eher wie eine Comedyroutine aus dem Borscht Belt als eine politische Rede. "`Ich möchte mich bei Microsoft bedanken, dass sie mir die Möglichkeit eröffnet haben, hier heute stehen zu dürfen"', frotzelt Stallman. "`In den letzten Wochen habe ich mich wie ein Autor gefühlt, dessen Buch zufällig irgendwo verboten wurde."'

Für die Uneingeweihten sei hier erwähnt, dass Stallman zur Aufwärmung ein Gleichnis zu freier Software bringt. Er vergleicht ein Softwareprogramm mit einem Kochrezept. Beide bieten nützliche Schritt-für-Schritt-Anleitungen, wie man ein gewünschtes Ziel erreicht, und können einfach verändert werden, wenn es auf Seiten des  Nutzers besondere Wünsche oder Umstände gibt. "`Man muss das Rezept nicht haargenau befolgen"', sagt Stallman. "`Man kann einige Zutaten weglassen; Pilze hinzugeben, weil man Pilze mag; weniger Salz verwenden, weil sein Arzt einem dazu geraten hat – wie man will."'

Am wichtigsten sei es, so  Stallman, dass Software und Rezepte einfach weiterzugeben sind. Wenn man einem Gast das Rezept vom Abendessen gibt, verliert der Koch nicht mehr als etwas Zeit und die Kosten für das Papier, auf das das Rezept geschrieben wird. Bei Software ist der Verlust noch geringer, meist nur ein paar Mausklicks und ein bisschen Elektrizität. In beiden Fällen gewinnt der Herausgeber der Information jedoch an zwei Dingen: einer festeren Freundschaft und der Möglichkeit, im Gegenzug selbst interessante Rezepte zu erhalten.

"`Man stelle sich vor, wie es wäre, wenn Rezepte in einem schwarzen Kasten verschlossen wären"', ändert Stallman das Tempo. "`Man könnte nicht sehen, welche Zutaten sie verwenden, und man könnte sie schon gar nicht ändern, und wenn man dann eine Kopie für einen Freund machen würde. Sie würden einen Pirat nennen und für Jahre ins Gefängnis bringen wollen. Das würde einen gewaltigen Aufschrei unter den Leuten verursachen, die es gewohnt sind, Rezepte auszutauschen. Aber genau so ist es in der Welt der proprietären Software. Einer Welt, in der etwas Höflichkeit und Anstand anderen Leuten gegenüber verboten oder verhindert wird."'

Als er diese einleitende Analogie hinter sich hat, setzt Stallman an, die Geschichte vom Xerox-Laserdrucker wieder zu erzählen. Wie das Rezeptgleichnis ist die Geschichte vom Laserdrucker ein nützlicher rhetorischer Kunstgriff. Durch ihre parabelartige Struktur verdeutlicht sie, wie schnell sich die Dinge in der Softwarewelt ändern können. Sie versetzt die Hörer in eine Zeit zurück, bevor es Amazons 1-Click, Microsoft Windows und Oracle-Datenbanken gab, und fordert vom Hörer, die Vorstellung vom Eigentum an Software ohne die heutige Firmenhegemonie zu überdenken.

Stallman führt die Geschichte mit dem Schliff und der Geübtheit eines Staatsanwalts aus, der sein Schlussplädoyer hält. Wenn er zu der Stelle über den Professor der Carnegie Mellon kommt, der ihm keine Kopie des Druckerquellcodes geben will, hält Stallman inne.

"`Er hat uns verraten"', sagt Stallman. "`Aber nicht nur uns. Er hat wahrscheinlich auch dich verraten."'
Bei dem Wort "`dich"' deutet Stallman mit seinem Zeigefinger vorwurfsvoll auf ein ahnungsloses Mitglied der Zuhörerschaft. Die Augenbrauen des Auditoriumsmitglieds zucken leicht, aber Stallmans Blick ist schon wieder von ihm abgewandt. Bedächtlich wählt er sich einen zweiten Hörer aus, unter Gekicher aus der Menge. "`Und, ich glaube, wahrscheinlich auch dich"', sagt er, und zeigt mit dem Finger auf einen Zuhörer drei Reihen hinter dem ersten.

Bis Stallman den dritten Zuhörer ausgewählt hat, ist das Kichern zu allgemeinem Gelächter geworden. Die Geste scheint etwas aufgesetzt, und sie ist es auch. Doch wenn die Zeit kommt, die Geschichte vom Xerox-Laserdrucker abzuschließen, macht Stallman sie mit dramatischem Schwung. "`Er hat wahrscheinlich die meisten Leute heute hier in diesem Saal verraten – ausgenommen vielleicht einige, die 1980 noch nicht geboren waren"', sagt Stallman und erregt noch mehr Gelächter. "`Weil er versprochen hat, Zusammenarbeit mit so ziemlich der gesamten Bevölkerung des Planeten Erde zu verweigern."' Stallman lässt den letzten Kommentar eine halbe Sekunde sickern. "`Er hatte eine Geheimhaltungserklärung unterschrieben"', fügt Stallman hinzu.

Richard Matthew Stallmans Aufstieg vom frustrierten Akademiker zu der politischen Größe über die letzten 20 Jahre spricht für viele Dinge. Er spricht für Stallmans dickköpfige Natur und seinen erstaunlichen Willen. Er spricht für die klar geäußerte Vision und die Werte der Free-Software-Bewegung, die Stallman mit aufgebaut hat. Er spricht für die qualitativ hochwertigen Programme, die Stallman erschaffen hat; Programme, die Stallmans Ruf als lebende Programmiererlegende zementiert haben. Er spricht für die wachsende Stoßkraft hinter der GPL, einer juristischen Neuerung, die viele als Stallmans bedeutendste Leistung ansehen.

Und, was am bemerkenswertesten ist, spricht er für die sich wandelnde Art politischer Macht in einer Welt, die immer abhängiger wird von Informationstechnik und der Software, die sie steuert.
Vielleicht ist das der Grund, warum zu einer Zeit, in der die meisten Sterne am Hightech-Himmel verblassen, Stallmans Stern heller denn je strahlt. Seit er 1984 das GNU Project\footnote{Die Abkürzung GNU steht für "`GNU's not Unix"'. In einem anderen Teil der Rede vom 29. Mai 2001 an der NYU fasst Stallman ihren Ursprung zusammen:
\begin{quote}
Wir Hacker suchen immer nach einem lustigen oder dreckigen Namen für ein Programm, weil die Namensgebung der halbe Spaß am Programmieren ist. Wir hatten auch eine Tradition der rekursiven Akronyme, um auszudrücken, dass das Programm, was man schreibt, Ähnlichkeiten zu einem bestehenden Programm hat... Ich habe nach einem rekursiven Akronym für Something Is Not UNIX gesucht. Und habe alle 26 Buchstaben durchprobiert, aber nichts ergab ein richtiges Wort. Dann habe ich mich entschieden, eine Auslassung zu machen. So konnte ich ein Akronym aus drei Buchstaben bekommen, Something's Not UNIX. Und ich habe wieder die Buchstaben durchprobiert, und bin auf das Wort "`GNU"' gekommen. Das war's.
\end{quote}

Obwohl er selbst ein Freund von Wortwitzen ist, empfiehlt Stallman den Nutzern der Software, das "`G"' am Anfang des Akronyms auszusprechen (also "`G-nu"'). Nicht nur vermeidet man so Verwechselungen mit ,,Gnu'', der afrikanischen Antilope, sondern auch mit dem Adjektiv "`new"' [US-Aussprache]. "`Wir arbeiten schon seit 17 Jahren daran, so gneu ist es jetzt nicht mehr"', sagt Stallman.

Quellen: Aufzeichnungen des Autors und \cite[Online-Mitschrift von][]{rmsnyu} }
gestartet hat, wurde Stallman manchmal ignoriert, persifliert, geschmäht und von innerhalb und außerhalb der Free-Software-Bewegung angegriffen. Trotzdem hat es das GNU Project geschafft, seine Meilensteine zu erreichen, abgesehen von einigen berüchtigten Verzögerungen, und bleibt bedeutsam in einem Softwaremarkt, der heute Größenordnungen komplexer ist als vor 18 Jahren. Ebenso die Ideologie hinter der freien Software, einer Ideologie peinlichst gepflegt von Stallman selbst.

Um die Gründe hinter ihrer Verbreitung zu verstehen, hilft es, Richard Stallman in seinen eigenen Worten zu untersuchen und in denen, die andere über ihn verlieren, die mit ihm gearbeitet und gekämpft haben. Die Charakterskizze Richard Stallmans ist nicht kompliziert. Wenn jemand das alte Motto "`what you see is what you get"' verkörpert, dann Stallman.

"`Wenn Sie Richard Stallman, den Menschen, verstehen wollen, glaube ich, muss man wirklich alles in seiner Gesamtheit betrachten"', rät Eben Moglen\index{Moglen, Eben}, Rechtsbeistand der Free Software Foundation und Professor der Rechtswissenschaften an der Columbia University Law School. "`All diese persönlichen Eigenarten, die viele Leute als Hindernis sehen, ihn wirklich gut kennenzulernen, \glq sind\grq{} Stallman: Richards starker Sinn von Unzufriedenheit, sein enormer Sinn für ethisches Engagement, seine Unfähigkeit zu Kompromissen, besonders bei Fragen, die er als fundamental ansieht. Das sind alles exakt die Gründe, warum Richard genau dann die Dinge getan hat, die er getan hat."'

Eine Reise zu erklären, die mit einem Laserdrucker begann und schließlich zu einem Sparringskampf mit dem reichsten Unternehmen der Welt führen sollte, ist keine leichte Aufgabe. Es bedarf sorgfältiger Untersuchung der Einflüsse, die das Softwareeigentum in der heutigen Gesellschaft so bedeutsam haben werden lassen. Es bedarf außerdem einer sorgfältigen Untersuchung des Mannes, der, wie viele politischen Größen vor ihm, die Formbarkeit des menschlichen Gedächtnisses versteht.
Es bedarf der Fähigkeit, Mythen und politisch beladene Schlüsselwörter zu verstehen, die sich mit der Zeit um Stallman herum aufgebaut haben. Und schließlich bedarf es des Verständnisses von Stallmans Genie als Programmierer und seinen Erfolgen und Misserfolgen, sein Genie auf andere Bereiche auszudehnen.

Wenn es um eine Zusammenfassung seiner eigenen Reise geht, bekennt sich Stallman zu der von Moglen beobachteten Verschmelzung aus Persönlichkeit und Prinzipientreue. "`Starrsinn ist meine Stärke"', sagt er. "`Die meisten Leute, die versuchen, etwas sehr schwieriges zu erreichen, geben am Ende entmutigt auf. Ich habe nie aufgegeben."'
Außerdem schreibt er dem Zufall Schuld zu. Wäre es nicht zu dem Zusammenstoß wegen des Xerox-Laserdruckers gekommen, hätte es nicht die persönlichen und politischen Konflikte gegeben, die seine Laufbahn als MIT-Angestellter erledigt haben, hätte es nicht das halbe Dutzend anderer Faktoren gegeben, kann Stallman sich leicht auch einen anderen möglichen Verlauf seines Lebens mit einer anderen beruflichen Karriere vorstellen. Auch wenn das so ist, ist Stallman dankbar für die Einflüsse und Umstände, die ihn in seine heutige Position versetzt haben\comment{that put him in the position to make a difference}.