\chapter{Kurzreise in die Hackerhölle}

[RMS: In diesem Kapitel habe ich nur einige Anmerkungen vorgenommen.]

Richard Stallman starrt unerschrocken durch die Frontscheibe eines Leihwagens, warted darauf, dass die Ampel umschaltet, als wir uns den Weg durch die Innenstadt von Kihei bahnen.

Wir zwei sind auf dem Weg zur nahe gelegenen Stadt Pa'ia, wo wir ein Treffen mit ein paar Programmierern und deren Frauen in etwa einer Stunde vereinbart haben.

Zwei Stunden sind seit der Rede Stallmans am Maui High Performance Center vergangen und Kihei, eine Stadt, die vor der Rede so einladend ausgesehen hatte, erscheint nun völlig unkooperativ. Wie die meisten Strandstädte ist Kihei ein eindimensionales Beispiel von vorstädtischer Zersiedlung. Beim Durchfahren der Hauptstraße mit seiner endlosen Reihe an Burgerbuden, Immobilienbüros und Bikiniläden fällt es schwer, sich nicht wie ein stahlbeschichteter Bissen zu fühlen, der durch den Verdauungstrakt eines riesigen Kommerzbandwurms wandert. Das Gefühl wird verstärkt durch das Fehlen von Seitenstraßen. With nowhere to go but forward bewegt sich der Verkehr stoßartig wie eine Druckfeder. 200 Meter weiter vorne wird eine Ampel grün. Bis wir vorwärtskommen, hat die Ampel schon wieder auf Gelb geschaltet.

Für Stallman als lebenslanger Ostküstenbewohner reicht die Vorstellung, den Großteil eines sonnigen hawaiianischen Nachmittags in zähem Verkehr zu stecken, um ein Embolie auszulösen. [RMS: Weil ich gefahren bin, habe ich auch Zeit verloren, in der ich E-Mails beantworten hätte können, und das war der echte Krampf, weil ich ohnehin schon kaum hinterherkomme.] Even worse is the knowledge that, with just a few quick right turns a quarter mile back, this whole situation easily could have been avoided. Unfortunately, we are at the mercy of the driver ahead of us, a programmer from the lab who knows the way and who has decided to take us to Pa'ia via the scenic route instead of via the nearby Pilani Highway.

"`Das ist schrecklich"', sagt Stallman zwischen frustrierten Seufzern. "`Warum sind wir nicht die andere Strecke gefahren?"'

Die ein Viertelkilometer entfernte Ampel wird wieder grün. Wieder kriechen wir ein paar Autolängen vorwärts. Der Vorgang wiederholt sich noch weitere 10 Minuten, bis wir endlich eine große Kreuzung erreichen, die Auffahrt auf den nahe gelegenen Highway verspricht.

Der Fahrer vor uns ignoriert es und fährt geradeaus über die Kreuzung.

"`Warum biegt er nicht ab?"', jammert Stallman, wirft frustriert die Hände hoch. "`Ich kann's nicht glauben."'

Ich entscheide mich, nichts dazu zu sagen. Ich finde den Umstand, dass ich hier mit Stallman in einem Auto sitze und er fährt, dazu noch in Maui, schon unglaublich genug. Vor zwei Stunden hatte ich nicht einmal gewusst, dass Stallman fahren kann. Jetzt höre ich Yo-Yo Mas Cellomusik mit den klagenden Bassnoten von "`Appalachian Journey"' vom Autoradio und sehe den Sonnenuntergang zu meiner Linken an uns vorbeiziehen, und versuche, in das Polster zu versinken I do my best to fade into the upholstery.

Als die nächste Wendemöglichkeit kommt, setzt Stallman den rechten Blinker, um den Fahrer vor uns einen Hinweis zu geben. Ohne Glück. Wieder kriechen wir langsam über die Kreuzung und halten gute 200 Meter vor der nächsten Ampel. Jetzt ist Stallman wütend.

"`Als ob er uns absichtlich ignorieren würde"', sagt er gestikulierend und fuchtelnd wie ein Flaggenwinker auf einem Flugzeugträger, während er vergebens versucht, unseren Vorfahrer auf sich aufmerksam zu machen. Der Vorfahrer bleibt unbeeindruckt und in den nächsten fünf Minuten sehen wir nur einen kleinen Teil seines Kopfs in seinem Rückspiegel.

I look out Stallman's window. Nearby Kahoolawe and Lanai Islands provide an ideal frame for the setting sun. It's a breathtaking view, the kind that makes moments like this a bit more bearable if you're a Hawaiian native, I suppose. I try to direct Stallman's attention to it, but Stallman, by now obsessed by the inattentiveness of the driver ahead of us, blows me off.

When the driver passes through another green light, completely ignoring a ``Pilani Highway Next Right,'' I grit my teeth. I remember an early warning relayed to me by BSD programmer Keith Bostic. ``Stallman does not suffer fools gladly,'' Bostic warned me. ``If somebody says or does something stupid, he'll look them in the eye and say, \glq That's stupid.\grq ''

Looking at the oblivious driver ahead of us, I realize that it's the stupidity, not the inconvenience, that's killing Stallman right now.

"`Als ob er sich die Strecke ausgesucht hätte, völlig ohne einen Gedanken, wie wir effizient dahin kommen"', sagt Stallman.

The word ``efficiently'' hangs in the air like a bad odor. Few things irritate the hacker mind more than inefficiency. It was the inefficiency of checking the Xerox laser printer two or three times a day that triggered Stallman's initial inquiry into the printer source code. It was the inefficiency of rewriting software tools hijacked by commercial software vendors that led Stallman to battle Symbolics and to launch the GNU Project. If, as Jean Paul Sartre once opined, hell is other people, hacker hell is duplicating other people's stupid mistakes, and it's no exaggeration to say that Stallman's entire life has been an attempt to save mankind from these fiery depths.

This hell metaphor becomes all the more apparent as we take in the slowly passing scenery. With its multitude of shops, parking lots, and poorly timed street lights, Kihei seems less like a city and more like a poorly designed software program writ large. Instead of rerouting traffic and distributing vehicles through side streets and expressways, city planners have elected to run everything through a single main drag. From a hacker perspective, sitting in a car amidst all this mess is like listening to a CD rendition of nails on a chalkboard at full volume.

``Imperfect systems infuriate hackers,'' observes Steven Levy, another warning I should have listened to before climbing into the car with Stallman. ``This is one reason why hackers generally hate driving cars -- the system of randomly programmed red lights and oddly laid out one-way streets causes delays which are so goddamn \textit{unnecessary} [Levy's emphasis] that the impulse is to rearrange signs, open up traffic-light control boxes\ldots redesign the entire system.''\footnote{See Steven Levy, \textit{Hackers} (Penguin USA [paperback], 1984): p. 40.}

More frustrating, however, is the duplicity of our trusted guide. Instead of searching out a clever shortcut -- as any true hacker would do on instinct -- the driver ahead of us has instead chosen to play along with the city planners' game. Like Virgil in Dante's \textit{Inferno}, our guide is determined to give us the full guided tour of this hacker hell whether we want it or not.

Before I can make this observation to Stallman, the driver finally hits his right turn signal. Stallman's hunched shoulders relax slightly, and for a moment the air of tension within the car dissipates. The tension comes back, however, as the driver in front of us slows down. ``Construction Ahead'' signs line both sides of the street, and even though the Pilani Highway lies less than a quarter mile off in the distance, the two-lane road between us and the highway is blocked by a dormant bulldozer and two large mounds of dirt.

It takes Stallman a few seconds to register what's going on as our guide begins executing a clumsy five-point U-turn in front of us. When he catches a glimpse of the bulldozer and the ``No Through Access'' signs just beyond, Stallman finally boils over.

``Why, why, why?'' he whines, throwing his head back. ``You should have known the road was blocked. You should have known this way wouldn't work. You did this deliberately.''  [RMS: I meant that he chose the slow road deliberately.  As explained below, I think these quotes are not exact.]

The driver finishes the turn and passes us on the way back toward the main drag. As he does so, he shakes his head and gives us an apologetic shrug. Coupled with a toothy grin, the driver's gesture reveals a touch of mainlander frustration but is tempered with a protective dose of islander fatalism. Coming through the sealed windows of our rental car, it spells out a succinct message: ``Hey, it's Maui; what are you gonna do?''

Stallman can take it no longer.

"`Lass dein scheiß Lachen!"', schreit er, die Scheibe beschlägt dabei. "`It's your fucking fault. This all could have been so much easier if we had just done it my way.'' [RMS: Diese Zitate scheinen ungenau zu sein, weil ich These quotes appear to be inaccurate, because I don't use ``fucking'' as an adverb.  This was not an interview, so Williams would not have had a tape recorder running.  I'm sure things happened overall as described, but these quotations probably reflect his understanding rather than my words.]

Stallman accents the words ``my way'' by gripping the steering wheel and pulling himself towards it twice. The image of Stallman's lurching frame is like that of a child throwing a temper tantrum in a car seat, an image further underlined by the tone of Stallman's voice. Halfway between anger and anguish, Stallman seems to be on the verge of tears.

Fortunately, the tears do not arrive. Like a summer cloudburst, the tantrum ends almost as soon as it begins. After a few whiny gasps, Stallman shifts the car into reverse and begins executing his own U-turn. By the time we are back on the main drag, his face is as impassive as it was when we left the hotel 30 minutes earlier.

It takes less than five minutes to reach the next cross-street. This one offers easy highway access, and within seconds, we are soon speeding off toward Pa'ia at a relaxing rate of speed. The sun that once loomed bright and yellow over Stallman's left shoulder is now burning a cool orange-red in our rearview mirror. It lends its color to the gauntlet wili wili trees flying past us on both sides of the highway.

For the next 20 minutes, the only sound in our vehicle, aside from the ambient hum of the car's engine and tires, is the sound of a cello and a violin trio playing the mournful strains of an Appalachian folk tune.
