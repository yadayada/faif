\chapter{Portrait des jungen Mannes}

Richard Stallmans Mutter, Alice Lippman, erinnert sich noch immer an die Momente, in denen ihr bewusst geworden ist, dass ihr Sohn ein besonderes Talent hat. "`Ich glaube, das war, als er 8 war"', sagt Lippman. Es war nachmittags an einem Wochenende im Jahr 1961 und Lippman, frisch geschiedene alleinerziehende Mutter, vertrieb sich die Zeit in der winzigen 1-Zimmer-Wohnung der Familie in Manhattans Upper West Side. Beim Blättern durch eine Ausgabe des Scientific American stieß Lippman auf ihren Lieblingsteil, einer Kolumne von Martin Gardner, "`Mathematical Games"'. Lippman, damals Vertretungslehrerin für Kunsterziehung, mochte Gardners Kolumne wegen der Denksportaufgaben. Ihr Sohn hatte es sich mit einem Buch auf dem Sofa gemütlich gemacht und sie entschied sich, sich am Rätsel der Woche zu versuchen.

"`Ich war nicht so gut beim Lösen der Rätsel"', gibt sie zu. "`Aber als Künstler fand ich, dass sie mir beim Durchbrechen von Konzeptbarrieren wirklich helfen."' Lippman sagt, bei ihrem Versuch, das Rätsel zu lösen, biss sie auf Granit. Als sie das Magazin weglegen wollte, bemerkte Lippman zu ihrem Erstaunen ein leichtes Ziehen an ihrem Ärmel.

"`Es war Richard."', entsinnt sie sich, "`Er wollte wissen, ob ich Hilfe brauche."' Sie schaute auf das Rätsel und dann auf ihren Sohn, und betrachtete das Angebot zuerst mit Skepsis. "`Ich fragte Richard, ob er das Magazin schon gelesen hätte"', sagt sie. "`Er sagte, ja, er hätte es gelesen, und er hätte das Rätsel schon gelöst. Und dann fing er auch schon an, mir zu erklären, wie man es löst."'

Als sie die Logik hinter dem Ansatz ihres Sohnes hörte, schlug Lippmans Skepsis in Staunen um. "`Ich meine, ich wusste immer schon, dass er ein intelligenter Junge ist"', sagt sie, "`aber das war das erste Mal, dass ich bemerkt habe, wie weit er eigentlich war."'
Dreißig Jahre danach unterbricht Lippman ihre Erzählung mit einem Lachen. "`Um ehrlich zu sein, ich glaube, ich habe nie verstanden, wie man das Rätsel löst"', sagt sie. "`Ich erinnere mich nur, dass ich verblüfft war, dass er die Antwort wusste."'

Sie sitzt am Esszimmertisch ihrer zweiten Wohnung in Manhattan – im selben geräumigen Drei-Zimmer-Komplex, in den sie nach der Heirat 1967 mit dem nun verstorbenen Maurice Lippman mit ihrem Sohn gezogen ist. Alice Lippman verströmt eine Mischung aus Stolz und Nachdenklichkeit einer jüdischen Mutter, wenn sie sich die jungen Jahre ihres Sohns ins Gedächtnis zurückruft. Auf der nahe stehenden Anrichte des Esszimmers steht ein Foto im 20er-Format vom finster blickenden, vollbärtigen Stallman im Talar eines Doktors. Das Bild stellt die anderen Photos von Lippmans Nichten und Neffen in den Schatten, aber bevor sich ein Besucher etwas daraus machen kann, wiegelt Lippman die herausstechende Position mit einer Witzelei ab.

"`Richard bestand darauf, dass ich das Bild bekomme, nachdem er seine Ehrendoktorwürde an der University of Glasgow bekommen hatte"', sagt Lippman. "`Er sagte zu mir: \glq Stell dir vor, das ist die erste Abschlussfeier, auf der ich jemals war.\grq "'\footnote{Eine der Hauptquellen für dieses Kapitel war das Interview von 
\cite[][]{rmshsm}, ein Interview zum dem Buch \citefield{shorttitle}{talkgen}, einer Sammlung von Interviews mit bedeutenden Persönlichkeiten aus der sogenannten "`Baby-Boom"'-Generation. Obwohl Stallman es nicht ins Buch geschafft hat, veröffentlichte Gross das Interview als Online-Beilage auf der Webseite des Buchs.}
Solche Kommentare bringen den Humor zum Ausdruck, den man entwickelt, wenn man ein Wunderkind großzieht. Für jede Geschichte, die Lippman über die Starrköpfigkeit und das unübliche Verhalten ihres Sohnes hört und liest, kann sie mit mindestens einem Dutzend kontern.

"`Er war früher so konservativ"', sagt sie und wirft die Hände hoch. "`Wir hatten früher die schlimmsten Streits genau hier an diesem Tisch. Ich war unter den ersten Lehrern der staatlichen Schulen, die für eine Gewerkschaft gestreikt haben und Richard war mir sehr böse. Er sagte, Gewerkschaften seien korrupt. Er war auch ein sehr starker Gegner der Wohlfahrt. Er dachte, die Leute könnten sehr viel mehr Geld verdienen, wenn sie es selbst investieren. Wer hätte gedacht, dass er in den nächsten 10 Jahren so idealistisch werden würde? Ich kann mich noch erinnern, dass seine Stiefschwester einmal zu mir gekommen ist und gesagt hat \glq Was soll aus dem mal werden, wenn er groß ist? Ein Faschist?{}\grq\,"'\footnote{RMS: Ich kann mich nicht erinnern, ihr das erzählt zu haben. Ich kann nur sagen, dass ich diesen Ansichten heute absolut nicht zustimme. Als Jugendlicher hatte ich kein Mitgefühl für die Probleme, die die meisten Leute in ihrem Leben haben; ich hatte andere Probleme. Ich habe nicht erkannt, dass die Reichen die meisten Leute in die Armut zwingen, wenn wir uns nicht auf allen Ebenen organisieren und sie davon abhalten. Ich habe nicht verstanden, wie schwer es für die meisten Leute ist, dem gesellschaftlichen Druck zu widerstehen, dumme Dinge zu tun, zum Beispiel sein ganzes Geld auszugeben, statt zu sparen, weil ich diesen Druck selbst kaum gespürt habe. Außerdem waren Gewerkschaften in den 60er, als sie sehr einflussreich waren, manchmal arrogant oder korrupt. Aber heutzutage sind sie viel schwächer, und als Resultat daraus profitieren vom Wirtschaftswachstum, wenn es ihn gibt, hauptsächlich die Reichen.}

Lippman als ein fast 10 Jahre alleinerziehender Elternteil – sie und Richards Vater, Daniel Stallman, hatten 1948 geheiratet, sich 1958 geschieden und das Sorgerecht an ihrem Sohn geteilt – kann die Abneigung ihres Sohns gegen Autorität bestätigen. Ebenso kann sie den Wissensdrang ihres Sohns bestätigen. Als  diese zwei Naturgewalten aufeinandertrafen, sagt Lippman, hatten sie und ihr Sohn die größten Auseinandersetzungen. 

"`Es war fast so, als ob er nie essen wollte"', erinnert sich Lippman an das Verhaltensmuster, das mit circa 8 eingesetzt hatte und bis zu seinem High-School-Abschluss 1970 nicht aufhörte. "`Ich rief ihn immer zum Abendessen, aber er hörte mich nie. Ich musste ihn neun- oder zehnmal rufen, bis er es mitbekommen hat. Er war völlig vertieft [in seine Bücher]."'

Stallman seinerseits erinnert sich ähnlich an diese Dinge, aber mit einer politischen Facette. "`Mir machte Lesen Spaß"', sagt er. "`Wenn ich lesen wollte und meine Mutter mir sagte, ich soll in die Küche kommen oder ins Bett gehen, dann habe ich nicht darauf gehört. Ich sah keinen Grund, warum ich nicht lesen sollte. Keinen Grund, warum sie mir zu sagen hatte, was ich tun soll. Punkt. Im Grunde habe ich das, wovon ich gelesen habe, Ideen wie Demokratie und Freiheit des Einzelnen, auf mich angewandt. Ich sah keinen Grund, Kinder von diesen Prinzipien auszuschließen."'

Der Glaube, dass die Freiheit des Einzelnen mehr wiegt als wahllose Autorität, erstreckte sich auch auf die Schule. Im Alter von 11 war er seinen Klassenkameraden um 2 Jahre voraus, und Stallman durchlebte die Frustration eines begabten Schülers an einer öffentlichen Schule. Nicht lange nach der Begebenheit mit dem Rätsel wurde seine Mutter zu einem Elterngespräch eingeladen. Viele weitere sollten noch folgen.

"`Er weigerte sich absolut, Hausarbeiten zu schreiben"', berichtet Lippman von der ersten Aussprache. "`Ich glaube, die letzte Hausarbeit, die er geschrieben hat, war vor seinem Abschlussjahr an der High School. Es war ein Aufsatz über die Geschichte des westlichen Zahlensystems\comment{for a fourth-grade teacher}."' Ein bestimmtes Thema wählen zu müssen, wenn es doch nichts gab, worüber er schreiben wollte, war für Stallman fast unmöglich und so qualvoll, dass er solche Situationen um jeden Preis vermeiden wollte. Mit seiner Begabung im analytischem Bereich war Stallman Mathe und den naturwissenschaftlichen Fächern zugeneigt, zu Lasten der anderen Fächer.

Was einige Lehrer als Zielstrebigkeit ansahen, hielt Lippman für Ungeduld. In Mathe und den Naturwissenschaften gab es einfach so viel zu lernen, besonders im Vergleich zu Fächern und Aktivitäten, für die ihrem Sohn die Veranlagung fehlte. Im Alter von 10 oder 11, als die Jungen anfingen, regelmäßig Touch-Football zu spielen, erinnert sie sich, dass ihr Sohn einmal aufgeregt nach Hause gekommen ist. "`Er wollte so gern mitspielen, aber er hatte einfach nicht das Koordinationsgefühl"', erinnert sich Lippman. "`Das hat ihn sehr geärgert."'

Der Ärger trieb ihn noch mehr zur Mathematik und den Naturwissenschaften. Selbst im Reich der Wissenschaften konnte die Ungeduld ihres Sohns problematisch sein. Als jemand, der im Alter von 7 über Analysis-Büchern hockte, sah Stallman wenig Anlass, das Niveau seiner Unterhaltungen für Erwachsene nach unten anzupassen. Einmal engagierte Lippman in seinen Mittelstufenjahren einen Studenten von der nahe gelegenen Columbia University, um für ihn den großen Bruder zu spielen. Der Student verließ nach dem ersten Treffen die Wohnung und kam nie wieder. "`Ich glaube, die Sachen, über die Richard redete, waren ihm zu hoch"', vermutet Lippman.
Ein anderer mütterlicher Lieblingsmoment geht zurück auf Anfang der 60er. \comment{,kurz nach der Sache mit dem Rätsel}
Etwa im Alter von sieben, zwei Jahre nach der Scheidung und dem Umzug von Queens, fing Richard an, als Hobby Modellraketen am nahe gelegenen Riverside Drive Park zu starten. Was als harmloser Spaß begann, wurde schnell ernst, als ihr Sohn begann, Daten über die Raketenstarts zu sammeln. Wie dem Interesse an mathematischen Spielen schenkte sie der Aktivität nicht viel Aufmerksamkeit, bis sie kurz vor einem wichtigen NASA-Start in sein Zimmer kam und fragte, ob er ihn mit anschauen wolle.
"`Er hat geschäumt"', sagt Lippman. "`Alles, was er noch sagen konnte, war \glq Aber ich hab' doch noch nichts publiziert.\grq ~ Anscheinend hatte er irgendwas, das er wirklich der NASA zeigen wollte."' Stallman erinnert sich nicht an den Vorfall, hält es aber für wahrscheinlicher, dass er verärgert war, weil er nichts vorzuzeigen hatte.

Solche Anekdoten liefern frühes Zeugnis von der Verbissenheit, die später zu seinem Markenzeichen werden sollte. Wenn andere Kinder zu Tisch kamen, blieb Stallman in seinem Zimmer und las. Andere Kinder wollten berühmte Footballspieler\comment{Johnny Unitas} sein, Stallman Raumfahrer. "`Ich war sonderlich"', fasst Stallman knapp seine frühen Jahre in einem Interview von 1999 zusammenfassen. "`Ab einem bestimmten Alter hatte ich nur noch Lehrer als Freunde."'\footcite{rmshsm} Stallman schämt sich seiner verschrobenen Eigenarten nicht, anders als die soziale Unbeholfenheit, die er als Schwäche ansieht. Beides zusammen trug zu seiner sozialen Ausgeschlossenheit bei.
Obwohl es wohl mehr Ärger mit der Schule bedeutete, entschied Lippman sich, die Hobbies ihres Sohns zu unterstützen. Im Alter von 12 nahm Richard im Sommer an Science Camps teil und besuchte in der Schulzeit eine Privatschule. Als ein Lehrer ihren Sohn für das Columbia Science Honors Program vorschlug, ein Programm für begabte Mittel- und Oberstufenschüler in New York City, hatte Stallman eine außerschulische Aktivität mehr und würde von da an samstäglich zum Campus der Columbia University pendeln.

Dan Chess, ein Klassenkamerad im Columbia Science Honors Program, erinnert sich, dass Richard Stallman selbst unter den Schülern, die sich ebenso für Mathe und Naturwissenschaften interessierten, etwas sonderbar erschien. "`Wir waren alle Geeks und Nerds, aber er war ungewöhnlich wenig sozialfähig\comment{schlecht angepasst?}"', erinnert sich Chess, heutiger Mathematikprofessor am Hunter College. "`Außerdem war er sauklug. Ich habe eine Menge kluge Leute kennengelernt, aber ich glaube, er ist der klügste Mensch, den ich jemals gekannt habe."'

%intensiv/ernst
Seth Breidbart\index{Breidbart, Seth|(}, ebenfalls Abgänger des Columbia Science Honors Program, liefert unterstützende Aussagen. Der Programmierer, der mit Stallman wegen der gemeinsamen Vorliebe für Science Fiction und SciFi-Konferenzen in Verbindung geblieben ist, erinnert sich an den 15 Jahre alten, Kurzhaarfrisur tragenden Stallman als "`unheimlich"', besonders für einen Gleichaltrigen.
"`Es ist schwer zu beschreiben"', sagt Breidbart\index{Breidbart, Seth|)}. "`Er war nicht unnahbar. Er war nur sehr ernst. [Er war] sehr gebildet, aber in einigen Dingen auch sehr stur."'

Diese Beschreibungen geben Anlass zur Spekulation:\comment{are judgment-laden adjectives like "`intense"' and "`hardheaded"' simply a way to describe traits that today might be categorized under juvenile behavioral disorder?}
 waren seine Eigenarten Ausdruck davon, was man heute bei Kindern "`Störung des Sozialverhaltens"' nennt? Ein Artikel vom Dezember 2001 namens \citefield{title}{geeksyndr} im \textit{Wired}-Magazine zeichnet ein Bild von wissenschaftlich begabten Kindern, die mit hochfunktionalem Autismus oder Asperger-Syndrom diagnostiziert wurden. In vielerlei Hinsicht sind die Erinnerungen der Eltern in dem Artikel verblüffend ähnlich zu denen Lippmans. Stallman spekuliert auch über dieses Thema. In einem Interview im Jahre 2000 für ein Profil über ihn im \textit{Toronto Star} sagt Stallman, er frage sich, ob er "`an der Grenze zum Autismus"' sei. In dem Artikel wird diese Spekulation als Fakt dargestellt.\footnote{\cite[Vgl.][]{frprophet}
\quote{Seine Vision von freier Software und sozialer Zusammenarbeit stehen im krassen Widerspruch zu seiner isolierten Natur. [Er ist] ein Exzentriker wie Glenn Gould, der kanadische Pianist, der ähnlich brillant, wortgewandt und einsam war. Stallman sieht sich selbst zu einem gewissen Grad von Autismus geplagt: einem Leiden, das es ihm schwierig macht, mit Menschen umzugehen.}}

Die Spekulationen profitieren natürlich von der losen Art, wie heutzutage Verhaltensstörungen definiert werden. Wie Steve Silberman, Autor des Artikels \citefield{title}{geeksyndr}, vermerkt, haben amerikanische Psychiater erst in letzter Zeit den Begriff "`Asperger-Syndrom"' als gültigen Oberbegriff für ein breites Spektrum an Verhaltenszügen akzeptiert. Die Merkmale umfassen schlechte motorische Fähigkeiten, geringe Sozialisation und hohe Intelligenz und eine fast zwanghafte Affinität für Zahlen, Computer und Ordnungssysteme.\footcite[Vgl.][]{geeksyndr}
"`Es ist gut möglich, dass ich so etwas ähnliches habe"', sagt Stallman. "`Andererseits ist ein Anzeichen des Syndroms die Schwierigkeit, Rhythmen zu folgen. Ich kann tanzen. Ich liebe es sogar, kompliziertesten Rhythmen zu folgen. Es ist nicht klar genug definiert, um sicher sein zu können."' Eine andere Möglichkeit ist, dass Stallman ein "`Schattensyndrom"' hat, welches sich ähnlich ausdrückt wie das Asperger-Syndrom, aber sich in den Grenzen der Normalität hält.\footcite[Vgl.][]{shadow}
Chess persönlich lehnt solche Rückdiagnosen ab. "`Ich habe nie daran gedacht, dass er so etwas haben könnte"', sagt er. "`Er war einfach sehr wenig sozial, aber das waren wir alle."'

Lippman andererseits hält es für möglich. Sie erinnert sich an einige Geschichten aus der frühen Kindheit ihres Sohns\comment{, die Anlass zur Spekulation geben}. Ein markantes Symptom von Autismus ist die Überempfindlichkeit gegenüber Geräuschen und Farben, und Lippman erzählt zwei Anekdoten, die in dieser Hinsicht herausstehen. "`Als Richard ein Säugling war, nahmen wir ihn mit zum Strand"', sagt sie. "`Er fing schon zwei, drei Blocks von der Brandung entfernt an zu schreien. Erst beim dritten Mal haben wir herausgefunden, was los war: der Klang der Brandung tat ihm in den Ohren weh."' Sie entsinnt sich auch an eine ähnliche Reaktion in Bezug auf Farbe: "`Meine Mutter hatte leuchtend rotes Haar, und jedes Mal, wenn sie sich herunterbeugte, um ihn auf den Arm zu nehmen, ließ er einen Schrei los."'

In den letzten Jahren hat Lippman angefangen, Bücher über Autismus zu lesen und glaubt, dass diese Vorfälle kein Zufall waren. "`Ich finde, dass Richard einige Merkmale von einem autistischen Kind hatte"', sagt sie. "`Es ist bedauerlich, dass damals so wenig über Autismus bekannt war."'
Mit der Zeit lernte ihr Sohn, sich anzupassen, so Lippmann. Im Alter von sieben stand ihr Sohn gerne hinter der Frontscheibe von U-Bahnen und kartographierte und prägte sich das labyrinthartige Schienensystem unter der Stadt ein. Es war ein Hobby, bei dem es erforderlich war, sich an die lauten Geräusche bei der Zugfahrt zu gewöhnen. "`Nur der Lärm am Anfang schien ihn zu stören"', sagt Lippman. "`Es war so, als ob er von dem Geräusch geschockt war, aber seine Nerven haben gelernt, sich daran anzupassen."'

Größtenteils erinnert sich Lippman aber, dass ihr Sohn die Aufgeregtheit, Energie und sozialen Fähigkeiten wie jeder normaler Junge hatte. Erst nach einer Reihe traumatischer Ereignisse, die den Stallman-Haushalt getroffen haben, sagt sie, sei ihr Sohn introvertiert und emotional abweisend geworden.

Das erste traumatische Ereignis war die Scheidung von Alice und Daniel Stallman\comment{, Richard’s father}. Obwohl Lippman sagt, sie und ihr Ex-Mann hätten versucht, Richard auf den Schlag vorzubereiten, sei er trotzdem vernichtend für ihn gewesen. "`Er hat das erste Mal nicht so richtig zugehört, als wir es ihm beibringen wollten", sagt Lippman. "`Aber die Realität hat ihn schnell eingeholt, als wir in die neue Wohnung gezogen sind. Das erste, was er gesagt hat, war \glq Wo sind Papas Möbel?{}\grq\,"'

Das nächste Jahrzehnt sollte Stallman unter der Woche in der Wohnung seiner Mutter in Manhattan verbringen und seine Wochenenden im Haus seines Vaters in Queens. Bei dem Hin und Her hatte er Gelegenheit, zwei unterschiedliche Erziehungsstile zu studieren, was Stallman bis zum heutigen Tag fest überzeugt hat, selbst keine Kinder großzuziehen. 

Seinen Vater, Veteran aus dem 2.~Weltkrieg, der Anfang 2001 verstorben ist, sieht Stallman mit Respekt und Wut. Einerseits war er ein Mann, dessen moralische Verpflichtung ihn dazu brachte, Französisch zu lernen, nur damit er den Alliierten hilfreicher sein konnte, wenn sie schließlich gegen die Nazis in Frankreich kämpfen würden. 
Andererseits war er der Elternteil, der immer wusste, wie er jemanden auf grausame Weise fertigmachen konnte.\footnote{Leider konnte ich Daniel Stallman nicht für dieses Buch befragen. Während der frühen Recherchephase zu diesem Buch informierte mich Stallman über die Alzheimererkrankung seines Vaters. Als ich die Recherche Ende 2001 fortsetzte, musste ich leider hören, das Daniel Stallman in diesem Jahr gestorben war.}
"`Mein Vater war schrecklich cholerisch"', sagt Stallman. "`Er hat nie geschrieen, aber er fand immer einen Weg, einen auf eine kalte, vernichtende Weise zu kritisieren."'

Für das Leben in der Wohnung seiner Mutter findet Stallman deutlichere Worte. "`Das war der reinste Krieg"', sagt er. "`Ich sagte immer in meiner Verzweiflung \glq Ich will nach Hause\grq, ein nicht existenter Ort, den ich nie haben werde."'
In den ersten Jahren nach der Scheidung fand Stallman den Frieden, den er zu Hause nicht hatte, bei seinen Großeltern väterlicherseits. Ein Großelternteil starb, als er 8 war, der andere, als er 10 war. Für Stallman war der Verlust vernichtend. "`Ich ging früher dort hin und fühlte mich in der liebevollen, gütigen Umgebung geborgen"', erinnert er sich. "`Das war für mich der einzige Ort dieser Art, bis ich ans College ging."'
Lippman zählt den Tod Richards Großeltern als das zweite traumatische Ereignis. "`Es hat ihn wirklich mitgenommen"', sagt sie. Er stand seinen Großeltern sehr nahe. Bevor sie starben, war er sehr kontaktfreudig, fast eine Art Gruppenführer unter den anderen Kindern. Nach ihrem Tod wurde er zunehmend emotional distanziert.

Aus Stallmans Sicht war der emotionale Rückzug nur der Versuch, mit den Qualen der Adoleszenz fertigzuwerden. Er beschreibt seine Jahre als Teenager als "`der reine Horror"' und sagt, er fühlte sich oft wie ein Gehörloser in einer Gruppe von schwatzenden Musikhörern. "`Ich hatte oft das Gefühl, dass ich nicht verstehen konnte, was andere Leute sagen"', erinnert sich Stallman an sein Gefühl der Abgeschlossenheit. "`Ich konnte die Wörter verstehen, aber irgendwas ging in den Gesprächen auf einer unteren Ebene vor, dass ich nicht verstand. Ich konnte nicht verstehen, warum die Leute interessiert daran waren, was die anderen sagen."'
Bei all der Agonie, die er in seiner Adoleszenz durchlebte, hatte sie doch einen festigenden Effekt auf seinen Sinn für Individualität. Zu einer Zeit, als die meisten seiner Klassenkameraden ihr Haar lang wachsen ließen, bevorzugte Stallman eine Kurzhaarfrisur. Zu der Zeit, als alle Teenager Rock'n'Roll hörten, bevorzugte Stallman klassische Musik. Als treuer Fan von Science Fiction, Mad und Sendungen im Spätabendprogramm kam Stallman zu seiner einzigartig ungewöhnlichen Persönlichkeit, die bei seinen Eltern und Gleichaltrigen auf Unverständnis traf.

"`Die Scherze"', klagt Lippman, immer noch verzweifelt über die Erinnerungen an Stallman als Teenager. "`Es gab nichts, was man am Tisch sagen konnte, das nicht direkt als Scherz zurückkam."' Außerhalb seines Zuhauses hob sich Stallman seine Witze für die Erwachsenen auf, die seine Begabungen eher zu schätzen wussten. Einer davon war der Ferienlagerbetreuer, der dem Acht- oder Neuntklässler Stallman ein Handbuch für die IBM 7094 gab. Für einen Teenager mit einer Faszination für Zahlen und Naturwissenschaften war das ein Geschenk des Himmels.\footnote{Stallman als Atheist würde diese Beschreibung wohl missfallen.\comment{Es reicht wohl, zu sagen, dass es etwas ist, dass Stallman mit Freude angenommen hat.} \citet{rmshsm}: "`Als ich das erste Mal von Computern gehört hatte, wollte ich einen sehen und damit rumspielen."'} Bald würde Stallman Programme auf Papier für die 7094 entwerfen. Es gab keinen Computer, auf dem er sie laufen hätte lassen können, und er hatte keine echte Verwendung für einen Computer, aber er wollte trotzdem ein Programm schreiben –  egal, was für ein Programm. Er bat den Betreuer um beliebige Aufgaben zum Programmieren.

Der erste PC lag noch ein Jahrzehnt in der Zukunft, und Stallman musste einige Jahre warten, bis er erstmals Zugang zu einem Computer bekommen sollte. Die Gelegenheit ergab sich endlich im Abschlussjahr der High School. Das IBM New York Scientific Center\index{IBM New York Scientific Center}, eine nun geschlossene Forschungseinrichtung im innerstädtischen Manhattan, gab Stallman die Möglichkeit, seine ersten richtigen Programme auszuprobieren. Er wollte einen Präprozessor für die Programmiersprache PL/I\index{PL/I} schreiben, der die Konvention für Tensoralgebra zur Sprache hinzufügt. "`Ich habe ihn erst in PL/I geschrieben und dann alles in Assembly umgeschrieben, weil das kompilierte PL/I-Programm zu groß für den Computer war."'
Das New York Scientific Center stellte ihn für den Sommer nach seinem High-School-Abschluss an. Er sollte ein Programm für numerische Analysis in Fortran schreiben und war nach ein paar Wochen damit fertig. Dabei war ihm die Fortran so zuwider geworden, dass er sich schwor, nie wieder etwas in dieser Sprache zu schreiben. Den Rest des Sommers verbrachte er damit, einen Texteditor in APL zu schreiben.

Gleichzeitig hatte Stallman eine Position als Laborassistent in der Biologie-Fakultät an der Rockefeller University inne. Obwohl er sich auf eine Karriere als Mathematiker oder Physiker vorbereitete, beeindruckte Stallmans analytischer Verstand den Laborleiter so sehr, dass er einige Jahre nach Stallmans Abgang von der High-School unerwartet bei seiner Mutter anrief. "`Es war der Professor von der Rockefeller-Uni"', sagt Lippman. "`Er wollte wissen, was Richard macht. Er war überrascht, als er gehört hat, dass er jetzt mit Computern arbeitet. Er dachte immer, Richard hätte eine große Zukunft als Biologe vor sich."'
Stallmans analytische Fähigkeiten beeindruckten auch die Dozenten der Columbia University, auch wenn er ihren Zorn auf sich zog. "`Typischerweise hatte [Stallman] ein oder zweimal in der Stunde einen Fehler in der Vorlesung gefunden"', sagt Breidbart\index{Breidbart, Seth}. "`Und er hat sich nicht zurückgehalten, es die Professoren sofort wissen zu lassen. Das hat ihm viel Respekt eingebracht, aber wenig Popularität."'

Breidbarts Anekdote zu hören, ruft ein gequältes Lächeln bei Stallman hervor. "`Ich hab mich manchmal ein bisschen wie ein Arsch verhalten"', gibt er zu. "`Aber ich habe unter den Lehrern welche gefunden, die ähnlich dachten wie ich, weil sie auch gerne lernen. Kinder lernen meistens nicht gern. Jedenfalls nicht auf dieselbe Art."'
Dennoch ermutigte das samstägliche Abhängen mit den begabteren Kindern Stallman dazu, mehr über die Vorteile des Sozialisierens nachzudenken. Die Hochschulzeit kam immer näher und Stallman hatte wie viele andere im Columbia Science Honors Program die Liste seiner favorisierten Hochschulen auf zwei heruntergebracht: Harvard und MIT. Als sie von dem Wunsch ihres Sohns hörte, an eine Eliteuniversität zu gehen, wurde Lippman beunruhigt.
In der elften Klasse hatte Stallman immer noch Krach mit seinen Lehrern und Schulleitern. Im Jahr zuvor hatte er glatte Einsen in Amerikanischer Geschichte, Chemie, Französisch und Algebra, aber eine fette Fünf in Englisch, Resultat seines andauernden Boykotts, Hausarbeiten zu schreiben. Solche Fehlschläge würden beim MIT vielleicht ein wissendes Schmunzeln hervorrufen, aber für Harvard waren sie ein Warnsignal.

Als er in der elften Klasse war, sagt Lippman, habe sie einen Termin mit einem Therapeuten vereinbart. Der Therapeut sah Anlass zur Besorgnis wegen Stallmans Weigerung, Hausarbeiten zu schreiben und seinen Problemen mit den Lehrern. Ihr Sohn hatte gewiss geistig das Zeug für ein erfolgreiches Harvard-Studium, aber hatte er auch die Geduld, an Kursen teilzunehmen, in denen man eine Semesterarbeit schreiben muss? Der Therapeut schlug einen Probelauf vor. Wenn Stallman es schaffen sollte, ein ganzes Jahr an einer staatlichen Schule in New York zu überstehen, einschließlich dem Englisch-Unterricht, für welchen eine Abschlussarbeit vorgeschrieben war, würde er es wahrscheinlich auch an der Harvard University schaffen. Nach dem Ende der 11.\,Klasse meldete sich Stallman bei einer staatlichen Sommerschule in der Innenstadt an, und holte seine geisteswissenschaftlichen Pflichtkurse nach, die vernachlässigt hatte\comment{had shunned earlier in his high-school career}.

Im Herbst war Stallman wieder unter der Alltagsbevölkerung der New Yorker Highschool-Schüler an der Louis D. Brandeis High School in der West 84th Street. Es war nicht leicht, die Kurse über sich ergehen zu lassen, die im Vergleich zum Science-Programm an der Columbia University wie Förderunterricht erschienen, und Lippman erinnert sich stolz daran, dass ihr Sohn sich unterordnen konnte\comment{toe the line}.
"`Er war gezwungen, zu einem gewissen Grad zu buckeln, und er hat es geschafft"', sagt Lippman. "`Ich wurde nur einmal herzitiert, was an sich schon eine kleine Sensation war. Es war der Analysis-Lehrer, der sich beschwerte, dass Richard den Unterricht stört. Ich fragte, inwiefern er störe. Er meinte, Richard würde den Lehrer immer beschuldigen, falsche Beweise zu führen. Ich fragte \glq Und, hat er recht?{}\grq{} Der Lehrer sagte, \glq Ja, aber das kann ich der Klasse doch nicht erzählen. Die würden das nicht verstehen.\grq\,"'

Zum Ende seines ersten Semesters an der Brandeis High ging alles glatt. 96 Punkte in Englisch machten das Stigma der 60 Punkte zwei Jahre vorher wieder wett. Zusätzlich unterstützte Stallman das mit Bestnoten in Amerikanischer Geschichte, Höherer Analysis und Mikrobiologie. Der krönende Abschluss waren 100 Punkte in Physik.
Er schloss nach 10 Monaten auf Brandeis als viertbester seines 789 Schüler starken Jahrgangs ab, immer noch als gesellschaftlicher Außenseiter. Außerhalb der Schule ging Stallman seinen Studien mit noch mehr Eifer nach; unter der Woche eilte er zur Rockefeller University, um seine Pflichten als Laborassistent zu erfüllen und auf dem Weg zum Samstagsunterricht an der Columbia University wich er den Vietnam-Protesten aus. Dort saßen einmal die restlichen Schüler vom Science Honors Program vor dem Unterricht und besprachen ihre Studienortwahlen, und Stallman stimmte schließlich an der Diskussion ein.

Breidbart\index{Breidbart, Seth} erzählt, "`Die meisten Schüler gingen natürlich nach Harvard und ans MIT, aber einige gingen an andere Eliteuniversitäten. Das Gespräch ging durch den Raum und es fiel auf, dass Richard noch nichts gesagt hatte. Und irgendeiner brachte den Mut auf, ihn zu fragen, was er geplant hatte."'
Dreißig Jahre später erinnert sich Breidbart noch deutlich an den Moment. Als Stallman die Nachricht verkündete, dass er auch nach Harvard gehen würde, überkam ein betretenes Schweigen die Schüler. Und fast wie auf Zeichen zog er langsam die Mundwinkel nach oben zu einem selbstzufriedenen Lächeln. Breidbart sagt, "`Es war seine Art, zu sagen \glq Richtig. Ihr seid mich immer noch nicht los.\grq\,"'