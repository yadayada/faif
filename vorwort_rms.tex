\addchap{Vorwort von Richard M. Stallman}


Ich habe versucht, in dieser Ausgabe mein Wissen mit den Gesprächen mit Williams und der Sichtweise von außen zu vereinen. Der geneigte Leser muss entscheiden, inwieweit mir das gelungen ist.

Ich habe den veröffentlichten Text der englischen Ausgabe erstmals 2009 gelesen, als ich um Mithilfe bei der französischen Übersetzung von \textit{Free as in Freedom} gebeten wurde. Es waren mehr als nur kleine Änderungen nötig. Viele Fakten mussten korrigiert werden, aber auch grundlegende Änderungen waren erforderlich. Williams als Nichtprogrammierer verwischte fundamentale technische und rechtliche Unterschiede wie das Verändern des Codes eines bestehenden Programms und das Implementieren einiger Ideen des Programms in einem anderen. So steht beispielsweise in der ersten Ausgabe, Gosmacs und GNU Emacs wären beide Modifikationen des ursprünglichen PDP-10-Emacs gewesen, was sie aber nicht sind. Weiterhin ist Linux fälschlicherweise als "`Version von Minix"' bezeichnet worden. SCO sollte später dieselbe Behauptung aufstellen, in ihrem berüchtigten Gerichtsverfahren gegen IBM; und Torvalds und Tanenbaum haben sie gemeinsam widerlegt. 

Die erste Ausgabe hat viele Ereignisse übertrieben dramatisch dargestellt, indem sie mit scheinbaren Emotionen verbunden wurden. 

Zum Beispiel stand geschrieben, ich habe 1992 "`Linux fast ignoriert"' und dann 1993 mit der Entscheidung, Debian GNU/Linux zu unterstützen, "`eine dramatische Kehrtwendung"' gemacht. Mein Interesse 1993 und mein Desinteresse 1992 waren Ausdruck derselben pragmatischen Vorgehensweise, die das Ziel verfolgte, das GNU-System zu vervollständigen. 
Der Start des GNU-Hurd-Kernels 1990 war ebensoein pragmatischer Schritt in diese Richtung.

Aus all diesen Gründen sind viele Aussagen in der Originalausgabe irrig oder stehen im falschen Zusammenhang. Es war notwendig, sie zu korrigieren, aber nicht direkt durch eine vollständige Überarbeitung, die aus anderen Gründen nicht wünschenswert war. Mir wurde vorgeschlagen, ausführliche Kommentare zu machen, aber in den meisten Kapiteln war das Ausmaß der Änderungen zu groß, um sie als Anmerkungen auszuführen. Einige Fehler waren zu tiefgehend oder verbreitet, um sie durch eine bloße Anmerkung richtigzustellen. Ergänzende Kommentare oder Fußnoten im Rest des Textes hätten das Textbild belastet und den Text schwer lesbar gemacht; Fußnoten wären von einigen Lesern übersprungen worden.
 
Ich habe deshalb die Korrekturen direkt im Text vorgenommen. Jedoch habe ich nicht alle Fakten und Zitate überprüft, die außerhalb meiner Expertise liegen; die meisten von ihnen habe ich auf Williams' Verantwortung so belassen. Williams’ Version enthielt viele Zitate, die kritisch meiner Person gegenüber sind. Diese habe ich alle nicht angetastet, nur bei Gelegenheit Gegendarstellungen angefügt. Ich habe keine Zitate entfernt, außer einige in Kapitel 11 über Open Source, die nichts mit meinem Leben oder Schaffen zu tun haben. Ebenso habe ich die meisten von Williams’ eigenen Interpretationen, die mich kritisieren, erhalten (und manche kommentiert), wenn sie keine falschen Auffassungen von Fakten oder über Technik enthielten. Aber ich habe die unzutreffenden Behauptungen zu meiner Arbeit und meinen Gedanken und Gefühlen frei berichtigt. Ich habe seine persönlichen Eindrücke beibehalten, wenn sie als solche dargestellt waren, und das "`ich"' im Text dieser Ausgabe bezieht sich auch immer auf Williams, ausgenommen bei den Anmerkungen, die mit "`RMS:"' eingeleitet werden.

In dieser Ausgabe ist das komplette System, das GNU und Linux umfasst, immer "`GNU/Linux"', und "`Linux"' allein bezieht sich auf Torvalds’ Kernel, ausgenommen in Zitaten, wo diese Verwendung des Begriffs mit "`[\textit{sic}]"' markiert wird. Siehe \url{http://www.gnu.org/gnu/gnu-linux-faq.html} für eine Erklärung, warum es falsch und ungerecht ist, das ganze System "`Linux"' zu nennen. 

Außerdem möchte ich John Sullivan für seine vielen nützlichen kritischen Gedanken und seine Vorschläge danken.
