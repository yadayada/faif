\chapter{Anhang C – GNU Free Documentation License (Übersetzung)}

\phantomsection

\begin{center}

[In der Übersetzung sind Anmerkungen wie diese in eckigen Klammern angegeben. Bitte beachten Sie Abschnitt~8 für die Gültigkeit dieser Übersetzung.]
\bigskip

Version 1.3, 3. November 2008

Copyright \copyright{} 2000, 2001, 2002, 2007, 2008  Free Software Foundation, Inc.
2011, Theo Walm
 
\bigskip
\url{http://fsf.org/} 
\bigskip

Es ist jedem gestattet, wortgleiche Kopien dieses Lizenzdokuments
zu kopieren und zu verbreiten, aber Änderungen
sind nicht erlaubt.

\bigskip

{\bf\large Präambel}
\end{center}

Der Zweck dieser Lizenz ist es, Handbücher, Lehrbücher oder andere
funktionale und nützliche Dokumente "`frei"' im Sinne von "`Freiheit"' zu machen:
jedem die effektive Freiheit zuzusichern, es zu kopieren und zu verbreiten,
egal ob bearbeitet oder nicht, gewerblich oder nichtgewerblich.
Weiterhin soll diese Lizenz für den Autor und Verleger die Möglichkeit erhalten,
Anerkennung für ihre Arbeit zu bekommen, ohne für Bearbeitungen zur Verantwortung 
gezogen werden zu können, die von anderen gemacht wurden.

Diese Lizenz ist eine Art "`Copyleft"', was bedeutet, dass abgeleitete Werke
des Dokuments selbst im selben Sinne frei sein müssen. Sie ist eine Ergänzung 
zur GNU General Public License, die eine für freie Software entworfene
Copyleft-Lizenz ist.

Wir haben diese Lizenz entworfen, um sie für Handbücher für freie Software
einzusetzen, weil freie Software freie Dokumentation braucht: ein freies
Programm sollte mit einem Handbuch geliefert werden, das dieselben Freiheiten bietet 
wie die Software. Aber diese Lizenz ist nicht auf Softwarehandbücher beschränkt;
sie kann für jedes textuelles Werk verwandt werden, unabhängig von dem Inhalt
oder ob es in Druckform erschienen ist. Wir empfehlen diese Lizenz
hauptsächlich für Werke, deren Hauptzweck es ist, eine Anleitung oder ein 
Nachschlagewerk zu bieten.

\section*{1. SCHUTZGEGENSTAND UND DEFINITIONEN}

Schutzgegenstand dieser Lizenz ist jedes Handbuch oder andere Werk in jedwedem Medium, das
einen Vermerk vom Copyrightinhaber enthält, der besagt, dass es unter den
Bedingungen dieser Lizenz verbreitet werden kann. So ein Vermerk gewährt eine
weltweite, gebührenfreie, zeitlich unbeschränkte Lizenz, das Werk unter den hier
genannten Bedingungen zu nutzen. Das "`\textbf{Dokument}"' bezeichnet im folgenden
jedes solche Handbuch oder Werk. Jedes Mitglied der Allgemeinheit ist ein Lizenznehmer
und wird mit "`\textbf{Sie}"' angesprochen. Sie akzeptieren die Lizenz, wenn
Sie das Werk kopieren, bearbeiten, oder auf eine Art weiterverbreiten, die unter
dem Copyrightgesetz eine Einwilligung erfordert.

Eine "`\textbf{bearbeitete Version}"' des Dokuments ist jedes Werk, dass das Dokument 
ganz oder teilweise enthält, entweder wortgleich kopiert oder mit 
Veränderungen und/oder in eine andere Sprache übersetzt.

Ein "`\textbf{sekundärer Abschnitt}"' ist ein benannter Anhang oder ein Abschnitt im Vorspann
des Dokuments, der sich ausschließlich mit dem Bezug des 
Verlegers oder Autors des Dokuments zum Gesamtthema des Dokuments befasst
(oder verwandten Themen) und nichts enthält, was direkt zum Gesamtthema
zählen könnte. (Wenn das Dokument also zum Teil ein Mathematik-Lehrbuch
ist, darf ein sekundärer Abschnitt keine mathematischen Sachverhalte erklären.)
Der Bezug kann eine Sache von geschichtlicher Verbindung zu dem Thema oder
verwandten Themen sein oder eine rechtliche, geschäftliche, philosophische, 
ethische oder politische Haltung zu ihnen.

"`\textbf{Unveränderliche Abschnitte}"' sind gewisse sekundäre Abschnitte, deren Titel
im Vermerk, der besagt, dass dieses Dokument unter dieser Lizenz veröffentlicht wird,
als unveränderliche Abschnitte bestimmt werden. Wenn ein Abschnitt nicht die obige
Definition als sekundär erfüllt, dann darf er nicht als unveränderlich bestimmt werden.
Das Dokument kann auch null unveränderliche Abschnitte enthalten.
Wenn das Dokument keine unveränderlichen Abschnitte benennt, dann gibt es auch keine.

Die "`\textbf{Einbandtexte}"' sind bestimmte kurze Textabschnitte, die im
Vermerk, der besagt, dass das Dokument unter dieser Lizenz veröffentlicht wird,
als Vorderdeckeltext und Rückdeckeltext bezeichnet werden. Ein Vorderdeckeltext
darf höchstens 5 Wörter umfassen, und ein Rückdeckeltext höchstens 25 Wörter.

Eine "`\textbf{transparente}"' Kopie des Dokuments ist eine maschinenlesbare Kopie,
vertreten in einem Format, dessen Spezifikation der allgemeinen Öffentlichkeit 
zugänglich ist, das zur unmittelbaren Überarbeitung mittels eines generischen 
Texteditors oder (für Rasterbilder) mittels generischen Bildbearbeitungsprogramms 
oder für (Vektorbilder) mittels eines weit verbreiteten Vektorzeichenprogramms geeignet
ist, und das als Eingabe für Textformatierungsprogramme geeignet ist
oder für eine automatische Umwandlung in eine Vielzahl von Formaten, die als Eingabe für 
Textformatierungsprogramme geeignet sind. Eine Kopie, die in einem sonst transparenten Dateiformat vorliegt, deren
Auszeichnungen [Mark-up] oder Fehlen von Auszeichnungen so gestaltet sind, um weitere Bearbeitung durch die
Leser zu vereiteln oder sie davon abzubringen, ist nicht transparent.
Ein Bildformat ist nicht transparent, wenn es für einen wesentlichen Teil des Texts
verwandt wird. Eine Kopie, die nicht "`transparent"' ist, nennt sich "`\textbf{opak}"'.

Beispiele für geeignete Formate für transparente Kopien sind einfaches
ASCII ohne Auszeichnungen, Texinfo-Eingabeformat, LaTeX-Eingabeformat, SGML
oder XML mit einer öffentlich verfügbaren DTD [Dokumenttypdefinition] und standardkonformes einfaches
HTML, PostScript oder PDF, die für menschliche Bearbeitung ausgelegt sind. Beispiele für
transparente Bildformate sind PNG, XCF und JPG. Opake Formate umfassen
proprietäre Formate, die nur mit einem proprietärem Textverarbeitungsprogramm gelesen und bearbeitet werden
können, SGML oder XML, für die die DTD und/oder die Bearbeitungswerkzeuge nicht allgemein zugänglich sind, und
von einem Textverarbeitungsprogramm nur zu Ausgabezwecken computererzeugtes HTML, PostScript oder PDF.

Die "`\textbf{Titelseite}"' bedeutet bei einem gedruckten Buch die Titelseite selbst,
dazu folgende Seiten, die benötigt werden, um das Material, das laut dieser Lizenz
auf der Titelseite erscheinen muss, lesbar aufzunehmen. Für Werke in Formaten,
die keine Titelseite als solche haben, bedeutet "`Titelseite"' den Text
nahe dem auffälligsten Auftreten des Werktitels vor dem Anfang des Textkörpers.

Der "`\textbf{Verleger}"' ist eine natürliche oder juristische Person, die Kopien
des Dokuments an die Öffentlichkeit verteilt.

Ein Abschnitt "`\textbf{namens XYZ}"' bezeichnet ein benanntes Segment des Dokuments, dessen
Titel entweder genau XYZ ist oder XYZ in Klammern enthält, worauf eine Übersetzung von
XYZ in eine andere Sprache folgt. (Hier steht XYZ für einen bestimmten
Abschnitttitel, der im weiteren genannt wird, z.\,B. "`\textbf{Acknowledgements}"',
"`\textbf{Dedications}"', "`\textbf{Endorsements}"' oder "`\textbf{History}"'.)  
Bei einem solchen Abschnitt den "`\textbf{Titel zu erhalten}"', bedeutet,
dass wenn Sie das Dokument bearbeiten, dieser ein Abschnitt "`namens XYZ"' 
laut dieser Definition bleibt.

Das Dokument darf Haftungsausschlussklauseln nach dem Vermerk enthalten, der 
besagt, dass das Dokument unter dieser Lizenz geschützt ist. Diese Haftungsausschlüsse
sind als referentiell in dieser Lizenz enthalten anzusehen, aber nur was
den Ausschluss von Haftungen betrifft: alle weiteren etwaigen Auswirkungen, die
diese Haftungsausschlüsse haben, sind nichtig und haben keinen Einfluss auf die
Bedeutung dieser Lizenz.

\section*{2. WORTGLEICHES KOPIEREN}

Sie dürfen dieses Dokument auf jedem/s Medium kopieren und verbreiten, gewerblich
oder nichtgewerblich, vorausgesetzt, dass diese Lizenz, die Copyright-Vermerke
und der Lizenzvermerk, der besagt, dass diese Lizenz das Werk schützt, in allen
Kopien wiedergegeben werden, und dass Sie keine irgendwie gestalteten Bedingungen 
zusätzlich zu denen in dieser Lizenz stellen. Sie dürfen keine technischen
Maßnahmen einsetzen, die das Lesen oder Weiterkopieren der Kopien, die Sie machen oder
verteilen, verhindern oder einschränken. Sie dürfen jedoch ein Entgelt für Kopien
verlangen. Wenn Sie eine ausreichend große Menge an Kopien verbreiten,
müssen Sie außerdem die Bedingungen des Abschnitts~3 erfüllen.

Sie dürfen auch Kopien unter den obenstehenden Bedingungen verleihen, und Sie dürfen
öffentlich Kopien ausstellen.
%you may publicly display copies


\section*{3. MASSENWEISES KOPIEREN}

Wenn Sie gedruckte Kopien (oder Kopien in Medien, die für gewöhnlich gedruckte 
Einbände/Umschläge/Cover haben [im folgenden wird vom Medium Buch ausgegangen]) des Dokuments veröffentlichen, in einer Menge von mehr als 100, und der
Lizenzvermerk des Dokuments Titeltexte fordert, dann müssen die Kopien in ihren Einbänden 
klar und deutlich lesbar folgende Einbandtexte enthalten: Vorderdeckeltexte auf
dem Vorderdeckel und Rückdeckeltexte auf dem Rückdeckel. Beide Einbandseiten müssen
Sie klar und deutlich lesbar als den Verleger dieser Kopien ausmachen. Der Vorderdeckeldext 
muss den vollen Titel mit allen Wörtern des Titels gleich auffallend und sichtbar darstellen.
Sie dürfen dem Einband weiteres Material hinzufügen.
Kopieren mit Änderungen, die sich auf den Einband beschränken, solange sie den Titel
des Dokuments erhalten und diese Bedingungen erfüllen, kann in den anderen
Belangen als wortgleiches Kopieren angesehen werden.

Wenn die für je eine der Einbandseiten vorgesehenen Texte zu umfangreich sind, um
leserlich aufgebracht werden zu können, sollten Sie die ersten aufgelisteten 
(so viele, wie sinnvoll) auf dem eigentlichem Einband unterbringen und
den Rest auf den angrenzenden Seiten.

Wenn Sie opake Kopien des Dokuments in einer Stückzahl größer 100 verkaufen oder verbreiten,
müssen Sie entweder jeder opaken Kopie eine maschinenlesbare transparente Kopie beifügen
oder in beziehungsweise zu jeder opaken Kopie einen Netzwerkort nennen, von dem die 
netzwerknutzende Allgemeinheit mittels eines öffentlich standardisierten Netzwerkprotokolls 
Zugriff zum Herunterladen einer vollständigen transparenten Kopie des Dokuments,
frei von Zusatzmaterial, hat.
Wenn Sie letztere Möglichkeit nutzen, müssen Sie sinnvolle umsichtige Schritte
unternehmen, um zu gewährleisten, dass diese transparente Kopie unter dem
angegebenen Ort mindestens bis ein Jahr nach der letztmaligen Verbreitung 
einer opaken Kopie (direkt oder durch Ihre Handelsverterter oder Einzelhändler)
dieser Ausgabe an die Allgemeinheit zugänglich bleibt.

Es ist erwünscht, aber nicht unerlässlich, dass Sie die Autoren des Dokuments reichlich
im voraus kontaktieren, bevor Sie Kopien des Dokument in großer Menge weiterverbreiten,
um ihnen die Gelegenheit zu geben, Ihnen eine aktualisierte Version des Dokuments bereitzustellen.

\section*{4. BEARBEITUNGEN}

Sie dürfen eine bearbeitete Version des Dokuments unter den Bedingungen der 
Abschnitte 2 und 3 kopieren und verbreiten, vorausgesetzt, Sie veröffentlichen
die bearbeitete Version unter genau dieser Lizenz, wobei die bearbeitete Version
die Rolle des Dokuments einnimmt und Sie so jedem, der eine Kopie besitzt, die 
Lizenz zur Verbreitung und Bearbeitung erteilen.
Zusätzlich müssen Sie folgende Dinge in der bearbeiteten Version befolgen:

\begin{itemize}
\item[A.] 
   Verwenden Sie auf der Titelseite (und auf dem Einband, falls vorhanden) einen vom Dokument
   und von vergangenen Versionen verschiedenen Titel (diese sollten, falls es welche gibt,
   im History-Abschnitt des Dokuments aufgeführt sein). Sie dürfen denselben Titel einer vergangenen Version
   verwenden, wenn der ursprüngliche Verleger dieser Version Ihnen die Erlaubnis dazu erteilt.
   
\item[B.]
   Führen Sie auf der Titelseite als Autoren eine oder mehrere natürliche oder juristische Personen auf,
   die für die Urheberschaft der Bearbeitungen in der bearbeiteten Version verantwortlich sind,
   zusätzlich mindestens fünf der Hauptautoren des Dokuments (alle Hauptautoren, wenn es weniger als fünf
   gibt), es sei denn, sie entheben Sie von dieser Bedingung.
   
\item[C.]
   Nennen Sie auf der Titelseite den Namen des Verlegers der bearbeiteten Version
   als den Verleger.
   
\item[D.]
   Erhalten Sie alle Copyright-Vermerke des Dokuments.
   
\item[E.]
   Fügen Sie einen angemessenen Copyright-Vermerk nahe den anderen Copyright-Vermerken
   für Ihre Bearbeitungen hinzu.
   
\item[F.]
   Fügen Sie direkt nach dem Copyright-Vermerk einen Lizenzvermerk bei, 
   der der Allgemeinheit die Erlaubnis erteilt, die bearbeitete Version unter den 
   Bedingungen dieser Lizenz zu nutzen, und zwar in der Form, wie im Addendum beschrieben.
   
\item[G.]
   Erhalten Sie im Lizenzvermerk die vollständige Liste der unveränderlichen
   Abschnitte und der geforderten Einbandtexte, die im Lizenzvermerk des Dokuments angegeben werden.
   
\item[H.]
   Fügen Sie eine unveränderte Kopie dieser Lizenz bei.
   
\item[I.]
   Erhalten Sie den Abschnitt namens "`History"', erhalten Sie seinen Titel und fügen Sie
   einen Eintrag hinzu, der mindestens den Titel, das Jahr, die neuen Autoren und den Verleger
   der bearbeiteten Version nennt, wie auf der Titelseite angegeben. Wenn
   es keinen Abschnitt namens "`History"' in dem Dokument gibt, erstellen Sie einen, in dem
   der Titel, das Jahr, die Autoren und den Verleger des Dokuments nennt, wie auf der Titelseite
   angegeben, und fügen Sie dann einen Eintrag für die bearbeitete Version, wie oben beschrieben, hinzu.
   
\item[J.]
   Erhalten Sie, falls im Dokument angegeben, den Netzwerkort für den öffentlichen Zugang
   zur transparenten Kopie des Dokuments und ebenso die Netzwerkorte, die im Dokument für 
   vergangene Versionen angegeben sind, auf denen es basiert.
   Diese dürfen in den Abschnitt "`History"' verschoben werden.
   Sie dürfen den Netzwerkort für ein Werk auslassen, das mindestens vier Jahre vor dem
   Dokument selbst veröffentlicht wurde, oder falls der darin angegebene ursprüngliche
   Verleger Ihnen dazu Erlaubnis erteilt.
   
\item[K.]
   Erhalten Sie etwaige Abschnitte namens "`Acknowledgements"' oder "`Dedications"'.
   Erhalten Sie für jeden solchen Abschnitt den Titel und erhalten Sie in dem Abschnitt den Gehalt und
   Ton aller in ihm enthaltenen Danksagungen an Mitwirkende und/oder Widmungen.
   
\item[L.]
   Erhalten Sie alle unveränderlichen Abschnitte des Dokuments,
   ungeändert in ihrem Text und ihren Titeln. Abschnittsnummern
   oder gleichartiges werden nicht als Teil des Abschnittstitels angesehen.
   
\item[M.]
   Entfernen Sie etwaige Abschnitte mit dem Titel "`Endorsements"'. Solche Abschnitte
   dürfen nicht in der bearbeiteten Version enthalten sein.
   
\item[N.]
   Benennen Sie keinen bestehenden Abschnitt in "`Endorsements"' um
   oder in einen konfliktiven Titel eines unveränderlichen Abschnitts.
   
\item[O.]
   Bewahren Sie alle Haftungsauschlussklauseln.
\end{itemize}

Wenn die bearbeitete Version neue Vorspannabschnitte oder Anhänge enthält, 
die die Definition als sekundäre Abschnitte erfüllen und kein aus dem Dokument
kopiertes Material enthalten, dürfen Sie nach Wahl einige oder alle dieser
Abschnitte als unveränderlich erklären. Dazu fügen Sie ihre Titel zu der Liste
der unveränderlichen Abschnitte in dem Lizenzvermerk der bearbeiteten Version hinzu. 
Diese Titel müssen sich von den Titeln aller anderer Abschnitte unterscheiden.

Sie dürfen einen Abschnitt namens "`Endorsements"' hinzufügen, vorausgesetzt, er enthält
ausschließlich Empfehlungen Ihrer bearbeiteten Version von verschiedenen
Parteien - z.\,B. Aussagen aus einer Beurteilung [eines Ebenbürtigen] oder dass der Text
von einer Organisation als maßgebliche Definition eines Standards anerkannt wurde.

Sie dürfen einen Passus von bis zu fünf Wörtern Länge als Vorderdeckeltext und einen
Passus von bis zu 25 Wörtern Länge als Rückdeckeltext am Ende der Liste der Einbandtexte
in der bearbeiteten Version hinzufügen. Es darf nur je ein Passus als Vorderdeckeltext
und Rückdeckeltext von jeder rechtsfähigen Person (oder auf Veranlassung von ihr) hinzugefügt werden.
Wenn das Dokument schon einen Einbandtext für denselben Einband enthält, der früher von Ihnen
hinzugefügt wurde oder auf Veranlassung derselben Rechtsperson, in dessen Auftrag Sie handeln,
dann dürfen Sie keinen weiteren hinzufügen; aber Sie dürfen den alten ersetzen, wenn Ihnen
der vorige Verleger, der ihn hinzugefügt hat, ausdrücklich dazu die Genehmigung erteilt.

Der/die Autor/en und Verleger des Dokuments erklären mit dieser Lizenz kein Einverständnis
dazu, dass Ihre Namen zu Werbezwecken für oder zur Behauptung oder Andeutung einer Empfehlung von 
jedweder bearbeiteten Version genutzt werden.

\section*{5. DOKUMENTE ZUSAMMENSTELLEN}

Sie dürfen das Dokument unter den in Abschnitt~4 definierten Bedingungen 
mit anderen Dokumenten zusammenstellen, die unter dieser Lizenz veröffentlicht
werden, vorausgesetzt, dass Sie in der Zusammenstellung alle unveränderlichen 
Abschnitte aller Ursprungsdokumente in unveränderter Form einschließen und
sie als unveränderliche Abschnitte Ihres zusammengestellten Werks in seinem
Lizenzvermerk auflisten und dass Sie alle Haftungsausschlussklauseln erhalten.

Das zusammengestellte Werk muss nur eine Kopie dieser Lizenz enthalten und
mehrere identische unveränderliche Abschnitte dürfen durch eine einzige Kopie
ersetzt werden. Wenn es mehrere unveränderliche Abschnitte mit demselben Titel, aber
unterschiedlichem Inhalt gibt, machen Sie den Titel jeder solcher Abschnitte eindeutig,
indem Sie an seinem Ende den Namen des ursprünglichen Autors oder Verlegers dieses Abschnitts, 
falls bekannt, in Klammern hinzufügen, ansonsten eine eindeutige Nummer.
Nehmen Sie dieselbe Anpassung bei den Abschnitttiteln in der Liste der unveränderlichen Abschnitte
im Lizenzvermerk des zusammengestellten Werks vor.

In der Zusammenstellung müssen Sie etwaige Abschnitte namens "`History"' der verschiedenen
Ursprungsdokumente zu einem einzigen Abschnitt namens "`History"' zusammenfügen;
ebenso müssen etwaige Abschnitte namens "`Acknowledgements"' und etwaige Abschnitte namens
"`Dedications"' zusammengefügt werden. Sie müssen alle Abschnitte namens 
"`Endorsements"' entfernen.

\section*{6. SAMMLUNGEN VON DOKUMENTEN}

Sie dürfen eine Sammlung, bestehend aus dem Dokument und anderen Dokumenten, die unter dieser Lizenz
veröffentlicht wurden, anfertigen und die einzelnen Kopien dieser Lizenz in den
verschiedenen Dokumenten mit einer einzigen Kopie ersetzen, die in der Sammlung
enthalten ist, solange Sie die Grundsätze für das wortgleiche Kopieren für jedes einzelne Dokument
in allen anderen Belangen befolgen.

Sie dürfen ein einzelnes Dokument aus so einer Sammlung entnehmen und es 
getrennt unter dieser Lizenz verbreiten, vorausgesetzt, Sie fügen dem entnommenen
Dokument eine Kopie dieser Lizenz bei und beachten diese Lizenz in allen anderen
Belangen bezüglich des wortgleichen Kopierens des Dokuments.

\section*{7. GESAMTWERKE MIT UNABHÄNGIGEN WERKEN}

Ein Sammelwerk bestehend aus dem Dokument oder einem abgeleitetem Werk mit anderen getrennten
und unabhängigen Dokumenten oder Werken in oder auf einem Speicher- oder Vertriebsmedium
wird "`Gesamtwerk"' genannt, wenn das aus das Sammelwerk ausgehende Copyright
nicht dazu eingesetzt wird, die Rechte der Nutzer des Sammelwerks weiter einzuschränken,
als wie es durch die Einzwelwerke gestattet ist.

Wenn das Dokument in einem Gesamtwerk enthalten ist, dann schützt diese Lizenz nicht die
anderen Werke dieses Gesamtwerks, die nicht selbst abgeleitete Werke dieses Dokuments sind.

Wenn die Einbandtextbedingung aus Abschnitt~3 auf diese Kopien des Dokuments
zutrifft, dann dürfen, wenn das Dokument weniger als die Hälfte des Gesamtwerks ausmacht,
die Einbandtexte des Dokuments auf den Einband, der das Dokument innerhalb des Gesamtwerks
umgibt, aufgebracht werden, oder auf dem elektronischen Pendant eines Einbands, wenn das
Dokument in elektronischer Form vorliegt. Andernfalls müssen sie auf dem Einband erscheinen,
der das vollständige Gesamtwerk umgibt.

\section*{8. ÜBERSETZUNG}

Übersetzungen werden als Art der Bearbeitung angesehen, also dürfen Sie
Übersetzungen des Dokuments unter den Bedingungen von Abschnitt~4 verbreiten.
Unveränderliche Abschnitte zu übersetzen, bedarf besonderer Erlaubnis der 
Copyrightinhaber, aber Sie dürfen Übersetzungen einiger oder aller unveränderlicher
Abschnitte zusätzlich zu den Ursprungsversionen der unveränderlichen Abschnitte beifügen.
Sie dürfen eine Übersetzung dieser Lizenz, aller Lizenzvermerke des Dokuments
und aller Haftungsausschlüsse beifügen, solange auch die englischen Ursprungsversionen dieser
Lizenz und die Ursprungsversionen der Vermerke und Ausschlüsse enthalten sind.
Im Falle einer eines Widerspruchs zwischen der Übersetzung und der Ursprungsversion
dieser Lizenz oder eines Vermerks oder Haftungsausschlusses ist die Ursprungsversion
maßgebend.

Wenn ein Abschnitt im Dokument "`Acknowledgements"', "`Dedications"'
oder "`History"' benannt ist, erfordert es die Bedingung (Abschnitt~4) zum Erhalt des Titels
(Abschnitt~1) normalerweise, den eigentlichen Titel zu ändern.


\section*{9. BEENDIGUNG}

Sie dürfen das Dokument nicht kopieren, bearbeiten, unterlizenzieren oder verteilen, 
außer auf die Weise, wie es in dieser Lizenz ausdrücklich genehmigt ist. Jeder anderweitige Versuch
zu kopieren, bearbeiten, unterzulizenzieren oder zu verteilen ist nichtig und hat
das automatische Erlöschen der Ihnen unter dieser Lizenz gewährten Rechte zur Folge.

Wenn Sie jedoch alle Lizenzverletzungen einstellen, dann lebt Ihre Lizenz
von einem bestimmten Copyrightsinhaber wieder auf, und zwar (a) einstweilig,
wenn nicht und bis der Copyrightinhaber Ihnen die Lizenz ausdrücklich und endgültig
entzieht und (b) dauerhaft, wenn der Copyrightinhaber es versäumt, Sie vor 60 Tage
nach der Einstellung auf angemessene Weise über die Verletzung in Kenntnis zu setzen.

Weiterhin lebt die Lizenz von einem bestimmten Copyrightinhaber dauerhaft wieder auf,
wenn er Sie auf angemessene Weise über die Verletzung in Kenntnis setzt, dies 
das erste Mal ist, dass Sie Kenntnis über die Verletzung dieser Lizenz (für ein beliebiges Werk) von diesem 
Copyrightinhaber erhalten und wenn Sie die Lizenzverletzung vor 30 Tage nach der
Kenntniserlangung heilen.

Der Entzug Ihrer Rechte unter diesem Abschnitt hat nicht den Entzug der Lizenzen
derer zur Folge, die Kopien oder Rechte von Ihnen unter dieser Lizenz erhalten haben. 
Wenn Ihre Rechte aufgehoben wurden und nicht wieder dauerhaft wiederhergestellt worden sind,
gibt Ihnen der Empfang einer Kopie desselben Gesamtmaterials oder eines Teils davon 
kein Nutzungsrecht daran.

\section*{10. KÜNFTIGE AUSGABEN DIESER LIZENZ}

Die Free Software Foundation kann von Zeit zu Zeit neue, überarbeitete Versionen 
der GNU Free Documentation License veröffentlichen. Solche neuen Versionen werden
der gegenwärtigen Version im Geiste ähnlich sein, können sich aber
im Detail unterscheiden, um neue Probleme oder Anliegen anzugehen. Siehe
\url{http://www.gnu.org/copyleft/}.

Jeder Version der Lizenz wird eine kennzeichnende Versionsnummer gegeben.
Wenn im Dokument bestimmt ist, dass es durch eine bestimmte benummerte Version dieser
Lizenz "`oder jeder künftigen Version"' geschützt ist, haben Sie die Möglichkeit,
entweder die Bedingungen der angegebenen Version zu befolgen oder die jeder
folgenden Version, die von der Free Software Foundation (nicht als Entwurf)
veröffentlicht wurde. Wenn das Dokument keine Versionsnummer zu dieser Lizenz
angibt, dann dürfen Sie jede (nicht als Entwurf) je von der Free Software Foundation 
veröffentlichte Version wählen. Wenn das Dokument angibt, dass ein Stellvertreter
entscheiden darf, welche künftigen Versionen dieser Lizenz genutzt werden dürfen, 
dann ermächtigt eine öffentliche Erklärung des Stellvertreters zur Billigung einer
Version Sie dauerhaft, diese Version für das Dokument zu nutzen.

\section*{11. RELIZENZIERUNG}

Eine "`großangelegte Gemeinschaftsarbeit mehrerer Autoren enthaltende Website"' 
(oder "`gGmAe Website"') bezeichnet jeden
Server im World Wide Web, der copyrechtlich schützbare Werke veröffentlicht
und außerdem auffällige Vorrichtungen für jedermann zur Bearbeitung dieser Werke bietet. Ein öffentliches
Wiki, das jeder bearbeiten kann, ist ein Beispiel für einen solchen Server.
Eine auf der Website enthaltene "`Großangelegte Gemeinschaftsarbeit mehrerer Autoren"' (oder "`GGMA"')
bezeichnet jede Sammlung copyrechtlich schützbarer Werke, die auf der gGmAe Website auf
diese Art veröffentlicht wurden.

Mit "`CC-BY-SA"' wird die Lizenz Creative Commons Attribution-Share Alike 3.0 bezeichnet,
veröffentlicht von der Creative Commons Corporation, einer gemeinnützigen
Gesellschaft mit Hauptsitz in San Francisco, California, sowie zukünftige
Copyleft-Versionen dieser Lizenz, die von derselben Organisation herausgegeben werden.

"`Einverleiben"' bedeutet, ein Dokument teilweise oder als Ganzes als Bestandteil 
eines anderen Dokuments zu veröffentlichen oder neu zu veröffentlichen.

Eine GGMA ist "`zur Relizenzierung infrage kommend"', wenn sie unter diese Lizenz fällt
und wenn alle Werke, die zuerst anderswo als in der GGMA unter dieser Lizenz veröffentlicht wurden,
und die dann als Ganzes oder zum Teil in die GGMA einverleibt wurden, 
(1.) keine Einbandtexte oder unveränderlichen Abschnitte hatten,
und (2.) vor dem 1. November 2008 einverleibt wurden.

Der Betreiber einer gGmAe Website darf eine auf der Website enthaltene GGMA 
auf derselben Website unter der CC-BY-SA jederzeit vor dem 1. August 2009 veröffentlichen,
vorausgesetzt, die GGMA ist zur Relizenzierung infrage kommend.

\section*{ADDENDUM: Wie man diese Lizenz für eigene Dokumente nutzt}

Um dieses Lizenz für ein Dokument zu verwenden, dass Sie geschrieben haben, 
fügen Sie dem Dokument eine Kopie der Lizenz bei und setzen Sie
folgende Copyright- und Lizenzvermerke gleich nach der Titelseite:

\bigskip
\begin{quote}
    Copyright \copyright{}  JAHR  IHR NAME.
    Ihnen ist es erlaubt, dieses Dokument unter den Bedingungen der GNU Free Documentation License Version 1.3
    oder jeder künftigen Version, die von der Free Software Foundation herausgegeben wird, zu kopieren,
    verbreiten und/oder bearbeiten; ohne unveränderliche Abschnitte, ohne
    Vorderdeckeltexte und Rückdeckeltexte.
    Eine Kopie dieser Lizenz ist im Abschnitt namens "`GNU
    Free Documentation License"' enthalten.
\end{quote}
\bigskip
    
Wenn Sie unveränderliche Abschnitte, Vorderdeckeltexte und Rückdeckeltexte haben,
ersetzen Sie den Satzteil "`ohne \dots\ Rückdeckeltexte."' durch folgendes:

\bigskip
\begin{quote}
    mit den unveränderlichen Abschnitten DEREN TITEL AUFLISTEN, mit den 
    Vorderdeckeltexten LISTE und den Rückdeckeltexten LISTE.
\end{quote}
\bigskip
    
Wenn Sie unveränderliche Abschnitte und keine Einbandtexte haben, oder eine 
andere Konstellation, dann führen Sie die zwei alternativen Texte so zusammen, dass
Sie Ihrer Situation entsprechen.

Wenn Ihr Dokument nichttriviale Programmcodebeispiele enthält, empfehlen wir, 
diese Beispiele parallel unter einer freien Softwarelizenz Ihrer Wahl zu lizenzieren,
zum Beispiel der GNU General Public License, um ihre Verwendung in freier Software zuzulassen.