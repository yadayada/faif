\chapter{Eine Pfützevoll Freiheit}

[RMS: In diesem Kapitel habe ich Tatsachenbehautptungen korrigiert, einschließlich Fakten über meine Gedanken und Gefühle und einige grundlose Feindseligkeiten in den Beschreibungen der Ereignisse entfernt. Ich habe Williams' Meinungsäußerungen erhalten, Ausnahmen davon sind gekennzeichnet.]

Egal, wen man fragt, der mehr als eine Minute in Stallmans Gegenwart verbracht hat, jeder hat dieselbe Erinnerung: die langen Haare kann man vergessen. Das merkwürdige Verhalten auch. Das erste, was man bemerkt, ist der starre Blick. Ein Blick in Stallmans grüne Augen, und man weiß sich in der Anwesenheit eines wahren Gläubigen.

Stallmans Blick intensiv zu nennen, ist eine Untertreibung. Stallmans Augen sehen einen nicht einfach an; sie durchdringen einen. Selbst wenn man kurz aus Höflichkeit seine Augen von ihm ab richtet, bleiben Stallmans Augen auf einem heften\comment{, sizzling away at the side of your head an den Ohren wie zwei Photonenstrahlen}.

Vielleicht ist das der Grund, warum die meisten Autoren bei der Beschreibung Stallmans die religiöse Sichtweise wählen. Laut einem \textit{Salon.com}-Artikel von 1998 namens \citefield{title}{saint} von Andrew Leonard "`strahlen [Stallmans grüne Augen] die Kraft eines Propheten aus dem Alten Testament aus."'\footcite[Vgl.][]{saint} Ein \textit{Wired}-Artikel aus dem Jahr 1999 beschreibt den Stallmanschen Bart als "`rasputingleich"'\footcite[Vgl.][]{forgotten} und ein Profil im \textit{London Guardian} nennt Stallmans Lächeln das Lächeln eines "`Jüngers, der Jesus sieht."'\footnote{\cite[Vgl.][]{moralhigh}

Dies ist nur eine kleine Auswahl aus den religiösen Vergleichen. Bis dato kommt der extremste\comment{hyperlativ} Vergleich von Linus Torvalds, der in seiner Autobiographie – \citefield[S.\,58]{shorttitle}{tojff} – schreibt "`Richard Stallman ist der Gott der freien Software."'

Erwähnt sei auch noch Lawrence Lessig, der in einer Fußnote in seinem Buch – \citefield[S.\,270]{title}{futofideas} – Stallman mit Moses vergleicht:

\begin{quote}
\ldots wie bei Moses gab es einen anderen Führer, Linus Torvalds, der die Bewegung schließlich ins Gelobte Land brachte\comment{WIRR, indem er die Entwicklung des letzten Teils des Betriebssystemrätsels lösen half} [\ldots] Wie Mose wird Stallman gleichermaßen von Verbündeten innerhalb der Bewegung angesehen und verunglimpft. Er ist [ein] unversöhnlicher, und deswegen für viele inspirierender, Anführer für einen kritisch wichtigen Aspekt in unser modernen Kultur. Ich habe tiefen Respekt für die Prinzipien und das Engagement dieser außergewöhnlichen Persönlichkeit, obwohl ich auch großen Respekt für jene habe, die mutig genug sind, seine Überlegungen zu hinterfragen und dann seinen Zorn erleiden.
\end{quote}

In einem der letzten Gespräche fragte ich Stallman über seine Gedanken zu den religiösen Vergleichen. "`Einige Leute vergleichen mich mit einem Propheten aus dem Alten Testament; [und zwar] aus dem Grund, weil die Propheten im Alten Testament gesagt haben, dass bestimmte gesellschaftliche Praktiken falsch sind. Sie ließen sich in moralischen Fragen nicht auf Kompromisse ein. Sie ließen sich nicht kaufen, und man behandelte sie meist mit Verachtung."'}

Diese Vergleiche dienen einem Ziel, aber sie hinken. Weil sie die verletzliche Seite von Stallmans Persönlichkeit nicht berücksichtigen. Wenn man Stallmans Blick eine längere Zeit beobachtet, wird man eine subtile Feststellung machen. Was aussieht, wie ein Einschüchterungs- oder Hypnoseversuch, stellt sich beim zweiten oder dritten Hinsehen als frustrierter Versuch heraus, Kontakt aufzubauen und zu halten. Wenn seine Persönlichkeit eine Spur oder einen "`Schatten"' von Autismus oder Asperger-Syndrom hat, eine Möglichkeit, die Stallman von Zeit zu Zeit hegt, bekräftigen seine Augen sicherlich die Diagnose. Selbst auf ihrer Fernlichtstufe haben sie eine Tendenz, trübe und abwesend zu wirken wie die Augen eines verwundeten Tiers, kurz bevor es den Geist aufgibt.

Meine erste Begegnung mit dem legendären Stallman-Blick geht zurück auf die LinuxWorld Convention and Expo in San Jose, California, im März 1999. Was als "`Coming-out-Party"' für die "`Linux"'-Softwaregemeinde angekündigt war, ist außerdem die Veranstaltung, die Stallman wieder in die Technikmedien gebracht hat. Entschlossen, seinen gerechten Teil an Anerkennung zu bekommen, nutzte Stallman die Veranstaltung, um die Besucher wie die Reporter über die Geschichte des GNU Projects und seine offenen politischen Zielen zu unterrichten.

Als Reporter, der über das Treffen berichten sollte, bekam ich meine persönliche Lektion von Stallman während einer Pressekonferenz, in der die Veröffentlichung von GNOME 1.0 angekündigt wurde, einer freien\comment{O-Ton: graphical user interface} Desktopumgebung. Unbeabsichtigt habe ich bei ihm ein heikles Thema angesprochen, als ich meine erste Frage an ihn richtete: "`Glauben Sie, dass die Fertigstellung von GNOME Auswirkungen auf den kommerziellen Erfolg von Linux haben wird?"'

"`Ich bitte Sie, das Betriebssystem nicht mehr \glq Linux\grq{} zu nennen"', erwidert Stallman, und seine Augen visieren sofort die meinen an. "`Der Linux-Kernel ist nur ein kleiner Teil des Betriebssystems. Viele Programme, die das Betriebssystem ausmachen, das Sie \glq Linux\grq{} nennen, wurden überhaupt nicht von Linus Torvalds entwickelt. Sie wurden von Freiwilligen\comment{Ehrenamtlichen?} des GNU Projects entwickelt, die ihre persönliche Freizeit dafür geopfert haben, damit Nutzer eines Tages ein freies Betriebssystem haben können, wie es heute der Fall ist. Die Beiträge dieser Programmierer nicht anzuerkennen ist gleichermaßen unhöflich wie eine falsche Darstellung historischer Tatsachen. Deswegen bitte ich Sie, wenn Sie sich auf das Betriebssystem beziehen, es bei seinem korrekten Namen zu nennen: \glq GNU/Linux\grq.\,"'\index{GNU/Linux}

Während ich diese Worte in mein Notizbuch schreibe, bemerke ich eine schaurige Stille in dem gefüllten Raum. Als ich schließlich aufsehe, warten Stallmans unverwandte Augen auf mich. Schüchtern stellt ein zweiter Reporter eine Frage, und achtet darauf, auch den Begriff "`GNU/Linux"' statt "`Linux"' zu verwenden. Miguel de Icaza, Leiter des GNOME-Projekts, antwortet auf die Frage. Aber erst als de Icazas die Hälfte seiner Antwort gegeben hat, lösen sich Stallmans Augen endlich von meinen. Als das passiert, läuft mir ein leichter Schauer den Rücken herunter. Als Stallman anfängt, einen anderen Reporter über einen Fehler in seiner Ausdrucksweise zu belehren, fühle ich einen Anflug von Erleichterung. Wenigstens schaut er mich nicht an, denke ich bei mir.

Für Stallman dienen solche direkten Konfrontationen ihrem Zweck. Bis zum Ende der ersten LinuxWorld-Konferenz wissen die meisten Reporter, den Begriff "`Linux"' in seiner Anwesenheit besser nicht zu verwenden, und Wired.com schreibt einen Artikel, in dem Stallman mit einem vorstalinistischen Revolutionär verglichen wird, der von Hackern und Unternehmern aus den Geschichtsbüchern getilgt wurde, um die allzu politischen Ziele des GNU Projects herunterzuspielen.\footcite[Vgl.][]{forgotten} Andere Artikel sollten folgen und während nur wenige Reporter das Betriebssystem im Print \glq GNU/Linux\grq{} nennen, schreiben prompt viele Stallman die Ehre zu, den Start zur Entwicklung eines freien Betriebssystems vor 15 Jahren angeschoben zu haben.

Ich sollte Stallman erst in 17 Monaten wiedersehen. In der Zwischenzeit besucht Stallman das Silicon Valley noch einmal im August 1999 zur LinuxWorld. Obwohl er keinen Vortrag hält, gelingt es Stallman, den besten Spruch auf der Veranstaltung zu liefern. Bei der Dankesrede zum Empfang des Linus Torvalds Awards für Community Service\footnote{gemeinnützige Arbeit} im Namen der Free Software Foundation\comment{ – einem Preis, der nach Linux-Schöpfer Linus Torvalds benannt ist –} witzelt Stallman: "`Den Linus Torvalds Award an die Free Software Foundation zu verleihen, ist wie als würde man den Han-Solo-Preis an die Allianz der Rebellen verleihen."'

Dieses mal jedoch erzeugt der Kommentar kaum Rauschen im Blätterwald. Mitte der Woche geht Red Hat, Inc., ein bekannter GNU/Linux-Distributor, an die Börse. Die Nachrichten bestätigen nur noch, was viele Reporter\comment{wie ich} schon vermuten: "`Linux"' ist ein Modewort an der Wall Street geworden, ähnlich wie davor "`E-Commerce"' und "`Dot-com"'. Die Rede von freier Software und Open Source als politisches Phänomen bleibt auf der Strecke, als sich die Börse der Jahr-2000-Umstellung nähert wie eine Hyperbel ihrer vertikalen Asymptote.

Vielleicht ist das der Grund, warum Stallman zur dritten LinuxWorld im August 2000 auffallend abwesend ist.

Meine zweite Begegnung mit Stallman und seinem bezeichnenden Blick kommt kurz nach der dritten LinuxWorld. Als ich höre, dass Stallman ins Silicon Valley kommt, arrangiere ich ein Interview und Mittagessen in Palo Alto, California. Das Treffen scheint ironisch, nicht nur weil er nicht auf der Konferenz erscheint, sondern auch wegen dem Hintergrund insgesamt. Außer Redmond, Washington, gibt es nur wenige Städte, die ein deutlicheres Zeugnis vom wirtschaftlichen Wert der proprietären Software ablegen. Neugierig, Stallman zu sehen, den Mann, der den Großteil seines Lebens damit verbracht hat, gegen die Schwäche für Gier und Selbstsucht in unserer Kultur zu wettern, der in einer Stadt zurechtkommen muss, wo selbst Bungalows in der Größe einer Garage eine halbe Million Doller kosten, mache ich mich in Oakland auf die Fahrt.

Ich folge der Wegbeschreibung, die Stallman mir gegeben hat, bis ich am Hauptsitz von Art.net ankomme, einem gemeinnützigen "`Kollektiv virtueller Künstler"'. Der heckenumrandete Hauptsitz im Nordteil der Stadt ist erfrischend heruntergekommen. Plötzlich scheint die Vorstellung, das Stallman im Herzen des Silicon Valley lauert, gar nicht mehr so erstaunlich.

Ich finde Stallman sitzend in einem abgedunkeltem Raum auf, er tippt auf seinem grauen Laptop herum. Er schaut auf, als ich den Raum betrete, und richtet seinen 200-Watt-Blick direkt auf mich. Als er mit einem beruhigendem "`Hallo."' grüßt, grüße ich zurück. Bevor ich etwas sagen konnte, waren seine Augen schon wieder auf den Bildschirm gerichtet.

"`Ich schreibe gerade einen Artikel über den Geist des Hackens zu Ende"', sagt Stallman, während er weitertippt. "`Sehen Sie sich's mal an."'
Ich sehe es mir an. Der Raum ist schwach beleuchtet und der Text erscheint in grünlich-weißen Lettern auf schwarzem Hintergrund, eine Umkehrung des Farbschemas, das die meisten Desktop-Textverarbeitungsprogramme nutzen.\comment{, so it takes my eyes a moment to adjust. When they do,} Ich lese einen Bericht Stallmanns über ein Essen in einem koreanischen Restaurant. Vor dem Essen macht Stallman eine interessante Entdeckung: die Person, die den Tisch gedeckt hat, hat sechs Essstäbchen statt der üblichen zwei zu seinem Gedeck gelegt. Während die meisten Restaurantbesucher die überflüssigen zwei Paar ignoriert hätten, sieht Stallman es als Herausforderung an, einen Weg zu finden, wie er alle sechs Essstäbchen auf einmal nutzen kann. Wie bei vielen Softwarehacks ist die Lösung clever und albern gleichermaßen. Daher verwendet Stallman sie als  Veranschaulichung.

Als ich die Geschichte lese, merke ich Stallman mich aufmerksam beobachten. Ich sehe hinüber und sehe ein stolzes, aber kindliches halbes Lächeln auf seinem Gesicht. Als ich das Essay lobe, hebt er nur ein wenig seine Augenbrauen.

"`Ich bin gleich fertig und wir können losgehen"', sagt er.
Stallman tippt wieder weiter auf seinem Laptop. Der Laptop ist grau und klobig, anders als die glatten, modernen Laptops, die auf der letzten LinuxWorld unter den Programmierern am beliebtesten schienen. Über der Tastatur ist eine kleinere, leichtere Tastatur, ein Beweis Stallmans alternder Hände. Mitte der 1990s sind die Schmerzen in Stallmans Händen so unerträglich geworden, dass er eine Schreibkraft anstellen musste. Heute ist er auf eine Tastatur angewiesen, deren Tastendruck weniger Kraft erfordert als eine normale Tastatur.

Stallman hat die Tendenz, alle äußeren Reize auszublenden, während er arbeitet. Wenn man seine Augen auf den Bildschirm fixiert und seine Finger tanzen sieht, macht das den Eindruck wie zwei alte Freunde, die sich in einem Gespräch vertieft haben.

Die Sitzung endet mit einigen lauten Tastendrücken und der langsamen Zerlegung des Laptops.
"`Fertig fürs Mittagessen?"', fragt Stallman.

Wir laufen zu meinem Auto. Mit seinem entzündeten Knöchel humpelt Stallman langsam vorwärts. Stallman meint, die Verletzung kommt von einer Sehne im linken Fuß. Er hat sie seit drei Jahren und sie ist so schlimm geworden, dass er als großer  Volkstanz-Fan gezwungen war, seine Tanzaktivitäten ganz aufzugeben. "`Ich liebe Volkstanz"', klagt Stallman. "`Nicht tanzen zu können ist eine Tragödie für mich."'

Stallmans Körper ist Zeugnis dieser Tragödie. Der Mangel an körperlicher Ertüchtigung hat ihm dicke Backen und einen Kugelbauch eingebracht, der noch im letzten Jahr sehr viel weniger sichtbar war. Man kann sehen, dass die Gewichtszunahme drastisch war, weil, wenn Stallman läuft, er seinen Rücken krümmt wie eine schwangere Frau, die mit der ungewohnten Last zurechtkommen muss.

Der Gang wird noch weiter gebremst, als er buchstäblich anhält, um die Rosen zu riechen. Er erblickt eine besonders schöne Blüte und reibt seine Nase gegen die innersten Blätter, holt tief Luft, macht einen Schritt zurück und seufzt zufrieden.
"`Mmmh, Rhinophytophilie"', sagt er, und reibt sich den Rücken.\footnote{Zu der Zeit dachte ich, Stallman würde sich auf den wissenschaftlichen Namen der Pflanze beziehen. Monate später fand ich heraus, dass \textit{Rhinophytophilie} in Wirklichkeit eine humorvolle Bezeichnung der Aktivität ist – d.\,h. wenn Stallman seine Nase in die Blume steckt und den Moment genießt – eine Darstellung als abartige nasale Sexualpraktik mit Pflanzen. Ein weiterer humorvoller Stallmanscher Vorfall Blumen betreffend: \cite{rmsdallas}.}

Die Fahrt zum Restaurant dauert weniger als drei Minuten. Auf Empfehlung Tim Neys, einstiger Executive Director der Free Software Foundation, habe ich Stallman das Restaurant aussuchen lassen. Während einige Reporter sich auf Stallmans mönchsgleichen Lebensstil konzentrieren, ist es doch so, dass Stallman ein überzeugter Genießer ist, wenn es ums Essen geht. Eine der Lohnnebenleistungen eines reisenden Missionärs für freie Software ist die Möglichkeit, delikates Essen aus aller Welt kosten zu dürfen. "`Du kannst fast jede größere Stadt der Welt besuchen und die Chancen stehen gut, dass Richard das beste Restaurant dort kennt"', sagt Ney. "`Richard ist auch sehr stolz darauf, zu wissen, was auf der Speisekarte steht und für den ganzen Tisch zu bestellen."' (Natürlich nur, wenn die anderen es wollen.)

Für das heutige Mahl hat Stallman ein Restaurant ausgesucht, dass Dim Sum auf kantonesische Art zubereitet, zwei Blocks von der University Avenue entfernt, der Hauptstraße Palo Altos. Die Wahl ist inspiriert von Stallmans jüngstem Chinabesuch mit Zwischenstop in Hong Kong, dazu kommt Stallmans persönliche Abneigung gegenüber der schärferen Hunanesischen und Sichuanesischen Küche. "`Ich bin kein großer Freund von scharfem Essen"', gibt Stallman zu.

Wir kommen ein paar Minuten nach 11 Uhr an und müssen 20 Minuten warten. Wegen der Abneigung gegenüber ungenutzter Zeit unter Hackern halte ich kurz den Atem an, und fürchte einen Wutanfall. Stallman nimmt es, anders als ich erwartet hatte, gelassen.

"`Es ist wirklich schade, dass wir niemand anderen finden konnten, der uns begleitet"', sagt er, "`Es macht immer mehr Spaß, mit einer Gruppe von Leuten zu essen."'

Während der Wartezeit übt Stallman einige Tanzschritte. Seine Bewegungen sind zaghaft, aber gekonnt. Wir reden über aktuelle Ereignisse. Stallman sagt, dass er es nur bedauert, nicht an der LinuxWorld teilgenommen zu haben, weil er so die Pressekonferenz verpasst hat, in der Gründung der GNOME Foundation angekündigt wurde. Die von Sun Microsystems und IBM unterstützte Stiftung ist in vielerlei Hinsicht eine Bestätigung für Stallman, der lange den Standpunkt vertreten hat, dass freie Software und freier Markt sich nicht notwendigerweise gegenseitig ausschließen. Trotzdem ist Stallman unzufrieden mit der Botschaft, die angekommen ist.

"`So wie es dargestellt war, haben die Firmen nur von Linux geredet, ohne das GNU Project überhaupt zu erwähnen"', sagt Stallman.

Solche Enttäuschungen stehen im Gegensatz zu den herzlichen Reaktionen aus dem Ausland, besonders Asien, bemerkt Stallman. Ein schneller Blick auf seinen Reiseplan von 2000 zeigt die wachsende Popularität der Idee von freier Software. Zwischen den letzten Reisen nach Indien, China und Brasilien hat Stallman nur 12 der letzten 115 Tage auf US-amerikanischen Boden verbracht. Seine Reisen haben ihm die Möglichkeit gegeben, zu sehen, wie sich das Konzept der freien Software in verschiedene Sprachen und Kulturen übertragen lässt.

"`In Indien sind die Leute sehr an freier Software interessiert, weil sie sie als Mittel beim Aufbau ihrer Computerinfrastruktur für wenig Geld betrachten"', sagt Stallman. "`In China setzt sich das Konzept viel langsamer durch. Freie Software mit freier Rede zu vergleichen, ist viel schwerer, wenn man nicht das Recht auf freie Rede hat. Trotzdem war das Interesse an freier Software bei meinem letzten Besuch enorm."'

Das Gespräch verlagert sich auf Napster\index{Napster|(}, das Softwareunternehmen in San Mateo, California, das in den letzten Monaten eine Cause célèbre in den Medien geworden war. Die Firma vertrieb\comment{O-Ton: vertreibt} ein umstrittenes Programm, mit dem Musikfreunde die Musikdateien anderer durchsuchen und kopieren konnten. Dank des Internets\comment{magnifying powers of the Internet hat sich dieses sogenannte "`Peer-to-peer"'-Programm de facto zu einer Online-Musicbox entwickelt, und es} eröffnet es dem normalen Musikfreund einen Weg, MP3-Dateien am Computer zu hören, ohne Tantiemen oder eine Gebühr zu bezahlen, sehr zum Leidwesen der Unternehmen.

\comment{Obwohl es proprietäre Software war, zieht }Napster bestätigt eine von Stallman lange gehegte Behauptung, dass wenn ein Werk einmal in das digitale Reich übertritt\comment{– in other words, once making a copy is less a matter of duplicating sounds or duplicating atoms and more a matter of duplicating information –}, der menschliche Drang, das Werk weiterzugeben, schwerer einzuschränken wird. Statt weitere Beschränkungen aufzuerlegen, hat der Napster-Vorsand sich entschlossen, diesen Impuls auszunutzen. Die Firma gab Musikliebhabern einen zentralen Ort, um Musik zu tauschen, und spekulierte darauf, die angezogenen Nutzer auf andere kommerzielle Angebote umzuleiten.

Der plötzliche Erfolg des Napster-Modells hatte die traditionellen Plattenfirmen in Schrecken versetzt, und das aus gutem Grund. Kurz vor meinem Treffen in Palo Alto mit Stallman hatte Marilyn Pate, Richterin am U.S. District Court, einem Antrag der RIAA\footnote{Recording Industry Association of America} auf eine Verfügung gegen den Filesharingdienst stattgegeben. Die Verfügung wurde später von einem Bundesberufungsgericht\comment{U.S. Ninth District Court of Appeals} aufgehoben, aber Anfang 2001 befand auch das Berufungsgericht die in San Mateo ansässige Firma für schuldig, Urheberrechte zu verletzen,\footcite[Vgl.][]{napinjunc} eine Entscheidung, die die RIAA-Sprecherin Hillary Rosen später als "`klaren Sieg für die kreative Gemeinschaft und den legitimen Online-Markt"' bezeichnen würde.\footcite[Vgl.][]{napstervictory}

Für Hacker wie Stallman war das Geschäftsmodell von Napster auf verschiedene Art problematisch.\comment{Die Bestrebungen der Firma, sich uralte Hackerprinzipien wie Filesharing und gemeinsames Informationseigentum anzueignen und gleichzeitig einen entgeltlichen Dienst auf Basis proprietärer Software anzubieten, geben ein erschreckend gemischtes Bild.} Als jemand, der es schon schwer genug hat, seine eigene sorgfältig formulierte Botschaft in die Medien zu bringen, ist er verständlicherweise zurückhaltend, wenn es darum geht, sich über die Firma zu äußern. Trotzdem gibt Stallman zu, ein oder zwei Dinge von der sozialen Seite des Napster-Phänomens gelernt zu haben.

"`Vor Napster dachte ich, es wäre vielleicht [genug] für die Leute, Unterhaltungswerke privat zu tauschen"', sagt Stallman. "`Aber die Anzahl an Leuten, die Napster\index{Napster|)} nützlich finden, sagt mir, dass das Recht zum Weiterverbreiten von Kopien nicht nur auf eine nachbarliche Weise, sondern unter der Allgemeinheit insgesamt, essenziell ist und deshalb nicht wegfallen darf."'

Gerade, als Stallman das sagt, öffnet sich die Restauranttür und wir werden vom Besitzer wieder hereingebeten. Kurz danach sitzen wir in einer Ecke\comment{side corner ???} neben einer großen Spiegelwand.
Die Speisekarte ist gleichzeitig auch Bestellformular und Stallman hat schnell seine Kreuze gemacht, bevor der Gastgeber das Wasser an den Tisch bringen konnte. "`Fritierte Shrimps in Tofu-Haut gewickelt"', sagt Stallman. "`Tofu-Haut, das ist so eine interessante Textur. Ich denke, das sollten wir bestellen."'

Diese Bemerkung führt zu einer spontanen Diskussion über chinesisches Essen und Stallmans Chinareise. "`Das Essen in China ist absolut exquisit"', sagt Stallman, und seine Stimme gewinnt zum ersten Mal diesen Morgen etwas an Gefühl. "`So viele verschiedene Dinge, die ich noch nie in den USA gesehen habe, lokale Sachen aus lokalen Pilzen und Gemüsesorten. Es ist dazu gekommen, dass ich angefangen habe, ein Tagebuch zu führen, nur um nicht den Überblick über jede dieser wundervollen Mahlzeiten zu verlieren."'

Das Gespräch geht nahtlos in eine Diskussion über die koreanische Küche über. Im Zuge seiner Vortragsreise durch Asien im Juni 2000 war in er Südkorea. Seine Ankunft entfachte einen kleinen Sturm in den lokalen koreanischen Medien wegen einer Softwarekonferenz, die Microsoft-Gründer und damaliger Vorsitzender Bill Gates in derselben Woche besucht hatte. Abgesehen davon, dass sein Bild in der führenden Zeitung Seouls über dem von Gates erschien, sagt Stallman, war das Beste an dem Trip das Essen. "`Ich habe eine Schale Naengmyeon gegessen, das sind kalte Nudeln. Die Nudeln haben sich sehr interessant angefühlt. An den meisten Orten bekommt man nicht dieselben Nudeln im Naengmyeon, also kann ich mit völliger Gewissheit sagen, dass das die exquisitesten Naengmyeon waren, die ich jemals hatte."'

Der Begriff "`exquisit"' ist ein großes Lob aus Stallmans Mund. Das weiß ich, weil er einige Momente nach seiner Schwärmerei über Naengmyun seinen laserscharfen Blick über meine rechte Schulter wirft.

"`Da sitzt die exquisiteste Frau hinter Ihnen"', sagt Stallman.
Ich drehe mich um, sehe kurz den Rücken der Frau. Die Frau ist jung, Mitte 20, und trägt ein weißes Paillettenkleid. Sie und ihre männliche Begleitung bezahlen gerade die Rechnung. Dass beide aufstehen und das Restaurant verlassen, weiß ich, ohne hinzusehen, weil Stallmans Blick plötzlich weniger intensiv wird.
"`O nein"', sagt er, "`Sie sind gegangen. Ich darf gar nicht daran denken, dass ich sie wahrscheinlich nie wieder sehen werde."'

Nach einem kurzen Moment erholt sich Stallman. Der Moment gibt mir Gelegenheit, Stallmans Ruf beim schönen Geschlecht vis-a-vis zu besprechen. Der Ruf ist manchmal etwas widersprüchlich. Eine Anzahl Hacker berichtet von Stallmans Vorliebe, Frauen mit einem Kuss auf den Handrücken zu begrüßen.\footcite[Vgl.][]{maeling}
%
\comment{Bis jetzt war Mak die einzige Person, die sich mir gegenüber öffentlich zu dieser Praktik äußern wollte, obwohl ich es auch von einigen anderen weiblichen Quellen gehört habe. Mak, obwohl anfangs davon angewidert, hat später ihre unguten Gefühle überwunden und 1999 mit Stallman auf einer Veranstaltung der LinuxWorld getanzt.} 

Ein Artikel vom 26. Mai 2000 auf \textit{Salon.com} stellt Stallman jedoch als einen kleinen Schwerenöter dar. Annalee Newitz berichtet über die Verbindung von freier Software und freier Liebe und präsentiert Stallman als Gegner der "`traditionellen Familienwerte"': "`Ich glaube an Liebe, aber nicht an Monogamie."'\footcite[Vgl.][]{freecodefreeme}

Stallman lässt seine Speisekarte etwas sinken, als ich das Thema anspreche. "`Die meisten Männer scheinen Sex zu wollen und scheinen eine ziemlich verachtende Haltung gegenüber Frauen zu haben"', sagt er. "`Selbst den Frauen gegenüber, mit denen sie eine Beziehung haben. Das kann ich überhaupt nicht verstehen."'

Ich erwähne eine Passage aus dem 1999 erschienenen Buch \textit{Open Sources}, in der Stallman zugibt, den GNU-Kernel nach einer damaligen Freundin benennen zu wollen. Der Name der Freundin war Alix\index{Alix}, ein Name, der perfekt in das Benennungsschema von Unix-Entwicklern passt, wobei dem Ende von Systemen und Kerneln ein "`x"' angehängt wird – z.\,B. "`Linux"'. Alix war eine Unix-Systemadministratorin und schlug ihren Freunden vor "`Jemand sollte einen Kernel nach mir benennen"'. Also entschloss sich Stallman, den GNU-Kernel als Überraschung "`Alix"' zu nennen. Der Hauptentwickler des Kernels nannte den Kernel in "`Hurd"' um, behielt aber den Namen "`Alix"' für ein Subsystem bei. Ein Freund von Alix bemerkte diesen Teil in einem Source-Snapshot und erzählte ihr davon, und sie war gerührt. Eine spätere Umgestaltung von Hurd eliminierte diesen Teil.\footcitet[Vgl.][S.\,65: Richard Stallman, \textit{The GNU Operating System and the Free Software Movement}]{opensrc} \comment{\footnote{[RMS: Williams interpretierte diese kurze Beschreibung so, als ob ich ein hoffnungsloser Romantiker wäre und dass meine Anstrengungen dazu dienen sollten, eine mir bis dahin unbekannte Frau zu beeindrucken. Kein MIT-Hacker würde so etwas glauben, weil wir in recht jungem Alter lernen, dass die meisten Frauen uns nicht für unsere Programmierarbeit beachten, und schon gar nicht lieben. Wir programmieren, weil es faszinierend war. Dieser Vorgang war nur möglich, weil ich zu dieser Zeit eine wohldefinierte Freundin hatte. Wenn ich ein Romantiker wäre, dann war ich damals weder ein hoffnungsloser Romantiker noch ein hoffnungsvoller Romantiker, sondern ein zeitweilig erfolgreicher.

Wegen dieser naiven Interpretation fing Williams dann an, mich mit Don Quijote zu vergleichen.

Der Vollständigkeit halber hier ein etwas wenig verständliches Zitat aus der ersten Ausgabe: "`Ich habe eigentlich nicht versucht, ein Romantiker zu sein. Es war eher ein Necken. Es war schon romantisch, aber auch neckisch, verstehen Sie? Es wäre eine reizende Überraschung geworden."']}}

Zum ersten Mal heute früh lächelt Stallman. Ich bringe das Thema mit den Handküssen auf. "`Ja, das mache ich"', sagt Stallman. "`Ich habe festgestellt, dass es ein Weg ist, etwas Zuneigung zu zeigen, und vielen Frauen gefällt es.\comment{ It's a chance to give some affection and to be appreciated for it.}"'

Zuneigung ist ein Faden, der sich rot durch Richard Stallmans Leben zieht, und er ist völlig offen, wenn solche Fragen aufkommen. "`Mir ist nicht viel Zuneigung widerfahren\comment{zugekommen} in meinem Leben, außer in meinem Kopf"', sagt er. Trotzdem wird das Gespräch schnell unangenehm. Nach einigen einsilbigen Antworten nimmt Stallman seine Speisekarte in die Hand und bricht das Verhör ab.
"`Wollen Sie ein paar Shumai?"', fragt er.

\comment{When the food comes out,}
Wie das Essen kommt, dreht sich das Gespräch um die jeweiligen Gänge. Wir reden über die oft festgestellte Vorliebe von Hackern für chinesisches Essen, die wöchentlichen Mittagessen in Bostons Chinatown-Bezirk während Stallmans Zeit als Programmierer am AI~Lab und die dem Chinesisch und seinem Schreibsystem zugrunde liegende Logik. Jeder Vorstoß meinerseits entlockt eine wohlinformierte Abwehr\comment{well-informed parry} von Seiten Stallmans.

\comment{tone: Klang oder Betonung?}
"`Ich habe ein paar Leute Shanghainesisch sprechen hören, als ich letztens in China war"', sagt Stallman. "`Das war interessant anzuhören. Es hört sich ziemlich anders an [als Mandarin]. Ich habe mir einige verwandte Wörter in Mandarin und Shanghainesisch sagen lassen. In einigen Fällen merkt man eine Ähnlichkeit, aber ich habe mich auch noch gefragt, ob die Betonung ähnlich sein würde. Ist sie nicht. Ich finde das interessant, weil es eine Theorie gibt, dass die Betonung sich aus zusätzlichen Silben entwickelt hat, die später verloren gingen und ersetzt wurden. Ihr Effekt bleibt in der Betonung erhalten. Wenn das wahr ist, und ich habe Behauptungen gehört, dass es zu historischen Zeiten passiert ist, dann müssen die Dialekte auseinandergegangen sein, bevor diese Endsilben verlorengingen."'

Der erste Gang, ein Teller gebratener Rübenkuchen, ist da. Stallman und ich brauchen einen Moment, die großen viereckigen Stücke zu zerschneiden, die nach gekochten Rüben riechen, aber wie in Speck fritierte Pfannkuchen schmecken.

Ich habe mich entschieden, das Außenseiter-Thema wieder aufzugreifen, und frage mich, ob Stallmans Teenagerjahre ihn konditioniert haben, unpopuläre Standpunkte einzunehmen, besonders seinen 1994 aufgenommenen Kampf gegen Computernutzer und die Medien, die gebräuchliche Bezeichnung "`Linux"' durch "`GNU/Linux"' zu ersetzen.

"`Ich glaube, [das Außenseiterdasein] hat mir geholfen, [mich nicht den populären Ansichten zu beugen]"', sagt Stallman auf einem Kloß kauend. "`Ich habe nie verstanden, warum gesellschaftlicher Druck so einen Einfluss auf die Leute hat. Ich glaube, bei mir lag der Grund darin, dass ich so völlig unpopulär war, deswegen hatte ich nichts zu gewinnen, wenn ich versucht hätte, den Trends zu folgen. Es hätte nichts geändert. Ich wäre genauso unbeliebt geblieben, also habe ich es nicht versucht."'

Stallman weist auf seinen Musikgeschmack als zentrales Beispiel seiner nonkonformen Tendenzen hin. Als Teenager hörten die meisten seiner Klassenkameraden in der High-School Motown und Acid Rock, Stallman bevorzugte klassische Musik. Die Erinnerung bringt ihn auf einen der wenigen witzigen Momente aus seinen Mittelstufenjahren. Nach dem Auftritt der Beatles in der Ed Sullivan Show 1964 sind die meisten von Stallmans Klassenkameraden losgestürzt, um die neuesten Beatles-Alben und -Singles zu kaufen. Genau dann, sagt Stallman, habe er sich entschieden, die Fab Four zu boykottieren.

"`Ich mochte einiges an Pop-Musik vor den Beatles"', sagt Stallman. "`Aber die Beatles konnte ich nicht leiden. Mir hat besonders die wilde Art missfallen, mit der die Leute auf sie reagiert haben. Es war so, als ob sie einen Wettbewerb darum veranstaltet haben, wer die Beatles am meisten vergöttert."'

Als der Boykott keinen Fuß fasste, suchte Stallman nach anderen Möglichkeiten, auf die Herdenmentalität seiner Gleichaltrigen aufmerksam zu machen. Stallman sagt, er habe kurz darüber nachgedacht, selbst eine Rockband zu gründen, die die Liverpooler Gruppe persifliert.

"`Ich wollte sie \glq Tokyo Rose and the Japanese Beetles\grq{} nennen."'

Bei seiner Vorliebe für internationale Folk-Musik frage ich Stallman, ob er ähnlich zu Bob Dylan und den anderen Folkmusikern der frühen 60er steht. Stallman schüttelt den Kopf. "`Ich mochte Peter, Paul and Mary"', sagt er. "`Das erinnert mich an ein großartiges Filk-Lied."'\index{Filk}

Als ich ihn nach der Definition von "`Filk"' frage, erklärt er mir, dass der Begriff im Bereich der Science Fiction für die Neuvertextung von Liedern steht.\comment{(In den letzten Jahrzehnten haben einige Filk-Schreiber auch Melodien geschrieben.)}  Klassische Filk-Lieder sind "`On Top of Spaghetti"', Neufassung von "`On Top of Old Smokey"'\footnote{Vgl. \url{http://de.wikibooks.org/wiki/Liederbuch:_On_Top_Of_Old_Smoky}} und "`Yoda"', eine an Star Wars angelehnte Interpretation des Lieds "`Lola"' der Kinks vom Filk-Meister "`Weird"' Al Yankovic.

Stallman fragt mich, ob ich Interesse hätte, einen Filk zu hören. Als ich bejahe, singt Stallman mit unerwartet klarer Stimme zu der Melodie von "`Blowin' in the Wind"':

\begin{verse}
How much wood could a woodchuck chuck,\\
If a woodchuck could chuck wood?\\
How many poles could a polak lock,\\
If a polak could lock poles?\\
How many knees could a negro grow,\\
If a negro could grow knees?\\
The answer, my dear,\\
is stick it in your ear.\\
The answer is, stick it in your ear\ldots
\end{verse}

Das Singen hört auf und Stallmans schürzt seine Lippen zu einem kindlichen halben Lächeln. Ich schaue umher zu den Nachbartischen. Die asiatischen Familien, die ihr sonntägliches Mittagessen genießen, scheinen dem bärtigen Altisten wenig Aufmerksamkeit zu schenken.\footnote{Stallmans eigenen Filk gibt es auf \url{http://www.stallman.org/doggerel.html}. Wer Stallman den "`The Free Software Song"' singen hören möchte, gehe auf \url{http://www.gnu.org/music/free-software-song.html}.} Nach einem kurzen Zögern lächele ich auch.

%?????????????
\comment{"`Do you want that last cornball?"' Stallman asks, eyes twinkling. Before I can screw up the punch line, Stallman grabs the corn-encrusted dumpling with his two chopsticks and lifts it proudly. "`Maybe I'm the one who should get the cornball,"' he says.}

Das Essen ist aufgegessen, unser Gespräch nimmt die Dynamik eines normalen Interviews an. Stallman lehnt sich in seinem Stuhl zurück und umschließt eine Tasse Tee mit seinen Händen. Wir sprechen weiter über Napster und den Bezug zur Free-Software-Bewegung. Sollten die Prinzipien der freien Software auf ähnliche Bereiche ausgedehnt werden, wie z.\,B. Musikveröffentlichungen, frage ich.

"`Es ist falsch, die Lösungen vom einen auf das andere zu übertragen"', sagt Stallman\comment{, contrasting songs with software programs}. "`Der richtige Ansatz ist, sich jede Art von Werken anzuschauen und zu sehen, zu welchen Schlussfolgerungen man kommt."'

Wenn es um urheberrechtlich geschützte Werke geht, nimmt Stallman eine Unterteilung in drei Kategorien vor. Die erste Kategorie umfasst "`funktionale"' Werke – d.\,h. Software, Wörterbücher und Lehrbücher. Die zweite umfasst Werke, die man als "`Zeugnisse"' bezeichnen kann – d.\,h., wissenschaftliche und historische Dokumente. Solche Werke dienen einem Zweck, der beschädigt würde, wenn sie später durch Leser oder Autoren verändert werden könnten. Eingeschlossen sind auch Werke von persönlichen Äußerungen – d.\,h. Tagebücher und Autobiographien. Solche Dokumente zu modifizieren wäre die Änderung der Erinnerungen oder Ansichten einer Person, was Stallman als ethisch nicht tragbar ansieht. Die dritte Kategorie beinhaltet Werke der Kunst und Unterhaltung.

Unter diesen Kategorien sollte die erste den Nutzern das uneingeschränkte Recht geben, modifizierte Versionen zu erstellen, und die zweite und dritte sollten dieses Recht nach der Intention des ursprünglichen Autors regulieren. Unabhängig von der Kategorie sollte die Freiheit zum nichtkommerziellen Kopieren und Weiterverbreiten zu jeder Zeit intakt bleiben, fordert Stallman. Wenn das bedeutet, dass Internetnutzer das Recht bekommen, Hundert Kopien eines Artikels, Bilds, Lieds oder Buchs zu machen und sie an Hundert Fremde per E-Mail zu verschicken, dann sei's drum. "`Es ist klar, dass gelegentliches privates Weiterverbreiten erlaubt sein muss, weil es nur ein Polizeistaat stoppen kann"', sagt Stallman. "`Es ist unsozial, sich zwischen Freunde zu stellen. Napster hat mich überzeugt, dass wir selbst nichtkommerzielle Verbreitung an die Allgemeinheit nur zum Vergnügen erlauben sollten, erlauben müssen. Weil es so viele Menschen wollen und es nützlich finden."'

Als ich frage, ob die Gerichte solch eine freizügige Einstellung akzeptieren würden, unterbricht mich Stallman.

"`Das ist die falsche Frage"', sagt er. "`Verstehen Sie, jetzt haben Sie komplett das Thema gewechselt von einem ethischen auf die Auslegung von Recht. Und das sind zwei völlig verschiedene Fragen im selben Bereich. Es hat keinen Sinn, von der einen zur anderen zu springen. Die Gerichte legen die bestehenden Gesetze ziemlich scharf aus, weil das so ist, wie die Verleger sie gekauft haben."'

Der Kommentar bringt Licht in Stallmans politische Philosophie: nur weil das Rechtssystem zur Zeit die Unternehmen darin bekräftigt, das Urheberrecht als Äquivalent von Grundbesitztum zu behandeln, heißt das nicht, dass Computernutzer diesen Regeln folgen müssen. Freiheit ist ein ethisches Thema, kein rechtliches. "`Ich sehe darüber hinaus, was das bestehende Recht ist, und darauf, wie es sein sollte"', sagt Stallman. "`Ich versuche nicht, Gesetze zu schreiben. Ich überlege mir, was das Gesetz tun sollte. Ich betrachte die Gesetze, die es verbieten, Kopien mit seinem Freund zu tauschen, als moralisches Äquivalent zu Jim Crow.\footnote{\textit{Jim Crow laws} bezeichnen ehemalige Rassentrennungsgesetze.} Sie verdienen keine Beachtung."'

Die Einbeziehung von Jim Crow wirft eine andere Frage auf. Inwieweit beeinflussen und inspirieren politische Größen der Vergangenheit Stallman? Wie die [Amerikanisch-Afrikanische] Bürgerrechtsbewegung in den 50ern und 60ern basiert sein Versuch, soziale Veränderungen voranzutreiben, auf einem Appell an ewige Werte: Freiheit, Gerechtigkeit und Fair play.

Stallman teilt seine Aufmerksamkeit zwischen meinem Vergleich und einer seiner besonders verknäuelten Haarsträhnen. Als ich die Analogie soweit treibe, ihn mit Dr. Martin Luther King zu vergleichen, unterbricht er mich, nachdem er ein gesplisstes Haarende abbricht und es sich in den Mund wirft.

"`Ich bin nicht in derselben Liga wie er, aber ich spiele dasselbe Spiel"', sagt er kauend.

Ich schlage Malcolm X als einen anderen Vergleichspunkt vor. Wie der frühere Führer der Nation of Islam hat sich Stallman einen Ruf dafür gemacht, Kontroversen zu suchen, mögliche Verbündete zu verprellen und eine Botschaft von Unabhängigkeit über die der kulturellen Integration zu stellen.

Auf dem nächsten Haar kauend weist Stallman auch diesen Vergleich ab. "`Meine Botschaft liegt näher an der von King"', sagt er. "`Es ist eine allgemeingültige Botschaft. Eine Botschaft von der starken Verurteilung gewisser Praktiken, die andere ungerecht behandeln. Es ist keine Botschaft vom Hass gegen irgendwen. Und sie ist nicht auf eine schmale Gruppe von Leuten gerichtet. Ich ermutige jeden, Freiheit zu schätzen und Freiheit zu haben."'

Viele kritisieren Stallman für seine Weigerung, naheliegende politische Bündnisse einzugehen; manche stellen das auf eine psychologische Ebene und bezeichnen es als einen Charakterzug. Bei seiner wohlbekannten Abneigung gegen den Begriff "`Open Source"' scheint die Weigerung\comment{to participate in recent coalition-building projects} verständlich. Als jemand, der die letzten zwei Jahrzehnte mit dem Wahlkampf für freie Software verbracht hat, ist Stallmans politisches Kapital tief in den Begriff versenkt. Trotzdem haben Bemerkungen wie der "`Han-Solo"'-Vergleich auf der LinuxWorld 1999 nur Stallmans Ruf\comment{amongst those who believe virtue consists of following the crowd,} verstärkt, ein missmutiger Reaktionär zu sein, der sich weigert, den politischen oder Marketingtrends zu folgen.

"`Ich bewundere und respektiere Richard für die Arbeit, die er geleistet hat"', fasst Red-Hat-Präsident Robert Young Stallmans paradoxes politisches Verhalten zusammen. "`Meine einzige Kritik ist, dass Richard manchmal seine Freunde schlimmer behandelt als seine Feinde."'

[RMS: Der Begriff "`Freunde"' passt nur teilweise auf Leute wie Young und Firmen wie Red Hat. Er passt zu einigem, was sie getan haben und noch tun: zum Beispiel trägt Red Hat zur Entwicklung freier Software bei, einschließlich einiger GNU-Programme. Aber Red Hat macht andere Dinge, die gegen die Ziele der Free-Software-Bewegung gehen – zum Beispiel enthalten ihre Versionen von GNU/Linux unfreie Software. Wechseln wir von den Taten zu den Worten; das ganze System als "`Linux"' zu bezeichnen ist eine unfreundliche Haltung gegenüber dem GNU Project und für "`Open Source"' statt "`freie Software"' zu werben, geht gegen unsere Werte. Ich könnte mit Young und Red Hat arbeiten, wenn sie in die gleiche Richtung gehen würden wie ich, aber das war nicht oft genug der Fall, um sie zu möglichen Verbündeten zu machen.]

Stallmans Widerstreben, die Free-Software-Bewegung mit anderen politischen Bewegungen zu verbünden, kommt nicht von einem fehlenden Interesse an ihnen. Wenn man in seinen Büros im MIT vorbeischaut, wird man sofort eine ganze Sammlung an linkslastigen Nachrichtenartikeln finden, die Bürgerrechtsverletzungen auf der ganzen Welt thematisieren. Besucht man seine persönliche Webseite \href{http://stallman.org}{stallman.org}, findet man Angriffe auf den Digital Millennium Copyright Act, den War on Drugs und die Welthandelsorganisation. Stallman erklärt "`Wir müssen vorsichtig sein, dass die Free-Software-Bewegung nicht Bündnisse mit politischen Organisationen eingeht, mit denen ein großer Teil der Unterstützer freier Software nicht übereinstimmt. Zum Beispiel vermeiden wir jede Verbindung der Free-Software-Bewegung mit einer politischen Partei, weil wir die Unterstützer und gewählten Amtspersonen anderer Parteien nicht vertreiben wollen."'

Bei seinen Tendenzen zum Aktivismus frage ich ihn, warum er sich nicht mehr Gehör verschafft. Warum er seine Präsenz\comment{visibility} in der Hackerwelt nicht als Plattform nutzt, um seinen politischen Ansichten besser Gehör zu verschaffen? [RMS: Aber das mache ich doch – wenn ich eine gute Gelegenheit sehe. Deswegen habe ich \href{http://stallman.org}{stallman.org} auf die Beine gestellt.]

Stallman lässt sein knotiges Haar fallen und überlegt einen Moment über die Frage. [RMS: Meine zitierte Antwort passt nicht zu der Frage. Sie passt aber zu einer anderen Frage: "`Warum konzentrieren Sie sich auf freie Software und nicht die anderen Dinge, an die Sie glauben?"'  Ich vermute, die gestellte Frage ging mehr in diese Richtung.]

"`Ich möchte ungern die Wichtigkeit dieser kleinen Pfützevoll Freiheit überbetonen"', sagt er. "`Weil die besser bekannten und herkömmlichen Bereiche bei der Arbeit für Freiheit und eine bessere Gesellschaft extrem wichtig sind. Ich würde nicht sagen, dass freie Software genauso wichtig ist wie sie. Es ist eine Verantwortung, die ich übernommen haben, weil sie mir in den Schoß gefallen ist und ich eine Möglichkeit gesehen habe, wie ich etwas in der Richtung unternehmen kann. Aber zum Beispiel der Polizeigewalt und dem War on Drugs ein Ende zu setzen, alle Arten Rassismus auszulöschen, die es noch gibt, jedem zu helfen, ein angenehmeres Leben zu führen, die Rechte der Leute zu beschützen, die abtreiben, uns vor einer Theokratie zu beschützen, das sind die immens wichtigen Probleme, viel wichtiger als das, was ich mache. Ich wünschte, ich wüsste, wie man da etwas unternehmen könnte."'

%% erster Satz ???
\comment{Wieder präsentiert Stallman seine politischen Aktivitäten als Funktion seines persönlichen Zutrauens.} Bei dem Zeitaufwand, den es ihn gebraucht hat, die Kernthesen der Free-Software-Bewegung zu entwickeln und auszufeilen, glaubt Stallman kaum, dass er andere politische Ziele vorantreiben kann, die er unterstützt.

"`Ich wünschte, ich wüsste, wie man bei den größeren Problemen etwas verändern kann, weil ich unglaublich stolz wäre, wenn ich das schaffen würde, aber sie sind einfach zu schwer und viele Leute, die das wahrscheinlich besser können, arbeiten daran und sind nicht sehr weit gekommen. Aber so wie ich das sehe, habe ich eine andere Bedrohung ohne Abwehr erkannt, während die anderen gegen die großen sichtbaren Gefahren gekämpft haben. Und dann bin ich gegen diese Bedrohung ins Feld gezogen. Es mag nicht die größte Bedrohung sein, aber ich war der einzige [um ihr entgegenzutreten]."'

Auf seinem letzten Spliss kauend, schlägt Stallman vor, zu zahlen. Bevor der Kellner das Geld mitnehmen kann, zieht Stallman eine weiße Dollarnote hervor und wirft sie auf den Stapel. Der Geldschein sieht so offensichtlich gefälscht aus, dass ich ihn mir ansehen muss. Natürlich kommt er nicht von der US-Notenbank. Statt dem Konterfei eines George Washington oder Abe Lincoln ist ein Cartoon-Schwein auf der Vorderseite abgebildet. Statt "`United States of America"' steht in dem Banner über dem Schwein "`Untied Status of Avarice"'.\footnote{loser Zustand der Habgier} Der Schein hat einen Wert von null Dollar und, als der Kellner das Geld nimmt, zieht Stallman ihm am Ärmel.\footnote{RMS: Williams irrt sich, wenn er die Note als "`gefälscht"' bezeichnet. Sie ist ein gesetzliches Zahlungsmittel mit einem Wert von null Dollar zur Bezahlung jedweder Geldschuld. \comment{Jeder US-Beamter wird den Gegenwert für die null Dollar in Gold auszahlen.}}

"`Ich habe Ihnen etwas extra zukommen lassen"', sagt Stallman und noch ein halbes Lächeln huscht über seine Lippen.
Der Kellner lächelt verständnislos oder durch den Geldschein getäuscht und eilt davon.
"`Ich glaube, das heißt, dass wir jetzt gehen dürfen"', sagt Stallman.
